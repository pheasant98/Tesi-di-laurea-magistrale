\chapter{Codice in Python}
\section{createTable}
Codice per la creazione del \emph{dataset} utilizzabile con i metodi di \emph{Machine Learning}.
\begin{lstlisting}[language=Python, caption={Codice per la creazione del dataset utilizabile con i metodi di Machine Learning.}, captionpos=b, label=code:a9]
	def createTable(df, dfM): 
		for i in range(0, df.shape[0], 1):
			if df.AtHome[i] == True: 
				for j in range(0, df.shape[0], 1): 
					if df.Team[j] == df.Vs[i] and df.Team[i] == df.Vs[j] and df.AtHome[j] == False:
					dfM2 = pd.DataFrame({'Date': [df.Date[i]],
						'Round': [df.Round[i]],
						'AtHome': [df.AtHome[i]],
						'Result': [df.Res[i]],
						'G_Home': [df.GF[i]],
						'G_Away': [df.GA[i]],
						'Home_Team': [df.Team[i]],
						'Away_Team': [df.Vs[i]],
						'Home_Poss': [df.Poss[i]],
						'Home_Sh': [df.Sh[i]],
						'Home_SoT': [df.SoT[i]],
						'Home_G/Sh': [df['G/Sh'][i]],
						'Home_Saves': [df.Saves[i]],
						'Home_PAtt': [df.PAtt[i]],
						'Home_PCmp%': [df['PCmp%'][i]],
						'Home_SPAtt': [df.SPAtt[i]],
						'Home_SPCmp%': [df['SPCmp%'][i]],
						'Home_MPAtt': [df.MPAtt[i]],
						'Home_MPCmp%': [df['MPCmp%'][i]],
						'Home_LPAtt': [df.LPAtt[i]],
						'Home_LPCmp%': [df['LPCmp%'][i]],
						'Home_ToDefPen': [df.ToDefPen[i]],
						'Home_ToDef3rd': [df.ToDef3rd[i]],
						'Home_ToMid3rd': [df.ToMid3rd[i]],
						'Home_ToAtt3rd': [df.ToAtt3rd[i]],
						'Home_ToAttPen': [df.ToAttPen[i]],
						'Home_TotDist': [df.TotDist[i]],
						'Home_Fls': [df.Fls[j]],
						'Home_Fld': [df.Fld[i]],
						'Home_Off': [df.Off[i]],
						'Home_Crs': [df.Crs[j]],
						'Home_Int': [df.Int[j]],
						'Home_TklWin': [df.TklWin[i]],
						'Home_Recov': [df.Recov[i]],
						'Away_Poss': [df.Poss[j]],
						'Away_Sh': [df.Sh[j]],
						'Away_SoT': [df.SoT[j]],
						'Away_G/Sh': [df['G/Sh'][j]],
						'Away_Saves': [df.Saves[j]],
						'Away_PAtt': [df.PAtt[j]],
						'Away_PCmp%': [df['PCmp%'][j]],
						'Away_SPAtt': [df.SPAtt[j]],
						'Away_SPCmp%': [df['SPCmp%'][j]],
						'Away_MPAtt': [df.MPAtt[j]],
						'Away_MPCmp%': [df['MPCmp%'][j]],
						'Away_LPAtt': [df.LPAtt[j]],
						'Away_LPCmp%': [df['LPCmp%'][j]],
						'Away_ToDefPen': [df.ToDefPen[j]],
						'Away_ToDef3rd': [df.ToDef3rd[j]],
						'Away_ToMid3rd': [df.ToMid3rd[j]],
						'Away_ToAtt3rd': [df.ToAtt3rd[j]],
						'Away_ToAttPen': [df.ToAttPen[j]],
						'Away_TotDist': [df.TotDist[j]],
						'Away_Fls': [df.Fls[j]],
						'Away_Fld': [df.Fld[j]],
						'Away_Off': [df.Off[j]],
						'Away_Crs': [df.Crs[j]],
						'Away_Int': [df.Int[j]],
						'Away_TklWin': [df.TklWin[j]],
						'Away_Recov': [df.Recov[j]]})
		dfM = dfM.append(dfM2, ignore_index=True) 
	return dfM
	
\end{lstlisting}

\section{Librerie}\label{cap:importPy}
\begin{itemize}
	\item \textbf{Sklearn} (\textit{\cite{sklearn}}) è una libreria \emph{open source} di \emph{Machine Learning} per il linguaggio di programmazione Python. La libreria Sklearn fornisce una serie di algoritmi di \emph{Machine Learning} per l'analisi dei dati, ovvero modelli di classificazione, regressione, clustering ma anche algoritmi di model selection, di dimensionality reduction e funzionalità di preprocessing dei dati. È progettato per lavorare con le librerie NumPy, pandas e matplotlib.
	\item \textbf{Matplotlib} (\textit{\cite{matplotlib}}) è una libreria \emph{open source} per la visualizzazione dei dati per il linguaggio di programmazione Python. La libreria Matplotlib fornisce una vasta gamma di opzioni per la creazione di grafici e figure. È particolarmente utile per la visualizzazione di dati multidimensionali e per la creazione di grafici di grandi dimensioni e complessi. 
	\item \textbf{NumPy} (\textit{\cite{numPy}}) è una libreria \emph{open source} per il linguaggio di programmazione Python. La libreria NumPy fornisce metodi per l'elaborazione di vettori multidimensionali e matrici. Offre una serie di funzionalità avanzate per l'elaborazione di dati, come ad esempio il calcolo di statistiche, l'algebra lineare e la generazione di numeri casuali. 
	\item \textbf{Pandas} (\textit{\cite{pandas}}) è una libreria \emph{open source} per il linguaggio di programmazione Python. La libreria Pandas fornisce funzionalità di manipolazione e analisi di dati di diverse forme e fonti.
\end{itemize}