% !TEX encoding = UTF-8
% !TEX TS-program = pdflatex
% !TEX root = ../tesi.tex

%**************************************************************
\chapter{Introduzione}
\label{cap:introduzione}
%**************************************************************
%MEMO: Spiegazione del problema affrontato (il suo dominio) alcune applicazioni fatte nell'ambito delle comparazioni sportive, con maggior attenzione a qui studi con approccio statistico, esporre tecnologie usate e tools (Packages R ecc), motivazione scelta argomento della tesi e esposizione struttura della tesi(capitoli) TO DO
\intro{Il seguente capitolo descrive il problema scientifico affrontato in questa tesi, riportando le maggiori applicazioni di ricerca svolte nell'ambito del problema affrontato. Inoltre, il capitolo illustra quali tecnologie sono state utilizzate, le motivazioni personali per cui è stata redatta la seguente tesi e infine la struttura che compone la tesi. } 

%**************************************************************
\section{Dominio del problema}

Negli ultimi anni lo sport del calcio è stato oggetto di un grande processo di rinnovamento tecnologico. Infatti, su spinta dell'evoluzione dei sistemi di comunicazioni, ora da una singola partita è possibile ottenere una grande quantità di dati, oltretutto, in modo semplice grazie all'evoluzione delle piattaforme di \emph{streaming} che permettono di usufruire di contenuti calcistici in ogni momento a livello globale. Questo rinnovamento tecnologico del calcio con la conseguente disponibilità di dati riguardanti partite e giocatori non è dovuto solo all'evoluzione dei sistemi di comunicazione, ma anche all'introduzione di sensori e sistemi di tracciamento sofisticati all'interno del campo da calcio, come ad esempio la \emph{Goal-Line Technology} (vedi \textit{\cite{glt}}), oppure la Video Assistant Referee (VAR) (vedi \textit{\cite{var}}). Inoltre, recentemente durante la World Cup Qatar 2022 è stato introdotto il fuorigioco semi-automatico (vedi \textit{\cite{offside}}) composto da una grande quantità di sensori e telecamere per il tracciamento.\\
Perciò, con un'incredibile quantità di dati e informazioni relative alle partite e ai giocatori, nuovi casi di studio sono nati. Solo per citare alcuni casi di studio nel calcio, vi è lo \textit{scouting} di giocatori emergenti (vedi \autocite{vilela2018towards}), oppure la scelta del ruolo più adatto per il giocatore in base alle sue caratteristiche (vedi \autocite{razali2017predicting}), oppure quanto un giocatore è soggetto ad infortuni (vedi \autocite{theron2020use}). Ovviamente, non esistono analisi che comprendo solo lo studio di statistiche relative al singolo giocatore ma esistono ambiti di analisi che analizzano informazioni relative alle squadre e alle partite nel loro complesso. Proprio sull'analisi delle partite questa tesi si concentrerà, ossia l'interesse di questa tesi è quello di identificazione quali fattori possono influenzare sull'esito della partita e comprendere in che misura essi lo fanno. Infatti, molto spesso ci si pone l'interrogativo se una statistica come il possesso della palla o il numero di falli fatti o il numero di tiri fatti ecc., sia rilevante per l'esito della partita, e se sì, con quale peso determina la vittoria o il pareggio o la sconfitta.\\
Fondamentale è la scelta delle metodologie da utilizzare per le analisi. Infatti si è scelto un'approccio statistico, ossia l'utilizzo di un modello matematico di \emph{data mining} ovvero il modello Bradley-Terry \autocite{bradley1952rank} per l'interpretazione dei dati, l'individuare di possibili legami tra le statistiche registrate durante una partita e l'esito della partita e infine attività di predizione per valutare le prestazioni. Successivamente si impiegheranno tecniche di \emph{machine learning} per implementare modelli matematici più complessi in grado di ottenere previsioni più accurate. L'analisi sarà condotta sulle partite del campionato italiano della Serie A della stagione 2021/2022.
\section{Applicazione}
Nella ricerca dei fattori che influenzano l'esito delle partite di calcio sono stati svolti numerosi lavori con approcci e metodologie differenti. Come già scritto precedentemente è stato scelto un'approccio statistico nell'analisi ovvero l'utilizzo del modello matematico Bradley-Terry \autocite{bradley1952rank} con l'aggiunta successiva di metodi di predizione appartenenti all'area dell'apprendimento automatico. %Come detto esistono altri tipi di approcci nell'identificazione dei fenomeni che influenzano l'esito delle partite di calcio, ad esempio si riporta il lavoro svolto da \textit{\cite{marchiori2020secrets}})
Il lavoro di tesi svolto è stato ispirato dagli studi presentati qui sotto.\\
La prima applicazione che si riporta è la ricerca che ha portato alla creazione del modello utilizzato, ossia il lavoro svolto da \textcite{bradley1952rank}. La loro ricerca consisteva nel creare un modello matematico che confrontasse tutti gli oggetti di un insieme attraverso confronti a coppie. L'obbiettivo perciò era quello di, per ogni confronto, stabilire quale dei due oggetti confrontati era il migliore sulla base di tratti latenti non osservati. Un esempio di tratto latente è l'effetto dell'ambiente dove avviene il confronto che nel calcio può essere tradotto come il vantaggio di giocare una partita in casa. Successivamente il modello Bradley-Terry è stato modificato con diverse estensioni. Un'importante estensione è stata introdotta da \textcite{davidson1970extending} il quale ha introdotto il pareggio nel confronto a coppie, elemento fondamentale per la nostra ricerca dato che il modello sviluppato da Bradley e Terry era binario. In seguito grazie ai lavoro di \textcite{francis2010} e di \textcite{Turner2012Firth} fu introdotta e approfondita l'inclusione di covariate per la valutazione nei confronti, ovvero l'utilizzo di attributi che descrivono i soggetti che eseguono i confronti tra oggetti. Nella nostro contesto i soggetti sono le partite di calcio mentre gli oggetti sono le squadre di calcio. Successivi lavori da parte di \textcite{thurner2000policy} e di \textcite{mauerer2015modeling} introdussero le covariate specifiche dell'oggetto e le covariate specifiche del soggetto e dell'oggetto.\\ %Nonostante non sia stata applicata all'interno di questa tesi si riporta l'esistenza di espansione del modello Bradley Terry elaborata da \textcite{cattelan2013dynamic} di una versione dinamica del modello ossia nel valutare i due oggetti viene presa in considerazione l'evoluzione che hanno avuto quest'ultimi.
Con l'introduzione delle covariate nei modelli di comparazioni a coppie aumentò la complessità dei modelli. Pertanto, furono presentate applicazioni di metodi di regolarizzazione per ridurre la complessità dei modelli, ad esempio \textcite{schauberger2019btllasso} svolsero un lavoro d'analisi su partite del campionato tedesco ovvero la Bundesliga applicando il metodo LASSO come metodo di regolarizzazione.\\
Nell'ambito delle predizioni degli esiti delle comparazioni attraverso il modello Bradley-Terry si riporta l’interessante lavoro svolto da \textcite{kang2015poisson} in cui vengono svolte predizioni sulle partite del videogioco League of Legends (LOL). Per quanto riguarda i modelli predittivi di \emph{machine learning} un lavoro un'interessante è l'analisi svolta da \textcite{xu2021prediction} in cui si ha l'applicazione del modello Decision Tree e Random Forest per la predizione degli esiti delle partite di calcio della Bundesliga.

\section{Tecnologie e Strumenti utilizzati}
Nella seguente sezione saranno illustrate le tecnologie e i strumenti utilizzati durante il lavoro di tesi 
\subsection{Tecnologie}
Le tecnologie utilizzate in questa tesi sono descritte di seguito.
\begin{itemize}
	\item \textbf{R} \autocite{R-language} è un linguaggio di programmazione per il calcolo statistico e l'analisi grafica. È stato sviluppato nel 1993 da Ross Ihaka e Robert Gentleman ed è diventato uno strumento molto popolare per l'analisi dei dati in molti campi, inclusa la scienza dei dati, l'economia, la genetica e la biologia computazionale. R offre un'ampia gamma di funzionalità per il trattamento dei dati, l'analisi statistica e la creazione di grafici e altre rappresentazioni visuali dei dati. Ad oggi è supportato dalla R Core Team e dalla R Foundation for Statistical Computing. È distribuito come software \emph{open source} e può essere facilmente esteso attraverso il \emph{download} di pacchetti di funzionalità aggiuntive sviluppati da una vasta comunità di utenti.\\
	Le librerie utilizzate sono indicate nella Appendice \ref{cap:importR}.
	\item \textbf{Python} \autocite{van2003introduction} è un linguaggio di programmazione general-purpose, interpretato e ad alto livello. È stato sviluppato da Guido van Rossum negli anni '90 ed è mantenuto dalla Python Software Foundation. Il linguaggio Python è utilizzato in molti ambiti, come il web development, il \emph{machine learning} e l'automazione. È distribuito come software \emph{open source} e viene fornito con una grande quantità di librerie standard che espandono le sue funzionalità base.\\
	Le funzioni utilizzate sono indicate nella Appendice \ref{cap:importPy}.
\end{itemize}


\begin{comment}
library(ggmosaic)
library(ggplot2)
library(gridExtra)
@article{marchiori2020secrets,
	title={Secrets of soccer: Neural network flows and game performance},
	author={Marchiori, Massimo and de Vecchi, Marco},
	journal={Computers \& Electrical Engineering},
	volume={81},
	pages={106505},
	year={2020},
	publisher={Elsevier}
}
\end{comment}

\subsection{Tools}
Gli strumenti utilizzati in questa tesi sono descritti di seguito.
\begin{itemize}
	\item \textbf{RStudio} (vedi \textit{\cite{rstudio}}) è un ambiente di sviluppo integrato (IDE) per il linguaggio di programmazione R. Fornisce un insieme di strumenti per facilitare la scrittura, il debugging e il \emph{testing} del codice R. RStudio include anche funzionalità per la visualizzazione e l'analisi dei dati, come il supporto per i grafici interattivi e la possibilità di eseguire il codice R direttamente nell'editor di testo. RStudio è distribuito come \emph{software} \emph{open source}. In questa tesi è stato utilizzato per implementare il modello Bradley-Terry in R.
	\item \textbf{PyCharm} (vedi \textit{\cite{pycharm}}) è un ambiente di sviluppo integrato (IDE) per il linguaggio di programmazione Python. Offre una serie di strumenti per facilitare la scrittura, il debugging e il \emph{testing} del codice Python. PyCharm include anche funzionalità per l'integrazione con altri strumenti e servizi comuni nello sviluppo web, come il supporto per il versionamento del codice con Git e il supporto per i \emph{framework} di sviluppo web come Django. PyCharm  è sviluppato da JetBrains. In questa tesi è stato utilizzato per implementare i modelli di \emph{machine learning} in Python.
\end{itemize}

\section{Motivazioni personali}
Durante il mio percorso di studio ho frequentato alcuni corsi legati al mondo dell'intelligenza artificiale e dello studio dei dati. Oltre a ciò, sono un appassionato dello sport del calcio. Infatti nel seguire questo sport a volte mi imbatto in articoli dove istituti di ricerca attraverso l'interpretazione dei dati e l'applicazione di algoritmi di \emph{machine learning} provano a predire le classifiche finali dei maggiori campionati europei in fase di svolgimento. Inoltre, in seguito alla pandemia di COVID19, sempre più club calcistici hanno iniziato ad analizzare dati e statistiche per migliorare le loro prestazioni in campo e nello \emph{scounting}. Perciò, su spinta delle mie passioni e dalle recenti applicazioni precedentemente descritte, ho individuato nell'identificazione dei fattori che influenzano l'esito di una partita, un campo di ricerca innovativo e interessante come lavoro di tesi di laurea magistrale.
\section{Struttura della tesi}
La struttura della tesi è riporta di seguito
\begin{description}

\item[{\hyperref[cap:dataset]{Il secondo capitolo}}] descrive in modo dettagliato la raccolta dati con una descrizione di essi e della struttura del dataset. 
\item[{\hyperref[cap:Analisi]{Il terzo capitolo}}] descrive in modo dettagliato l'analisi grafica dei dati e il \emph{preprocessing} dei dati. 
\item[{\hyperref[cap:BT]{Il quarto capitolo}}] descrive il modello Bradley-Terry e le sue estensioni che sono state utilizzante durante l'analisi.
\item[{\hyperref[cap:risultatiDM]{Il quinto capitolo}}] illustra i risultati registrati con il modello Bradley-Terry e le sue estensioni. Inoltre vengono riportate le predizioni eseguite dai vari modelli Bradley-Terry confrontate con le predizioni dei \emph{bookmakers}.
\item[{\hyperref[cap:ML]{Il sesto capitolo}}] descrive gli algoritmi di apprendimento automatico che sono stati utilizzanti durante l'analisi.
\item[{\hyperref[cap:RisML]{Il settimo capitolo}}] riporta le predizioni calcolate dagli algoritmi di apprendimento automatico.
\item[{\hyperref[cap:extraDM]{Il ottavo capitolo}}] descrive una nuova applicazione di un modello BT già utilizzato nel Capitolo \ref{cap:risultatiDM} ma con una variabile risposta non più a tre ma a cinque categorie.
\item[{\hyperref[cap:precls]{Il nono capitolo}}] vengono discussi e confrontati i risultati riportati dai Capitoli \ref{cap:risultatiDM} \ref{cap:extraDM} e \ref{cap:RisML}. 
\item[{\hyperref[cap:conclusioni]{Il 10 capitolo}}] riporta un riassunto di quanto è stato svolto, sottolineando possibili sviluppi applicabili al lavoro di tesi.
\end{description}





\begin{comment}
\begin{figure}[h]
	\begin{center}
		\includegraphics[scale=0.5]{Logo_azzurrodigite.png}
		\caption{Logo di AzzurroDigitale}
	\end{center}
\end{figure}	contenuto...
\end{comment}


%**************************************************************


%\gls{AWMS} \g{machine learning}


%\begin{description}
    
   % \item[{\hyperref[cap:descrizione-stage]{Il secondo capitolo}}] descrive in modo dettagliato lo stage svolto, indicandone obiettivi, prodotti attesi, pianificazione delle attività, strumenti e tecnologie utilizzate e motivazioni personali.
    
    
%\end{description}
