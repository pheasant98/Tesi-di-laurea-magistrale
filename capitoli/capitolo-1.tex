% !TEX encoding = UTF-8
% !TEX TS-program = pdflatex
% !TEX root = ../tesi.tex

%**************************************************************
\chapter{Introduzione}
\label{cap:introduzione}
%**************************************************************
MEMO: Spiegazione del problema affrontato (il suo dominio) alcune applicazioni fatte nell'ambito delle comparazioni sportive, con maggior attenzione a qui studi con approccio statistico, esporre tecnologie usate e tools (Packages R ecc), motivazione scelta argomento della tesi e esposizione struttura della tesi(capitoli) TO DO

%\begin{itemize}

%\end{itemize}

%**************************************************************
\section{Dominio del problema}

\section{Applicazione}

\section{Tecnologie e Tools usati}

\subsection{Tecnologie}
\begin{comment}
library(ggmosaic)
library(ggplot2)
library(gridExtra)
\end{comment}

\subsection{Tools}

\section{Motivazioni personali}

\section{Struttura della tesi}






\begin{comment}
\begin{figure}[h]
	\begin{center}
		\includegraphics[scale=0.5]{Logo_azzurrodigite.png}
		\caption{Logo di AzzurroDigitale}
	\end{center}
\end{figure}	contenuto...
\end{comment}


%**************************************************************


%\gls{AWMS} \g{machine learning}


%\begin{description}
    
   % \item[{\hyperref[cap:descrizione-stage]{Il secondo capitolo}}] descrive in modo dettagliato lo stage svolto, indicandone obiettivi, prodotti attesi, pianificazione delle attività, strumenti e tecnologie utilizzate e motivazioni personali.
    
    
%\end{description}
