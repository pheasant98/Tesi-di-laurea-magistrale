%**************************************************************
\chapter{Conclusioni}
\label{cap:conclusioni}
%**************************************************************

%**************************************************************
In questa tesi, partendo dalla domanda posta inizialmente ovvero, cosa influenza il successo o il fallimento delle singole squadre durante una partita di calcio, si sono analizzate le possibili relazioni tra l'esito di una partita di calcio e le principali statistiche raccolte durante la partita. Perciò, è stata una naturale applicazione per l'analisi, l'utilizzo di un modello di confronto a coppie ovvero, il modello Bradely-Terry. I dati utilizzati per il lavoro di tesi sono stati reperiti dal sito web \textit{\cite{fbref}}, il quale mette a disposizione un enorme quantità di statistiche riguardanti le partite delle maggiori leghe di calcio di più stagioni. Per il nostro lavoro i dati che sono stati ricavati si riferiscono alla Seria A italiana della stagione 2021/2022.\\
L'analisi, attraverso strumenti grafici, inizia con lo studio dei dati individuando sia le possibili relazioni tra l'esito della partita e le singole 29 variabili esplicative e sia le possibili interazioni tra covariate. Successivamente ad alcune operazioni di \emph{prepocessing}, l'analisi è continuata con la modellazione dei modelli Bradley-Terry. La modellazione parte con il modello Bradley-Terry standard per stimare l'abilità delle singole squadre e l'effetto di giocare in casa per poi spingersi sempre più in profondità, introducendo 26 variabili esplicative fino a analizzarne il loro effetto sulle partite per ogni singola squadra. Successivamente, per verificare che l'uso dei dati è stato svolto correttamente, si è svolto l'attività di predizione con i modelli confrontando le loro predizioni con le predizioni fatte dai \emph{bookmakers}. Per rendere il lavoro di tesi più completo e approfondito, l'analisi è passata allo studio delle predizioni fatte da metodi di apprendimento ovvero il K-Nearest-Neighbors (K-NN), la Support Vector Machine (SVM), il Decision Tree, la Random Forest e l'AdaBoost. Dopo un breve confronto tra i vari metodi di \emph{machine learning} dal punto di vista delle prestazioni registrate durante la fase di predizione, i metodi di apprendimento automatico sono stati utilizzati come riferimento dal punto di vista dell'accuratezza, della precisione, della sensibilità e della specificità, per valutare la bontà dei modelli BT in fase di predizione. Successivamente, l'analisi prosegue confrontando i metodi Decision Tree e Random Forest con i modelli BT, sull'identificazione delle statistiche che influenzano l'esito di una partita. Infine, per un maggior approfondimento è stato riapplicato il modello BT con covariate specifiche per ogni partita e per ogni squadra, estendendo la variabile risposta da tre a cinque categorie, ottenendo un risultato più raffinato ma in linea con quanto ricavato precedentemente.\\
I risultati ricavati mettono in luce l'importanza in positivo dei tiri, l'effetto di giocare in casa, il mantenimento del controllo della palla nell'area della squadra in possesso della palla e dell'utilizzo di lanci lunghi. Viceversa, viene individuato che i cross, i passaggi filtranti e un alto numero di tocchi del pallone nell'area dell’avversario, influiscono negativamente sull'esito della partita. Per le squadre con la maggior abilità stimata subire un fuorigioco incide negativamente sul loro successo sportivo. Infine, viene ricavato che il possesso della palla non influisce nell'esito della partita.\\
Chiaramente quanto ricavato vale per la stagione 2021/2022 del campionato italiano di Serie A. È naturale quindi che l'analisi può essere estesa su un’altra stagione della Serie A oppure analizzando gli altri principali campionati europei ovvero, la Premier League, la Liga spagnola, la Ligue 1 e la Bundesliga. Occorre fare attenzione al fatto che sia nella Ligue 1 e sia nella Bundesliga negli ultimi anni rispettivamente Paris Saint Germain e Bayern München hanno monopolizzato la vittoria del campionato. Quindi c'è da aspettarsi ampi dislivelli di abilità tra le due queste squadre e le altre squadre dei due rispettivi campionati. Per quanto riguarda la Premier League è possibile aspettarsi una stima delle abilità più bilanciata ma sopratutto di una stima fortemente positiva riguardo al possesso della palla grazie allo stile di gioco proposto dall'allenatore del Manchester City, Pep Guardiola (vedi \textit{\cite{futbol}}), vincitore di quattro delle ultime cinque edizioni del campionato inglese.\\
Sicuramente d'interesse potrebbe essere l'applicazione di un modello Bradley-Terry dinamico, ovvero che vada a valutare la variazione dell'abilità di ogni singola squadra durante la stagione sportiva, in modo da individuare possibili fenomeni che possano influire positivamente o negativamente sulle prestazioni delle squadre. Naturalmente, oltre a valutare l'abilità di una squadra durante la stagione un ulteriore analisi che consideri più di una stagione può essere una interessante estensione del lavoro svolto. Occorre sottolineare che le covariate vengono incorporate in modo lineare nelle abilità delle squadre nelle specifiche partite, e quindi l'aggiunta di effetti non lineari è una possibile estensione del lavoro come lo può essere l'utilizzo di metodi di \emph{machine learning} non supervisionati. Certamente anche l'aggiunta di altre covariate come, ad esempio, la distanza percorsa dai giocatori o il numero di calci d'angolo battuti, potrebbe far sì di individuare nuove statistiche chiave che determinano l'esito della partita.\\
In altre parole, questo lavoro di tesi ha molte opportunità di utilizzo e sviluppo futuro, poiché i modelli utilizzati sono versatili e l'analisi può offrire molte prospettive diverse.
