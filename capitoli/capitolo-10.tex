%**************************************************************
\chapter{Conclusioni}
\label{cap:conclusioni}
%**************************************************************

%**************************************************************
\intro{Questo capitolo conclude la tesi riassumendo brevemente i risultati ottenuti dai modelli Bradley-Terry e dalle tecniche di machine learning. Infine, si presentano alcune possibili estensioni dell'analisi svolta.}\\

In questa tesi, partendo dalla domanda posta inizialmente "Cosa influenza il successo o il fallimento delle singole squadre durante una partita di calcio?" si sono analizzate le possibili relazioni tra l'esito di una partita di calcio e le principali statistiche raccolte durante la partita.  Successivamente sono state svolte delle predizioni sui risultati delle partite di calcio esaminate.
Un riassunto delle tecniche utilizzate è riportato nella Tabella \ref{tab:fine}.
\begin{table}[]
\begin{longtable}{|C{0.18\textwidth}|C{0.3\textwidth}|C{0.5\textwidth}|}
	\hline	
	\rowcolor{SchoolColor2}
	\textbf{Modello o algoritmo} & \textbf{Scopo utilizzo} & \textbf{Risultati ricavati}\\
	\hline			
	BTOE & Stima dell'abilità delle squadre e dell'effetto di giocare in casa + Predizione. & Classifica delle abilità molto simile alla classifica finale. \\
	\hline	
	BTMO & Come BTOE + stima del valore di ogni statistica associato all'esito di una partita, comune per tutte le squadre + Predizione. & Sono fortemente associate a un aumento della probabilità di vittoria le statistiche \texttt{AtHome, G/Sh, Sh, SoT, ToMid3rd} e \texttt{PCmp\%}, Viceversa, le statistiche \texttt{MPAtt, MPCmp\%, SPAtt, Recov, LPCmp\%} e \texttt{Crs} sono associate a una diminuzione. Le statistiche rimanenti sono poco significative.\\
	\hline	
	BTM3SO + Lasso & Come BTOE + stima del valore di ogni covariata associato all'esito di una partita per ogni squadra + Predizione. & Sono fortemente associate a un aumento della probabilità di vittoria le statistiche \texttt{G/Sh, Sh, SoT, ToMid3rd, Saves} e \texttt{ToDefPen}, Viceversa, le statistiche \texttt{ToAtt3rd, MPCmp\%, Recov} e \texttt{Crs} sono associate a una diminuzione. Le statistiche rimanenti sono poco significative. \\
	\hline	
	BTM5SO + Lasso & Come BTM3SO ma con \emph{Y} = 5 + Predizione. & Sono fortemente associate a un aumento della probabilità di vittoria le statistiche \texttt{G/Sh, Sh, SoT, ToMid3rd, Saves, LPAtt} e \texttt{ToDefPen}, Viceversa, le statistiche \texttt{ToAtt3rd, MPCmp\%} e \texttt{Crs} sono associate a una diminuzione. Le statistiche rimanenti sono poco significative. \\
	\hline	
	BTMNI3SO + Lasso & Per approfondire il peso delle singole statistiche associato all'esito di una partita a seconda della squadra in esame + Predizione. & Sono fortemente associate a un aumento della probabilità di vittoria le statistiche \texttt{G/Sh, Sh, SoT, Saves, LPAtt} e \texttt{ToDefPen}, Viceversa, le statistiche \texttt{MPCmp\%} e \texttt{Crs} sono associate a una diminuzione. Le statistiche rimanenti sono poco significative. \\
	\hline	
	K-NN & Predizione. & Predizione con una accuratezza pari a 0.58\\
	\hline	
	SVM & Predizione. & Predizione con una accuratezza pari a 0.78\\
	\hline	
	Decision Tree & Predizione + Individuazione statistiche significative. & Predizione con una accuratezza pari a 0.71. Vengono individuate come significative solo le statistiche legate ai tiri e ai passaggi medi.\\
	\hline	
	Random Forest & Come Decision Tree. & Predizione con una accuratezza pari a 0.72. Vengono individuate come significative solo le statistiche legate ai tiri, i passaggi di tutti i tipi completati i recuperi della palla e i tocchi del pallone nelle diverse zone del campo da calcio. \\
	\hline	
	AdaBoost & Predizione. & Predizione con una accuratezza pari a 0.72. \\
	\hline
\end{longtable}	
\vspace*{5mm}

\caption{La tabella riporta ogni modello o algoritmo utilizzato nell'analisi indicato per ognuno lo scopo di utilizzo e i risultati ottenuti. BTMOE = modello Bradley-Terry standard con effetto d'ordine. BTMO = modello BT con covariate specifiche dell'oggetto. BTM3SO + Lasso = modello BT con variabile risposta \emph{Y} a tre categorie, covariate specifiche del soggetto dell'oggetto, con LASSO. Con BTMNI3SO + Lasso = modello BT senza intercette, con variabile risposta \emph{Y} a tre categorie, covariate specifiche del soggetto dell'oggetto, con LASSO. K-NN = K-Nearest-Neighbors. SVM = Support Vector Machine.} \label{tab:fine}
\end{table}
L'utilizzo di un modello di confronto a coppie, ovvero il modello Bradely-Terry è un naturale strumento di analisi che è stato applicato sui dati che si riferiscono agli incontri della Serie A italiana 2021/2022. 
Inoltre, si utilizzano i metodi di apprendimento automatico per effettuare predizioni. In aggiunta, Decision Tree e Random Forest permettono di individuare le statistiche più significative. \\

\begin{comment}
	L'analisi, attraverso strumenti grafici, inizia con lo studio dei dati individuando le possibili relazioni tra l'esito della partita e le singole 29 variabili esplicative, e le possibili interazioni tra covariate. Successivamente ad alcune operazioni di \emph{prepocessing}, l'analisi è continuata con la modellazione dei modelli Bradley-Terry. La modellazione parte con il modello Bradley-Terry standard per stimare l'abilità delle singole squadre e l'effetto di giocare in casa per poi spingersi sempre più in profondità, introducendo 26 variabili esplicative fino a analizzarne il loro effetto sulle partite per ogni singola squadra. Successivamente, per verificare che l'uso dei dati è stato svolto correttamente, si è svolto l'attività di predizione con i modelli confrontando le loro predizioni con le predizioni fatte dai \emph{bookmakers}. Per rendere il lavoro di tesi più completo e approfondito, l'analisi è passata allo studio delle predizioni fatte da metodi di apprendimento automatico ovvero, il K-Nearest-Neighbors (K-NN), la Support Vector Machine (SVM), il Decision Tree, la Random Forest e l'AdaBoost. Dopo un breve confronto tra i vari metodi di \emph{machine learning} dal punto di vista delle prestazioni registrate durante la fase di predizione, i metodi di apprendimento automatico sono stati utilizzati come riferimento dal punto di vista dell'accuratezza, della precisione, della sensibilità e della specificità, per valutare la bontà dei modelli BT in fase di predizione. Successivamente, l'analisi prosegue confrontando i metodi Decision Tree e Random Forest con i modelli BT, sull'identificazione delle statistiche che influenzano l'esito di una partita. Infine, per un maggior approfondimento, è stato riapplicato il modello BT con covariate specifiche per ogni partita e per ogni squadra, estendendo la variabile risposta da tre a cinque categorie, ottenendo un risultato più raffinato ma in linea con quanto ricavato precedentemente.\\
\end{comment}
I risultati ricavati dai cinque modelli di Bradley-Terry e dai algoritmi Decision Tree e Random Forest, mettono in luce l'importanza dei tiri, l'effetto di giocare in casa, il mantenimento del controllo della palla nell'area della squadra in possesso della palla e dell'utilizzo di lanci lunghi al fine di una vittoria. Viceversa, viene individuato che i cross, i passaggi filtranti e un alto numero di tocchi del pallone nell'area dell’avversario sono associati negativamente all'esito della partita. Per le squadre con la maggior abilità stimata subire un fuorigioco incide negativamente sul loro successo sportivo. Infine, viene ricavato che il possesso della palla non viene associato all'esito della partita.\\
Chiaramente quanto ricavato vale per la stagione 2021/2022 del campionato italiano di Serie A. È naturale quindi che l'analisi possa essere estesa su un’altra stagione della Serie A oppure ad gli altri campionati europei quali, la Premier League, la Liga spagnola, la Ligue 1 e la Bundesliga. Occorre fare attenzione al fatto che sia nella Ligue 1 e nella Bundesliga negli ultimi anni Paris Saint Germain e Bayern München hanno monopolizzato la vittoria del campionato. Quindi c'è da aspettarsi ampi dislivelli di abilità tra queste squadre e le altre dei rispettivi campionati. Per quanto riguarda la Premier League è possibile aspettarsi una stima delle abilità più bilanciata, ma soprattutto una stima fortemente positiva riguardo al possesso della palla grazie allo stile di gioco proposto dall'allenatore del Manchester City, Pep Guardiola (vedi \textit{\cite{futbol}}), vincitore di quattro delle ultime cinque edizioni del campionato inglese.\\
Sicuramente d'interesse potrebbe essere l'applicazione di un modello Bradley-Terry dinamico \autocite{cattelan2013dynamic}, ovvero che vada a valutare la variazione dell'abilità di ogni singola squadra durante la stagione sportiva, in modo da individuare possibili fenomeni che possano ripercuotersi positivamente o negativamente sulle prestazioni delle squadre.\\
Naturalmente, oltre a valutare l'abilità di una squadra durante la stagione un'ulteriore estensione dell'analisi è considerare più di una stagione \autocite{tsokos2019modeling}, sebbene ci sia un atteso aumento di complessità computazionale. \\
Occorre sottolineare che nelle analisi svolte, le covariate vengono incorporate in modo lineare nelle abilità delle squadre nelle specifiche partite. L'aggiunta di effetti non lineari attraverso l'utilizzo di metodi di \emph{machine learning} non supervisionati come ad esempio, il Clustering \autocite{dunn1974well}, può essere una possibile estensione del lavoro. Infatti come riportato nel lavoro svolto da \textcite{shin2014novel}, il Clustering è utilizzato per individuare caratteristiche non lineari comuni tra squadre di calcio come ad esempio, le strategie di gioco. Infatti, gli algoritmi di apprendimento non supervisionato hanno l'obiettivo di raggruppare e interpretare i dati basandosi solo sull'input ricevuto, attraverso l'individuazione di \emph{features} non lineari. Questi algoritmi permettono di migliorare l'analisi ma con lo svantaggio di un maggior costo computazionale da sostenere.\\
Certamente anche l'aggiunta di altre covariate come, ad esempio, la distanza percorsa dai giocatori o il numero di calci d'angolo battuti, potrebbe permettere di individuare nuove statistiche chiave che determinano l'esito della partita.\\

