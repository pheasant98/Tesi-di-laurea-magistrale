% !TEX encoding = UTF-8
% !TEX TS-program = pdflatex
% !TEX root = ../tesi.tex

%**************************************************************
\chapter{Serie A 2021/2022 dataset }
%\label{cap:dataset}
%**************************************************************

\intro{Nel seguente capitolo verrà descritto in dettaglio la raccolta dati effettuata per costruire il dataset riguardante le partite di calcio della Serie A italiana della stagione 2021/2022 e di come tale dataset è strutturato descrivendone le variabili e i dati al suo interno, utilizzati per l'analisi descritta precedentemente.}\\

%**************************************************************
\section{Serie A 2021/2022}

L'analisi che è stata effettuata ha preso in considerazione le partite della Serie A italiana della stagione 2021/2022. La Serie A è un torneo che comprende 20 squadre sparse per tutta l'Italia, alcune anche della stessa città ad esempio, Milan e Inter sono due squadre di Milano. Tale torneo è organizzato con una struttura Double-Round-Robin, dove ogni squadra affronta due volte le altre 19 avversarie del torneo. Vi è quindi una partita di andata e una di ritorno che in base al sorteggio della creazione del calendario delle partite decide quale delle due partite sarà giocata in casa oppure fuori casa (in casa dell'avversario). Tale torneo nella stagione 2021/2022 è iniziato il 22 Agosto con Inter - Genoa e si è concluso il 22 Maggio con le partite Salernitana - Udinese e Venezia - Cagliari, per un totale 380 partite giocate suddivise in 38 turni dove ogni turno è composto da 10 partite.

\subsection{Ranking}
Le squadre di calcio sono classificate in base all'ordine dei punti che hanno totalizzato al termine della stagione. In un torneo calcistico, per ogni partita vinta la squadra vincente guadagna 3 punti, per ogni pareggio le due squadre avversarie guadagnano entrambe un punto, mentre per ogni sconfitta la squadra perdente non guadagna punti. Nel torneo della Serie A chi guadagna più punti vince il campionato, mentre chi si classifica tra le ultime tre retrocede alla lega inferiore, la Serie B, dove il posto delle tre squadre retrocesse verrà presso da tre squadre della Serie B che hanno guadagnato la promozione alla Serie A.\\ 
La classifica della stagione 2021/2022 è mostrata nella Tabella \ref{tab:ranking}.

	\begin{table}[!h]%
	\rowcolors{2}{grigetto}{white}
	\renewcommand{\arraystretch}{1.7}
	\centering
	\begin{tabular}{c c c c}
		\hline	
		\rowcolor{SchoolColor}
		\intest{Posizione} & \intest{Squadra} & \intest{Punti} & \intest{ \% casa}  \\	
		\hline			
		1 & Milan & 86 & 0.47\\
		2 & Inter & 84 & 0.54\\
		3 & Napoli & 79 & 0.46\\
		4 & Juventus & 70 & 0.50\\
		5 & Lazio & 64 & 0.56\\
		6 & Roma & 63 & 0.57\\
		7 & Fiorentina & 62 & 0.66\\
		8 & Atalanta & 59 & 0.33\\
		9 & Hellas Verona & 53 & 0.57\\
		10 & Torino & 50 & 0.58\\
		11 & Sassuolo & 50 & 0.48\\
		12 & Udinese & 47 & 0.53\\
		13 & Bologna & 46 & 0.61\\
		14 & Empoli & 41 & 0.42\\
		15 & Sampdoria & 36 & 0.58\\
		16 & Spezia & 36 & 0.50\\
		17 & Salernitana & 31 & 0.48\\
		18 & Genoa & 30 & 0.50\\
		19 & Cagliari & 28 & 0.61\\
		20 & Venezia & 27 & 0.52\\
			\hline
		 & & & \\
	
	\end{tabular} \hbox{}

	\caption{La tabella mostra i punti guadagnati da ogni squadra con il loro piazzamento. Inoltre viene mostrata la percentuale di punti guadagnati in casa.} \label{tab:ranking}
\end{table}

%**************************************************************
\section{Costruzione del dataset}

Al giorno d'oggi, nelle partite di calcio professionistico viene raccolta un'enorme quantità di variabili. Ad esempio, per ogni squadra è noto il tempo in percentuale del possesso della palla o il numero di tiri in porta prodotto dalla squadra in una determinata partita. L'obiettivo principale di questo lavoro è determinare l'influenza di queste variabili specifiche della partita. Per creare il dataset per tale scopo, sono state raccolte un gran numero di variabili che a primo avviso possono essere significative, tali dati sono stati offerti dal sito web FBref (https://fbref.com/).\\ FBref è un sito web dedicato al tracciamento delle statistiche relative ai calciatori e alle squadre di calcio di tutto il mondo. \\

FBref mette a disposizione i dati sotto forma di tabelle che possono essere modificate per mantenere solo i dati di nostro interesse, in più per rendere più facile l'esportazione, tali tabelle possono essere convertite in formato di CSV per poter essere poi trasportate in un file Excel. 

Quindi per ogni squadra che ha partecipato alla stagione 2021/2022 di Serie A si è esportato per ogni partita giocata alcune variabili che ci interessavano, selezionando per prima cosa la macro aree dove si trovavano le variabili d'interesse e poi, modificando le tabelle per ottenere solo i dati di tali variabili. Ogni tabella generata veniva poi riconvertita in CSV per essere poi unita con tutte le altre in un file Excel che una volta completato, divenne il dataset per le nostre analisi. Per rendere più leggibile il file Excel, dato che le stringhe in CSV separavano i dati con il carattere separatore virgola, si è utilizzata la funzione di Excel "trasforma testo in colonne" per inserire tutti i dati in modo ordinato nelle celle del foglio Excel.

\subsection{Struttura dataset}
Il dataset risultate dalla raccolta dati è composto da 760 righe e 35 colonne. Ogni riga riguarda una specifica partita di calcio giocata dalla squadra indicata nella colonna \textsf{Team} contro la squadra indicata nella colonna \textsf{Vs}. Ogni riga perciò contiene informazioni riguardati solo la squadra indicata in \textsf{Team} fatta eccezioni per la data della partita (\textsf{Date}), il turno (\textsf{Round}), e gli spettatori (\textsf{Spec}). Quindi per ogni partita esistono due righe, una per ognuna delle due squadre coinvolte. Perciò ogni squadra appare nella colonna \textsf{Team} 38 e dato che si hanno 20 squadre si hanno perciò 760 righe totali. Per quanto riguarda le colonne se ne discuterà nella prossima sotto sezione. \\
La Tabella \ref{tab:db} mostra un breve estratto dei dati riguardanti le prime tre partite della stagione. 
	\begin{table}[!h]%
	\rowcolors{2}{grigetto}{white}
	\renewcommand{\arraystretch}{1.7}
	\centering
	\begin{tabular}{c c c c c c c c c  }
		\hline	
		\rowcolor{SchoolColor}
		\intest{Date} & \intest{AtHome} & \intest{Res} & \intest{GF} & \intest{GA} & \intest{Team} & \intest{Vs} & \intest{Poss} & \intest{...}   \\	
		\hline	
		21/08/2021 & TRUE & 1 & 4 & 0 & Inter & Genoa & 0,59 & ... \\
		... & ... & ... & ... & ... & ... & ... & ... & ... \\
		22/08/2021  & TRUE & 1 & 2 & 0 & Napoli & Venezia & 0,56 & ... \\
		... & ... & ... & ... & ... & ... & ... & ... & ...  \\
		23/08/2021  & FALSE & 1 & 1 & 0 & Milan & Sampdoria & 0,51 & ... \\		
		... & ... & ... & ... & ... & ... & ... & ... & ... \\
		21/08/2021  & FALSE & -1 & 0 & 4 & Genoa & Inter & 0,41 & ... \\
		... & ... & ... & ... & ... & ... & ... & ... & ...  \\
		22/08/2021  & FALSE & -1 & 0 & 2 & Venezia & Napoli & 0,44 & ... \\
		... & ... & ... & ... & ... & ... & ... & ... & ...  \\
		23/08/2021 1 & TRUE & 1 & 0 & 1 & Sampdoria & Milan & 0,49 & ... \\
		... & ... & ... & ... & ... & ... & ... & ... & ...  \\
		\hline
		& & & & & & & & \\
		
		
		
	\end{tabular} \hbox{}
	
	\caption{La tabella mostra un estratto del dataset utilizzato i cui dati sono stati ricavati da FBref.} \label{tab:db}
\end{table}




\subsection{Covariate}

Come scritto precedentemente all'interno del dataset sono presenti 35 colonne. Oltre alle già citate \textsf{Date}, \textsf{Round} e \textsf{Spec} che hanno solo un valore di completezza dei dati, le restanti 32 colonne saranno le possibili candidate a essere le covariate che costituiranno il modello. Ovviamente non è detto che tutte queste variabili saranno inserite nel modello perché prima di costruire un modello, ci sarà un analisi per verificare se sia sensato o no l'utilizzo di ognuna delle variabili verificando attraverso grafici (analisi grafica) e individuando possibili problemi di multicollinearità o di bassa significatività delle variabili.\\ Le possibili cavariate sono le seguenti:
\begin{itemize}
	\item \textsf{AtHome}: Tale variabile indica se la squadra indicata sulla variabile \textsf{Team} gioca nel suo stadio, quindi in casa oppure fuori casa. Per indicare se la squadra gioca in casa viene messo come valore \texttt{TRUE} altrimenti \texttt{FALSE}. 
	
	Come mostrato nella terza colonna della tabella \ref{tab:ranking}, che indica in percentuale quante partite sono state vinte in casa per ogni singola squadra, ci sono 11 squadre che hanno avuto un leggero vantaggio nel giocare in casa le partite di calcio rispetto a altre sei squadre che hanno avuto l'effetto opposto, mentre le rimanti tre hanno avuto un effetto nullo. Alla luce di questo è stato deciso di inserire tale variabile per via del suo effetto nell'esito di una partita in generale.
	\item \textsf{Res}: Tale variabile indica se la squadra indicata sulla variabile \textsf{Team} ha vinto o ha pareggiato o ha perso. Per indicare se ha vinto viene inserito il valore 1, se ha pareggiato 0, altrimenti se ha perso -1. Chiaramente questa variabile sarà la nostra Y cioè la variabile risposta che il modello deve riuscire a prevedere.
	\item \textsf{GF}: Tale variabile indica il numero di gol fatti dalla squadra indicata sulla variabile \textsf{Team}. 
	
	Questa variabile è stata inserita perché può permettere di valutare la qualità della fase offensiva della squadra e quindi essere significativa ai fine dell'analisi.
	\item \textsf{GA}: Tale variabile indica il numero di gol subiti dalla squadra indicata sulla variabile \textsf{Team} e quindi fatti dalla squadra indicata nella variabile \textsf{Vs}. 
	
	Questa variabile è significativa perché subire pochi gol incide positivamente nell'esito della partita, infatti non espone la squadra a doversi sbilanciare in attacco per poter recuperare lo svantaggio e quindi non rischiare di subire altri gol dai avversari. Inoltre è un fatto riconosciuto che aver la miglior difesa del campionato porta con molta probabilità a vincere il campionato
	\item \textsf{Team}: Tale variabile indica il nome della squadra a cui i dati della riga fanno riferimento. É necessaria per il funzionamento del modello, nel prossimo paragrafo verrà approfondito il suo utilizzo nel modello.
	\item \textsf{Vs}: Tale variabile indica il nome della squadra avversaria. É necessaria per il funzionamento del modello, nel prossimo paragrafo verrà approfondito il suo utilizzo nel modello.
	\item \textsf{Poss}: Tale variabile indica in percentuale, la quantità di tempo di possesso della palla durante una partita di calcio della squadra indicata sulla variabile \textsf{Team}. Nel gioco del calcio con il termine “possesso palla” si intende un’azione manovrata di due o più giocatori che riescono a passarsi la palla evitando i contrasti degli avversari. In poche parole durante la partita, ogni volta che una squadra ha il dominio della palla si dice che questa squadra è in fase di “possesso palla”, quindi in questa variabile viene indicato quanto questa fase è durata nell'intera partita.\\
	Il metodo più comune utilizzato per calcolare il possesso palla di una squadra si basa sull'utilizzo di tre cronometri: uno per ciascuna formazione più uno per i tempi morti. Quando un giocatore della squadra A tocca un pallone che prima era in possesso della squadra B, il cronometro della squadra A parte e quello della squadra B si ferma e così via. Il terzo cronometro registra il tempo in tutte le situazioni di palla inattiva cioè ad esempio: rimesse laterali, calci di punizione ecc.. I tempi vengono poi trasformati in percentuali. Per una registrazione più sofisticata, si può utilizzare 22 cronometri, uno per ogni giocatore, in modo da registrare anche il possesso palla di ogni singolo giocatore per avere una registrazione più precisa.
	
	Tale variabile è stata inserita perché, la supremazia nel possesso palla è solitamente desiderabile e utile infatti si possono avere i seguenti vantaggi:
	\begin{itemize}
		\item Spingere l’avversario a muoversi verso la palla per allontanarlo dalla difesa della propria porta per poi sorprenderlo negli spazi lasciati incustoditi.   
		\item Modulare il ritmo della gara, ad esempio la squadra A sta vincendo con un gol di scarto e per non rischiare attacchi dalla squadra B, "congela" il risultato mantenendo il possesso della palla.
	\end{itemize}
	Il possesso palla però non garantisce certo la vittoria, infatti produrre un possesso palla "sterile" cioè senza che questo porti alla produzioni di azioni offensive, può esporre la squadra in possesso della palla a possibili contropiedi nel caso in cui perde la palla e quindi all'alto rischio di subito gol perché sbilanciata e non ben posizionata. Vedremmo di seguito quali variabili possono essere utili per capire se il possesso palla fatto dalla squadra è "sterile" oppure no.

	\item \textsf{Sh}: Tale variabile indica il numero di tiri totali fatti dalla squadra indicata sulla variabile \textsf{Team}. Quindi vengono conteggiati il numero di tiri in porta più i tiri fuori dalla porta. 
	
	Una squadra che effettua tanti tiri ha più probabilità di segnare un gol. Occorre pero capire quanto è precisa una squadra nel centrare la porta.
	\item \textsf{SoT}: Tale variabile indica il numero di tiri in porta totali fatti dalla squadra indicata sulla variabile \textsf{Team}. 
	
	Una squadra con un alto valore di tiri in porta è più probabile che possa segnare un gol. Tale variabile permette di capire quanto è precisa in combinazione con \texttt{Sh} la squadra di calcio nel centrare la porta nei suoi tiri.
	\item \textsf{G/Sh}: Tale variabile indica la proporzione tra gol e tiri fatti dalla squadra indicata sulla variabile \textsf{Team}. 
	
	Tale variabile perciò permette di capire quanto la produzioni di tiri della squadra è efficace o meno. Con \texttt{Sh} e \texttt{SoT} si riesce a valutare quanto è offensiva la squadra cioè, se essa gioca costantemente in attacco o utilizza la tattica difesa e contropiede. Inoltre permette di capire quanto la squadra è precisa nel effettuare i tiri in porta.
	\item \textsf{Saves}: Tale variabile indica il numero di parate fatte del portiere della squadra indicata sulla variabile \textsf{Team}. 
	
	La variabile è stata inserita perché permette di valutare se la squadra subisce tanti tiri dai avversari e la qualità del portiere nel salvare la squadra da un possibile gol subito.
	\item \textsf{PAtt}: Tale variabile indica il numero di tutti i passaggi tentati dai giocatori della squadra indicata sulla variabile \textsf{Team}. 
	
	Utile a capire quanto la squadra sia incline a tentare i passaggi. Si studierà nell'analisi se tale variabile è significativa ma sicuramente ha un maggior significato se messa a confronto con la percentuale di passaggi riusciti \texttt{PCmp\%}.
	\item\textsf{PCmp\%}: Tale variabile indica la percentuale di passaggi riusciti ai giocatori della squadra indicata sulla variabile \textsf{Team}. 
	
	Questa variabile è stata inserita perché permette di capire quanti passaggi sono andati a buon fine tra tutti quelli tentati e quindi qual'è la precisione dei giocatori della squadra.
	\item \textsf{SPAtt}: Tale variabile indica il numero di passaggi corti tentati dai giocatori della squadra indicata sulla variabile \textsf{Team}. Per passaggi corti si intendono quelli effettuati all'interno di una lunghezza tra i tre e 14 metri.
	
	Questa variabile è stata inserita per capire se un alto numero di passaggi corti possono essere determinanti ai fini dell'esito della partita. Ovviamente analogamente a \texttt{PAtt} occorre fare un confronto con la sua percentuale di passaggi corti riusciti \texttt{SPCmp\%}.
	\item \textsf{SPCmp\%}: Tale variabile indica la percentuale di passaggi corti riusciti ai giocatori della squadra indicata sulla variabile \textsf{Team}. 
	
	Questa variabile è stata inserita perché permette di capire quanti passaggi sono andati a buon fine tra tutti quelli tentati e quindi qual'è la precisione dei giocatori della squadra.
	\item \textsf{MPAtt}: Tale variabile indica il numero di passaggi medi tentati dai giocatori della squadra indicata sulla variabile \textsf{Team}. Per passaggi medi si intendono quelli effettuati all'interno di una lunghezza tra i 13 e 27 metri. Questi passaggi possono essere considerati come passaggi filtranti cioè un tipo di passaggio non diretto direttamente al proprio compagno di squadra ma verso un area del campo dove il compagno di squadra deve andare a prendere la palla, spesso questi passaggi vengono fatti per sorprendere la difesa avversaria e evitare che intercettino la palla. Nella Figura \ref{fig:filt} viene mostrato l'esecuzione di un passaggio filtrante.
	\begin{figure}[h]
		\begin{center}
			\includegraphics[scale=0.51]{filtrante2.jpg}
			\caption{Esecuzione di un passaggio filtrante} \label{fig:filt}
		\end{center}
	\end{figure}

	Questa variabile è stata inserita per capire se un alto numero di passaggi medi possono essere determinanti ai fini dell'esito della partita. Ovviamente analogamente a \texttt{PAtt} occorre fare un confronto con la sua percentuale di passaggi corti riusciti \texttt{MPCmp\%}.

	\item \textsf{MPCmp\%}: Tale variabile indica la percentuale di passaggi medi riusciti ai giocatori della squadra indicata sulla variabile \textsf{Team}. Questa variabile è stata inserita perché permette di capire quanti passaggi sono andati a buon fine tra tutti quelli tentati e quindi qual'è la precisione dei giocatori della squadra.
	\item \textsf{LPAtt}: Tale variabile indica il numero di passaggi lunghi tentati dai giocatori della squadra indicata sulla variabile \textsf{Team}. Per passaggi corti si intendono quelli effettuati all'interno di una lunghezza superiore ai 27 metri. Questi passaggi possono essere considerati come lanci lunghi per cambi di gioco o per lanciare le punte, cioè i giocatori che giocano come attaccanti, in profondità. Una rappresentazione di passaggio lungo è mostrata nella Figura \ref{fig:cambio}.
	\begin{figure}[h]
		\begin{center}
			\includegraphics[scale=0.53]{cambio-di-gioco.png}
			\caption{Esecuzione di un cambio di gioco} \label{fig:cambio}
		\end{center}
	\end{figure}
	
	Questa variabile è stata inserita per capire se un alto numero di passaggi lunghi possono essere determinanti ai fini dell'esito della partita. Ovviamente analogamente a \texttt{PAtt} occorre fare un confronto con la sua percentuale di passaggi corti riusciti \texttt{LPCmp\%}.

	\item \textsf{LPCmp\%}: Tale variabile indica la percentuale di passaggi lunghi riusciti ai giocatori della squadra indicata sulla variabile \textsf{Team}. 
	
	Questa variabile è stata inserita perché permette di capire quanti passaggi sono andati a buon fine tra tutti quelli tentati e quindi qual'è la precisione dei giocatori della squadra.
	\item \textsf{ToDefPen}: Tale variabile indica il numero di tocchi fatti dai giocatori della squadra indicata sulla variabile \textsf{Team} nella propria area di rigore. 
	
	Questa variabile è stata inserita perché può essere utile per capire come il possesso della palla viene gestito, cioè se vi è un alto numero di tocchi vuol dire che la squadra subisce molto la pressione della squadra avversaria, viceversa cerca di fare un gioco più offensivo. Questa variabile in combinazione con \textsf{ToDef3rd}, \textsf{ToMid3rd}, \textsf{ToAtt3rd} e \textsf{ToAttPen} permette di capire se il possesso della palla fatto della squadra è utile e porta benefici ai fini del risultato oppure è sterile. Inoltre si vuole capire attraverso l'analisi in che misura può influenzare il risultato della partita con un alto o un basso valore di numero di tocchi nella propria area di rigore la cui area nel campo da calcio è indicata nella Figura \ref{fig:penalty}.
	
	\begin{figure}[!h]
		\begin{center}
			\includegraphics[scale=0.60]{rigore.jpg}
			\caption{In rosso l'area di rigore in un campo da calcio.} 
			\label{fig:penalty}
		\end{center}
	\end{figure}
	

	\item \textsf{ToDef3rd}: Tale variabile indica il numero di tocchi fatti dai giocatori della squadra indicata sulla variabile \textsf{Team} nella propria mediana o trequarti difensiva. 
	
	Questa variabile è stata inserita perché può essere utile per capire come il possesso della palla viene gestito, cioè se vi è un alto numero di tocchi vuol dire che la squadra cerca di mantenere il possesso palla creando poche azioni offensive, viceversa cerca di fare un gioco più offensivo. Questa variabile in combinazione con \textsf{ToDef3rd}, \textsf{ToMid3rd}, \textsf{ToAtt3rd} e \textsf{ToAttPen} permette di capire se il possesso della palla fatto della squadra è utile e porta benefici ai fini del risultato oppure è sterile. Inoltre si vuole capire attraverso l'analisi in che misura può influenzare il risultato della partita con un alto o un basso valore di numero di tocchi nella propria mediana la cui area nel campo da calcio è indicata nella Figura \ref{fig:def}.
	
		\begin{figure}[!h]
		\begin{center}
			\includegraphics[scale=0.60]{mid.jpg}
			\caption{In rosso la mediana nel campo da calcio.} 
			\label{fig:def}
		\end{center}
	\end{figure}
	\item \textsf{ToMid3rd}: Tale variabile indica il numero di tocchi fatti dai giocatori della squadra indicata sulla variabile \textsf{Team} a centrocampo. 
	
	Questa variabile è stata inserita perché può essere utile per capire come il possesso palla viene gestito, cioè se vi è un alto numero di tocchi vuol dire che la squadra cerca di mantenere il possesso palla cercando di creare delle azioni offensive, viceversa cerca di fare un gioco più difensivo. Questa variabile in combinazione con \textsf{ToDef3rd}, \textsf{ToMid3rd}, \textsf{ToAtt3rd} e \textsf{ToAttPen} permette di capire se il possesso della palla fatto dalla squadra è utile e porta benefici ai fini del risultato oppure è sterile. Inoltre si vuole capire attraverso l'analisi in che misura può influenzare il risultato della partita con un alto o un basso valore di numero di tocchi a centrocampo la cui area nel campo da calcio è indicata nella Figura \ref{fig:cen}.
	
		\begin{figure}[!h]
		\begin{center}
			\includegraphics[scale=0.60]{cen.jpg}
			\caption{In rosso il centrocampo nel campo da calcio.} 
			\label{fig:cen}
		\end{center}
	\end{figure}

	\item \textsf{ToAtt3rd}: Tale variabile indica il numero di tocchi fatti dai giocatori della squadra indicata sulla variabile \textsf{Team} a nella trequarti dell'avversario. 
	
	Questa variabile è stata inserita perché può essere utile per capire come il possesso della palla viene gestito, cioè se vi è un alto numero di tocchi vuol dire che la squadra cerca di mantenere il possesso palla per effettuare una pressione sulla squadra avversaria affinché si possano creare degli spazi per delle azioni offensive, viceversa cerca di fare un gioco molto più difensivo. Questa variabile in combinazione con \textsf{ToDef3rd}, \textsf{ToMid3rd}, \textsf{ToAtt3rd} e \textsf{ToAttPen} permette di capire se il possesso della palla fatto della squadra è utile e porta benefici ai fini del risultato oppure è sterile. Inoltre si vuole capire attraverso l'analisi in che misura può influenzare il risultato della partita con un alto o un basso valore di numero di tocchi nella trequarti dell'avversario la cui area nel campo da calcio è indicata nella Figura \ref{fig:treq}.
	
		\begin{figure}[!h]
		\begin{center}
			\includegraphics[scale=0.60]{treq.jpg}
			\caption{In rosso la trequarti dell'avversario nel campo da calcio.} 
			\label{fig:treq}
		\end{center}
	\end{figure}

	\item \textsf{ToAttPen}: Tale variabile indica il numero di tocchi fatti dai giocatori della squadra indicata sulla variabile \textsf{Team} a nell'area di rigore dell'avversario. 
	
	Questa variabile è stata inserita perché può essere utile per capire come il possesso della palla viene gestito, cioè se vi è un alto numero di tocchi vuol dire che la squadra cerca di mantenere il possesso palla applicando un'alta pressione sulla squadra avversaria affinché si possano creare molte occasioni da gol in area, viceversa o la squadra subisce troppo la pressione dell'avversario oppure tende ad avere un gioco molto difensivo. Questa variabile in combinazione con \textsf{ToDef3rd}, \textsf{ToMid3rd}, \textsf{ToAtt3rd} e \textsf{ToAttPen} permette di capire se il possesso della palla fatto della squadra è utile e porta benefici ai fini del risultato oppure è sterile. Inoltre si vuole capire attraverso l'analisi in che misura può influenzare il risultato della partita con un alto o un basso valore di numero di tocchi nell'area di rigore dell'avversario.
	
	\item \textsf{ToDist}: Tale variabile indica la distanza totale, espressa in metri, in cui un giocatore della squadra indicata sulla variabile \textsf{Team}, si è mosso con la palla in qualsiasi direzione, controllandola con i piedi.
	
	Variabile che è stata inserita perché permette di ricavare se il possesso della palla sia stato statico cioè i giocatori si sono mossi poco senza avanzare, oppure no. Sarà di interesse analizzare se con un alto valore di metri percorsi con palla al piede, possa essere utile a ottenere la vittoria.
	\item \textsf{Fls}: Tale variabile indica il numero di falli dai giocatori della squadra indicata sulla variabile \textsf{Team}. 
	
	Questa variabile è stata inserita per capire se una squadra adotta un gioco più fisico/tattico. In questo caso sarà più propensa a interrompere il gioco della squadra avversaria e a commettere più falli. Si vuole perciò capire come questa variabile può andare ad influire sull'esito della partita, ricordando però che una che commette molti falli è più soggetta a ricevere cartellini gialli o rossi che condizionano la prestazione dei giocatori.
	\item \textsf{Fld}: Tale variabile indica il numero di falli subiti dai giocatori della squadra indicata sulla variabile \textsf{Team} da parte della squadra avversaria indicata sulla variabile \textsf{Vs}. 
	
	Si è deciso di inserire questa covariata perché un alto numero di falli può portare a molte interruzione della manovra di gioco e quindi permettere alla squadra avversaria di riorganizzarsi. Si vuole perciò capire come questa variabile può andare ad influire sull'esito della partita.
	\item \textsf{Off}: Tale variabile indica il numero di volte che la squadra indicata sulla variabile \textsf{Team} è finita in fuorigioco. Un calciatore si trova in posizione di fuorigioco quando una qualsiasi parte del suo corpo, fatta eccezione per braccia e mani perché non possono essere usate per controllare il pallone; si trova nella nella metà campo avversaria ed è più vicina alla linea di porta avversaria sia rispetto al pallone e sia rispetto al penultimo giocatore difendente avversario; tale penultimo avversario può essere anche il portiere, se un compagno di questi è più vicino di lui alla linea di porta. Una rappresentazione grafica del fuorigioco è mostrata nell'immagine \ref{fig:offside}.
	
	\begin{figure}[!h]
		\begin{center}
			\includegraphics[scale=0.55]{var.png}
			\caption{Rappresentazione del fuorigioco} \label{fig:offside}
		\end{center}
	\end{figure}

	Tale variabile è stata inserita perché, se una squadra viene colta molte volte in fuorigioco allora il suo gioco sarà interrotto e darà un vantaggio alla squadra avversaria che farà ripartire la sua azione a suo favore.
	
	
	\item \textsf{Crs}: Tale variabile indica il numero di cross effettuati dalla squadra indicata sulla variabile \textsf{Team}. Un cross in italiano traversone, è un tipo di passaggio medio o lungo, solitamente effettuato sulle fasce laterali dell'area avversaria o comunque vicino all'area avversaria, che se eseguito permette al compagno di squadra posizionato vicino alla porta avversaria, di colpire la palla al volo di testa oppure di piede per segnare un possibile gol. Quindi se eseguito correttamente, il cross può diventare un assist, cioè l'ultimo passaggio per la realizzazione del gol. 
	
	\begin{figure}[!h]
		\begin{center}
			\includegraphics[scale=0.43]{cross.jpg}
			\caption{Rappresentazione di un cross} \label{fig:cross}
		\end{center}
	\end{figure}
	
	Per tale motivo si è deciso di inserire una variabile specifica per questo tipo di passaggio nell'analisi. Una rappresentazione di un cross è mostrata nella Figura \ref{fig:cross}.
	\item \textsf{Int}: Tale variabile indica il numero di intercettazioni fatte dai giocatori della squadra indicata sulla variabile \textsf{Team}. Per intercettazione della palla si intende interrompere un passaggio della squadra avversaria entrando in possesso del pallone che era stato lanciato per un passaggio ma che una volta intercettato non è andato a buon fine cioè non è arrivato al compagno del giocatore avversario che ha effettuato il passaggio. 
	
	Quindi si è deciso di inserire questa variabile perché indica quante volte si è tolto il possesso della palla all'avversario interrompendone il suo gioco.
	\item \textsf{TklWin}: Tale variabile indica il numero di contrasti vinti dai giocatori della squadra indicata sulla variabile \textsf{Team}. Per contrasto si intende il tentativo da parte di un giocatore difendente di sottrarre il possesso della palla all'avversario. Quindi chi ha in possesso la palla viene attaccato da chi ne è privo. Se si riesce a prendere il pallone all'avversario allora si avrà vinto il contrasto. Si sottolinea che i contrasti vengono anche effettuati per allontanare dalle zone pericolose l'avversario. La Figura \ref{fig:tackle} mostra un contrasto di gioco.
	
	\begin{figure}[!h]
		\begin{center}
			\includegraphics[scale=0.40]{tackle.jpg}
			\caption{Rappresentazione di un contrasto in scivolata} \label{fig:tackle}
		\end{center}
	\end{figure}
	
	Visto che tale intervento senza palla va a modificare il gioco dell'avversario, si è deciso di inserire i contrasti vinti come variabile. 
	
	\item \textsf{Recov}: Tale variabile indica il numero di palle vaganti recuperate dalla squadra indicata sulla variabile \textsf{Team}. Per palle vaganti si intendono quei palloni che a seguito di un contrasto di gioco, non sono stati recuperati dalla squadra che ha effettuato il contrasto ma chi ha subito il contrasto ne ha comunque perso il controllo. Si ha che nessuno ha in possesso il pallone e quindi si ha una palla vagante.
	
	Dato che questa variabile sembra essere legata al possesso del pallone sembra essere interessante per l'analisi.
	
\end{itemize}
\section{Adattamento del dataset al modello Bradley-Terry}
Nella sezione precedente si è descritto come si è costruito il dataset che verrà utilizzato per l'analisi e come esso è stato strutturato. Tale struttura ha il vantaggio di essere di facile interpretazione per un essere umano ma vi sono alcune criticità che non lo permettono di essere utilizzato correttamente all'interno del modello messo a disposizione dal pacchetto \texttt{BradleyTerry2}.\\ Dopo aver importato il dataset sul software R, sono state perciò necessarie apportare alcune modifiche attraverso la scrittura di codice che andasse a modificare la struttura del dataset per poi essere correttamente utilizzabile nel modello. \\

Innanzitutto il modello richiede per il suo funzionamento che le due variabili che indicano quali delle due squadra hanno partecipato alla partita in esame e quindi alla comparazione che si sta analizzando; devono essere o di tipo fattore oppure un \textsf{data.frame}. Una variabile fattore è un variabile non numerica, espressa in termini verbali ad esempio una categoria. Un \textsf{data.frame} è una lista di vettori, che devono avere tutti la stessa lunghezza, ma possono essere di tipo diverso: variabili nominali (fattori), variabili cardinali (vettori numerici); un \textsf{data.frame} può essere visto come una matrice ma con il tipo dei valori che può essere diverso.\\ Dato che le due variabili in questione \textsf{Team} e \textsf{Vs} erano solo di tipo \texttt{character} e si voleva trovare un modo per far capire al modello quelli valori erano legati alla squadra indicata in \textsf{Team} e quali in \textsf{Vs} nella stessa partita; si è deciso perciò di trasformare \textsf{Team} e \textsf{Vs} in \texttt{data.frame} inserendo al loro interno tutte le covariate descritte nella sezione precedente, ad esempio \textsf{Poss}, \textsf{Int} ecc..

Si sottolinea inoltre che sono state necessarie ulteriore modifiche per quanto riguarda la variabile \textsf{AtHome}, che sebbene si possa pensare che i valori scritti siano di tipo logico in realtà essi venivano interpretati come stringhe, quindi si sono subito trasformati in valori logici con il commando \textsf{as.logical(soccern\$AtHome)}. Ciononostante pero il valore logico non era accettato dal modello ma era accettato un valore numerico per indicare se la squadra giocava in casa o no; si è quindi convertito il valore \texttt{TRUE} in 1 mentre FALSE in 0. 
\subsection{Codice per l'adattamento del dataset}
Di seguito viene mostrato il codice applicato per adeguare il dataset con le modifiche scritte precedentemente.

\begin{lstlisting}
PossVs <- c()
ShVs <- c()
ShTVs <- c()
G.ShVs <- c()
SavesVs <- c()
PAttVs <- c()
PCmp.Vs <- c()
SPAttVs <- c()
SPCmp.Vs <- c()
MPAttVs <- c()
MPCmp.Vs <- c()
LPAttVs <- c()
LPCmp.Vs <- c()
ToDefPenVs <- c()
ToDef3rdVs <- c()
ToMid3rdVs <- c()
ToAtt3rdVs <- c()
ToAttPenVs <- c()
ToDistVs <- c()
FlsVs <- c()
FldVs <- c()
OffVs <- c()
CrsVs <- c()
IntVs <- c()
TklWinVs <- c()
RecovVs <- c()
del <-c()
k <- 1
z <- 1
for(i in 1:nrow(soccern)){
   if(soccern$AtHome[i] == TRUE){
     for(j in 1:nrow(soccern)){
	 if((soccern$Team[j] == soccern$Vs[i]) && (soccern$Team[i] == soccern$Vs[j]) && (soccern$AtHome[j] == FALSE)){
		PossVs[k] <- soccern$Poss[j]
		ShVs[k] <- soccern$Sh[j]
		ShTVs[k] <- soccern$SoT[j]
		G.ShVs[k] <- soccern$G.Sh[j]
		SavesVs[k] <- soccern$Saves[j]
		PAttVs[k] <- soccern$PAtt[j]
		PCmp.Vs[k] <- soccern$PCmp.[j]
		SPAttVs[k] <- soccern$SPAtt[j]
		SPCmp.Vs[k] <- soccern$SPCmp.[j]
		MPAttVs[k] <- soccern$MPAtt[j]
		MPCmp.Vs[k] <- soccern$MPCmp.[j]
		LPAttVs[k] <- soccern$LPAtt[j]
		LPCmp.Vs[k] <- soccern$LPCmp.[j]
		ToDefPenVs[k] <- soccern$ToDefPen[j]
		ToDef3rdVs[k] <- soccern$ToDef3rd[j]
		ToMid3rdVs[k] <- soccern$ToMid3rd[j]
		ToAtt3rdVs[k] <- soccern$ToAtt3rd[j]
		ToAttPenVs[k] <- soccern$ToAttPen[j]
		ToDistVs[k] <- soccern$TotDist[j]
		FlsVs[k] <- soccern$Fls[j]
		FldVs[k] <- soccern$Fld[j]
		OffVs[k] <- soccern$Off[j]
		CrsVs[k] <- soccern$Crs[j]
		IntVs[k] <- soccern$Int[j]
		TklWinVs[k] <- soccern$TklWin[j]
		RecovVs[k] <- soccern$Recov[j]
		k <- k + 1
	   }      
	}
   }
   else{
	del[z] <- i
	z <- z + 1
   }
}
\end{lstlisting}

Con il codice precedente si ha l'obbiettivo di prendere tutti i dati delle partite giocate fuori casa (\textsf{AtHome} = FALSE) dalle squadre indicate in \textsf{Team} per poi unirli con i dati delle righe che contengono i dati delle partite giocate in casa (\textsf{AtHome} = TRUE) dalle squadre indicate in \textsf{Team}. Le righe che contengono solo i dati delle partite fuori casa saranno eliminate.\\
Perciò si è creato un vettore vuoto per ogni covariata presente nel dataset, ad eccezione di \textsf{AtHome} che verrà gestita in un modo diverso. Il vettore \texttt{del} è il vettore che tiene traccia di quali righe saranno da eliminare. \texttt{k} è l'indice usato per scorre il dataset per trovare i dati dell'avversario; \texttt{z} l'indice usato pe per inserire un nuovo elemento nel vettore \texttt{del}.\\
Il primo ciclo \texttt{for} scorre tutto il dataset alla ricerca delle righe con i dati delle partite giocate in casa dalla squadra indicata in \texttt{Team}, infatti al suo interno il primo costrutto \texttt{if} controlla se la partita è in casa per \texttt{Team} se sì, parte un secondo ciclo \texttt{for} che anche esso scorre tutto il dataset per cercare la riga con la partita giocata della squadra indicata in \texttt{Vs}; giocata ovviamente fuori casa. Perciò all'interno del secondo ciclo \texttt{for} vi è un costrutto \texttt{if} che controlla se la j-esima riga si riferisce alla stessa partita indicata nella i-esima riga, se sì allora si salvano tutti i dati nei vettori e si incrementa l'indice \texttt{k}. Se il primo \texttt{if} da esito negativo allora si andrà a inserire l'indice dell'i-esima riga perché contiene informazioni di una partita giocata fuori casa dalla squadra indicata in \textsf{Team} e viene incrementato l'indice di uno \texttt{z}.\\

Di seguito vengono riportati i comandi fatti per applicare le modifiche al dataset.

\begin{lstlisting}
> soccern3 <- soccern2[-del,]
\end{lstlisting}

Con il precedente commando si va a creare un nuovo dataset con 380 righe, eliminando tutte quelle righe con valore \texttt{FALSE} su \textsf{AtHome}. 

\begin{lstlisting}
> soccern3$Team <- data.frame(team = soccern3$Team, GF = soccern3$GF, GA = soccern3$GA,  at.home = 1, Poss = soccern3$Poss, Sh = soccern3$Sh, SoT = soccern3$SoT, G.Sh = soccern3$G.Sh, Saves = soccern3$Saves, PAtt = soccern3$PAtt, PCmp. = soccern3$PCmp., SPAtt = soccern3$SPAtt, SPCmp. = soccern3$SPCmp., MPAtt = soccern3$MPAtt, MPCmp. = soccern3$MPCmp., LPAtt = soccern3$LPAtt, LPCmp. = soccern3$LPCmp., ToDefPen = soccern3$ToDefPen, ToDef3rd = soccern3$ToDef3rd, ToAtt3rd = soccern3$ToAtt3rd, ToAttPen = soccern3$ToAttPen, TotDist = soccern3$TotDist, Fls = soccern3$Fls, Fld = soccern3$Fld, Off = soccern3$Off, Crs = soccern3$Crs, Int = soccern3$Int, TklWin = soccern3$TklWin, Recov = soccern3$Recov)

\end{lstlisting}
Con il precedente commando si va a modificare \textsf{Team} rendendolo un \texttt{data.frame}, andando a inserire i dati della riga relativi alla squadra che gioca in casa. Si inserisce come chiave \texttt{team = soccern3\$Team} e si indica che la partita è in casa per la squadra di riferimento con \texttt{at.home = 1}.

\begin{lstlisting}
> soccern3$Vs <- data.frame(team = soccern3$Vs, GF = GFVs, GA = GAVs, at.home = 0, Poss = PossVs, Sh = ShVs, SoT = ShTVs, G.Sh = G.ShVs, Saves = SavesVs, PAtt = PAttVs, PCmp. = PCmp.Vs, SPAtt = SPAttVs, SPCmp. = SPCmp.Vs, MPAtt = MPAttVs, MPCmp. = MPCmp.Vs, LPAtt = LPAttVs, LPCmp. = LPCmp.Vs, ToDefPen = ToDefPenVs, ToDef3rd = ToDef3rdVs, ToAtt3rd = ToAtt3rdVs, ToAttPen = ToAttPenVs, TotDist = ToDistVs, Fls = FlsVs, Fld = FldVs, Off = OffVs, Crs = CrsVs, Int = IntVs, TklWin = TklWinVs, Recov = RecovVs)

\end{lstlisting}
Con il precedente commando si va a modificare \textsf{Vs} rendendolo un \texttt{data.frame}, andando a inserire i dati della riga relativi alla squadra che gioca fuori casa. Si inserisce come chiave \texttt{team = soccern3\$Vs} e si indica che la partita è fuori casa per la squadra \texttt{Vs} con \texttt{at.home = 0}.\\ Per quanto riguarda il resto dei dati vengo riportati attraverso l'inserimento dei vettori costruiti e riempiti precedentemente.
\newpage

%\label{cap:obbiettivi}
\begin{comment}
	\begin{table}[h]%
		\rowcolors{2}{grigetto}{white}
		\renewcommand{\arraystretch}{1.7}
		\centering
		\begin{tabularx}{\textwidth}{c c X}
			\hline	
			\rowcolor{heavenly}
			\intest{Durata in ore} & \intest{Date (inizio - fine)} & \intest{Attività} \\	
			\hline			
			24 &  01/07/2020 - 03/07/2020 & Studio delle tecnologie Angular 2+ e Ionic, da utilizzare durante lo stage.\\
			
			40 &  06/07/2020 - 10/07/2020 & Studio di componenti dell'architettura di sistema di Azzurra, creazione di \emph{test} per la \emph{dashboard} di \gls{AWMS} e per l'applicazione \emph{mobile}. \\
			
			40 &  13/07/2020 - 17/07/2020 & Continuazione studio delle componenti del sistema di Azzurra, analisi, progettazione e implementazione di flussi conversazionali.\\
			
			40 &  20/07/2020 - 24/07/2020 & Documentazione per le componenti di Azzurra.\\
			
			40 &  27/07/2020 - 31/07/2020 & Continuazione studio di altre componenti di \gls{AWMS}.\\
			
			40 &  03/08/2020 - 07/08/2020 & Documentazione delle componenti \gls{AWMS} e implementazione \glslink{notifica push}{notifiche push}\textcolor{SchoolColor}{\ap{[g]}}.\\
			
			40 &  17/08/2020 - 21/08/2020 & Progettazione, implementazione e documentazione di \emph{template engine} multi-lingua.\\
			
			40 &  24/08/2020 - 28/08/2020 & Studio della gestione dei comportamenti \emph{mobile} \emph{application} in condizioni di mancanza di connettività.\\
			
			16 &  31/08/2020 - 01/09/2020 & Continuazione ottava settimana. \\
			\hline
		\end{tabularx} \hbox{}
		
		\caption{Tabella riassuntiva delle attività pianificate per il progetto di stage}
	\end{table}
\end{comment}
