% !TEX encoding = UTF-8
% !TEX TS-program = pdflatex
% !TEX root = ../tesi.tex

%**************************************************************
\chapter{Analisi dei dati}
%\label{cap:flow engine}
%**************************************************************
\intro{Nel seguente capitolo verrà illustrata la fase di preprocessing e le analisi grafiche dei dati. Le analisi verranno svolte usando il linguaggio di programmazione di R \autocite{R}. }\\

%*************************************************************

\section{Preprocessing dei dati}
Dopo aver importato il dataset utilizzando il linguaggio di programmazione R \autocite{R}, il primo step da effettuare durante il prepocessing è individuare e risolvere possibili anomalie nei dati.
Il dataset è stato importato in modo che la prima riga contenga l'intestazione, mentre le restanti righe tutte le osservazioni. Il comando usato per importare il dataset è il seguente:\\

\begin{lstlisting}[language=R]
> soccer<-read.xlsx("SerieA.xlsx", 1, header=TRUE)
\end{lstlisting}
\bigskip
Il dataset non ha valori mancanti. Questo è stato possibile grazie a FBref che ha messo a disposizione dati quasi sempre completi; in quei rari casi di mancanza di dati sono stati reperiti manualmente da altre fonti altrettanto attendibili.\\ 
Sono state inoltre tolte le variabili \texttt{Date} e \texttt{Round}.\\
Il passo successivo è stato controllare che le variabili fossero interpretate correttamente. \texttt{Team} e \texttt{Vs} vengono interpretate erroneamente come tipo \texttt{character}. \texttt{Team} e \texttt{Vs} devono essere interpretate come un fattore cioè è un valore non numerico, espresso in termini verbali, ad esempio una categoria; quindi ogni squadra sarà un livello del fattore. Analogamente, \texttt{AtHome} è stata fatta trasformata in un fattore a due livelli. Invece, \texttt{Res} è stata trasformata in un fattore ordinato con i livelli: -1 = sconfitta <  0 = pareggio < 1 = vittoria.


\section{Analisi grafica dei dati}
In questa sezione attraverso il supporto di grafici, si analizzerà graficamente i dati disponibili e le loro relazione per avere una prima visione dei dati raccolti. Si valuteranno le relazione tra covariate e la variabile di risposta, le relazioni tra due covariate. Tutto ciò per individuare quali covariate possano essere significative per la variabile risposta e quali interazioni emergono dall'analisi grafica.\\

Come primo passo, è stata valutata la distribuzione della variabile risposta \texttt{Res}, come è mostrato in Figura \ref{fig:res}.

\begin{figure}[htbp]
	\begin{center}
		\includegraphics[scale=0.40]{barRes.png}
		\caption{Barplot della distribuzione della variabile di risposta \texttt{Res}} \label{fig:res}
	\end{center}
\end{figure}

Si può notare come le classi sembrino ben distribuite, dato che abbiamo 196 pareggi e 282 vittorie e altrettante sconfitte. Si ha quindi un campione abbastanza ampio, distribuito e privo di classi povere.\\

 La Figura \ref{fig:team} mostra la distribuzione delle vittorie, dei pareggi e delle sconfitte per ogni squadra.

\newpage
\paperwidth=\pdfpageheight
\paperheight=\pdfpagewidth
\pdfpageheight=\paperheight
\pdfpagewidth=\paperwidth
\headwidth=\textheight

\begingroup 
\vsize=\textwidth
\hsize=\textheight


\pagestyle{empty}
\begin{figure}[htbp]
	\begin{center}
		\includegraphics[height = 15cm, width = 25cm]{ResTeam.png}
		\caption{Barplot della distribuzione della variabile di risposta per squadra\texttt{Res}} \label{fig:team}
	\end{center}
\end{figure}

\textwidth=\hsize
\textheight=\vsize

\endgroup
\newpage
\paperwidth=\pdfpageheight
\paperheight=\pdfpagewidth
\pdfpageheight=\paperheight
\pdfpagewidth=\paperwidth
\headwidth=\textwidth


\subsection{Relazione tra la variabile risposta e le covariate}

La prima relazione che si analizza riguarda la variabile categorica \texttt{AtHome}. Nella Figura \ref{fig:AtHome} viene riportato il mosaicplot tra la variabile risposta e \texttt{AtHome}. Tale grafico è un particolare tipo di diagramma a barre impilate che mostra la relazione che c'è tra due fattori. Il numero di colonne è uguale al numero livelli della variabile inserita sull'asse orizzontale. L'altezza delle barre in verticale, invece, è proporzionale al numero di osservazioni della variabile inserita sull'asse verticale per ciascun livello della variabile nell'asse orizzontale.
In sostanza, il mosaicplot è una rappresentazione grafica di una tabella di contingenza che permette un confronto visivo tra gruppi. Nella Figura \ref{fig:AtHome} c'è una leggera variazione dei risultati tra la squadra che gioca in casa e l'avversaria, infatti per le squadre che giocano in casa, c'è una maggior presenza di vittorie e di minor sconfitte. Naturalmente non c'è alcuna variazione per il pareggio dato che entrambe le squadre lo ottengono.

\begin{figure}[htbp]
	\begin{center}
		\includegraphics[scale=0.45]{AtHomeRes.png}
		\caption{Mosaicplot che mostra la distribuzione degli esiti rispetto alle partite giocate in casa e fuori casa} \label{fig:AtHome}
	\end{center}
\end{figure}

Nella Figura \ref{fig:Poss} viene riportato il boxplot della distribuzione della variabile \texttt{Poss} rispetto ai valori della variabile risposta \texttt{Res}. Il boxplot è un grafico che consente di visualizzare il centro e la distribuzione dei dati. Inoltre, può essere un strumento visivo per la verifica della normalità o per l'identificazione di possibili outlier. Dal grafico si nota che \texttt{Poss} sembra essere significativa per l'esito. Infatti i valori crescono dal boxplot della sconfitta al boxplot della vittoria. C'è una buona distribuzione dei dati perché la lunghezza dei baffi per ogni boxplot è simmetrica. Si segnala che la mediana della sconfitta è più vicina al 3$^{\circ}$ quantile mentre la mediana della vittoria è più vicina al 1$^{\circ}$ quantile. Non sono presenti outliers.\\

\begin{figure}[htbp]
	\begin{center}
		\includegraphics[scale=0.50]{Poss.png}
		\caption{Boxplot della distribuzione della variabile \texttt{Poss} rispetto ai valori della variabile risposta \texttt{Res}} \label{fig:Poss}
	\end{center}
\end{figure}

Nella Figura \ref{fig:sot} viene riportato il boxplot della distribuzione della variabile \texttt{SoT} rispetto ai valori della variabile risposta \texttt{Res}. Valori più alti sono presenti nella vittoria mentre valori molto più bassi sono presenti nella sconfitta. C'è una buona distribuzione dei valori nella vittoria dato che i baffi sono simmetrici, viceversa per le altri due boxplot non c'è simmetria infatti, il baffo inferiore è molto più corto rispetto al baffo superiore, segno che la maggior parte dei valori sono bassi e simili tra loro. Inoltre alcuni outliers si discostano dalla distribuzione di tutti e tre i boxplot, questo perché ci sono state squadre che hanno tirato molte volte in porta. Le mediane dei boxplot pareggio e vittoria non sono equidistanti dai quantili ma più vicine al 1$^{\circ}$ quantile. Il boxplot della sconfitta ha una bassa varianza. In conclusione, avere un valore alto di tiri in porta sembra essere utile ai fini della vittoria.\\

Per la relazione tra la variabile risposta e la variabile \texttt{Sh}, si ha un boxplot molto simile al boxplot mostrato nella Figura \ref{fig:Poss}. Il grafico di \texttt{Sh} rispetto al grafico di \texttt{Poss}, ha degli outliers e la mediana della sconfitta non è equidistante dai quantili ma più vicina al 1$^{\circ}$ quantile.\\

\begin{figure}[htbp]
	\begin{center}
		\includegraphics[scale=0.50]{SoT.png}
		\caption{Boxplot della distribuzione della variabile \texttt{SoT} rispetto ai valori della variabile risposta \texttt{Res} } \label{fig:sot}
	\end{center}
\end{figure}

Nella Figura \ref{fig:g} viene riportato il boxplot della distribuzione della variabile \texttt{G/Sh} rispetto ai valori della variabile risposta \texttt{Res}. Si nota che ci sono valori molto bassi ma leggermente più alti per la vittoria. La distribuzione non è buona perché i baffi sono asimmetrici infatti, tutti i valori sono concentrati in basso e pochi verso il baffo superiore, segno che la maggior parte dei valori sono bassi e simili tra loro. C'è una bassa varianza tra i valori. C'è la presenza di outliers perché alcune squadre sono riuscite a ottenere il massimo da ogni tiro. I risultati mostrati, nonostante la distribuzione, sono comunque coerenti dato che non ci si aspetta dal rapporto tiri-gol un numero alto ma comunque una tendenza che favorisca la vittoria.\\

\begin{figure}[htbp]
	\begin{center}
		\includegraphics[scale=0.50]{g.png}
		\caption{Boxplot della distribuzione della variabile \texttt{G/Sh} rispetto ai valori della variabile risposta \texttt{Res} } \label{fig:g}
	\end{center}
\end{figure}

Nella Figura \ref{fig:saves} viene riportato il boxplot della distribuzione della variabile \texttt{Saves} rispetto ai valori della variabile risposta \texttt{Res}. Come si può notare sembra che \texttt{Saves} sia poco significativa ai fini del risultato. Infatti c'è poca variazione tra un boxplot e l'altro perché sembra che avere un alto numero di parate non è determinante a fini del risultato.\\

\begin{figure}[htbp]
	\begin{center}
		\includegraphics[scale=0.50]{saves.png}
		\caption{Boxplot della distribuzione della variabile \texttt{Saves} rispetto ai valori della variabile risposta \texttt{Res} } \label{fig:saves}
	\end{center}
\end{figure}

La Figura \ref{fig:pass} viene riportato a sinistra il boxplot della variabile numerica \texttt{PAtt} rispetto ai valori della variabile risposta \texttt{Res} e a destra il boxplot della variabile numerica \texttt{PCmp\%} rispetto ai valori della variabile risposta \texttt{Res}. Per entrambi sembra significativo l'elevato numero di passaggi tentati ma soprattutto quelli completati ai fini della vittoria. Nel grafico a sinistra, nel secondo e terzo boxplot il baffo superiore è più lungo rispetto al baffo inferiore, segno che molti valori sono bassi e simili tra loro, viceversa il primo boxplot ha una buona distribuzione perché i baffi sono simmetrici. Il boxplot della vittoria ha una maggiore varianza rispetto agli altri due e in più ha valori più alti; sia la mediana del boxplot della vittoria e sia quello del pareggio sono più vicine al 1$^{\circ}$ quantile, viceversa quella della sconfitta. I dati nel primo boxplot sembrano essere coerenti con l'esito della partita perché, maggior numero di passaggi si prova ad effettuare, maggiori sono le possibilità di vittoria. Occorre però sapere quanto è precisa la squadra e questo lo si può scoprire con la variabile \texttt{PCmp\%}\\

Nel grafico a destra, si notano valori alti e molti outliers con valori bassi dovuti al fatto che ci sono state partite dove alcune squadre sono state poco precise nei passaggi. I baffi superiori di tutti e tre i boxplot sono molto meno lunghi rispetto ai baffi inferiori segno che molti valori sono alti e simili tra loro, inoltre, le varianze dei box sembrano essere uguali tra di loro. Sorprendentemente l'andamento invece di essere sempre crescente, prima scende da sconfitta a pareggio e poi sale da pareggio a vittoria.\\

Per la relazione tra la variabile risposta e la variabile \texttt{SPAtt}, si ha un grafico molto simile al grafico a sinistra della Figura \ref{fig:pass}. Il grafico di \texttt{SPAtt} rispetto al grafico di \texttt{PAtt}, ha un maggior numero di outliers soprattutto per la sconfitta rispetto al grafico \texttt{PAtt} inoltre, c'è una minor varianza per tutti i tre boxplot oltre a valori più bassi in generale, questo è naturale perché \texttt{PAtt} contiene tutti i passaggi tentati e non solo quelli corti.\\

Per la relazione tra la variabile risposta e la variabile \texttt{SPCmp\%}, si ha un grafico molto simile al grafico a destra della Figura \ref{fig:pass}. Il grafico di \texttt{SPCmp\%} rispetto al grafico di \texttt{PCmp\%}, il boxplot della sconfitta ha una maggior varianza, viceversa per la vittoria, che ha una minor varianza.\\

Per la relazione tra la variabile risposta e la variabile \texttt{MPAtt}, si ha un grafico molto simile al grafico a sinistra della Figura \ref{fig:pass}. Il grafico di \texttt{MPAtt} rispetto al grafico di \texttt{PAtt}, il boxplot della sconfitta ha una maggior varianza. In generale i valori sono più bassi rispetto al grafico di \texttt{PAtt} ma questo è naturale perché \texttt{PAtt} contiene tutti i passaggi tentati e non solo quelli medi.\\

Per la relazione tra la variabile risposta e la variabile \texttt{MPCmp\%}, si ha un grafico molto simile al grafico a destra della Figura \ref{fig:pass}. Il grafico di \texttt{MPCmp\%} rispetto al grafico di \texttt{PCmp\%}, ha valori più alti e molti più outliers, inoltre i baffi inferiore dei boxplot della sconfitta e della vittoria sono più corti.\\

Per la relazione tra la variabile risposta e la variabile \texttt{LPAtt}, si ha un grafico molto simile al grafico a sinistra della Figura \ref{fig:pass}. Il grafico di \texttt{LPAtt} rispetto al grafico di \texttt{PAtt}, ha per il boxplot della sconfitta valori più bassi rispetto agli boxplot del pareggio e della vittoria inoltre, il boxplot del pareggio ha una maggior varianza valori mentre il boxplot della vittoria ha una minor varianza.\\
In generale i valori sono più bassi rispetto al grafico di \texttt{PAtt} ma questo è naturale perché \texttt{PAtt} contiene tutti i passaggi tentati e non solo quelli lunghi.\\

Per la relazione tra la variabile risposta e la variabile \texttt{LPCmp\%}, si ha un grafico molto simile al grafico a destra della Figura \ref{fig:pass}. Il grafico di \texttt{LPCmp\%} rispetto al grafico di \texttt{PCmp\%}, ha valori più bassi, la distribuzione dei valori per il boxplot della sconfitta è ben equilibrata perché i baffi sono della stessa lunghezza e in più la mediana è equidistante dai due quantili, analogamente anche il boxplot del pareggio ha una distribuzione equilibrata ma con più varianza e una mediana equidistante dai quantili.\\

\begin{figure}[htbp]
	\begin{center}
		\includegraphics[height=8cm,width=13cm]{pass.png}
		\caption{A sinistra il boxplot della variabile numerica \texttt{PAtt} rispetto ai valori della variabile risposta \texttt{Res} e a destra il boxplot della variabile numerica \texttt{PCmp\%} rispetto ai valori della variabile risposta \texttt{Res}} \label{fig:pass}
	\end{center}
\end{figure}

Nella Figura \ref{fig:defp} viene riportato il boxplot della distribuzione della variabile \texttt{ToDefPen} rispetto ai valori della variabile risposta \texttt{Res}. Si nota che non c'è nessuna variazione dei tre boxplot, oltre ad avere la stessa varianza. L'esito può essere giustificato dal fatto che le squadre cercano di rimane fuori il più possibile dalla propria area di rigore per non portare troppo vicino alla porta l'avversario. Da ciò si può ipotizzare che \texttt{ToDefPen} non è significativa per la variabile risposta. Prima di escluderla si andrà ad analizzare se c'è qualche interazione con altre variabili che la fanno diventare significativa.\\

\begin{figure}[htbp]
	\begin{center}
		\includegraphics[scale=0.50]{def.png}
		\caption{Boxplot della distribuzione della variabile \texttt{ToDefPen} rispetto ai valori della variabile risposta \texttt{Res} } \label{fig:defp}
	\end{center}
\end{figure}

Nella Figura \ref{fig:att} viene riportato il boxplot della distribuzione della variabile \texttt{ToAttPen} rispetto ai valori della variabile risposta \texttt{Res}. Contrariamente quanto visto con la Figura \ref{fig:defp} qui si nota una certa variazione tra i boxplot infatti, i valori crescono dal boxplot della sconfitta fino al boxplot della vittoria. C'è una maggior varianza per il boxplot della vittoria rispetto agli altri due boxplot. Per tutti e tre i boxplot i baffi inferiori sono leggermente meno lunghi rispetto ai baffi superiori, segno che i valori sono bassi e simili tra loro infatti, ci sono alcuni outliers sopra al baffo superiore, segno che alcune squadre in qualche partita, si sono particolarmente rese note nel produrre un quantitativo di tocchi maggiore rispetto alla distribuzione, ciò però non sembra influenzare l'esito. Le mediane sono equidistanti.\\

Per la relazione tra la variabile risposta e la variabile \texttt{ToDef3rd}, si ha un grafico molto simile a quello mostrato nella Figura \ref{fig:att}. Il grafico di \texttt{ToDef3rd} rispetto al grafico di \texttt{ToAttPen}, ha un minore numero di outliers soprattutto per il boxplot del pareggio, tale boxplot ha inoltre una varianza simile al boxplot della sconfitta. Il boxplot della vittoria invece, ha una distribuzione ben equilibrata.\\

Per la relazione tra la variabile risposta e la variabile \texttt{ToMid3rd}, si ha un grafico molto simile a quello mostrato nella Figura \ref{fig:att}. Il grafico di \texttt{ToMid3rd} rispetto al grafico di \texttt{ToAttPen}, ha un minore numero di outliers e la varianza del boxplot della sconfitta è molto simile alla mediana del boxplot del pareggio ma con la mediana più vicina al 3$^{\circ}$ quantile.\\

Per la relazione tra la variabile risposta e la variabile \texttt{ToAtt3rd}, si ha un grafico molto simile a quello mostrato nella Figura \ref{fig:att}. Il grafico di \texttt{ToAtt3rd} rispetto al grafico di \texttt{ToAttPen}, ha una minor varianza in generale per tutti e tre i boxplot e una distribuzione sbilanciata verso valori più bassi dato che tutti i baffi inferiori sono più corti rispetto ai baffi superiori. L'andamento però rimane lo stesso presente nella Figura \ref{fig:att}.\\

\begin{figure}[htbp]
	\begin{center}
		\includegraphics[scale=0.50]{att.png}
		\caption{Boxplot della distribuzione della variabile \texttt{ToAttPen} rispetto ai valori della variabile risposta \texttt{Res} } \label{fig:att}
	\end{center}
\end{figure}

Nella Figura \ref{fig:falli} vengono riportati a sinistra il boxplot della variabile numerica \texttt{Fls} rispetto ai valori della variabile risposta \texttt{Res} e a destra il boxplot della variabile numerica \texttt{Fld} rispetto ai valori della variabile risposta \texttt{Res}. Nel boxplot a sinistra si può notare che i valori più alti sono nel boxplot del pareggio e della vittoria ma nel boxplot del pareggio ci sono più valori alti. Ciò fa ipotizzare che subire molti falli può impedire la vittoria alla squadra che li subisce. Per quanto riguarda la distribuzione sembra essere buona; c'è una minor varianza per quanto riguarda il boxplot della sconfitta. \\

\begin{figure}[htbp]
	\begin{center}
		\includegraphics[height=8cm,width=13cm]{falli.png}
		\caption{A sinistra il boxplot della variabile numerica \texttt{Fls} rispetto ai valori della variabile risposta \texttt{Res} e a destra il boxplot della variabile numerica \texttt{Fld} rispetto ai valori della variabile risposta \texttt{Res}} \label{fig:falli}
	\end{center}
\end{figure}

Nel secondo boxplot si hanno valori valori più alti nel boxplot della vittoria e una maggior varianza rispetto al boxplot della sconfitta. Sembra perciò che dal grafico si può intuire che se la squadra non commette dei falli allora sarà più soggetta a perdere.\\

Per la relazione tra la variabile risposta e la variabile \texttt{Off}, si ha un grafico molto simile a quello mostrato nella Figura \ref{fig:saves}. Il grafico di \texttt{Off} rispetto al grafico di \texttt{Saves}, ha un numero minore di valori per il boxplot della sconfitta rispetto agli altri due boxplot inoltre, le mediane del boxplot della sconfitta e del pareggio sono attaccate al 1$^{\circ}$ quantile.\\

Per la relazione tra la variabile risposta e la variabile \texttt{Crs}, si ha un grafico molto simile a quello mostrato nella Figura \ref{fig:int}. Il grafico di \texttt{Crs} rispetto al grafico di \texttt{Saves}, ha per il boxplot della sconfitta maggior varianza e il baffo inferiore dei boxplot della sconfitta e della vittoria sono più corti rispetto ai baffi superiori.\\

Nella Figura \ref{fig:int} viene riportato il boxplot della distribuzione della variabile \texttt{Int} rispetto ai valori della variabile risposta \texttt{Res}. Sorprendentemente valori più alti sono registrati nel boxplot della sconfitta, anche se la mediana risulta essere più vicina al 1$^{\circ}$ quantile sottolineando che c'è un maggior numero di valori bassi piuttosto che alti. Le mediane dei restanti boxplot invece, sono ben equilibrate ma il boxplot del pareggio risulta avere meno varianza. Sembra perciò che effettuare troppi intercettazioni dei passaggi avversari contrariamente da quanto si pensi sia controproducente per la vittoria. Si segnala inoltre la presenza di alcuni outliers con valori alti di intercettazioni, che si discostano dalle distribuzioni.\\

\begin{figure}[htbp]
	\begin{center}
		\includegraphics[scale=0.50]{int.png}
		\caption{Boxplot della distribuzione della variabile \texttt{Int} rispetto ai valori della variabile risposta \texttt{Res}} \label{fig:int}
	\end{center}
\end{figure}

Nella Figura \ref{fig:tkl} viene riportato il boxplot della distribuzione della variabile \texttt{TklWin} rispetto ai valori della variabile risposta \texttt{Res}. Come si può notare, vincere più contrasti possibili evita di subire una sconfitta. Infatti ci sono valori più alti nei boxplot del pareggio e della vittoria rispetto al boxplot della sconfitta. Nello specifico però si nota che: nella distribuzione ci sono maggior valori alti nella vittoria rispetto al pareggio, graficamente lo si vede dalla mediana che nel boxplot del pareggio è più vicina al 1$^{\circ}$ quindi ha valori più bassi e lo si nota anche dal baffo inferiore che è meno lungo rispetto a quello superiore viceversa, la mediana del boxplot della vittoria risulta più vicina al 3$^{\circ}$ oltre ad avere il baffo superiore più corto rispetto a quello inferiore. C'è inoltre qualche outliers con valori più alti di contrasti vinti ma sembrano non influenzare la classificazione.\\

\begin{figure}[htbp]
	\begin{center}
		\includegraphics[scale=0.50]{tklwin.png}
		\caption{Boxplot della distribuzione della variabile \texttt{TklWin} rispetto ai valori della variabile risposta \texttt{Res}} \label{fig:tkl}
	\end{center}
\end{figure}

\begin{figure}[htbp]
	\begin{center}
		\includegraphics[scale=0.50]{recov.png}
		\caption{Boxplot della distribuzione della variabile \texttt{Recov} rispetto ai valori della variabile risposta \texttt{Res}} \label{fig:recov}
	\end{center}
\end{figure} 

Infine nella Figura \ref{fig:recov} viene riportato il Boxplot della distribuzione della variabile \texttt{Recov} rispetto ai valori della variabile risposta \texttt{Res}. Per entrambi i boxplot la distribuzione sembra più sbilanciata verso valori bassi quindi ad una loro maggior presenza, infatti entrambe i baffi inferiori sono più corti rispetto a quelli superiori. Per quanto riguarda la mediana sembra equidistante dai quantili per entrambi i tre boxplot. Si nota che il boxplot del pareggio presenta minor varianza rispetto agli altri due boxplot ma valori più alti soprattutto nei confronti del boxplot della vittoria. Sembra perciò che un eccessivo numero di recuperi non porti alla vittoria. Si nota inoltre che ci sono numerosi outliers.

\subsection{Analisi possibili interazioni} 
Per concludere l'attività di preprossening, non resta che analizzare le relazioni tra covariate per individuare possibili interazioni tra di loro che possono influenzare la variabile risposta. Chiaramente dato che ci sono più di 30 variabili e dunque, un grandissimo numero di combinazioni, non si sono esaminate tutte le relazioni ma sono state selezionate solo alcune per l'analisi, basandosi su teorie calcistiche esaminate durante la fase di studio del problema.\\
Di seguito si riporteranno le interazioni che sono state individuate come significative. Si sottolinea che nei grafici di dispersione si è inserito come terza variabile, la variabile risposta \texttt{Res}, infatti ogni punto è colorato in tre possibili colori che rappresenta una delle tre categorie di \texttt{Res}. Tale scelta è stata fatta per capire se i tre gruppi di classi sono ben separati e quindi capire se un'interazione può spiegare l'andamento dei punti della variabile risposta.\\


\bigskip
Sono state individuate le seguenti tre interazioni con la variabile \texttt{Sh}:
\begin{itemize}
	\item Interazione tra \texttt{Sh} e \texttt{SoT}. Chiaramente si può dedurre facilmente che possa esserci una buona correlazione tra queste due variabili perché teoricamente più tiri vengono effettuati maggiori saranno i tiri in porta. \\
	La relazione viene anche mostrata graficamente, infatti nella Figura \ref{fig:ShSoT} si può notare che c'è un'andamento positivo tra le due variabili, al cresce di una c'è un aumento quasi lineare dell'altra.
	Come si può notare dai colori inseriti nel grafico per indicare le tre classi della variabile risposta, l'effetto combinato delle due variabili è utile a spiegare la variabile risposta dato che, valori più bassi sono quasi sempre classificati come sconfitta, un po' più alti come pareggio, mentre quelli più alti sono quasi sempre classificati come vittoria.
	Molte volte i valori vengo ripetuti per molte osservazioni, quindi i valori nel grafico sono disposti in colonne e non sparsi.
	\begin{figure}[htbp]
		\begin{center}
			\includegraphics[scale=0.50]{sh-sot.png}
			\caption{Scatter plot tra \texttt{Sh} e \texttt{SoT}} \label{fig:ShSoT}
		\end{center}
	\end{figure} 
	\item Interazione tra \texttt{Sh} e \texttt{ToAtt3rd}. È ragionevole ipotizzare che il numero di tocchi fatti nella trequarti avversaria possano creare azioni che portano ad effettuare un tiro verso la porta avversaria; è quindi possibile che tra le due variabili possa esserci una relazione. L'ipotesi è avvalorata dalla Figura \ref{fig:shtreq} dove è presente una tendenza positiva quasi lineare tra le due variabili oltre a tre distribuzioni differenti dei dati in base alla loro classificazione.
	\begin{figure}[htbp]
		\begin{center}
			\includegraphics[scale=0.50]{sh-toatt3rd.png}
			\caption{Scatter plot tra \texttt{Sh} e \texttt{ToAtt3rd}} \label{fig:shtreq}
		\end{center}
	\end{figure}
	\item Interazione tra \texttt{Sh} e \texttt{ToAttPen}. Per la stessa ipotesi esposta nel punto precedente si è ipotizzato a tale interazione. La Figura \ref{fig:shpen} mostra che l'interazione è giustificata da una tendenza positiva nel cresce delle due variabili oltre a tre distribuzioni differenti dei dati in base alla loro classificazione. Si nota graficamente una maggior linearità rispetto alla Figura \ref{fig:shtreq}; ciò è coerente con il fatto che i tocchi vengono effettuati all'interno dell'area di rigore avversaria e quindi ad una distanza ravvicinata dalla porta, ne consegue una maggior possibilità di effettuare tiri in porta.
	\begin{figure}[htbp]
		\begin{center}
			\includegraphics[scale=0.50]{sh-toattpen.png}
			\caption{Scatter plot tra \texttt{Sh} e \texttt{ToAttPen}}  \label{fig:shpen}
		\end{center}
	\end{figure}
\end{itemize}

Sono state individuate le seguenti tre interazioni con la variabile \texttt{Poss}:
\begin{itemize}
	\item Interazione tra \texttt{Poss} e \texttt{PAtt}. È ragionevole ipotizzare che il possesso della palla possa incidere su quanto una squadra tenti di effettuare passaggi, cioè da un alto possesso della palla ci si aspetta un alto numero di passaggi tentati, viceversa con un valore basso di possesso. L'ipotesi è confermata dalla Figura \ref{fig:posspatt} che mostra una relazione positiva è fortemente lineare tra le due ipotesi, oltre ad essere utili per spiegare l'andamento delle tre classi della variabile risposta.
	\begin{figure}[htbp]
		\begin{center}
			\includegraphics[scale=0.50]{poss-patt.png}
			\caption{Scatter plot tra \texttt{Poss} e \texttt{PAtt}}  \label{fig:posspatt}
		\end{center}
	\end{figure}
	\item Interazione tra \texttt{Poss} e \texttt{TotDist}. Appare naturale ipotizzare che il possesso della palla e la distanza percorsa con il pallone siano in relazione tra loro. È altrettanto naturale aspettarci da un alto possesso della palla un alto numero di metri percorsi con la palla in possesso, viceversa con un valore basso di possesso. L'ipotesi è confermata dalla Figura \ref{fig:posstotdist} che mostra una relazione positiva abbastanza lineare tra le due ipotesi. Si segnala però che dal grafico sembra che non ci sia una chiara divisione delle osservazioni in tre gruppi, tale aspetto sarà tenuto in considerazione nella modellazione.
	\begin{figure}[htbp]
		\begin{center}
			\includegraphics[scale=0.50]{poss-totdist.png}
			\caption{Scatter plot tra \texttt{Poss} e \texttt{TotDist}}  \label{fig:posstotdist}
		\end{center}
	\end{figure}
\end{itemize}
\pagebreak
Sono state individuate le seguenti tre interazioni con la variabile \texttt{TotDist}:
\begin{itemize}
	\item Interazione tra \texttt{TotDist} e \texttt{PAtt}. Dato che per poter tentare di effettuare passaggi è possibile farlo solo se ci si muove con la palla, allora è possibile ipotizzare che ci sia una relazione tra queste variabili. Dalla Figura \ref{fig:totdistpatt} si può notare che tra le due variabili c'è una forte relazione lineare e con una correlazione positiva.
	\begin{figure}[htbp]
		\begin{center}
			\includegraphics[scale=0.50]{TotDist-PAtt.png}
			\caption{Scatter plot tra \texttt{TotDist} e \texttt{PAtt}}  \label{fig:totdistpatt}
		\end{center}
	\end{figure}
	\item Interazione tra \texttt{TotDist} e \texttt{PCmp\%}. Dato che per poter tentare di effettuare passaggi e completarli è possibile farlo solo se ci si muove con la palla, allora è possibile ipotizzare che ci sia una relazione tra queste variabili. Dalla Figura \ref{fig:totdistpatt} si può notare che tra le due variabili c'è una relazione con correlazione positiva, con una andamento simile a una funzione esponenziale, ciò sarà tenuto conto nella modellazione per valutare se inserire oppure no una delle variabili con un grado superiore.
	\begin{figure}[htbp]
		\begin{center}
			\includegraphics[scale=0.50]{TotDist-PCmp.png}
			\caption{Scatter plot tra \texttt{TotDist} e \texttt{PCmp\%}}  \label{fig:totdistpcmp}
		\end{center}
	\end{figure}
\end{itemize}

Infine sono state individuate le seguenti interazioni:
\begin{itemize}
	\item Interazione tra \texttt{ToAtt3rd} e \texttt{ToAttPen}. Dato che le due variabili si riferiscono a due zone di campo adiacenti e interessanti a fini del l'esito della partita, si ipotizza che ci sia un'interazione. Nella Figura \ref{fig:toatt} si può notare un correlazione positiva molto lineare tra le due variabile che prova l'ipotesi. Si nota all'inizio che tutti i dati sono molto vicini ma che via via diventano più sparsi. Tale interazione sembra perciò utile a spiegare la variabile risposta.
	\begin{figure}[htbp]
		\begin{center}
			\includegraphics[scale=0.50]{ToAtt3rd-ToAttPen.png}
			\caption{Scatter plot tra \texttt{ToAtt3rd} e \texttt{ToAttPen}}  \label{fig:toatt}
		\end{center}
	\end{figure}
	\item Interazione tra \texttt{PAtt} e \texttt{PCmp\%}. Data la loro naturale correlazione si ipotizza che ci sia un'interazione tra loro. Infatti tale interazione è possibile vederla nella Figura \ref{fig:pp} la quale sembra simile all'interazione \texttt{TotDist}*\texttt{PCmp\%}.
	\begin{figure}[htbp]
		\begin{center}
			\includegraphics[scale=0.50]{PAtt-PCmp.png}
			\caption{Scatter plot tra \texttt{PAtt} e \texttt{PCmp\%}}  \label{fig:pp}
		\end{center}
	\end{figure}

\end{itemize}

\subsection{Collinearità}
Per collinearità si intende quel fenomeno per il quale se più variabili esplicative altamente correlate vengono inserite nel modello, allora la loro alta correlazione andrà a nasconde la loro associazione con la variabile risposta. La soluzione per risolvere questo problema è quella di scegliere soltanto una sola variabile della relazione da inserire nel modello.\\
Nella Figura \ref{fig:cor} viene mostrato il valore della correlazione per ogni possibile interazione tra variabile numeriche.\\
Come si può notare una maggior correlazione tra le variabile è concentra nella prima parte del triangolo. Dal grafico possiamo vedere come tutte le interazioni che sono state descritte nella sottosezione precedente abbiano un alta correlazione ma non eccessivamente alta. Si hanno i seguenti valori di correlazione:
\begin{itemize}
	\item Le interazioni con la variabile \texttt{Sh}:
	\begin{itemize}
		\item L'interazione tra \texttt{Sh} e \texttt{SoT} ha come valore 0,70 tale da giustificare l'inserimento dell'interazione.
		\item L'interazione tra \texttt{Sh} e \texttt{ToAtt3rd} ha come valore 0,72 tale da giustificare l'inserimento dell'interazione.
		\item L'interazione tra \texttt{Sh} e \texttt{ToAttPen} ha come valore 0,82 è un valore alto con un possibile rischio di collinearità. Tale valore però giustifica l'inserimento dell'interazione.
	\end{itemize}
	\item Le interazioni con la variabile \texttt{Poss}:
	\begin{itemize}
		\item L'interazione tra \texttt{Poss} e \texttt{PAtt} ha come valore 0,88 è un valore molto alto con un possibile rischio di collinearità. Tale valore però giustifica l'inserimento dell'interazione.
		\item L'interazione tra \texttt{Poss} e \texttt{PAtt} ha come valore 0,81 è un valore alto con un possibile rischio di collinearità. Tale valore però giustifica l'inserimento dell'interazione.
	\end{itemize}
	\item Le interazioni con la variabile \texttt{TotDist}:
	\begin{itemize}
		\item L'interazione tra \texttt{TotDist} e \texttt{PAtt} ha come valore 0,87 è un valore molto alto con un possibile rischio di collinearità. Tale valore però giustifica l'inserimento dell'interazione.
		\item L'interazione tra \texttt{TotDist} e \texttt{PCmp\%} ha come valore 0,75 tale da giustificare l'inserimento dell'interazione.
	\end{itemize}

	\item L'interazione tra \texttt{ToAtt3rd} e \texttt{PAttPen} ha come valore 0,79 tale da giustificare l'inserimento dell'interazione.
	\item L'interazione tra \texttt{PAtt} e \texttt{PCmp\%} ha come valore 0,74 tale da giustificare l'inserimento dell'interazione.
\end{itemize}  


Nella sottosezione precedente si poteva pensare di inserire interazioni abbastanza naturali ad esempio: \texttt{PAtt}*\texttt{SPAtt}, \texttt{PAtt}*\texttt{MPAtt} e, \texttt{PCmp\%}*\texttt{MPCmp\%} e \texttt{PAtt}*\texttt{LPCmp\%}. Tali interazioni però sono composte da variabili che hanno un alta correlazione tra loro, e quindi si ha il rischio di incombere in un problema di collinearità. Una correlazione cosi alta era prevedibile dato che c'è una ridondanza dei dati tra le variabili. In questa fase dell'analisi non si hanno abbastanza elementi per poter scegliere quale variabile tenere e quale no perciò tale scelta verrà rinviata alla fase di modellazione.\\
Il grafico suggerisce alcune interazioni che non sono state descritti ad esempio:\\ \texttt{Poss}*\texttt{ToAtt3rd}, \texttt{Poss}*\texttt{SPAtt},  \texttt{TotDist}*\texttt{ToAtt3rd}, \texttt{PCmp\%}*\texttt{SPAtt} e \texttt{PCmp\%}*\texttt{MPAtt}.
Tali interazioni saranno analizzate durante la fase di modellazione per verificare se effettivamente sono significative per il modello.\\
Infine si nota una buona correlazione tra \texttt{ToDefPen} e \texttt{ToDef3rd}, l'interazione può essere inserita perché va a giustificare il fatto che la variabile \texttt{ToDefPen} combinata con \texttt{ToDef3rd} diventa significativa per la variabile risposta.

\begin{figure}[htbp]
	\begin{center}
		\includegraphics[scale=0.47]{Rplot.png}
		\caption{Grafico delle correlazioni di ogni coppia di variabili}  \label{fig:cor}
	\end{center}
\end{figure}

%\begin{figure}[htbp]
%	\begin{center}
%		\includegraphics[scale=0.70]{cov.png}
%		\caption{Grafico riassuntivo delle variabili rimaste dopo il Prepossesing}  \label{fig:cov}
%	\end{center}
%\end{figure}

\section{Adattamento dataset al modello}

Nelle sezioni precedenti si è descritto come si è costruito il dataset e come esso è stato strutturato. Tale struttura ha il vantaggio di rendere il dataset di facile interpretazione per un essere umano ma ci sono alcune criticità che non lo permettono di essere utilizzato correttamente all'interno del modello messo a disposizione dal pacchetto \texttt{BradleyTerry2}.\\ 
Sono state apportare le seguenti modifiche.\\

Innanzitutto il modello richiede per il suo funzionamento che le due variabili \texttt{Team} e \texttt{Vs} devono essere o di tipo fattore oppure un \textsf{data.frame}. Un \textsf{data.frame} è una lista di vettori, che devono avere tutti la stessa lunghezza, ma possono essere di tipo diverso: variabili nominali cioè fattori, variabili cardinali cioè vettori numerici; un \textsf{data.frame} può essere visto come una matrice ma con il tipo dei valori che può essere diverso.\\ 
Le variabili \textsf{Team} e \textsf{Vs} sono state trasformate in \texttt{data.frame} in modo da poter inserire al loro interno tutte le covariate descritte nella sezione precedente, ad esempio \textsf{Poss}, \textsf{Int} ecc.., cosi che il modello capisca quali valori sono legati alla squadra indicata in \textsf{Team} e quali in \textsf{Vs} nella stessa partita.\\

Inoltre per indicare nel modello se la squadra giocava in casa o no, i valori della variabile \texttt{AtHome} non erano accettati, si è quindi convertito il valore \texttt{TRUE} in 1 mentre FALSE in 0.

\subsection{Implementazione dell'adattamento del dataset}
Nella Sezione \ref{sec:a1} viene mostrato il codice applicato per adeguare il dataset con le modifiche scritte precedentemente.\\
Tale codice ha l'obbiettivo di prendere le due righe di ogni partita e di unirle insieme formando un unica riga per ogni partita. Successivamente si elimineranno le righe delle partite giocate fuori casa (\textsf{AtHome} = FALSE) dalle squadre indicate in \textsf{Team} mentre le righe delle partite giocate in casa (\textsf{AtHome} = TRUE) dalle squadre indicate in \textsf{Team} conteranno il risultato della fusione.\\
Perciò si è creato un vettore vuoto per ogni covariata presente nel dataset, ad eccezione di \textsf{AtHome} che verrà gestita in un modo diverso. Il vettore \texttt{del} è il vettore che tiene traccia di quali righe saranno da eliminare. \texttt{k} è l'indice usato per scorre il dataset per trovare i dati dell'avversario; \texttt{z} l'indice usato per inserire un nuovo elemento nel vettore \texttt{del}.\\
Il primo ciclo \texttt{for} scorre tutto il dataset alla ricerca delle righe con i dati delle partite giocate in casa dalla squadra indicata in \texttt{Team}, infatti al suo interno il primo costrutto \texttt{if} controlla se la partita è in casa per \texttt{Team} se sì, parte un secondo ciclo \texttt{for} che anche esso scorre tutto il dataset per cercare la riga con la partita giocata della squadra indicata in \texttt{Vs}; giocata ovviamente fuori casa. Perciò all'interno del secondo ciclo \texttt{for} c'è un costrutto \texttt{if} che controlla se la j-esima riga si riferisce alla stessa partita indicata nella i-esima riga, se sì allora si salvano tutti i dati nei vettori e si incrementa l'indice \texttt{k}. Se il primo \texttt{if} da esito negativo allora si andrà a inserire l'indice dell'i-esima riga nel vettore \texttt{del} perché contiene informazioni di una partita giocata fuori casa dalla squadra indicata in \textsf{Team} e viene incrementato l'indice di uno \texttt{z}.\\

Di seguito vengono riportati i comandi fatti per applicare le modifiche al dataset.
\bigskip
\begin{lstlisting}[language=R]
> soccern3 <- soccern2[-del,]
\end{lstlisting}
\bigskip
Con il precedente comando si va a creare un nuovo dataset con 380 righe, eliminando tutte quelle righe con valore \texttt{FALSE} su \textsf{AtHome}. \\

Con il comando mostrato nella Sezione \ref{sec:a2} si va a modificare \textsf{Team} rendendolo un \texttt{data.frame}, andando a inserire i dati della riga relativi alla squadra che gioca in casa. Si inserisce come chiave \texttt{team = soccern3\$Team} e si indica che la partita è in casa per la squadra di riferimento con \texttt{at.home = 1}.\\

Con il comando mostrato nella Sezione \ref{sec:a3} si va a modificare \textsf{Vs} rendendolo un \texttt{data.frame}, andando a inserire i dati della riga relativi alla squadra che gioca fuori casa. Si inserisce come chiave \texttt{team = soccern3\$Vs} e si indica che la partita è fuori casa per la squadra \texttt{Vs} con \texttt{at.home = 0}.\\ Per quanto riguarda il resto dei dati, vengono riportati attraverso l'inserimento dei vettori costruiti e riempiti precedentemente.\\
