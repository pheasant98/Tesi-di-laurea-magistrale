% !TEX encoding = UTF-8
% !TEX TS-program = pdflatex
% !TEX root = ../tesi.tex

%**************************************************************
\chapter{Risultati dei modelli Bradley-Terry}
\label{cap:risultatiDM}
%**************************************************************

\intro{In questo capitolo vengono presente le stime e i risultati ottenuti dai modelli Bradley-Terry presentati nel Capitolo 4. Inoltre, sarà illustrata un'analisi sui risultati ottenuti e verrà riportata l'applicazione del metodo\emph{LASSO} con relativi risultati.}\\

\section{Risultati del modello Bradley-Terry con effetto dell'ordine}
Le analisi dell'indagine iniziano con l'applicazione del modello \hyperref[for:3.9]{(4.9)}. Tale modello è abbastanza semplice, infatti la stima dell'abilità delle squadre tiene conto solo degli esiti osservarti delle varie partite e del vantaggio di giocare in casa. Ovviamente da tali stime si basa la distribuzione di probabilità degli esiti delle partite.\\
La stima del parametro soglia $\theta_k$ è di -0.669 mentre il parametro $\delta$ globale per tutte le squadre è di 0.099 con uno \emph{standard error} di 0.126. Si nota che il vantaggio di giocare in casa effettivamente è un vantaggio anche secondo il modello, infatti la stima del parametro è positiva, quindi generalmente ha un effetto positivo per la squadra in casa. Nella Tabella \ref{tab:BTH} vengono riportati i risultati ottenuti.
	\begin{table}[!htb]%
	
	\renewcommand{\arraystretch}{1.7}
	\centering
	\begin{tabular}{c c c c c c}
		\hline	
		
		\textbf{Squadra} & \textbf{$eta_i$} & \textbf{SE} & \textbf{QSE} & \textbf{QV} & \textbf{Rank}   \\	
		\hline			
		Milan & 1.492 & 0.557 & 0.359 & 0.129 & 1\\
		Inter & 1.4 & 0.537 & 0.400 & 0.160 & 2\\
		Napoli & 1.17 & 0.530 & 0.389 & 0.152 & 3 \\
		Juventus & 0.825 & 0.520& 0.373& 0.139& 4\\
		Lazio & 0.459 & 0.516 & 0.368 & 0.135 & 5\\
		Roma & 0.413 & 0.516& 0.368& 0.135& 6\\
		Fiorentina & 0.339 & 0.511& 0.357& 0.127& 7\\
		Atalanta & 0.312 & 0.000 & 0.368& 0.135& 8 \\
		Hellas Verona & 0.048 & 0.513& 0.356& 0.127& 9\\
		Torino & -0.012 & 0.512 & 0.355& 0.126& 10 \\
		*Udinese & -0.072 & 0.512& 0.355 & 0.126& 12\\
		*Sassuolo & -0.145 & 0.511& 0.355 & 0.126& 11\\
		Bologna & -0.233 & 0.515& 0.359& 0.128& 13\\
		Empoli & -0.549 & 0.518& 0.362& 0.131& 14\\
		Sampdoria & -0.775 & 0.527& 0.372& 0.138& 15\\
		Spezia & -0.831 & 0.527& 0.372& 0.138& 16\\
		*Genoa & -0.879 & 0.532& 0.378& 0.143& 19 \\
		Cagliari & -0.897 & 0.532& 0.378& 0.143& 18\\
		*Salernitana & -0.91 & 0.527& 0.372& 0.138& 17\\
		Venezia & -1.156 & 0.538& 0.387& 0.149& 20\\
		\hline
		& & & \\
		
	\end{tabular} \hbox{}
	
	\caption{Per ogni squadra viene riportata l'abilità stimata $\eta_i$, lo \emph{Standard 
		Error} (SE), il \emph{Quasi Standard Error} (QSE) e il \emph{Quasi Variance} (QV).} \label{tab:BTH}
\end{table}
Nonostante, la semplicità del modello, viene offerta una stima delle abilità delle squadre che rispecchia molto il piazzamento mostrato nella Tabella \ref{tab:ranking}. Infatti, solo quattro squadre hanno un piazzamento diverso da quello reale. L'Udinese e il Sassuolo hanno il piazzamento invertito e la loro stima dell'abilità è molto simile. Ciò è un bene dato che nella stagione in esame il loro distacco è stato solo di tre punti. Anche Genoa e Salernitana hanno un piazzamento differente da quello reale. Per quanto riguarda il Genoa tale risultato può essere spiegato dal fatto che all'inizio del campionato ha avuto un buon andamento (vedi \textit{\cite{site:storyGenoa}}) e dall'ottenimento di punti contro Juventus, Inter, Roma e Atalanta, cioè squadre considerate tra le più forti del campionato. Per quanto la stima al ribasso della Salernitana è determinata dal suo pessimo andamento per la maggior parte del campionato fatta eccezione l'ultima parte, dove sono stati guadagnati la maggior parte dei punti, tanto da permettere alla squadra di guadagnare all'ultima giornata la salvezza (vedi \textit{\cite{site:storySal}}). \\
Come si può notare oltre allo \emph{Standard Error} (SE) sono state riportate altre due misurazioni, il \emph{Quasi Standard Error} (QSE) \autocite{firth2004quasi} e il \emph{Quasi Variance} (QV)\autocite{firth2004quasi}. Il \emph{Quasi Variance} (QV)\autocite{firth2004quasi} è un metodo che fornisce un'approssimazione della varianza, ed è utilizzato per confrontare livelli differenti di un fattore. Il tipo fattore è stato illustrato nel Capitolo \ref{cap:Analisi}. Il QV è stato introdotta da \autocite{firth2004quasi} per risolvere il problema della categoria di riferimento. Tale problema si riferisce al fatto che risulta essere semplice confrontare un livello qualsiasi del fattore con il suo livello di riferimento ma confrontare tra loro due livelli entrambi non di riferimento non è possibile. Grazie a il QV cioè è possibile, infatti permette di confrontare tra di loro diversi livelli che non sono di riferimento con il vantaggio di non dover riportare tutta la matrice delle varianze e delle covarianze per effettuare i confronti. Nel nostro caso abbiamo la variabile \texttt{team} di tipo fattore con la squadra Atalanta come livello di riferimento. Grazie al QV ci viene fornita il QSE, una stima dello SE che verrà utilizzata per confrontare le abilità stimate dei diversi livelli, ovvero le squadre, per poter dedurre se la differenza di abilità tra due squadre è significativa dal punto di vista statistico. Con il QSE le squadre vengono trattate come variabili indipendenti. Esempio di applicazioni del QSE e del QV su modelli Bradley-Terry è possibile trovarli in \textcite{firth2004quasi} e in \textcite{turner2012bradley}
%**************************************************************
