% !TEX encoding = UTF-8
% !TEX TS-program = pdflatex
% !TEX root = ../tesi.tex

%**************************************************************
\chapter{Risultati dei modelli Bradley-Terry}
\label{cap:risultatiDM}
%**************************************************************

\intro{In questo capitolo vengono presentate le stime e i risultati ottenuti dai modelli Bradley-Terry (BTM) descritti nel Capitolo \ref{cap:BT}. Inoltre, sarà riportata l'applicazione del metodo \emph{LASSO} con relativi risultati. Infine si riporteranno le predizioni sugli esiti delle partite prodotte dai modelli per essere poi confrontate con le predizioni dei \emph{bookmakers}.}\\
%Seguirà poi un’analisi conclusiva sulle variabili esplicative alla luce dei risultati ottenuti.
\section {Premesse}
I risultati che verranno esposti non tengono in considerazione le variabili esplicative del numero di gol fatti \texttt {GF} e dei gol subiti \texttt{GA}. Questo perché provocano la non convergenza del modello. Infatti, le librerie usate non sono in grado di interpretare correttamente i dati. Una possibile soluzione è allargare il numero di categoria \emph{K} inserendo per ogni possibile risultato finale una categoria.  Data l'elevata complessità che raggiunge il modello esteso Bradley-Terry, non sono state inserite le interazioni illustrate nel Capitolo \ref{cap:Analisi}.

\section {BTM con effetto dell'ordine}
Le analisi dello studio iniziano con l'applicazione del modello (\hyperref[for:3.9]{4.9}). Tale modello presenta una struttura abbastanza semplice, in cui la stima dell'abilità delle squadre tiene conto solo degli esiti osservati delle varie partite e del vantaggio di giocare in casa. 
La stima dei parametri soglia $\theta_1$ e $\theta_2$ è pari rispettivamente, a -0.669 e 0.669 mentre il parametro $\delta$ globale per tutte le squadre è di 0.099 con uno \emph{standard error} (SE) di 0.126. Si nota che la possibilità di giocare in casa è effettivamente un vantaggio in quanto la stima del parametro è positiva. Nella Figura \ref{tab:BTH} vengono riportati i risultati ottenuti in ordine dell'abilità stimata. Inoltre viene riportato lo \emph{Standard Error} (SE) il \emph{Quasi Standard Error} (QSE) \autocite{firth2004quasi} e il \emph{Quasi Variance} (QV)\autocite{firth2004quasi} per ogni squadra.\\
Nonostante, la semplicità del modello, viene offerta una stima delle abilità delle squadre che rispecchia molto il piazzamento mostrato nella Figura \ref{fig:ranking}. Infatti, solo quattro squadre hanno un piazzamento diverso da quello reale. L'Udinese e il Sassuolo hanno il piazzamento invertito con una stima dell'abilità che è molto simile. Ciononostante, il risultato è comunque soddisfacente dato che nella stagione in esame il distacco tra Udinese e Sassuolo è stato solo di tre punti. Anche Genoa e Salernitana hanno un piazzamento differente da quello reale.

\begin{sidewaysfigure} 
	\begin{center}
		\includegraphics[height = 13cm, width = 23cm]{rank.png}
		\caption{Barplot che indica per ogni squadra l'abilità stimata dal modello (\hyperref[for:3.9]{4.9}). A fianco al grafico viene riportato lo \emph{Standard Error} (SE), il \emph{Quasi Standard Error} (QSE) e il \emph{Quasi Variance} (QV). Nel grafico viene indicato con un asterisco le squadre con un piazzamento stimato diverso da quello reale, anche esso riportato a destra del grafico.} \label{tab:BTH}
	\end{center}
\end{sidewaysfigure} 


\begin{comment}
	\newpage
	\paperwidth=\pdfpageheight
	\paperheight=\pdfpagewidth
	\pdfpageheight=\paperheight
	\pdfpagewidth=\paperwidth
	\headwidth=\textheight
	
	\begingroup 
	\vsize=\textwidth
	\hsize=\textheight
	
	
	\pagestyle{empty}
	\textwidth=\hsize
	\textheight=\vsize
	
	\endgroup
	\newpage
	\paperwidth=\pdfpageheight
	\paperheight=\pdfpagewidth
	\pdfpageheight=\paperheight
	\pdfpagewidth=\paperwidth
	\headwidth=\textwidth
	\begin{table}[!htb]%
		
		\renewcommand{\arraystretch}{1.7}
		\centering
		\begin{tabular}{c c c c c c}
			\hline	
			
			\textbf{Squadra} & \textbf{Abilità} & \textbf{SE} & \textbf{QSE} & \textbf{QV} & \textbf{Rank}   \\	
			\hline			
			Milan & 1.492 & 0.557 & 0.359 & 0.129 & 1\\
			Inter & 1.4 & 0.537 & 0.400 & 0.160 & 2\\
			Napoli & 1.17 & 0.530 & 0.389 & 0.152 & 3 \\
			Juventus & 0.825 & 0.520& 0.373& 0.139& 4\\
			Lazio & 0.459 & 0.516 & 0.368 & 0.135 & 5\\
			Roma & 0.413 & 0.516& 0.368& 0.135& 6\\
			Fiorentina & 0.339 & 0.511& 0.357& 0.127& 7\\
			Atalanta & 0.312 & 0.000 & 0.368& 0.135& 8 \\
			Hellas Verona & 0.049 & 0.513& 0.356& 0.127& 9\\
			Torino & -0.012 & 0.512 & 0.355& 0.126& 10 \\
			*Udinese & -0.072 & 0.512& 0.355 & 0.126& 12\\
			*Sassuolo & -0.145 & 0.511& 0.355 & 0.126& 11\\
			Bologna & -0.233 & 0.515& 0.359& 0.128& 13\\
			Empoli & -0.549 & 0.518& 0.362& 0.131& 14\\
			Sampdoria & -0.775 & 0.527& 0.372& 0.138& 15\\
			Spezia & -0.831 & 0.527& 0.372& 0.138& 16\\
			*Genoa & -0.879 & 0.532& 0.378& 0.143& 19 \\
			Cagliari & -0.897 & 0.532& 0.378& 0.143& 18\\
			*Salernitana & -0.91 & 0.527& 0.372& 0.138& 17\\
			Venezia & -1.156 & 0.538& 0.387& 0.149& 20\\
			\hline
			& & & & &\\
			
		\end{tabular} \hbox{}
		
		\caption{Per ogni squadra viene riportata l'abilità stimata, lo \emph{Standard 
				Error} (SE), il \emph{Quasi Standard Error} (QSE) e il \emph{Quasi Variance} (QV). Nella tabella viene indicato con un asterisco le squadre con un piazzamento stimato diverso da quello reale.} \label{tab:BTH}
	\end{table}
\end{comment}

Per quanto riguarda il Genoa tale risultato può essere spiegato dal fatto che all'inizio del campionato ha avuto un buon andamento (vedi \textit{\cite{storyGenoa}}) e dall'ottenimento di punti contro Juventus, Inter, Roma e Atalanta, cioè squadre considerate tra le più forti del campionato. Per quanto riguarda la stima al ribasso della Salernitana è determinata dal suo pessimo andamento per la maggior parte del campionato fatta eccezione per l'ultima parte, dove sono stati guadagnati la maggior parte dei punti, tanto da permettere alla squadra di guadagnare all'ultima giornata la salvezza (vedi \textit{\cite{storySal}}). \\
Il \emph{Quasi Variance} (QV)\autocite{firth2004quasi} è un metodo che fornisce un'approssimazione della varianza, ed è utilizzato per confrontare livelli differenti di un fattore. Il tipo fattore è stato illustrato nel Capitolo \ref{cap:Analisi}. Il QV è stato introdotto da \textcite{firth2004quasi} per risolvere il problema della categoria di riferimento. Tale problema si riferisce al fatto che risulta essere semplice confrontare un livello qualsiasi del fattore con il suo livello di riferimento ma confrontare tra loro due livelli entrambi non di riferimento non è possibile. Grazie al QV cioè è possibile, infatti permette di confrontare tra di loro diversi livelli che non sono di riferimento con il vantaggio di non dover riportare tutta la matrice delle varianze e delle covarianze per effettuare i confronti. Nel nostro caso abbiamo la variabile \texttt{team} di tipo fattore con la squadra Atalanta come livello di riferimento. Grazie al QV ci viene fornita il QSE, una stima dello SE che verrà utilizzata per confrontare le abilità stimate dei diversi livelli, ovvero le squadre, per poter dedurre se la differenza di abilità tra due squadre sia significativa dal punto di vista statistico. Con il QSE le squadre vengono trattate come unità indipendenti. Esempio di applicazioni del QSE e del QV sul BTM è possibile trovarli in \textcite{firth2004quasi} e in \textcite{turner2012bradley}.\\
Perciò, confrontiamo le stime dei valori delle abilità delle squadre classificatesi nelle prime due posizioni, rispettivamente il Milan e Inter. Il QSE per Milan è di 0.359, mentre per l'Inter è di 0.400. La differenza assoluta tra le loro abilità è pari a |1.492 - 1.4| = 0.092. Applicando il calcolo pitagorico è possibile calcolare lo QSE, cioè un SE approssimato, relativo alla differenza tra abilità. Questo risulta essere ($0.359^2 + 0.400^2)^\frac{1}{2}=0.537 > 0.092$. Perciò, la differenza in termini di abilità tra le due squadre non è significativa da un punto di vista statistico. Infatti, le due squadre hanno una differenza di soli due punti.
%**************************************************************

\section{BTM con covariate specifiche dell'oggetto}
Si consideri l'estensione del modello Bradley-Terry con covariate specifiche dell'oggetto. Il modello applicato è il seguente
\begin{align}
	P(Y_{p(i,j)}\leq k) =  \frac{exp(\delta + \theta_{k} + \beta_{i0} - \beta_{j0} + x^T_{pi}\tau - x^T_{pj}\tau)}{1 + exp(\delta + \theta_{k} + \beta_{i0} - \beta_{j0} + x^T_{pi}\tau - x^T_{pj}\tau)}, \label{for:5.1}
\end{align}
dove l'effetto dell'ordine $\delta$, cioè il vantaggio di giocare la partita in casa, ha ancora un effetto globale per tutte le squadre, mentre $x^T_{pi}$ è il vettore con tutti i valori delle ventisei covariate per l'i-esima squadra e per la p-esima partita. Il parametro $\tau$ è il peso medio stimato di ogni covariata. Le covariate, perciò, sono specifiche del soggetto e dell'oggetto ma con un effetto specifico dell'oggetto.\\
La stima dei parametri soglia $\theta_1$ e $\theta_2$ è pari, rispettivamente a -1.113 e 1.113, mentre il parametro $\delta$ globale per tutte le squadre è salito a 0.27, con uno SE di 0.142. Nella Figura \ref{tab:BTC} e nella Tabella \ref{tab:BTC2} vengono riportate le stime delle abilità delle squadre con i relativi SE, QSE e QV, e le stime di ogni covariata sul modello con relativo SE.
	\begin{sidewaysfigure} 
	\centering
	{\includegraphics[height = 12cm, width = 23cm]{rank2.png}}\qquad\qquad 
	\caption{Barplot che indica per ogni squadra l'abilità stimata dal modello (\ref{for:5.1}). A fianco al grafico vengono riportati i relativi \emph{Standard 
			Error} (SE), \emph{Quasi Standard Error} (QSE) e \emph{Quasi Variance} (QV). Nel grafico viene indicato con un asterisco le squadre con un piazzamento stimato diverso da quello reale, anche esso riportato a destra del grafico} \label{tab:BTC}  
	
	
\end{sidewaysfigure}

\begin{table}[htbp]
	
	\renewcommand{\arraystretch}{1.7}
	\centering
	\begin{tabular}{c c c }
		\hline	
		
		\textbf{Covariata} & \textbf{Stima} & \textbf{SE} \\	
		\hline
		ToMid3rd & 1.57 & 0.025\\
		G/Sh & 1.135 & 0.317 \\
		Sh & 0.787 & 0.085 \\  
		SoT &  0.536 & 0.324 \\  
		PCmp\% & 0.534 & 0.300 \\
		ToDefPen & 0.375 & 0.027 \\      
		ToDef3rd & 0.347 & 0.026 \\
		ToAtt3rd & 0.283 & 0.025 \\     	     	 
		Saves & 0.280 & 0.312 \\ 
		Fls & 0.138 & 0.204  \\     
		Fld & 0.100 & 0.204  \\
		TklWin &  0.082 & 0.049  \\    
		LPAtt & 0.078 & 0.049  \\ 		
		Poss & 0.032 & 0.169 \\ 
		ToAttPen & 0.027 & 0.044 \\  
		TotDist & -0.039 & 0.001 \\  	
		Off & -0.054 & 0.144  \\
		PAtt & -0.080 & 0.053 \\ 
		Int & -0.082 & 0.057 \\  
		SPCmp\% & -0.100 & 0.136 \\ 
		Crs & -0.199 & 0.062\\  
		LPCmp\% & -0.309 & 0.380 \\ 
		Recov &  -0.512 & 0.030 \\        
		SPAtt & -0.650 & 0.053 \\     
		MPCmp\% & -0.748 & 0.126 \\
		MPAtt & -1.011 & 0.050 \\     		     		   		    
		\hline
		& &  \\
		
	\end{tabular} \hbox{}
	\caption{Stime delle covariate con relativo \emph{Standard 
			Error} (SE), stimate dal modello \ref{for:5.1}.} \label{tab:BTC2} 
	
\end{table}
Dai risultati si nota che alcune variabili esplicative sono fortemente associate all'esito della partita. Come ci si aspetta le variabili esplicative legate ai tiri quindi, tiri \texttt{Sh}, tiri in porta \texttt{SoT} e il rapporto gol/tiri \texttt{G/Sh} hanno un peso stimato molto alto e positivo. Sono perciò fortemente decisive per aumentare la probabilità di vittoria. Da notare che sia \texttt{G/Sh} e sia \texttt{Sh} hanno un alto SE, quindi un' elevata variabilità. Sarà interessante, perciò, analizzare nel prossimo modello, il peso di queste covariate per ogni singola squadra.
Sorprendentemente la variabile esplicativa \texttt{ToMid3rd} ovvero, il numero di tocchi con la palla fatti a centrocampo ha una forte associazione positiva con la probabilità di vittoria. Le altre covariate legate ai tocchi nelle altre zone del campo quindi \texttt{ToDefPen}, \texttt{ToDef3rd}, \texttt{ToAtt3rd} e \texttt{ToAttPen} hanno un'associazione positiva seppur minore rispetto a \texttt{ToMid3rd}. Sembra perciò che avere il controllo del centrocampo sia fondamentale per costruire azioni da gol, ma anche per mantenere un risultato positivo dalla partita. Anzi, mantenere il pallone in zone difensive con meno transizioni in zone d'attacco sembra che dia maggior probabilità di vittoria. Infatti, si può notare che un elevato numero di tocchi in area di rigore avversaria \texttt{ToAttPen} aumenti di molto poco la probabilità di vittoria. A sostegno di ciò, si consideri che il campionato italiano è spesso considerato un campionato difensivista e tattico (vedi \textit{\cite{speculazione}}), dove si spinge l'avversario a sbilanciarsi per poi attaccarlo in contropiede.\\
Un aspetto difensivo chiave sembra essere rappresentato dalle parate fatte \texttt{Saves}. Inoltre, anche il numero di contrasti vinti \texttt{TklWin} è positivo quando associato alla probabilità di vittoria. Sorprendentemente, le altre variabili esplicative difensive rispettivamente, numero di intercetti \texttt{Int} e numero di recuperi \texttt{Recov} hanno un'associazione negativa con la probabilità di vittoria. Al contrario di quanto si pensi il possesso della palla non sembra essere un elemento chiave per la vittoria. Infatti, la sua stima fa aumentare di molto poco la probabilità di vittoria. Analogamente, anche la distanza percorsa con la palla \texttt{TotDist} non sembra essere un elemento chiave per la vittoria, anzi va a diminuire la probabilità di vittoria. Perciò sembra che stia emergendo dall'analisi, una tendenza ad avere il controllo del gioco nei momenti giusti e nelle zone giuste del campo per aver maggior probabilità di vittoria.\\
Per quanto riguarda l'aggressività della squadra, la stima del modello indica che commettere falli \texttt{Fld} aumenti le probabilità di vittoria, d'altra parte subire falli \texttt{Fls} è più conveniente.\\ 
Si nota che subire un fuorigioco \texttt{Off} ha un impatto negativo sulle probabilità di vittoria.\\
Per quanto riguarda le covariate legate ai passaggi notiamo che solo la percentuale dei passaggi completati \texttt{PCmp\%} e il numero di lanci lunghi tentati \texttt{LPAtt} hanno un’associazione positiva con la probabilità di vittoria, le restanti covariate invece presentano un'associazione negativa. Un abuso di passaggi filtrati \texttt{MPAtt} o di cross \texttt{Crs} sembra essere controproducente per la vittoria. Una buona precisione sui passaggi \texttt{PCmp\%} e tentare i cambi di gioco \texttt{LPAtt} invece, suggeriscono la possibilità di maggiori probabilità di vittoria. \\
Come fatto nella sezione precedente è possibile anche qui confrontare tra loro le squadre utilizzando i loro QSE relativi alla loro abilità stimata.
Confrontando ancora le prime due squadre cioè Milan e Inter, la loro differenza assoluta di abilità è pari |1.406 - 1.097| = 0.309 e il relativo QSE è pari a ($0.455^2 + 0.433^2)^\frac{1}{2}=0,628 > 0.309$. Si ottiene per ciò che la differenza di abilità tra le due squadre è ancora non significativa anche con l’effetto delle covariate.
\begin{comment}
\begin{table}[!htb]%
	
	\renewcommand{\arraystretch}{1.7}
	\centering
	\begin{tabular}{c c c c c c}
		\hline	
		
		\textbf{Squadra} & \textbf{Abilità} & \textbf{SE} & \textbf{QSE} & \textbf{QV} & \textbf{Rank}   \\	
		\hline			
		Milan & 1.406 & 0.644 & 0.455 & 0.239 & 1\\
		Inter & 1.097 & 0.685 & 0.433 & 0.286 & 2\\
		Napoli & 1.067 & 0.595 & 0.423 & 0.236 & 3 \\		
		Juventus & 0.892 & 0.623 & 0.417& 0.226& 4\\
		Lazio & 0.399 & 0.645 & 0.467 & 0.276 & 5\\
		Roma & 0.377 & 0.634 & 0.469 & 0.279 & 6\\
		*Atalanta & 0.317 & 0.000 & 0.423& 0.238& 8 \\
		*Fiorentina & 0.236 & 0.596 & 0.383 & 0.235& 7\\
		*Torino & 0.092 & 0.591 & 0.427 & 0.165 & 10 \\
		*Hellas Verona & 0.013 & 0.561 & 0.427& 0.164& 9\\
		Sassuolo & -0.023 & 0.587 & 0.435 & 0.253& 11\\
		*Bologna & -0.045 & 0.657& 0.459& 0.128& 13\\
		*Empoli & -0.094 & 0.618& 0.432& 0.211 & 14\\
		*Udinese & -0.178 & 0.642& 0.478 & 0.281& 12\\
		Sampdoria & -0.426 & 0.600 & 0.453& 0.288& 15\\
		*Salernitana & -0.854 & 0.544& 0.429& 0.219& 17\\
		*Spezia & -0.922 & 0.587& 0.452 & 0.249 & 16\\
		Cagliari & -1.01 & 0.612 & 0.498& 0.269 & 18\\
		Genoa & -1.026 & 0.632 & 0.456 & 0.214& 19 \\
		Venezia & -1.318 & 0.592 & 0.434 & 0.231 & 20\\
		
		\hline
		& & & & & \\
		
	\end{tabular} \hbox{}
	\caption{Stime delle abilità con relativi \emph{Standard 
			Error} (SE), \emph{Quasi Standard Error} (QSE) e \emph{Quasi Variance} (QV). Nella tabella viene indicato con un asterisco le squadre con un piazzamento stimato diverso da quello reale.} \label{tab:BTC}  
\end{table}
\end{comment}



\section{Modello Bradley-Terry e LASSO}
Nella sezione precedente si sono presentati i risultati ottenuti nel modello Bradley-Terry con l'inserimento di covariate con effetto specifico dell'oggetto. È però di interesse per le nostre analisi capire in che modo ogni singola covariata sia determinante nell'esito della partita a seconda della squadra in esame. Per esempio, è possibile che il possesso della palla possa essere determinante per una squadra mentre per un'altra no. A tale scopo si applicherà il modello (\ref{for:4.9}) utilizzando covariate specifiche del soggetto e dell'oggetto. Ovviamente con l'inserimento di questo tipo di covariate il modello sarà estremante complesso, essendo basato su 520 covariate. Di conseguenza sarà applicata una selezione delle covariate operata attraverso il metodo \emph{LASSO} illustrato nel Capitolo \ref{cap:BT}. Sempre attraverso il \emph{LASSO} sarà di interesse individuare clusters di squadre che per una certa covariata hanno un effetto simile. Allo stesso tempo si cercherà di individuare quali squadre invece si discostano maggiormente da questi clusters.\\
Purtroppo, non è stato possibile riportare gli SE delle stime a causa dell'elevata complessità del procedimento di calcolo. Questo è possibile solo attraverso la procedura di tipo \emph{bootstrap} \autocite{henderson2005bootstrap}, molto onerosa in termini computazionali, soprattutto con un numero elevato di covariate.\\
Nella Figura \ref{tab:BTCL} e nelle Tabelle \ref{tab:BTCL2} e \ref{tab:BTCL3} vengono riportate le stime dei parametri delle abilità e delle covariate per ogni singola squadra. Si noti che non tutte le covariate hanno un’unica stima per tutte le squadre, ma in alcuni casi ci sono più stime per alcune covariate. Perciò, per ogni stima del parametro di una covariata verrà indicata quale squadra ha tale valore stimato. Nell'analisi dei risultati spesso si farà un confronto con i risultati ottenuti con il modello della sezione precedente.\\%stacca
Nella Figura \ref{tab:BTCL} si può notare che le abilità stimate tramite \emph{LASSO} sono quasi sempre  in linea con il piazzamento reale, migliorando perciò le prestazione del modello rispetto al modello con effetto specifico dell'oggetto stimato nella sezione precedente.\\
Purtroppo, l'abilità dell'Atalanta viene sovrastimata nonostante al termine della stagione si sia classificata dietro a Roma e Fiorentina. Tale fenomeno può essere spiegato dal fatto che l'Atalanta per larga parte della stagione militava tra il terzo e il quarto posto, ma nell'ultima parte della stagione l'Atalanta è crollata di prestazione (vedi \textit{\cite{storyAta}}). Si nota che l'abilità della Sampdoria viene sottostimata, infatti in generale, non ha avuto un buon rendimento soprattutto verso la fine della stagione (vedi \textit{\cite{storySamp}}).\\
Nella Tabella \ref{tab:BTCL2} e nella Tabella \ref{tab:BTCL3} alcune variabili esplicative sono state portate a zero, quindi eliminate per effetto della regolarizzazione tramite \emph{LASSO}, mentre altre hanno diversi valori a seconda della squadra in considerazione. \\
Tra le covariate eliminate c'è il numero di passaggi tentati \texttt{PAtt} che nella Tabella \ref{tab:BTC2} del modello precedente aveva un valore stimato quasi nullo. Sorprendentemente anche la percentuale di passaggi tentati \texttt{PCmp\%} viene eliminata dal modello nonostante per il modello precedente avesse un valore stimato alto del parametro. Anche il numero di tocchi nella trequarti di difesa \texttt{ToDef3rd} viene tolta dal modello nonostante un valore stimato alto nella Tabella \ref{tab:BTC2}. Infine l'ultima variabile esplicativa eliminata interamente del modello è la distanza percorsa con la palla \texttt{TotDist} rimanendo in linea con quanto visto nella Tabella \ref{tab:BTC2}.
Anche qui viene confermato che giocare le partite in casa \texttt{Home} ha un effetto positivo stimato pari a 0.310. Per quanto riguarda invece il possesso della palla \texttt{Poss}, come visto dal precedente modello viene stimato con un peso nullo per la maggior parte delle squadre ad eccezione di Lazio e Torino, il quale ha una stima positiva.

\begin{sidewaysfigure} 
	\centering
	\begin{center}
		\includegraphics[height=13cm, width=23cm]{rank3.png}
		\caption{Barplot che indica per ogni squadra l'abilità stimata dal modello (\ref{for:4.9}). Viene indicato con un asterisco le squadre con un piazzamento stimato diverso da quello reale anche esso riportato a destra del grafico.} \label{tab:BTCL} 
	\end{center}
\end{sidewaysfigure}

\begin{comment}
	\begin{table}[!htb]%
		
		\renewcommand{\arraystretch}{1.7}
		\centering
		\begin{tabular}{c c c }
			\hline	
			
			\textbf{Squadra} & \textbf{Abilità} & \textbf{Rank}   \\	
			\hline			
			Milan & 1.673 & 1\\
			Inter & 1.443 &  2\\
			Napoli & 1.436 & 3 \\		
			Juventus & 1.003 & 4\\
			Lazio & 0.641 & 5\\
			*Atalanta & 0.594 & 8 \\
			*Roma & 0.555 &  6\\
			*Fiorentina & 0.227 &  7\\
			Hellas Verona & 0.126 & 9 \\
			Torino & -0.042 & 10 \\	
			Sassuolo & -0.171 & 11\\
			Udinese & -0.262 & 12\\
			Bologna & -0.292 &  13\\
			Empoli & -0.386 & 14\\
			*Spezia & -0.869 &  16\\
			*Salernitana & -0.876 & 17\\
			*Sampdoria & -1.095 &  15\\
			Cagliari & -1.136 &  18\\
			Genoa & -1.231 & 19 \\
			Venezia & -1.338 &  20\\
			
			\hline
			& &  \\
			
		\end{tabular} \hbox{}
		\caption{Stime delle abilità per ogni squadra stimate dal modello (\ref{for:4.9}). Nella tabella viene indicato con un asterisco le squadre con un piazzamento stimato diverso da quello reale.} \label{tab:BTCL}  
	\end{table}	
\end{comment}

\begin{table}[!htbp]
	
	\renewcommand{\arraystretch}{1.7}
	\centering
	\begin{tabular}{ccp{10cm}}
		\hline	
		
		\textbf{Covariata} & \textbf{Stima} & \textbf{Squadra} \\	
		\hline
		Home & 0.310 & Tutti\\
		Poss & 0.239 & Lazio \\
		Poss & 0.171 & Torino\\
		Poss & 0.000 & Tutti tranne Lazio e Torino\\
		Sh & 0.520 & Tutti \\
		SoT & 0.596 & Atalanta, Cagliari, Empoli, Genoa, Verona, Juventus, Lazio, Milan, Napoli, Salernitana, Sampdoria, Sassuolo, Spezia, Torino, Venezia\\
		SoT & 0.495 & Inter, Roma \\
		SoT & 0.361 & Bologna \\
		SoT & 0.263 & Fiorentina\\
		SoT & 0.007 & Udinese \\
		G/Sh & 1.107 & Tutti \\
		Saves & 0.260 & Tutti \\
		PAtt & 0.000 & Tutti \\
		PCmp\% & 0.000 & Tutti \\
		SPAtt & 0.124 & Napoli \\
		SPAtt & 0.000 & Tutti tranne Napoli \\
		SPCmp\% & 0.067 & Tutti tranne Genoa \\ 
		SPCmp\% & -0.235 & Genoa \\	
		MPAtt & -0.058 & Tutti \\ 
		MPCmp\% & -0.246 & Tutti tranne Bologna e Genoa \\
		MPCmp\% & -0.255 & Bologna e Genoa \\
		LPAtt & 0.077 & Tutti \\
		LPCmp\% & 0.199 & Hellas Verona \\
		LPCmp\% & 0.000 & Tutti tranne Bologna e Verona \\
		LPCmp\% & -0.303 & Bologna \\	     		   		    
		\hline
		& &  \\
		
	\end{tabular} \hbox{}
	\caption{Stime delle covariate stimate dal modello (\ref{for:4.9}).} \label{tab:BTCL2} 
	
\end{table}
\begin{table}[!htbp]%
	
	\renewcommand{\arraystretch}{1.7}
	\centering
	\begin{tabular}{ccp{10cm}}
		\hline			
		\textbf{Covariata} & \textbf{Stima} & \textbf{Squadra} \\	
		\hline
		ToDefPen & 0.135 & Tutti \\      
		ToDef3rd & 0.000 & Tutti \\
		ToMid3rd & 0.147 &Tutti\\
		ToAtt3rd & -0.154 & Tutti \\  
		ToAttPen & 0.000 & Tutti tranne Atalanta \\    
		ToAttPen & -0.311 & Atalanta \\ 	     	 
		TotDist & 0.000 & Tutti \\	
		Fls & 0.219 & Bologna  \\
		Fls & 0.012 & Tutti tranne Bologna, Napoli, Genoa e Salernitana  \\ 		
		Fls & -0.001 & Napoli  \\
		Fls & -0.030 & Genoa, Salernitana  \\
		Fld & 0.100 & Spezia \\
		Fld & 0.015 & Tutti tranne Spezia e Udinese  \\
		Fld & -0.005 & Udinese \\
		Off & 0.055 & Hellas Verona\\
		Off & 0.002 & Tutti tranne Verona, Inter, Juventus, Milan e Napoli\\
		Off & -0.097 & Inter, Juventus, Milan e Napoli  \\
		Crs & 0.000 & Torino\\
		Crs & -0.180 & Tutti tranne Milan, Roma, Torino, Atalanta e Napoli\\
		Crs & -0.359 & Roma\\
		Crs & -0.391 & Milan \\
		Crs & -0.639 & Napoli\\
		Crs & -0.671 & Atalanta \\
		Int & 0.012 & Tutti\\
		TklWin &  0.225 & Empoli  \\
		TklWin &  0.086 & Tutti tranne Empoli  \\ 
		Recov &  -0.120& Genoa \\ 
		Recov &  -0.132& Tutti tranne Udinese e Genoa \\ 
		Recov &  -0.189& Udinese \\ 
		\hline
		& &  \\
		
	\end{tabular} \hbox{}
	\caption{Stime delle covariate stimate dal modello (\ref{for:4.9}).} \label{tab:BTCL3} 
\end{table}
 
Il risultato della stima legata alla Lazio è un risultato in realtà non è sorprendente, infatti il \textit{\cite{sarrismotr}}, neologismo per indicare il gioco applicato dall'allenatore Maurizio Sarri, allenatore della Lazio nella stagione 2021/2022, ha tra le sue caratteristiche il mantenimento del possesso della palla, oltre a una propensione offensiva (vedi \textit{\cite{sarrismo}}). Analogamente anche il gioco del Torino si fonda sul possesso palla, ma con minor propensione offensiva (vedi \textit{\cite{torino}}).\\
Come era atteso, il numero di tiri \texttt{Sh}, il numero di tiri in porta \texttt{SoT}, il rapporto gol/tiri \texttt{G/Sh} e il numero di parate \texttt{Saves} sono fortemente associate all'aumento della probabilità di vittoria. Si nota che per \texttt{SoT} ci sono ben cinque stime, ciò poteva essere atteso dato che nella Tabella \ref{tab:BTC2} era stato stimato un SE pari a 0.324 che giustifica la variazione di stima da squadra a squadra. \\
Per quanto riguarda le variabili legate ai passaggi non ancora illustrate, abbiamo che, il numero di passaggi corti tentati \texttt{SPAtt} non risulta essere associato alla probabilità di vittoria per le squadre ad eccezione del Napoli, per la quale invece la stima risulta positiva. La percentuale di passaggi corti completati \texttt{SPCmp\%}, invece, presenta una stima del parametro molto bassa per tutte le squadre ad eccezione del Genoa, squadra per la quale il valore stimato è associato ad una diminuzione della probabilità di vittoria. Il numero di passaggi medi tentati \texttt{MPAtt} diminuisce le probabilità di vittoria per tutte le squadre. Analogamente anche per la percentuale di passaggi medi riusciti \texttt{MPCmp\%} ha il parametro stimato fortemente negativo.\\
Si nota che il numero di passaggi lunghi tentati \texttt{LPAtt} ha la stessa stima calcolata con il modello precedente per tutte le squadre. È interessante notare come la percentuale di passaggi lunghi riusciti \texttt{LPCmp\%} per la maggior parte delle squadre non ha alcuna associazione con l'esito della partita. Per l'Hellas Verona è associato ad un aumento della probabilità di vittoria, al contrario al Bologna è associato ad una diminuzione delle probabilità di vittoria. Infine, si nota che al crescere del numero di cross \texttt{Crs} si ha una diminuzione della probabilità di vittoria, principalmente per Atalanta e Napoli. Resta escluso il Torino, con una stima pari a 0.\\
Per quanto riguarda le variabili legate al possesso, al crescere del numero di tocchi in area di rigore \texttt{ToDefPen} e a centrocampo \texttt{ToMid3rd} si ha una crescita della probabilità di vittoria. Viceversa, il numero di tocchi fatti nella trequarti avversaria \texttt{ToAtt3rd} e nell'area di rigore avversaria \texttt{ToAttPen} è associato ad una riduzione della probabilità di vittoria.\\
Al subire falli \texttt{Fls} viene associato ad un aumento della probabilità di vittoria per molte squadre, specialmente per il Bologna, mentre la relazione si inverte per Napoli, Genoa e Salernitana. Per quanto riguarda l'effettuare falli \texttt{Fld} si associa una leggera probabilità di vittoria per la maggior parte delle squadre, soprattutto per lo Spezia. Tendenza inversa per l'Udinese.\\
Il numero di fuorigioco \texttt{Off} in generale ha una associazione non rilevante con l'esito della partita. Curiosamente per le quattro squadre con la maggior abilità stimata, cioè Milan, Inter, Napoli e Juventus, \texttt{Off} ha un impatto negativo sull'esito della partita. Tale risultato può essere spiegato dal fatto che le squadre più forti ad esempio Milan, Inter ecc., creano più azioni d'attacco, mentre le squadre meno forti per difendersi da esse, utilizzano la trappola del fuorigioco per fermarle.\\
Per quanto riguarda le variabili esplicative difensive, il numero di intercetti \texttt{Int} e il numero di contrasti vinti \texttt{TklWin} sono associati ad un aumento della probabilità di vittoria. Viceversa, il numero di recuperi \texttt{Recov} si associa ad una diminuzione della probabilità di vittoria a tutte le squadre. Anche qui rispetto al modello precedente è cambiato la stima delle soglie $\theta_1$ e $\theta_2$ che valgono rispettivamente -1.075 e 1.075.\\
Si ricorda scelto per ottenere i risultati illustrati precedentemente si è scelto il parametro di tuning ottimo, indicato con il simbolo $\lambda$, attraverso la procedura di K-Fold Cross Validation spiegata nel Capitolo \ref{cap:BT}. Nella Figura \ref{fig:lambda1} è mostrato l'andamento delle prestazioni del modello su tutti i valori assunti dal parametro di tuning $\lambda$ durante l'operazione di K-Fold Cross Validation.

\begin{figure}[htbp]
	\begin{center}
		\includegraphics[height=8cm, width=13cm]{CVBTL.png}
		\caption{Grafico dell'andamento delle prestazioni del modello (\ref{for:4.9}) su tutti i trenta valori assunti dal parametro di tuning, indicato con il simbolo $\lambda$, durante l'applicazione di K-Fold Cross Validation. L'andamento viene valutato in termini di Ranked Probability Score (RPS). La linea rossa tratteggiata indica il parametro di tuning $\lambda$ ottimo da utilizzare.} \label{fig:lambda1}
	\end{center}
\end{figure}
La K-Fold Cross Validation utilizzata prevedeva l'uso di 10 gruppi (k = 10) e di trenta valori diversi per $\lambda$. Successivamente, i risultati ottenuti dal modello applicando i trenta diversi valori del parametro di tuning, sono stati confrontati in termini di RPS. Si nota che con il diminuire della penalizzazione il modello registra una RPS che migliora fino a quando la penalizzazione diventa troppo debole, causando un peggioramento delle prestazioni. Perciò, il parametro di tuning ottimo $\lambda$ risulta essere pari a 1.196, come indicato dalla linea tratteggiata di color rosso.\\
In alcuni casi nelle stime dei parametri delle variabili esplicative c'è un alta variabilità delle stime tanto da essere negative per alcune squadre mentre per altre nulle o positive. In altri casi invece, si vengono a formare dei clusters per alcune covariate. Questo fenomeno lo si può osservare chiaramente dalla Figura \ref{fig:possL} alla Figura \ref{fig:recovL}. Nei grafici vengono mostrati come cambiano le stime dei parametri associati ad ogni covariata e per ogni squadra al variare del parametro di tuning espresso in scala logaritmica. Ovviamente con un valore alto di penalizzazione si vede all'inizio che tutte le stime sono spinte a zero, ma con il diminuire della penalizzazione si iniziano ad ottenere stime diverse per la stessa covariata. La linea rossa tratteggiata indica il parametro di tuning ottimo utilizzato per ottenere i risultati illustrati precedentemente.
\begin{figure}[htbp]
	\begin{center}
		\includegraphics[height=8cm, width=15cm]{possL.png}
		\caption{Grafico che riporta l'andamento stimato dal modello (\ref{for:4.9}) della stima del possesso della palla per ogni squadra al variare del parametro di tuning $\lambda$. La linea rossa tratteggiata indica il parametro di tuning $\lambda$ ottimo che è stato scelto per ottenere i risultati finali.} \label{fig:possL}
	\end{center}
\end{figure}

In Figura \ref{fig:possL} viene mostrato l'andamento relativo alla stima della covariata del possesso della palla \texttt{Poss}, si diversificano i risultati per Lazio e Torino che si discostano nettamente dall'andamento nullo tenuto dalla maggior parte delle squadre.

\begin{figure}[htbp]
	\begin{center}
		\includegraphics[height=8cm, width=15cm]{sotL.png}
		\caption{Grafico che riporta l'andamento stimato dal modello (\ref{for:4.9}) della stima del numero di tiri in porta per ogni squadra al variare del parametro di tuning $\lambda$. La linea rossa tratteggiata indica il parametro di tuning $\lambda$ ottimo che è stato scelto per ottenere i risultati finali.} \label{fig:sotL}
	\end{center}
\end{figure}

In Figura \ref{fig:sotL} viene mostrato l'andamento relativo alla stima della covariata del numero di tiri in porta \texttt{SoT}. Si notano cinque clusters con stima positiva. Il cluster con la stima più alta contiene la maggioranza delle squadre, seguito dal secondo cluster per stima contenente Inter e Roma. Il terzo cluster per stima contiene solo il Bologna, anche il quarto cluster per stima contiene solo una squadra, la Fiorentina. Infine, il quinto cluster per stima contiene l'Udinese che ha un valore positivo prossimo a zero.

\begin{figure}[htbp]
	\begin{center}
		\includegraphics[height=8cm, width=15cm]{spattL.png}
		\caption{Grafico che riporta l'andamento stimato dal modello (\ref{for:4.9}) della stima del numero di passaggi corti tentati per ogni squadra al variare del parametro di tuning $\lambda$. La linea rossa tratteggiata indica il parametro di tuning $\lambda$ ottimo che è stato scelto per ottenere i risultati finali.} \label{fig:spattL}
	\end{center}
\end{figure}

In Figura \ref{fig:spattL} viene mostrato l'andamento relativo alla stima della covariata del numero di passaggi corti tentati \texttt{SPAtt}. Si nota che il Napoli ha un andamento positivo che si discosta nettamente dall'andamento nullo tenuto dalla maggior parte delle squadre.

\begin{figure}[htbp]
	\begin{center}
		\includegraphics[height=8cm, width=15cm]{spcmpL.png}
		\caption{Grafico che riporta l'andamento stimato dal modello (\ref{for:4.9}) della stima della percentuale di passaggi corti riusciti per ogni squadra al variare del parametro di tuning $\lambda$. La linea rossa tratteggiata indica il parametro di tuning $\lambda$ ottimo che è stato scelto per ottenere i risultati finali.} \label{fig:spcmpL}
	\end{center}
\end{figure}

In Figura \ref{fig:spcmpL} viene mostrato l'andamento relativo alla stima della covariata della percentuale di passaggi corti riusciti \texttt{SPCmp\%}. Si nota che il Genoa ha un andamento negativo che si discosta nettamente dall'andamento leggermente positivo tenuto dalla maggior parte delle squadre.

\begin{figure}[htbp]
	\begin{center}
		\includegraphics[height=8cm, width=15cm]{mpcmpL.png}
		\caption{Grafico che riporta l'andamento stimato dal modello (\ref{for:4.9}) della stima della percentuale di passaggi medi riusciti per ogni squadra al variare del parametro di tuning $\lambda$. La linea rossa tratteggiata indica il parametro di tuning $\lambda$ ottimo che è stato scelto per ottenere i risultati finali.} \label{fig:mpcmpL}
	\end{center}
\end{figure}

In Figura \ref{fig:mpcmpL} viene mostrato l'andamento relativo alla stima della covariata della percentuale di passaggi medi riusciti \texttt{MPCmp\%}. Si nota che il Genoa e il Bologna hanno un andamento leggermente più negativo rispetto all’andamento, comunque, negativo tenuto dalla maggior parte delle squadre.

\begin{figure}[htbp]
	\begin{center}
		\includegraphics[height=8cm, width=15cm]{lpcmpL.png}
		\caption{Grafico che riporta l'andamento stimato dal modello (\ref{for:4.9}) della stima della percentuale di passaggi lunghi riusciti per ogni squadra al variare del parametro di tuning $\lambda$. La linea rossa tratteggiata indica il parametro di tuning $\lambda$ ottimo che è stato scelto per ottenere i risultati finali.} \label{fig:lpcmpL}
	\end{center}
\end{figure}

In Figura \ref{fig:lpcmpL} viene mostrato l'andamento relativo alla stima della covariata della percentuale di passaggi lunghi riusciti \texttt{LPCmp\%}. Sono presenti tre clusters. Il primo contenente solo l'Hellas Verona con un percorso positivo, il secondo che è il cluster più grande contiene quasi tutte le squadre ha un andamento nullo. Infine, il terzo cluster con il Bologna che presenta un andamento negativo.

\begin{figure}[htbp]
	\begin{center}
		\includegraphics[height=8cm, width=15cm]{toattpenL.png}
		\caption{Grafico che riporta l'andamento stimato dal modello (\ref{for:4.9}) della stima del numero di tocchi fatti nell'area di rigore avversaria per ogni squadra al variare del parametro di tuning $\lambda$. La linea rossa tratteggiata indica il parametro di tuning $\lambda$ ottimo che è stato scelto per ottenere i risultati finali.} \label{fig:toattpenL}
	\end{center}
\end{figure}

In Figura \ref{fig:toattpenL} viene mostrato l'andamento relativo alla stima della covariata del numero di tocchi fatti nell'area di rigore avversari \texttt{ToAttPen}. Si nota che l'Atalanta ha un andamento negativo che si discosta nettamente dall'andamento nullo tenuto dalla maggior parte delle squadre.

\begin{figure}[htbp]
	\begin{center}
		\includegraphics[height=8cm, width=15cm]{flsL.png}
		\caption{Grafico che riporta l'andamento stimato dal modello (\ref{for:4.9}) della stima del numero di falli subiti per ogni squadra al variare del parametro di tuning $\lambda$. La linea rossa tratteggiata indica il parametro di tuning $\lambda$ ottimo che è stato scelto per ottenere i risultati finali.} \label{fig:flsL}
	\end{center}
\end{figure}

In Figura \ref{fig:flsL} viene mostrato l'andamento relativo alla stima della covariata del numero di falli subiti \texttt{Fls}, in cui si notano quattro clusters. Il primo cluster contenente il Bologna che con un percorso positivo è seguito dal cluster più grande che contiene quasi tutte le squadre il quale ha un andamento leggermente positivo. Invece, il cluster che contiene il Napoli ha un andamento leggermente negativo, mentre ancora più negativo è il cluster contenente Genoa e Salernitana.

\begin{figure}[htbp]
	\begin{center}
		\includegraphics[height=8cm, width=15cm]{fldL.png}
		\caption{Grafico che riporta l'andamento stimato dal modello (\ref{for:4.9}) della stima del numero di falli fatti per ogni squadra al variare del parametro di tuning $\lambda$. La linea rossa tratteggiata indica il parametro di tuning $\lambda$ ottimo che è stato scelto per ottenere i risultati finali.} \label{fig:fldL}
	\end{center}
\end{figure}

In Figura \ref{fig:fldL} viene mostrato l'andamento relativo alla stima della covariata del numero di falli fatti, in cui si notano tre clusters. C'è il cluster contenete lo Spezia che ha un percorso positivo, il cluster più grande che contiene quasi tutte le squadre che ha un andamento leggermente positivo. Invece il cluster che contiene l'Udinese ha un andamento leggermente negativo.

\begin{figure}[htbp]
	\begin{center}
		\includegraphics[height=8cm, width=15cm]{offL.png}
		\caption{Grafico che riporta l'andamento stimato dal modello (\ref{for:4.9}) della stima del numero di fuorigioco fatti per ogni squadra al variare del parametro di tuning $\lambda$. La linea rossa tratteggiata indica il parametro di tuning $\lambda$ ottimo che è stato scelto per ottenere i risultati finali.} \label{fig:offL}
	\end{center}
\end{figure}

In Figura \ref{fig:offL} viene mostrato l'andamento relativo alla stima della covariata del numero di fuorigioco fatti \texttt{Off}. Ci sono tre clusters. Il primo cluster contenente l'Hellas Verona ha un percorso positivo. Il secondo cluster è il cluster più grande dato che contiene quasi tutte le squadre e ha un andamento leggermente positivo. Infine, il cluster terzo che contiene Milan, Inter, Napoli e Juventus ha un andamento negativo.

\begin{figure}[htbp]
	\begin{center}
		\includegraphics[height=8cm, width=15cm]{crsL.png}
		\caption{Grafico che riporta l'andamento stimato dal modello (\ref{for:4.9}) della stima del numero di cross fatti per ogni squadra al variare del parametro di tuning $\lambda$. La linea rossa tratteggiata indica il parametro di tuning $\lambda$ ottimo che è stato scelto per ottenere i risultati finali.} \label{fig:crsL}
	\end{center}
\end{figure}

In Figura \ref{fig:crsL} viene mostrato l'andamento relativo alla stima della covariata del numero di cross fatti \texttt{Crs}. Ci sono sei clusters. Il primo cluster contenente il Torino ha un andamento nullo. Il secondo cluster è il cluster più grande dato che contiene quasi tutte le squadre e ha un percorso negativo. Ancora più negativi sono i percorsi dei clusters contenenti rispettivamente Roma e Milan, secondi solo ai clusters contenenti Napoli e Atalanta che si discostano nettamente da tutti gli altri clusters.

\begin{figure}[htbp]
	\begin{center}
		\includegraphics[height=8cm, width=15cm]{tklwinL.png}
		\caption{Grafico che riporta l'andamento stimato dal modello (\ref{for:4.9}) della stima del numero di contrasti vinti per ogni squadra al variare del parametro di tuning $\lambda$. La linea rossa tratteggiata indica il parametro di tuning $\lambda$ ottimo che è stato scelto per ottenere i risultati finali.} \label{fig:tklwinL}
	\end{center}
\end{figure}

\begin{figure}[htbp]
	\begin{center}
		\includegraphics[height=8cm, width=15cm]{recovL.png}
		\caption{Grafico che riporta l'andamento stimato dal modello (\ref{for:4.9}) della stima del numero di recuperi per ogni squadra al variare del parametro di tuning $\lambda$. La linea rossa tratteggiata indica il parametro di tuning $\lambda$ ottimo che è stato scelto per ottenere i risultati finali.} \label{fig:recovL}
	\end{center}
\end{figure}
In Figura \ref{fig:tklwinL} viene mostrato l'andamento relativo alla stima della covariata del numero di contrasti vinti \texttt{TklWin}, in cui si nota che l'Empoli ha un percorso positivo che si discosta nettamente dall'andamento comunque positivo ma in minor misura tenuto dalla maggior parte delle squadre.\\
In Figura \ref{fig:recovL} viene mostrato l'andamento relativo alla stima della covariata del numero di recuperi \texttt{Recov}. Si nota che l'Udinese ha un percorso leggermente più negativo rispetto all'andamento negativo tenuto dalla maggior parte delle squadre. Viceversa, Genoa che ha un andamento meno negativo rispetto a tutte le altre squadre.\\

Per riassumere quanto visto finora, la Figura \ref{fig:l2BTCL} mostra i percorsi delle norme L2 che rappresentano l'importanza complessiva dei singoli effetti delle covariate.

\begin{figure}[htbp]
	\begin{center}
		\includegraphics[height=8cm, width=15cm]{L2.png}
		\caption{Grafico che riporta l'importanza delle covariate rispetto alle norme L2 al variare del parametro di tuning $\lambda$ secondo le stime del modello (\ref{for:4.9}). La linea rossa tratteggiata indica il parametro di tuning $\lambda$ ottimo che è stato scelto per ottenere i risultati finali.} \label{fig:l2BTCL}
	\end{center}
\end{figure}
Dal grafico si nota che il rapporto gol/tiri \texttt{G/Sh} è la variabile che incide maggiormente nella determinazione dell'esito di una partita. Analogamente, il numero di tiri in porta \texttt{SoT} e il numero di parate \texttt{Saves} sono determinanti sull'esito di una partita, ma con un minor peso rispetto a \texttt{G/Sh}. Anche il numero di cross \texttt{Crs} è significativamente associato all'esito della partita, ma al contrario di \texttt{G/Sh} sappiamo che contribuisce a diminuire le probabilità di vittoria. Si nota anche qui che il possesso della palla ha un ruolo molto marginale nel determinare il risultato di una partita. Viene confermata la tendenza che mantenere il pallone in zone difensive con meno transizioni in zone d'attacco sembra che si associ maggior probabilità di vittoria. Si riconferma importante il numero di lanci lunghi tentati \texttt{LPAtt} per l'esito favorevole della partita. Dal grafico si nota che sia il numero di contrasti vinti \texttt{TklWin} e il numero di fuorigioco \texttt{Off} sono determinanti per l'ottenimento della vittoria. Diversamente, il numero di recuperi \texttt{Recov}, la distanza percorsa con la palla \texttt{TotDist} e il numero di intercetti \texttt{Int} sono poco significativi. Infine, notiamo che il numero di falli fatti \texttt{fld} è più determinante di quelli subiti \texttt{fls}.\\
Perciò, la maggior parte delle stime ottenute sembrano essere in linea con i risultati osservati nel precedente modello con covariate specifiche dell'oggetto senza l'applicazione del metodo \emph{LASSO}.

\section{BTM senza l'intercetta e con LASSO}
Come era stato accennato nel Capitolo \ref{cap:BT}, l'intercetta spiega la maggior parte dell'abilità relativa alla squadra. Per cui le covariate possono essere viste come estensioni contenenti effetti aggiuntivi dell'abilità della squadra che non sono spiegati dall'intercetta. In tal senso, gli effetti della covariata possono aiutare a spiegare i risultati imprevisti di una partita. Nelle tre precedenti applicazione del modello Bradley-Terry è sempre stata inserita un intercetta per ogni squadra. Perciò, di seguito verranno mostrati i risultati relativi a un modello Bradley-Terry della stessa forma del modello \hyperref[for:4.9]{(4.12)} ma senza le intercette, con lo scopo di capire quale sia l'effetto che ogni variabile esplicativa sull'abilità della squadra senza l'interferenza dell'intercetta. Ovviamente dato il numero elevato di covariate è stata applicata una selezione attraverso il metodo \emph{LASSO}. Il modello applicato è il seguente
\begin{align}
	P(Y_{p(i,j)}\leq k) =  \frac{exp(\delta_i + \theta_{k} + x^T_{pi}\eta_i - x^T_{pj}\eta_j)}{1 + exp(\delta_i + \theta_{k} + x^T_{pi}\eta_i - x^T_{pj}\eta_j)}.\label{for:5.2}
\end{align}

Nella Tabella \ref{tab:BTCLI2} e nella Tabella \ref{tab:BTCLI3} vengono riportati i risultati nella stessa modalità utilizza nella precedente sezione.
\begin{table}[]%
	
	\renewcommand{\arraystretch}{1.7}
	\centering
	\begin{tabular}{ccp{10cm}}
		\hline	
		
		\textbf{Covariata} & \textbf{Stima} & \textbf{Squadra} \\	
		\hline
		Home & 0.270 & Tutti\\
		Poss & 0.299 & Lazio \\
		Poss & 0.047 & Tutti tranne Lazio\\
		Sh & 0.317 & Tutti \\
		SoT & 0.495 & Atalanta, Cagliari, Empoli, Genoa, Verona, Juventus, Lazio, Milan, Napoli, Roma, Salernitana, Sampdoria, Sassuolo, Spezia, Torino e Venezia\\
		SoT & 0.438 & Inter\\
		SoT & 0.399 & Bologna, Fiorentina e Udinese \\
		G/Sh & 0.867 & Tutti \\
		Saves & 0.242 & Tutti \\
		PAtt & 0.000 & Tutti \\
		PCmp\% & 0.000 & Tutti \\
		SPAtt & 0.000 & Tutti \\
		SPCmp\% & 0.000 & Tutti tranne Genoa \\ 
		SPCmp\% & -0.076 & Genoa \\	
		MPAtt & 0.000 & Tutti \\ 
		MPCmp\% & -0.230 & Udinese \\
		MPCmp\% & -0.236 & Tutti tranne Udinese \\		
		LPAtt & 0.178 & Tutti \\
		LPCmp\% & 0.016 & Hellas Verona \\
		LPCmp\% & 0.000 & Tutti tranne Hellas Verona \\
		ToDefPen & 0.080 & Tutti \\      
		ToDef3rd & 0.024 & Tutti \\
		
		
		\hline
		& &  \\
		
	\end{tabular} \hbox{}
	\caption{Stime delle covariate stimate dal modello (\ref{for:5.2}).} \label{tab:BTCLI2} 
	
\end{table}

\begin{table}[]%
	
	\renewcommand{\arraystretch}{1.7}
	\centering
	\begin{tabular}{ccp{10cm}}
		\hline	
		
		\textbf{Covariata} & \textbf{Stima} & \textbf{Squadra} \\	
		\hline
		ToMid3rd & 0.002 & Tutti tranne Inter e Sampdoria\\
		ToMid3rd & 0.000 & Inter e Sampdoria\\
		ToAtt3rd & -0.013 & Tutti \\  
		ToAttPen & 0.035 & Tutti tranne Atalanta \\    
		ToAttPen & -0.083 & Atalanta \\ 	     	 
		TotDist & 0.000 & Tutti \\	
		Fls & 0.256 & Bologna  \\
		Fls & 0.088 & Tutti tranne Bologna \\ 		
		Fld & 0.066 & Tutti tranne Udinese \\
		Fld & 0.023 & Udinese \\
		Off & 0.000 & Tutti tranne Juventus\\
		Off & -0.085 & Juventus  \\
		Crs & -0.190 & Tutti tranne Atalanta\\
		Crs & -0.464 & Atalanta \\
		Int & 0.000 & Tutti\\
		TklWin &  0.117 & Empoli  \\
		TklWin &  0.000 & Tutti tranne Empoli  \\ 
		Recov &  0.000 & Tutti \\ 
		\hline
		& &  \\
		
	\end{tabular} \hbox{}
	\caption{Stime delle covariate stimate dal modello (\ref{for:5.2}).} \label{tab:BTCLI3} 
	
\end{table}
Anche in questa applicazione sono state eliminate alcune covariate. Come già visto nel modello precedente, vengono confermate l'eliminazione del numero di passaggi tentati \texttt{PAtt}, della percentuale dei passaggi completati \texttt{PCmp\%} e della distanza percorsa con la palla \texttt{TotDist}. Viene tolta dal modello la variabile esplicativa del numero di passaggi corti tentati \texttt{SPAtt} la quale nel modello precedente andava ad aumentare le probabilità di vittoria solo per il Napoli. Viene eliminata la covariata del numero di passaggi medi tentati \texttt{MPAtt} e quella del numero di intercettazioni \texttt{Int}. Infine, si rileva l'eliminazione della variabile esplicativa che indica il numero di recuperi \texttt{Recov} che nel precedente modello era valutata come una covariata che incideva negativamente sulla probabilità di vittoria.\\
In questa nuovo tipo di modello la stima del parametro del possesso palla \texttt{Poss} ha subito una piccola variazione. Infatti, ora la stima non è più nulla per la maggior parte delle squadre ma e leggermente positiva. Ciononostante, la significatività si riconferma ancora bassa. Si riconferma però significativa solo per il gioco della Lazio, ma non più per il Torino come nello scorso modello. Tale risultato è visibile nella Figura \ref{fig:possLI}.\\
\begin{figure}[]
	\begin{center}
		\includegraphics[height=8cm, width=15cm]{possLI.png}
		\caption{Grafico che riporta l'andamento stimato dal modello (\ref{for:5.2}) della stima del possesso della palla per ogni squadra al variare del parametro di tuning $\lambda$. La linea rossa tratteggiata indica il parametro di tuning $\lambda$ ottimo che è stato scelto per ottenere i risultati finali.} \label{fig:possLI}
	\end{center}
\end{figure}
Il numero di tiri \texttt{Sh}, il rapporto gol/tiri \texttt{G/Sh} e il numero di parate \texttt{Saves} sono ancora determinati per aumentare la probabilità di vittoria. Analogamente anche il numero di tiri in porta \texttt{SoT} mantiene una stima positiva del parametro, con una variabilità più ristretta rispetto al modello precedente. Infatti nella Figura \ref{fig:sotLI} è possibile individuare tre clusters. Il più grande con la maggiore stima contiene la maggior parte delle squadre. Il secondo contiene solo l'Inter e infine il terzo contiene Bologna, Fiorentina e Udinese. Il risultato è visibile nella Figura \ref{fig:sotLI}.\\
\begin{figure}[htbp]
	\begin{center}
		\includegraphics[height=8cm, width=15cm]{sotLI.png}
		\caption{Grafico che riporta l'andamento stimato dal modello (\ref{for:5.2}) della stima del numero di tiri in porta per ogni squadra al variare del parametro di tuning $\lambda$. La linea rossa tratteggiata indica il parametro di tuning $\lambda$ ottimo che è stato scelto per ottenere i risultati finali.} \label{fig:sotLI}
	\end{center}
\end{figure}
Nella percentuale di passaggi corti riusciti \texttt{SPCmp\%} ora viene a crearsi un cluster con un percorso nullo contenente quasi tutte le squadre eccetto il Genoa che è contenuto in un cluster con un percorso negativo. Tali risultati sono visibili nella Figura \ref{fig:spcmpLI}.\\
\begin{figure}[htbp]
	\begin{center}
		\includegraphics[height=8cm, width=15cm]{spcmpLI.png}
		\caption{Grafico che riporta l'andamento stimato dal modello (\ref{for:5.2}) della stima della percentuale di passaggi corti riusciti per ogni squadra al variare del parametro di tuning $\lambda$. La linea rossa tratteggiata indica il parametro di tuning $\lambda$ ottimo che è stato scelto per ottenere i risultati finali.} \label{fig:spcmpLI}
	\end{center}
\end{figure}
Per la variabile esplicativa della percentuale di passaggi medi riusciti \texttt{MPCmp\%} ora viene a crearsi un cluster con un percorso fortemente negativo contenente quasi tutte le squadre eccetto l'Udinese dove si distingue per avere un percorso leggermente meno negativo. Tali risultati sono visibili nella Figura \ref{fig:mpcmpLI}.\\
\begin{figure}[]
	\begin{center}
		\includegraphics[height=8cm, width=15cm]{mpcmpLI.png}
		\caption{Grafico che riporta l'andamento stimato dal modello (\ref{for:5.2}) della stima della percentuale di passaggi medi riusciti per ogni squadra al variare del parametro di tuning $\lambda$. La linea rossa tratteggiata indica il parametro di tuning $\lambda$ ottimo che è stato scelto per ottenere i risultati finali.} \label{fig:mpcmpLI}
	\end{center}
\end{figure}
Il numero di passaggi lunghi tentati è ancora una covariata con una stima del parametro che aumenta la probabilità di vittoria. Si nota che la stima della covariata della percentuale di passaggi lunghi riusciti \texttt{LPCmp\%} ha un cluster con un percorso nullo contenente quasi tutte le squadre eccetto l'Hellas Verona, che si distingue per un percorso positivo. Tali risultati sono visibili nella Figura \ref{fig:lpcmpLI}.\\
\begin{figure}[htbp]
	\begin{center}
		\includegraphics[height=8cm, width=15cm]{lpcmpLI.png}
		\caption{Grafico che riporta l'andamento stimato dal modello (\ref{for:5.2}) della stima della percentuale di passaggi lunghi riusciti per ogni squadra al variare del parametro di tuning $\lambda$. La linea rossa tratteggiata indica il parametro di tuning $\lambda$ ottimo che è stato scelto per ottenere i risultati finali.} \label{fig:lpcmpLI}
	\end{center}
\end{figure}
Sia il numero di tocchi fatti in area di rigore \texttt{ToDefPen} sia il numero di tocchi fatti nella trequarti di difesa \texttt{ToDef3rd} sono associate ad un aumento della probabilità di vittoria. Nella stima del parametro della covariata che indica il numero di tocchi fatti a centrocampo \texttt{ToMid3rd}, viene a crearsi un cluster con un percorso poco significativo contenente quasi tutte le squadre eccetto Inter e Sampdoria, le quali formano un cluster con un percorso non significante.\\
Il numero di tocchi fatti nella trequarti offensiva \texttt{ToAtt3rd} si conferma essere associato ad una riduzione della probabilità di vittoria.\\
Per la stima della variabile esplicativa che indica il numero di tocchi fatti nell'area di rigore avversaria \texttt{ToAttPen}, c'è un cluster con un percorso negativo contenente quasi tutte le squadre eccetto l'Atalanta che si distingue per un percorso positivo. Tali risultati sono visibili nella Figura \ref{fig:toattpenLI}.\\
\begin{figure}[htbp]
	\begin{center}
		\includegraphics[height=8cm, width=15cm]{toattpenLI.png}
		\caption{Grafico che riporta l'andamento stimato dal modello (\ref{for:5.2}) della stima del numero di tocchi fatti nell'area di rigore avversaria per ogni squadra al variare del parametro di tuning $\lambda$. La linea rossa tratteggiata indica il parametro di tuning $\lambda$ ottimo che è stato scelto per ottenere i risultati finali.} \label{fig:toattpenLI}
	\end{center}
\end{figure}
Per quanto riguarda l'aggressività della squadra, il numero di falli fatti \texttt{Fld} si associa ad un aumento della probabilità di vittoria per tutte le squadre. Come mostrato della Figura \ref{fig:fldLI} però, la stima per l'Udinese è minore rispetto a tutte le altre squadre. Analogamente anche il numero di falli subiti \texttt{Fls} si associa ad una aumento della probabilità di vittoria di tutte le squadre, in particolare il Bologna si distingue con una stima maggiore come mostrato nella Figura \ref{fig:flsLI}.
\begin{figure}[]
	\begin{center}
		\includegraphics[height=8cm, width=15cm]{fldLI.png}
		\caption{Grafico che riporta l'andamento stimato dal modello (\ref{for:5.2}) della stima del numero di falli fatti per ogni squadra al variare del parametro di tuning $\lambda$. La linea rossa tratteggiata indica il parametro di tuning $\lambda$ ottimo che è stato scelto per ottenere i risultati finali.} \label{fig:fldLI}
	\end{center}
\end{figure}
\begin{figure}[htbp]
	\begin{center}
		\includegraphics[height=8cm, width=15cm]{flsLI.png}
		\caption{Grafico che riporta l'andamento stimato dal modello (\ref{for:5.2}) della stima del numero di falli subiti per ogni squadra al variare del parametro di tuning $\lambda$. La linea rossa tratteggiata indica il parametro di tuning $\lambda$ ottimo che è stato scelto per ottenere i risultati finali.} \label{fig:flsLI}
	\end{center}
\end{figure}
Nella stima della covariata che indica il numero di fuorigioco fatti \texttt{Off}, viene a crearsi un cluster con un percorso nullo contenente quasi tutte le squadre eccetto la Juventus, la quale forma un cluster con un percorso negativo. Tali risultati sono visibili nella Figura \ref{fig:offLI}.
\begin{figure}[htbp]
	\begin{center}
		\includegraphics[height=8cm, width=15cm]{offLI.png}
		\caption{Grafico che riporta l'andamento stimato dal modello (\ref{for:5.2}) della stima del numero di fuorigioco fatti per ogni squadra al variare del parametro di tuning $\lambda$. La linea rossa tratteggiata indica il parametro di tuning $\lambda$ ottimo che è stato scelto per ottenere i risultati finali.} \label{fig:offLI}
	\end{center}
\end{figure}
La stima della variabile esplicativa del numero di cross fatti \texttt{Crs} si conferma essere ancora determinante per diminuire la probabilità di vittoria. L'Atalanta inoltre si distingue dalle altre squadre con un percorso ancora pù negativo rispetto, come mostrato nella Figura \ref{fig:crsLI}.
\begin{figure}[htbp]
	\begin{center}
		\includegraphics[height=8cm, width=14.8cm]{crsLI.png}
		\caption{Grafico che riporta l'andamento stimato dal modello (\ref{for:5.2}) della stima del numero di cross fatti per ogni squadra al variare del parametro di tuning $\lambda$. La linea rossa tratteggiata indica il parametro di tuning $\lambda$ ottimo che è stato scelto per ottenere i risultati finali.} \label{fig:crsLI}
	\end{center}
\end{figure}
Si nota che nella stima della covariata che indica il numero di contrasti vinti \texttt{TklWin}, viene a crearsi un cluster con un percorso nullo contenente quasi tutte le squadre eccetto l'Empoli, il quale forma un cluster con un percorso positivo. Tali risultati sono visibili nella Figura \ref{fig:tklwinLI}.
\begin{figure}[htbp]
	\begin{center}
		\includegraphics[height=8cm, width=15cm]{tklwinLI.png}
		\caption{Grafico che riporta l'andamento stimato dal modello (\ref{for:5.2}) della stima del numero di contrasti vinti per ogni squadra al variare del parametro di tuning $\lambda$. La linea rossa tratteggiata indica il parametro di tuning $\lambda$ ottimo che è stato scelto per ottenere i risultati finali.} \label{fig:tklwinLI}
	\end{center}
\end{figure}
Infine, viene confermato che giocare le partite in casa \texttt{Home} ha un effetto positivo stimato in 0.270, mentre è cambiata la stima delle soglie $\theta_1$ e $\theta_2$ che valgono rispettivamente -0.803  e 0.803 .\\
Nella Figura \ref{fig:lambda2} viene mostrato l'andamento delle prestazioni del modello su tutti i valori assunti dal parametro di tuning $\lambda$ durante l'operazione di K-Fold Cross Validation.
\begin{figure}[]
	\begin{center}
		\includegraphics[height=8cm, width=13cm]{CVBTLI.png}
		\caption{Grafico dell'andamento delle prestazioni del modello (\ref{for:5.2}) su tutti i trenta valori assunti dal parametro di tuning, indicato con il simbolo $\lambda$, durante l'applicazione di K-Fold Cross Validation. L'andamento viene valutato in termini di Ranked Probability Score (RPS). La linea rossa tratteggiata indica il parametro di tuning $\lambda$ ottimo da utilizzare.} \label{fig:lambda2}
	\end{center}
\end{figure}
Anche qui, la K-Fold Cross Validation utilizzata prevedeva l'uso di 10 gruppi (k = 10) e di trenta valori diversi per $\lambda$. Successivamente, i risultati ottenuti dal modello applicando i trenta diversi valori del parametro di tuning, sono stati confrontati in termini di RPS. Come indicato dal grafico con la linea tratteggiata in rosso, il parametro di tuning ottimo $\lambda$ è pari a 1.458.\\

Un'ulteriore analisi che può essere condotta è la valutazione dell'effetto medio dei valori assunti della covariata per ogni partita e per ogni squadra, insieme alle stime dei singoli parametri per squadra. Si utilizzeranno i grafici a \emph{effetto stella} proposti da \textcite{tutz2013visualization}. In questi grafici è possibile visualizzare i valori medi per squadra e per covariata moltiplicati per le rispettive stime riportate precedentemente. Quindi verrà illustrato graficamente il contributo medio di una variabile esplicativa sull'abilità di una singola squadra. Il grafico funziona nel seguente modo: esso mostra il prodotto esponenziale tra la media dei valori assunti da una covariate e le sue stime per ogni squadra. Per ogni grafico, viene creato un cerchio con raggio \emph{exp(0) = 1} il quale rappresenta il caso con stima nulla. I valori oltre il cerchio indicano che la covariata ha effetto positivo in media sulla squadra. Viceversa, i valori all'interno del cerchio indicano che la variabile esplicativa si associa a effetti negativi in media sulla squadra. Nella Figura \ref{fig:effstar1} nella Figura \ref{fig:effstar2} e nella Figura \ref{fig:effstar3} vengono mostrati i grafici a \emph{effetto stella}.
Nella Figura \ref{fig:effstar1} si possono vedere tutte le variabili esplicative in cui la maggioranza delle squadre ha ricevuto una stima dei parametri delle variabili esplicative   uguale a zero. Si possono comunque notare alcune squadre distinguersi in alcune covariate.
Ad esempio la Lazio si differenzia dalle altre squadre con il possesso palla \texttt{Poss} mentre l'Empoli con il numero di contrasti vinti \texttt{TklWin}.\\
\begin{figure}[htbp]
	\begin{center}
		\includegraphics[height=8.5cm, width=15cm]{effstar.png}
		\caption{Grafico a effetto stella che riporta il contributo medio di una covariata sull'abilità di una singola squadra secondo il modello (\ref{for:5.2}).} \label{fig:effstar1}
	\end{center}
\end{figure}
Per la Figura \ref{fig:effstar2} abbiamo due particolari grafici. Entrambi rappresentano l'effetto negativo delle covariate che indicano, rispettivamente, il numero di tocchi nella trequarti avversaria fatti \texttt{ToAtt3rd} e il numero di cross fatti \texttt{Crs}. Notiamo che a subire più gli effetti negativi sono Inter e Atalanta per entrambe le variabili esplicative. 
\begin{figure}[htbp]
	\begin{center}
		\includegraphics[scale = 0.38]{estarL2.png}
		\caption{Grafico a effetto stella che riporta il contributo medio di una covariata sull'abilità di una singola squadra secondo il modello (\ref{for:5.2}).} \label{fig:effstar2}
	\end{center}
\end{figure}
Infine, risultati più interessanti si hanno nella Figura \ref{fig:effstar3}. Innanzitutto si vede che nel numero di tiri \texttt{Sh} l'Inter ha un grosso beneficio. In minor misura ne beneficiano Milan, Roma, Atalanta, Napoli, Juventus e Sassuolo. Analoghi risultati sono visibili con il numero di tiri in porta \texttt{SoT}, con l'aggiunta della Lazio tra le squadre che ricevano più benefici. In generale, nel grafico relativo al numero di parate \texttt{Saves}, tutte le squadre ottengo benefici. Stesso risultato si rileva per il numero di passaggi lunghi tentati \texttt{LPAtt}. Nel grafico del numero di tocchi in area di rigore \texttt{ToDefPen} c'è un particolare beneficio ottenuto dal Venezia, cosi come Inter, Lazio, Empoli e Sassuolo. Analoghi risultati anche per il numero di tocchi nella trequarti difensiva \texttt{ToDef3rd}, ma con la differenza di minori benefici per il Venezia. Pertanto, si nota la tendenza delle squadre italiane ad attuare tattiche che prediligono di giocare nella propria metà campo. I tocchi a centrocampo \texttt{ToMid3rd} vengono in media effettuati molto dalle squadre, tranne per Inter e Sampdoria per le quali l'effetto è nullo. Analogo effetto anche per il numero di tocchi fatti in area di rigore \texttt{ToAttPen}, con l'unica differenza che si registra ora un effetto positivo per Inter e Sampdoria e un effetto negativo solo per Atalanta. In generale i falli subiti \texttt{Fls} portano benefici alle squadre, come si era notato dalle stime del modello. Per i falli fatti, invece abbiamo che l'Udinese ha minori benefici rispetto a tutte le altre squadre. \\
Dai grafici \emph{effetto stella} emerge la tendenza da parte delle squadre italiane a ricorrere alla cosiddetta \textit{\cite{costrdalbasso}}. Ossia la costruzione dell'azione partendo dal proprio portiere e l'utilizzo di lanci lunghi. Inoltre, sapendo che ai parametri del numero di tocchi in area di rigore \texttt{ToDefPen}, del numero di tocchi nella trequarti difensiva \texttt{ToDef3rd}, del numero di tocchi a centrocampo \texttt{ToMid3rd} e del numero di passaggi lunghi tentati \texttt{LPAtt} sono associati degli aumenti della probabilità di vittoria, è possibile affermare che la costruzione dal basso offre maggiori probabilità di vittoria.\\
\begin{figure}[!htbp]
	\begin{center}
		\includegraphics[height=16cm, width=16cm]{estarL.png}
		\caption{Grafico che riporta il contributo medio di una covariata sull'abilità di una singola squadra secondo il modello \ref{for:5.2}.} \label{fig:effstar3}
	\end{center}
\end{figure}
Infine, come fatto nella sezione precedente, si analizzano i percorsi delle norme L2 che rappresentano l'importanza complessiva dei singoli effetti delle covariate. Tali percorsi sono visibili nella Figura \ref{fig:IL2}.

\begin{figure}[!htbp]
	\begin{center}
		\includegraphics[height=8cm, width=15cm]{IL2.png}
		\caption{Grafico che riporta l'importanza delle covariate rispetto alle norme L2 al variare del parametro di tuning $\lambda$} \label{fig:IL2}
	\end{center}
\end{figure}

Gli andamenti ottenuti nella Figura \ref{fig:IL2} sono molto simili a quelli visti nella Figura \ref{fig:l2BTCL}, con un aumento di importanza per la covariata \texttt{ToAttPen} in termini di diminuzione della probabilità di vittoria. Pertanto, quanto ricavato del modello (\ref{for:4.9}) ora trova conferma anche nel modello (\ref{for:5.2}).
\pagebreak

\section{Predizioni}
In questa sezione si vuole valutare le prestazioni dei quattro modelli presentati nelle precedenti sezioni. I modelli vengono valutati in base alle loro capacità di predizioni dell'esito di una partita, sulla base delle informazioni (covariate) disponibili. Per predizione si intende che il modello stabilisce l'esito di una partita senza conoscerne il risultato reale. Per rendere più interessante il confronto si è aggiunto un quinto elemento nel confronto, ossia le predizioni fatte dai \emph{bookmakers}, ad esempio Bet365, William Hill ecc... I dati dei \emph{bookmakers} sono stati presi da \textit{\cite{bet}}, il quale fornisce la media delle probabilità dei \emph{bookmakers} per ogni risultato, su un gran numero di campionati di calcio, tra cui la Serie A italiana. Si è quindi preso come predizione il risultato più probabile secondo i \emph{bookmakers}.\\
Le predizioni dei modelli sono state eseguite nel seguente modo: il \emph{dataset} è stato diviso in modo casuale in due parti chiamate solitamente \emph{training set} e \emph{test set}. Il \emph{training set} contiene quasi l'80\% delle 38 giornate, ossia 30 giornate per un totale di 300 partite. Invece il \emph{test set} contiene circa il restante 20\% ossia 8 giornate per un totale di 80 partite. Il \emph{training set} è utilizzato per stimare i parametri del modello mentre il \emph{test set} è utilizzato per fare predizione. Perciò una parte delle osservazioni è stata utilizzata per allenare il modello, mentre la restante parte per predire l'esito delle restanti osservazioni. \\
Prima di discutere delle misurazioni e delle predizioni ottenute, è importante tener presente che i modelli (\ref{for:3.1}), (\ref{for:5.1}), (\ref{for:4.9}) e (\ref{for:5.2}) utilizzano informazioni e statistiche che sono disponibili solo dopo il termine delle partite, cioè non disponibili per i \emph{bookmakers}. Infatti, i \emph{bookmakers} calcolano le loro predizioni prima che le partite comincino. Certamente i quattro modelli non sono utilizzabili per poter fare predizioni, ma l'obbiettivo del confronto è quello di capire se le informazioni sono state impiegate nel modo opportuno per acquisire maggior conoscenza. Ossia se i modelli ottengono delle prestazioni peggiori rispetto ai \emph{bookmakers}, nonostante abbiano più informazioni sulle partite allora, non sono state utilizzate nel modo corretto le informazioni, viceversa se le prestazioni dei modelli sono migliori rispetto ai \emph{bookmakers} allora le informazioni sono state utilizzate correttamente. \\
La capacità predittiva di un modello sarà valutata come segue:
\begin{itemize}
	\item \texttt{Accuratezza}. Indica il rapporto tra il numero di istanze classificate correttamente e il numero totale delle osservazioni in esame.
	\item \texttt{Sensibilità}. Indica il rapporto tra il numero di istanze identificate correttamente con la categoria \emph{k} e il numero totale delle osservazioni di categoria \emph{k} con \emph{k $\in$ \{1,....,K\}}.
	\item \textsf{Precisione}. Misura il grado di correttezza del sistema. Indica il rapporto tra il numero di istanze identificate correttamente con la categoria \emph{k} e il numero totale delle osservazioni classificate con la categoria \emph{k} con \emph{k $\in$ \{1,....,K\}}.
	\item \texttt{Specificità}. Indica il rapporto tra il numero di predizioni identificate correttamente con una categoria diversa dalla categoria \emph{k} e il numero totale delle osservazioni classificate con una categoria diversa dalla categoria \emph{k} con \emph{k $\in$ \{1,....,K\}}.
\end{itemize}
Nella Figura \ref{fig:pre} sono mostrate le classificazione ottenute sulle 80 partite del \emph{test set} per ogni modello e per la predizione dei \emph{bookmakers}.
\begin{figure}[h]
	\begin{center}
		\includegraphics[scale = 0.28]{tabpre.png}
		\caption{La prima tabella indica le predizioni di 80 partite fatte dal modello (\ref{for:3.1}), la seconda dal modello (\ref{for:5.1}), la terza dal modello (\ref{for:4.9}), la quarta dal modello (\ref{for:5.2}) e la quinta dai \emph{bookmakers}. La classe 1 indica la vittoria della squadra in casa, la classe 2 indica il pareggio tra le due squadre, la classe 3 indica la vittoria della squadra ospite. Con \textsf{True} si indicano le classificazioni effettive mentre con \textsf{Predicted} le classificazioni effettuate.}\label{fig:pre}
	\end{center}
\end{figure}
Dai risultati ottenuti, l'accuratezza dei quattro modelli è rispettivamente 0.65, 0.6125, 0.6625 e 0.6375, mentre per i \emph{bookmakers} è di 0.55. Si può subito notare che tutti e quattro i modelli sono migliori delle predizioni dei \emph{bookmakers}. In particolare, il modello (\ref{for:4.9}) risulta essere quello che produce più predizioni corrette. Sorprendentemente il modello (\ref{for:3.1}) che utilizza solo le abilità medie delle squadre e quindi senza l'utilizzo delle variabili esplicative risulta essere migliore di tutti eccetto del modello (\ref{for:4.9}). In particolare, si conferma quanto enunciato nel Capitolo \ref{cap:BT} riguardo al ruolo dell'intercetta e delle covariate, infatti il modello senza intercetta (\ref{for:5.2}) risulta essere leggermente peggiore del modello con l'intercetta (\ref{for:4.9}). A tal proposito si può confermare l'esistenza di una differente relazione delle variabili esplicative da squadra a squadra. Infatti, tra i quattro modelli confrontati, il modello (\ref{for:5.1}) ottiene le peggiori prestazioni dato che ha ignorato le diverse relazioni delle variabili esplicative con ogni singola squadra.\\
Nella Figura \ref{fig:recall} vengono mostrate le misurazioni della sensibilità per ognuna delle tre categorie per i quattro modelli e per i \emph{bookmakers}.\\
\begin{figure}[h]
	\begin{center}
		\includegraphics[scale = 0.60]{recall.png}
		\caption{La prima tabella indica le sensibilità delle predizioni del modello (\ref{for:3.1}), la seconda del modello (\ref{for:5.1}), la terza del modello (\ref{for:4.9}), la quarta del modello (\ref{for:5.2}) e la quinta dei \emph{bookmakers}. La classe 1 indica la vittoria della squadra in casa, la classe 2 indica il pareggio tra le due squadre, la classe 3 indica la vittoria della squadra ospite. }\label{fig:recall}
	\end{center}
\end{figure}
Come si può notare i \emph{bookmakers} hanno molte difficoltà a identificare le partite che terminano con un pareggio. Infatti, vediamo che la sensibilità è pari a zero. Questo perché ci sono zero partite identificate con il pareggio, ma dai dati osservati si sa che ci sono delle partite che terminano con tale esito. Tuttavia, sono molto affidabili nell’identificare la vittoria della squadra in casa contro la squadra ospite. I modelli (\ref{for:4.9}) e (\ref{for:5.2}) sanno identificare correttamente la vittoria della squadra ospite sulla squadra che gioca in casa. Infatti, la sensibilità associata è pari a uno. Infine, notiamo che i modelli (\ref{for:3.1}) e (\ref{for:5.1}) fanno registrare delle buone prestazione sulla sensibilità delle partite di classe pareggio infatti, sbagliano a identificare rispettivamente quattro e tre partite su 20. Viceversa, i modelli (\ref{for:4.9}) e (\ref{for:5.2}) non hanno buone prestazioni sulla sensibilità della classe pareggio infatti, molte partite di classe pareggio non vengono identificate correttamente con tali dai due modelli. Per tutti i modelli Bradley-Terry calcolati la sensibilità della classe vittoria della squadra in casa è discreta.\\ 
Nella Figura \ref{fig:precision} vengono mostrate le misurazioni della precisione per ognuna delle tre categorie per i quattro modelli e per i \emph{bookmakers}.
\begin{figure}[]
	\begin{center}
		\includegraphics[scale = 0.60]{precision.png}
		\caption{La prima tabella indica la precisione delle predizioni del modello (\ref{for:3.1}), la seconda del modello (\ref{for:5.1}), la terza del modello (\ref{for:4.9}), la quarta del modello (\ref{for:5.2}) e la quinta dei \emph{bookmakers}. La classe 1 indica la vittoria della squadra in casa, la classe 2 indica il pareggio tra le due squadre, la classe 3 indica la vittoria della squadra ospite.}\label{fig:precision}
	\end{center}
\end{figure}
Come già riportato, i \emph{bookmakers} hanno molte difficoltà a etichettare correttamente le partite che terminano con un pareggio. Infatti, sono state classificate zero partite con la classe pareggio. Inoltre, la precisione registrata nelle classi vittoria della squadra in casa e vittoria della squadra ospite è discreta. I modelli (\ref{for:4.9}) e (\ref{for:5.2})
sono estremamente precisi nell'etichettare le partite con le classi vittoria della squadra in casa e pareggio infatti, non hanno commesso alcun errore di classificazione. Purtroppo, le prestazioni dei modelli (\ref{for:4.9}) e (\ref{for:5.2}) calano per quanto riguarda la precisione sulla classe vittoria della squadra ospite. Anche i modelli (\ref{for:3.1}) e (\ref{for:5.1}) sono estremamente precisi nell'etichettare le partite con la classe vittoria della squadra in casa infatti, non hanno commesso alcun errore di classificazione. Buone prestazioni si registrano per quanto riguarda le partite che terminano con la vittoria della squadra ospite. Purtroppo, i modelli hanno registrato molti errori di classificazione nella classe pareggio.\\ 
Nella Figura \ref{fig:speci} vengono mostrate le misurazioni della specificità per ognuna delle tre categorie per i quattro modelli e per i \emph{bookmakers}.\\
\begin{figure}[]
	\begin{center}
		\includegraphics[scale = 0.60]{specificity.png}
		\caption{La prima tabella indica le specificità delle predizioni del modello (\ref{for:3.1}), la seconda del modello (\ref{for:5.1}), la terza del modello (\ref{for:4.9}), la quarta del modello (\ref{for:5.2}) e la quinta dei \emph{bookmakers}. La classe 1 indica la vittoria della squadra in casa, la classe 2 indica il pareggio tra le due squadre, la classe 3 indica la vittoria della squadra ospite.}\label{fig:speci}
	\end{center}
\end{figure}
Ovviamente in questa misurazione i \emph{bookmakers} ottengono il miglior risultato per quanto riguarda l'identificazione delle partite che non terminano in un pareggio. Si nota che tutti e quattro i modelli Bradley-Terry riescono a predire quando l'esito della partita non è la vittoria della squadra in casa, infatti la misurazione ottenuta è uno. Inoltre, i modelli (\ref{for:4.9}) e (\ref{for:5.2}) riescono a identificare correttamente le partite che non terminano in un pareggio. D’altra parte, i modelli (\ref{for:3.1}) e (\ref{for:5.1}) ottengono le migliori misurazione nell'identificare le partite che non terminano con la vittoria della squadra ospite. Viceversa, i modelli (\ref{for:4.9}) e (\ref{for:5.2}) che ottengono le peggiori prestazioni nell'identificare le partite che non terminano con la vittoria della squadra ospite.\\
%\pagebreak
In conclusione, dai risultati ottenuti si evince che i modelli mostrano una maggiore accuratezza nelle previsioni rispetto a quanto fornito dai \emph{bookmakers}. Se ne deduce quindi un buon utilizzo delle informazioni a disposizione nei modelli.



