% !TEX encoding = UTF-8
% !TEX TS-program = pdflatex
% !TEX root = ../tesi.tex

%**************************************************************
\chapter{Metodi di Machine Learning}
\label{cap:ML}
%**************************************************************
\intro{Questo capitolo illustrerà i metodi di \emph{Machine Learning} che sono stati utilizzati per la predizione degli esiti delle partite di calcio della Seria A italiana della stagione 2021/2022. Purtroppo, non è stato possibile applicare metodi di \emph{Machine learning} che corrispondessero al modello \emph{Bradley-Terry} perché, nonostante esistano metodi in \emph{Machine learning} che forniscono modelli basati sul modello \emph{Bradley-Terry}, essi non sono in grado di gestire l'esito del pareggio ma solo un esito binario. Ne consegue che tali metodi non sono adatti per contesti come il calcio ma ad altri tipi di sport dove il pareggio non è previsto come il \emph{baseball}. I metodi di \emph{Machine learning} considerati sono: il K-Nearest-Neighbors (KNN), la Support Vector Machine (SVM), gli alberi di decisione per la classificazione, la Random Forests e in fine l'Adaboost.
}
\section{Componenti essenziali}
In questa sezione vengono definite alcune misure e tecniche che sono necessarie per il funzionamento dei metodi di \emph{Machine Learning} applicati.
\subsection{Distanza di Minkowski}
La \textit{\cite{minkdist}} è una misura utilizzata per la valutazione della distanza ovvero, nel nostro contesto della somiglianza tra due punti in spazio di \textit{n}-dimensioni. La distanza di Minkowski di ordine \emph{d} tra due punti A = (a$_1$,...a$_n$) e B = (b$_1$,...b$_n$) vale
\begin{center}
	$Dist(A,B) =  \left(\sum_{i = 1}^{n}|a_i-b_i|^d\right)^{1/d} $
\end{center}

Si sottolinea che quando l'ordine d = 1, la distanza utilizzata è la \textit{\cite{manhattan}} ovvero la distanza tra due punti è la somma del valore assoluto delle differenze delle loro coordinate. Quando l'ordine d = 2 è applicata la \textit{\cite{euclidea}} dove la distanza tra due punti è la lunghezza del segmento con agli estremi i due punti d'interesse.
Tale misura sarà utilizzata nel metodo K-Nearest-Neighbors (KNN).
\subsection{Funzione kernel}
Nel contesto dell'apprendimento automatico, la \textit{\cite{kernel}} permette di trasformare uno spazio di input non linearmente separabile in uno nuovo spazio delle istanze di input detto \emph{feature space} di dimensione superiore rispetto a quello originale tale da diventare linearmente separabile. Per spazio linearmente separabile si intende che esiste un iperpiano in grado di separare correttamente i dati in due gruppi distinti. Perciò aumentando la dimensionalità dello spazio d'interesse è possibile trovare la dimensione opportuna che permetta di separare linearmente i dati. Tale applicazione è chiamata kernel trick. Perciò, una funzione kernel è una funzione \emph{K} che per ogni \emph{x}, \emph{y} $\in \chi$ dove $\chi$ è lo spazio di input di dimensione \emph{n}, vale 
\begin{center}
	$K(x,y) =  \langle\psi(x),\psi(y)\rangle $.
\end{center}
Dove $\psi$ è la funzione che mappa i punti di uno spazio di dimensione \emph{n} in uno spazio di dimensione \emph{m} con \emph{m>n}, invece, $\langle . \rangle$ indica il prodotto scalare.\\
Nelle nostre predizioni saranno usati questi kernel:
\begin{itemize}
	\item Linear kernel: è la funzione precedentemente definita.
	\item Polynomial kernel: $K(x,y) =  \left(1 + \sum_{i = 1}^{p}x_iy_i\right)^{d} $ dove \emph{p} è il numero di istanze di input presenti in $\chi$ mentre \emph{d} la dimensione del spazio (l'ordine).
	\item Gaussian Radial Basis kernel (RBF): $K(x,y) = exp(-\gamma||x-y||^2) $ con $\gamma=\frac{1}{2\sigma^2}$ mentre $\sigma$ è un paramento libero. 
\end{itemize}
La funzione kernel sarà utilizza nella Support Vector Machine (SVM).
