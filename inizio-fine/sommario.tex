% !TEX encoding = UTF-8
% !TEX TS-program = pdflatex
% !TEX root = ../tesi.tex

%**************************************************************
% Sommario
%**************************************************************
\cleardoublepage
\phantomsection
\pdfbookmark{Sommario}{Sommario}
\begingroup
\let\clearpage\relax
\let\cleardoublepage\relax
\let\cleardoublepage\relax

\chapter*{Abstract}

Viviamo nell'era dei cosiddetti \emph{Big Data}, dove grazie all'interconnessione, un grande flusso di informazioni e di dati può essere ricavato da ogni possibile attività. \\
Non fa eccezione il calcio in cui da un paio d'anni, le società calcistiche si affidano a sistemi di analisi per produrre tattiche di gioco ma anche per effettuare \textit{scouting} di giocatori emergenti. Nel calcio moderno, perciò, numerose variabili ad esempio il possesso palla, il numero di tiri effettuati da una squadra ecc. vengono raccolte durante una partita di calcio.\\
Tale fatto scaturisce l'attenzione su un ulteriore tematica d'analisi: dato che si hanno a disposizione un gran numero di dati sulle prestazioni delle squadre nelle loro partite, è possibile individuare quali variabili vanno ad influenzare in modo significativo il successo o il fallimento sportivo delle singole squadre? \\
Da questo quesito nasce la tesi qui presentata.  L’ obbiettivo è quello di presentare un'analisi che risponda a tale quesito, attraverso l'utilizzo di tecniche di \textit{Data Mining}, in particolare lo sfruttamento di un modello a comparazione a coppie per le partite di calcio che sia in grado di tenere conto delle variabili esplicative specifiche per le partite. Il modello scelto per l’analisi sarà il modello \emph{Bradley-Terry} con le sue estensioni.  Infine, verrà presentata un’applicazione di metodi di \textit{Machine Learning} per la predizione dei risultati delle singole partite. \\
Lo studio prenderà in considerazione i dati relativi alle partite della Serie A italiana della stagione 2021/2022.

TO DO + POSSIBLE ADDITIONS






%\vfill
%
%\selectlanguage{english}
%\pdfbookmark{Abstract}{Abstract}
%\chapter*{Abstract}
%
%\selectlanguage{italian}

\endgroup			

\vfill

