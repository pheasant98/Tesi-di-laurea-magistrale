% !TEX encoding = UTF-8
% !TEX TS-program = pdflatex
% !TEX root = ../tesi.tex

%**************************************************************
% Sommario
%**************************************************************
\cleardoublepage
\phantomsection
\pdfbookmark{Sommario}{Sommario}
\begingroup
\let\clearpage\relax
\let\cleardoublepage\relax
\let\cleardoublepage\relax

\chapter*{Abstract}

Viviamo nell'era dei cosiddetti \emph{Big Data} in cui, grazie all'interconnessione, è possibile ottenere un grande flusso di informazioni da ogni attività. \\
Non fa eccezione il calcio in cui da un paio d'anni, le società calcistiche si affidano a sistemi di analisi per produrre tattiche di gioco ma anche per effettuare \textit{scouting} di giocatori emergenti. Nel calcio moderno, perciò, numerose statistiche ad esempio il possesso della palla, il numero di tiri effettuati da una squadra ecc. vengono raccolte durante una partita di calcio.\\
Questo porta alla domanda: poiché disponiamo di una grande quantità di dati sulle prestazioni delle squadre nelle loro partite, è possibile identificare quali statistiche influiscono significativamente sul successo o sul fallimento sportivo delle singole squadre? \\
Da qui nasce la tesi che verrà presentata. L'obiettivo è quello di fornire un'analisi che risponda a questa domanda utilizzando tecniche di \emph{Data Mining}, in particolare attraverso l'utilizzo di un modello di confronto a coppie per le partite di calcio che tenga conto delle statistiche inserite. Il modello scelto per l'analisi sarà il modello Bradley-Terry con le sue estensioni.
Successivamente i modelli Bradley-Terry saranno utilizzati per predire l’esito delle partite e confrontati con le predizioni dei principali \emph{bookmakers} e degli algoritmi di \emph{Machine Learning}:  K-Nearest-Neighbors (K-NN), Support Vector Machine (SVM), Decision Tree, Random Forest e AdaBoost. Infine, Decision Tree e Random Forest verranno ulteriormente approfonditi per individuare quali statistiche sono importanti.\\
Lo studio prenderà in considerazione i dati relativi alle partite della Serie A italiana della stagione 2021/2022.






%\vfill
%
%\selectlanguage{english}
%\pdfbookmark{Abstract}{Abstract}
%\chapter*{Abstract}
%
%\selectlanguage{italian}

\endgroup			

\vfill

