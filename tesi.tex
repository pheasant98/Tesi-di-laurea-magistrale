        %%******************************************%%
%%                                          %%
%%        Modello di tesi di laurea         %%
%%            di Andrea Giraldin            %%
%%                                          %%
%%             2 novembre 2012              %%
%%                                          %%
%%******************************************%%


% I seguenti commenti speciali impostano:
% 1. 
% 2. PDFLaTeX come motore di composizione;
% 3. tesi.tex come documento principale;
% 4. il controllo ortografico italiano per l'editor.

% !TEX encoding = UTF-8
% !TEX TS-program = pdflatex
% !TEX root = tesi.tex
% !TEX spellcheck = it-IT

\documentclass[10pt,                    % corpo del font principale
a4paper,                 % carta A4
twoside,                 % impagina per fronte-retro
openright,               % inizio capitoli a destra
english,                                  
]{book}    

%**************************************************************
% Importazione package
%************************************************************** 

%\usepackage{amsmath,amssymb,amsthm}    % matematica

\usepackage[T1]{fontenc}                % codifica dei font:
% NOTA BENE! richiede una distribuzione *completa* di LaTeX

\usepackage[utf8]{inputenc}             % codifica di input; anche [latin1] va bene
% NOTA BENE! va accordata con le preferenze dell'editor

\usepackage[english, italian]{babel}   		 	% per scrivere in italiano e in inglese;
% l'ultima lingua (l'italiano) risulta predefinita

\usepackage{bookmark}                   % segnalibri

\usepackage{caption}                    % didascalie

\usepackage{chngpage,calc}              % centra il frontespizio

\usepackage{csquotes}                   % gestisce automaticamente i caratteri (")

\usepackage{emptypage}                  % pagine vuote senza testatina e piede di pagina

\usepackage{epigraph}					% per epigrafi

\usepackage{eurosym}                    % simbolo dell'euro

%\usepackage{indentfirst}               % rientra il primo paragrafo di ogni sezione

\usepackage{graphicx}                   % immagini

\usepackage{hyperref}                   % collegamenti ipertestuali

\usepackage[binding=5mm]{layaureo}      % margini ottimizzati per l'A4; rilegatura di 5 mm

\usepackage{listings}                   % codici

\usepackage{microtype}                  % microtipografia

\usepackage{mparhack,fixltx2e,relsize}  % finezze tipografiche

\usepackage{nameref}                    % visualizza nome dei riferimenti                                      

\usepackage[font=small]{quoting}        % citazioni

\usepackage{subfig}                     % sottofigure, sottotabelle

\usepackage{comment}

\usepackage[italian]{varioref}          % riferimenti completi della pagina

\usepackage[dvipsnames]{xcolor}         % colori

\usepackage{booktabs, caption}                   % tabelle                                       
\usepackage{tabularx}                   % tabelle di larghezza prefissata                                    
\usepackage{longtable}                  % tabelle su più pagine      
\usepackage{ltxtable}                   % tabelle su più pagine e adattabili in larghezza
\usepackage{multirow}
\usepackage{rotating}
\usepackage{adjustbox}
\usepackage{tablefootnote}
\usepackage{threeparttablex}
\usepackage{etoolbox}
\usepackage{lmodern}
\usepackage{array}
\usepackage{colortbl}
\usepackage{verbatim}
\usepackage[explicit]{titlesec}
\usepackage{lipsum}% just to generate text
\usepackage{fancyhdr}
\usepackage{dsfont}
\usepackage{amsmath}
\usepackage{pdflscape}
\usepackage{ragged2e}

%\usepackage[square,sort,comma,numbers]{natbib}
%*******************Per gestire un'immagine in backgorund*********************************
\usepackage{graphicx}
\usepackage{watermark}
\usepackage{transparent}
%*****************************************************************************************
%\usepackage{fourier}


\usepackage[toc, acronym]{glossaries}   % glossario
% per includerlo nel documento bisogna:
% 1. compilare una prima volta tesi.tex;
% 2. eseguire: makeindex -s tesi.ist -t tesi.glg -o tesi.gls tesi.glo
% 3. eseguire: makeindex -s tesi.ist -t tesi.alg -o tesi.acr tesi.acn
% 4. compilare due volte tesi.tex.

\usepackage[backend=biber,citestyle=numeric,style=authoryear,hyperref,backref=false]{biblatex}
% eccellente pacchetto per la bibliografia; 
% produce uno stile di citazione autore-anno; 
% lo stile "numeric-comp" produce riferimenti numerici
% per includerlo nel documento bisogna:
% 1. compilare una prima volta tesi.tex;
% 2. eseguire: biber tesi
% 3. compilare ancora tesi.tex.
%style=   authoryear

\DeclareLabeldate[online]{
	\field{date}
	\field{year}
	\field{eventdate}
	\field{origdate}
	\field{urldate}
}                                        


\DefineBibliographyStrings{english}{
	andothers = {\mkbibemph{et\addabbrvspace al\adddot}}
}           

\renewcommand*{\nameyeardelim}{\addcomma\space}

%\makeatletter
%\input{numeric.bbx}
%\makeatother

\renewbibmacro{in:}{}
\renewbibmacro*{volume+number+eid}{%
	\printfield{volume}%
	%  \setunit*{\adddot}% DELETED
	\setunit*{\addnbspace}% NEW (optional); there's also \addnbthinspace
	\printfield{number}%
	\setunit{\addcomma\space}%
	\printfield{eid}}                                    

\usepackage{amsmath}
\usepackage{amsfonts} 
\usepackage{bm}

\usepackage{url}
\makeatletter
\g@addto@macro{\UrlBreaks}{\UrlOrds}
\makeatother

\def\hat{\mathaccent "705E\relax}

%**************************************************************
% file contenente le impostazioni della tesi
%**************************************************************

%**************************************************************
% Frontespizio
%**************************************************************

% Autore
\newcommand{\myName}{Federico Perin}                                    
\newcommand{\myTitle}{Modellazione di un modello Bradley-Terry per l'individuazione delle variabili significative per l'esito di una partita di calcio nella Seria A italiana }

% Tipo di tesi                   
\newcommand{\myDegree}{Tesi di laurea magistrale}

% Università             
\newcommand{\myUni}{Università degli Studi di Padova}

% Facoltà       
\newcommand{\myFaculty}{Corso di Laurea Magistrale in Informatica}

% Dipartimento
\newcommand{\myDepartment}{Dipartimento di Matematica "Tullio Levi-Civita"}

% Titolo del relatore
\newcommand{\profTitle}{Prof.}

% Relatore
\newcommand{\myProf}{ Annamaria Guolo}

% Luogo
\newcommand{\myLocation}{Padova}

% Anno accademico
\newcommand{\myAA}{2022-2023}

% Data discussione
\newcommand{\myTime}{Febbraio 2023}

\newcommand{\paginavuota}{\newpage\null\thispagestyle{empty}}

%**************************************************************
% Impostazioni di impaginazione
% see: http://wwwcdf.pd.infn.it/AppuntiLinux/a2547.htm
%**************************************************************

\setlength{\parindent}{0pt}   % larghezza rientro della prima riga
\setlength{\parskip}{0pt}   % distanza tra i paragrafi


%**************************************************************
% Impostazioni di biblatex
%**************************************************************
\bibliography{bibliografia} % database di biblatex 

\defbibheading{bibliography} {
    \cleardoublepage
    \phantomsection 
    \addcontentsline{toc}{chapter}{\bibname}
    \chapter*{\bibname\markboth{\bibname}{\bibname}}
}

\setlength\bibitemsep{1.5\itemsep} % spazio tra entry

\DeclareBibliographyCategory{opere}
\DeclareBibliographyCategory{web}

%\addtocategory{opere}{womak:lean-thinking}
\addtocategory{web}{site:agile-manifesto}
\addtocategory{web}{site:css}
\addtocategory{web}{site:html}
\addtocategory{web}{site:typescript}
\addtocategory{web}{site:webstorm}
\addtocategory{web}{site:jiras}
\addtocategory{web}{site:confluence}
\addtocategory{web}{site:selenium}
\addtocategory{web}{site:appium}
\addtocategory{web}{site:protractor}
\addtocategory{web}{site:cucumber}
\addtocategory{web}{site:handlebars}
\addtocategory{web}{site:gitlab}
\addtocategory{web}{site:angula}
\addtocategory{web}{site:ionic}
\addtocategory{web}{site:cordova}
\addtocategory{web}{site:webstorm}

\defbibheading{opere}{\section*{Riferimenti bibliografici}}
\defbibheading{web}{\section*{Siti Web consultati}}


%**************************************************************
% Impostazioni di caption
%**************************************************************
\captionsetup{
    tableposition=top,
    figureposition=bottom,
    font=small,
    format=hang,
    labelfont=bf
}

%**************************************************************
% Impostazioni di glossaries
%**************************************************************

%**************************************************************
% Acronimi
%**************************************************************
\renewcommand{\acronymname}{Acronimi e abbreviazioni}

\newacronym[description={\glslink{apig}{\textit{Application Program Interface}}}]
    {api}{API}{Application Program Interface}
    
\newacronym[description={\textit{Advanced Workforce Management System}}]
	{AWMS}{AWMS}{Advanced Workforce Management System}

\newacronym[description={\glslink{umlg}{\textit{Unified Modeling Language}}}]
    {uml}{UML}{Unified Modeling Language}
    
\newacronym[description={\textit{Test End to End}}]
	{test e2e}{E2E}{test End to End}

\newacronym[description={\textit{Internazionalizzazione}}]
	{i18n}{i18n}{internazionalizzazione}

\newacronym[description={\glslink{JSONg}{\textit{JavaScript Object Notation}}}]
	{JSON}{JSON}{JavaScript Object Notation}
	
\newacronym[description={\glslink{restg}{\textit{Representational State Transfer}}}]
	{rest}{REST}{Representational State Transfer}	

\newacronym[description={\glslink{httpg}{\textit{Hyper Text Transfer Protocol}}}]
{http}{HTTP}{Hyper Text Transfer Protocol}	
	
\newacronym[description={\textit{Hyper Text Markup Language}}]
	{HTML}{HTML}{HyperText Markup Language}	
	
\newacronym[description={\textit{Cascading Style Sheets}}]
	{CSS}{CSS}{Cascading Style Sheets}
	
\newacronym[description={\textit{Syntactically Awesome StyleSheets}}]
	{Sass}{Sass}{Syntactically Awesome StyleSheets}
	
\newacronym[description={\textit{Hypertext Preprocessor}}]
	{PHP}{PHP}{Hypertext Preprocessor}	
	
\newacronym[description={\glslink{urlg}{\textit{Uniform Resource Locator}}}]
	{url}{URL}{Uniform Resource Locator}	
			
\newacronym[description={\glslink{httpsg}{\textit{Hyper Text Transfer Protocol over Secure Socket Layer}}}]
	{https}{HTTPS}{Hyper Text Transfer Protocol over Secure Socket Layer}	
		
\newacronym[description={\glslink{dbmsg}{\textit{Database Management System}}}]
	{DBMS}{DBMS}{Database Management System}

\newacronym[description={\glslink{domg}{\textit{Document Object Model}}}]
	{DOM}{DOM}{Document Object Model}

\newacronym[description={\textit{World Wide Web Consortium}}]
	{W3C}{W3C}{World Wide Web Consortium}

\newacronym[description={\glslink{gdprg}{\textit{General Data Protection Regulation}}}]
	{GDPR}{GDPR}{General Data Protection Regulation}	

\newacronym[description={\glslink{sdkg}{\textit{Software Development Kit}}}]
	{sdk}{SDK}{Software Development Kit}

\newacronym[description={\textit{eXtensible Markup Language}}]
	{XML}{XML}{eXtensible Markup Language}	

\newacronym[description={\textit{eXtensible Hyper Text Markup Language}}]
	{XHTML}{XHTML}{eXtensible Hyper Text Markup Language}
	
\newacronym[description={\textit{behavior-driven development}}]
	{BDD}{BDD}{behavior-driven development}
%**************************************************************
% Glossario
%**************************************************************
\renewcommand{\glossaryname}{Glossario}

\newglossaryentry{apig}
{
	name=\glslink{api}{\textit{API}},
	text=Application Program Interface,
	sort=api,
	description={In informatica con il termine \emph{Application Programming Interface (API)} (ing. interfaccia di programmazione di un'applicazione) si indica ogni insieme di procedure disponibili al programmatore, di solito raggruppate a formare un set di strumenti specifici per l'espletamento di un determinato compito all'interno di un certo programma. La finalità è ottenere un'astrazione, di solito tra l'\emph{hardware} e il programmatore o tra \emph{software} a basso e quello ad alto livello semplificando così il lavoro di programmazione}
}

\newglossaryentry{umlg}
{
	name=\glslink{uml}{\textit{UML}},
	text=UML,
	sort=uml,
	description={in ingegneria del software \emph{UML, Unified Modeling Language} (ing. linguaggio di modellazione unificato) è un linguaggio di modellazione e specifica basato sul paradigma object-oriented. L'\emph{UML} svolge un'importantissima funzione di ``lingua franca'' nella comunità della progettazione e programmazione a oggetti. Gran parte della letteratura di settore usa tale linguaggio per descrivere soluzioni analitiche e progettuali in modo sintetico e comprensibile a un vasto pubblico}
}

\newglossaryentry{Electrolux}
{
	name={\textit{Electrolux}},
	text=Electrolux,
	sort=electrolux,
	description={Electrolux è una multinazionale svedese produttrice di elettrodomestici con sede a Stoccolma}
}

\newglossaryentry{machine learning}
{
	name={\textit{Machine learning}},
	text=machine learning,
	sort=machine learning,
	description={Nell'ambito dell'informatica, (ing.apprendimento automatico) l'apprendimento automatico è una variante alla programmazione tradizionale nella quale in una macchina si predispone l'abilità di apprendere qualcosa dai dati in maniera autonoma, senza istruzioni esplicite}
}

\newglossaryentry{plant manager}
{
	name={\textit{Plant manager}},
	text=plant manager,
	sort=plant manager,
	description={È colui (ing.responsabile di stabilimento) che presiede e organizza le operazioni quotidiane degli impianti di produzione aziendali, di cui deve assicurare il funzionamento ottimale ed efficiente. Si occupa dei lavoratori, assegnando funzioni e ruoli, definendo orari di lavoro e produzione. Raccoglie e analizza i dati di produzione per trovare eventuali spazi di miglioramento. Si occupa della sicurezza dei lavoratori e quella degli impianti inoltre, monitora le apparecchiature di produzione e, in caso di necessità, della loro riparazione o sostituzione}
}

\newglossaryentry{bot}
{
	name={\textit{Bot}},
	text=bot,
	sort=bot,
	description={È un \emph{software} progettato per simulare una conversazione con un essere umano. Lo scopo principale di questi \emph{software} è quello di simulare un comportamento umano e sono a volte definiti anche agenti intelligenti e vengono usati per vari scopi come la guida in linea, per rispondere alle FAQ degli utenti che accedono a un sito. In alcuni utilizzano sofisticati sistemi di elaborazione del linguaggio naturale, ma molti si limitano a eseguire la scansione delle parole chiave nella finestra di input e fornire una risposta con le parole chiave più corrispondenti}
}

\newglossaryentry{QR code}
{
	name={\textit{QR code}},
	text=QR code,
	sort=qr-code,
	description={È un codice a barre bidimensionale (o codice 2D), ossia a matrice, composto da moduli neri disposti all'interno di uno schema bianco di forma quadrata, impiegato tipicamente per memorizzare informazioni generalmente destinate a essere lette tramite uno \emph{smartphone}}
}

\newglossaryentry{WebSocket}
{
	name={\textit{WebSocket}},
	text=WebSocket,
	sort=websocket,
	description={È una tecnologia web che fornisce canali di comunicazione a due direzioni cioè gli interlecutori possono sia inviare sia ricevere contemporaneamente attraverso una singola connessione TCP}
}

\newglossaryentry{SCRUM}
{
	name={\textit{SCRUM}},
	text=SCRUM,
	sort=scrum,
	description={È un \emph{framework} agile per la gestione del ciclo di sviluppo del \emph{software}, iterativo ed incrementale, concepito per gestire progetti e prodotti software o applicazioni di sviluppo. Nel proprio manifesto prevede i seguenti punti che lo caratterizzano, le persone e le interazioni sono più importanti dei processi e degli strumenti, meglio avere da subito \emph{software} funzionante che documentazione ampia, meglio una collaborazione con il cliente piuttosto che fare una negoziazione del contratto, essere in grando di rispondere ai cambiamenti piuttosto che rispettare un piano. 
		I progetti Scrum progrediscono attraverso una serie di sprint che hanno una durata massima di un mese. Negli sprint vengono decisi quali requisiti devono essere soddisfatti, e quindi successivamente, progettati, implementati e testati}
}

\newglossaryentry{framework}{
	name={\textit{Framework}},
	text=framework,
	sort=framework,
	description={In informatica con il termine framework 
		si indica un insieme di elementi \emph{software} che un programmatore può usare o modificare per realizzare un programma. Rappresenta un’astrazione composta da elementi universali e riutilizzabili con lo scopo di facilitare lo sviluppo di un programma e di far applicare buone norme di programmazione. Inoltre, un framework può offrire programmi di supporto, librerie, compilatori e documentazione per l'utilizzo}
}

\newglossaryentry{iOS}{
	name={\textit{iOS}},
	text=iOS,
	sort=iOS,
	description={È un sistema operativo \emph{mobile} sviluppato da Apple per iPhone, iPod touch e iPad. Le versioni principali di iOS vengono distribuite ogni anno. L'attuale versione, iOS 13, è stata distribuita al pubblico il 19 settembre 2019}
}

\newglossaryentry{Android}{
	name={\textit{Android}},
	text=Android,
	sort=Android,
	description={È un sistema operativo per dispositivi \emph{mobile} sviluppato da Google e basato sul kernel Linux, progettato principalmente per \emph{smartphone} e \emph{tablet}, interfacce utente specializzate per televisori (Android TV), automobili (Android Auto), orologi da polso (Wear OS), occhiali (Google Glass). L'attuale ultima versione è Android 11}
}

\newglossaryentry{architettura}{
	name={\textit{Architettura}},
	text=architettura,
	sort=architettura,
	description={In informatica con il termine architettura, in questo caso intesa come architettura \emph{software}, è l'organizzazione fondamentale di un sistema, definita dai suoi componenti, dalle relazioni reciproche tra i componenti e con l'ambiente, e i principi che ne governano la progettazione e l'evoluzione}
}

\newglossaryentry{linguaggio di markup}{
	name={\textit{Linguaggio di markup}},
	text=linguaggio di markup,
	sort=linguaggio di markup,
	description={In informatica con il termine linguaggio di markup, si intende un gruppo di regole detti marcatori, attraverso le quali vengono descritti i meccanismi di rappresentazione di un testo}
}

\newglossaryentry{browser web}{
	name={\textit{Browser web}},
	text=browser web,
	sort=browser web,
	description={In informatica si intende un'applicazione per l'acquisizione, la presentazione e la navigazione di risorse sul web. Permette la visualizzazione dei contenuti ipertestuali, e la riproduzione di contenuti multimediali. Tra i browser più popolari vi sono Google Chrome, Internet Explorer, Mozilla Firefox, Microsoft Edge, Safari, Opera}
}

\newglossaryentry{applicazione nativa}{
	name={\textit{Applicazione nativa}},
	text=applicazione nativa,
	sort=applicazione nativa,
	description={In informatica si intende un'applicazione scritta e compilata per una specifica piattaforma utilizzando i linguaggi di programmazione e librerie supportati dal particolare sistema operativo \emph{mobile}}
}

\newglossaryentry{applicazione web mobile}{
	name={\textit{Applicazione web mobile}},
	text=applicazione web mobile,
	sort=applicazione web mobile,
	description={In informatica si intende pagine web ottimizzate per dispositivi \emph{mobile} scritte utilizzando tecnologie web, in particolare \gls{HTML}, JavaScript e \gls{CSS}. Inoltre, le applicazioni web non possono accedere alle funzionalità del dispositivo ad'esempio la fotocamera}
}

\newglossaryentry{applicazione ibrida}{ %va a benzina ma anche con l'elettricità
	name={\textit{Applicazione ibrida}},
	text=applicazione ibrida,
	sort=applicazione ibrida,
	description={In informatica si intende applicazioni sviluppate con tecnologie web e vengono eseguite localmente all’interno di un’applicazione nativa. Grazie a ciò possono interagire con il dispositivo ad'esempio utilizzare la fotocamera}
}

\newglossaryentry{notifica push}{ 
	name={\textit{Notifica push}},
	text=notifica push,
	sort=notifica push,
	description={In informatica si intende un tipo di messaggistica istantanea grazie alla quale il messaggio perviene al destinatario senza che questo debba effettuare un'operazione di scaricamento. Tale modalità è quella tipicamente usata da applicazioni come \emph{WhatsApp} o da servizi di sistemi operativi come \g{Android}, oppure da numerose applicazioni derivate da siti web come, ad esempio, il servizio meteo o quello delle notizie}
}

\newglossaryentry{licenza MIT}{ 
	name={\textit{Licenza MIT}},
	text=licenza MIT,
	sort=licenza MIT,
	description={La Licenza MIT è una licenza di \emph{software} libero. È una licenza permissiva, cioè permette il riutilizzo nel \emph{software} proprietario sotto la condizione che la licenza sia distribuita con tale \emph{software}}
}

\newglossaryentry{JSONg}{ 
	name=\glslink{JSON}{\textit{JSON}},
	text=JavaScript Object Notation,
	sort=JSON,
	description={In informatica con il termine \emph{JavaScript Object Notation (JSON)}(ing.Notazione degli oggetti JavaScript) si intende un formato testuale standard, usato per rappresentare dati strutturati basati sulla sintassi degli oggetti in JavaScript. È comunemente utilizzato per l'interscambio di dati fra applicazioni client/server. Risulta essere facile da comprendere e da scrivere per le persone mentre per le macchine risulta essere un formato leggero e veloce da analizzare}
}

\newglossaryentry{pooling}{ 
	name={\textit{Pooling}},
	text=pooling,
	sort=pooling,
	description={In informatica con il termine pooling si intende una procedura attraverso la quale periodicamente viene eseguita una operazione. Nel caso delle comunicazioni tra \g{client} e \g{server} il pooling è la richiesta periodica del \g{client} di dati al \g{server} per controllare sei i dati che ha sono aggiornati}
}

\newglossaryentry{client}{ 
	name={\textit{Client}},
	text=client,
	sort=client,
	description={In informatica con il termine client si intende un entita presente in un rete di comunicazione che accede ai servizi o alle risorse messe a dispozione da un'altra componente detta \g{server}, la cui comunicazione tra client e \g{server} è regolata da insieme di regole e norme detti protocolli di comunicazione. Insieme al \g{server} forma l'architettura client/server}
}

\newglossaryentry{server}{ 
	name={\textit{Server}},
	text=server,
	sort=server,
	description={In informatica con il termine client si intende un entita presente in un rete di comunicazione che offre dei servizi o dalle risorse a un'altri componenti presenti nella rete detti \g{client}, la cui comunicazione tra \g{client} e server è regolata da insieme di regole e norme detti protocolli di comunicazione. Insieme al \g{client} forma l'architettura client/server}
}

\newglossaryentry{httpg}{ 
	name=\glslink{http}{\textit{HTTP}},
	text=Hyper Text Transfer Protocol,
	sort=http,
	description={In informatica con il termine \emph{Hyper Text Transfer Protocol (HTTP)} (ing. protocollo di trasferimento di un ipertesto) si intende un insieme di regole e norme che regolano la trasmissione e la comunicazione d'informazione nella rete Internet. Questo trasmissione d'informazioni avviene sotto forma di scambi di messaggi tipicamente tra il client che puo essere un \g{browser web}, e un server}
}

\newglossaryentry{open-source}{ 
	name={\textit{Open-source}},
	text=open-source,
	sort=open-source,
	description={In informatica con il termine open-source (ing. sorgente libero) si intende un \emph{software} per cui chi lo ha sviluppato rinuncia alla propretà del \emph{software} dando libero accesso a tutto il codice sorgente a chiunque, e quindi è permesso a tutti di contribuire nello sviluppo del codice al fine di migliorarlo, aggiungere nuove funzionalità o  correggere errori all'interno del codice}
}

\newglossaryentry{restg}{ 
	name=\glslink{rest}{\textit{REST}},
	text=Representational State Transfer,
	sort=rest,
	description={In informatica con il termine \emph{Representational State Transfer (REST)} (ing. trasferimento di stato rappresentativo) si intende un approccio architetturale alla creazione di web \g{api} basato sul protocollo di comunicazione \g{http}. Viene imposto che le \g{api} devono permettere di accedere a delle risorse attraverso un \g{url}, utilizzare il formato \gls{JSON} e \gls{XML}, non avere uno stato cioè non deve essere memorizzato cioè che è stato fatto e infine, utilizzare i metodi del \g{http}, GET, POST, PUT, DELETE}
}

\newglossaryentry{urlg}{ 
	name=\glslink{url}{\textit{URL}},
	text=Uniform Resource Locator,
	sort=url,
	description={In informatica con il termine \emph{Uniform Resource Locator (URL)} (ing. localizzare di risorse uniformi) in intede una sequenza di caratteri che identifica univocamente l'indirizzo di una risorsa su una rete di computer, come ad esempio un documento, un'immagine, un video, tipicamente presente su un \g{server} e resa accessibile a un \g{client}}
}

\newglossaryentry{front-end}{ 
	name={\textit{Front-end}},
	text=front-end,
	sort=front-end,
	description={In informatica con il termine front-end si intende la parte visibile all'utente di un programma e con cui egli può interagire solitamente è un'interfaccia utente. Perciò il front-end è la parte di un sistema \emph{software} che gestisce l'interazione con l'utente, ricevendo da esso un input da cui viene prodotto (dal \g{back-end}) un output da mostrare all'utente}
}

\newglossaryentry{back-end}{ 
	name={\textit{Back-end}},
	text=back-end,
	sort=back-end,
	description={In informatica con il termine back-end si intende la parte che si occuppa di ricevere in input dati inseriti dall'utenti e di eleborarli per rispondere alle richieste dell'utente. Dopo l'elaborazione dei dati il back-end produce un risultato che sarà compito del \g{front-end} mostrarlo}
}

\newglossaryentry{httpsg}{ 
	name=\glslink{https}{\textit{HTTPS}},
	text=Hyper Text Transfer Protocol over Secure Socket Layer,
	sort=https,
	description={In informatica con il termine \emph{Hyper Text Transfer Protocol over Secure Socket Layer (HTTPS)} (ing. protocollo di trasferimento di un ipertesto basato su un strato di sicurrezza) si intende un insieme di regole e norme che regolano la trasmissione e la comunicazione d'informazione nella rete Internet in modo sicuro cioè il contenuto della trasmissione non è interpretabile da entità diverse dal mittente o dal/dai destinataro/i.
		Per la comunicazione viene utilizzato il protocollo \g{http} all'interno di una connessione criptata dal protocollo \emph{Secure Sockets Layer (SSL)} garantendo così riservatezza dei dati cioè il contenuto della tramissione e visibile solo al mittente e al destinatario, integrità dei dati cioè il contenuto della trasmessione non viene alterato e autenticazione di comunica}
}

\newglossaryentry{database}{ 
	name={\textit{Database}},
	text=database,
	sort=database,
	description={In informatica con il termine database (ing. base di dati) si intende una collezione di dati ben organizzati e ben strutturati, gestiti in modo integrato da un sistema per la gestione delle basi di dati, costituiscono una base di lavoro per utenti diversi con programmi diversi. I prodotti \emph{software} per la gestione dei database sono indicati con il termine \g{DBMS}}
}

\newglossaryentry{dbmsg}{ 
	name=\glslink{DBMS}{\textit{DBMS}},
	text=Database Management System,
	sort=dbms,
	description={In informatica con il termine \emph{Database Management System (DBMS)} (ing. sistema di gestione di basi di dati) in intede un sistema \emph{software} progettato per consentire la creazione, la manipolazione e l'interrogazione efficiente di \g{database}}
}

\newglossaryentry{firebase}{ 
	name={\textit{Firebase}},
	text=firebase,
	sort=firebase,
	description={È la piattaforma \emph{mobile} di Google che aiuta nello sviluppare applicazione \emph{mobile}. Firebase offre tutto ciò che dovrebbe offrire un \g{back-end} quindi, funzionalità di autenticazione, un \g{database} (quello di Firebase è di tipo NoSQL), servizi di \emph{hosting} e algoritmi di \g{machine learning} per l'apprendimento automatico}
}

\newglossaryentry{foreground}{ 
	name={\textit{Esecuzione in foreground}},
	text=foreground,
	sort=foreground,
	description={In informatica con il termine foreground (ing. primo piano) si intende l'esecuzione dei processi di un \emph{software} dove può essere richiesta l'interazione dell'utente ma che comunque l'utente sa dell'esecuzioni di tali processi. Nel caso di un'applicazione \emph{mobile} significa che nello schermo viene visualizzata l'applicazione in esecuzione con la quale l'utente può interagire}
}

\newglossaryentry{background}{ 
	name={\textit{Esecusione in background}},
	text=background,
	sort=background,
	description={In informatica con il termine background, si intede l'esecuzione dei processi di un \emph{software} dove non viene richiesto l'intervento dell'utente, tanto da non essere a lui visibile tale esecuzione. Nel caso di un'applicazione \emph{mobile} significa che nello schermo non viene visualizzata l'applicazione in esecuzione. Resta comunque attiva ma non interagibile con l'utente finché è in background}
}

\newglossaryentry{base64}{ 
	name={\textit{Base64}},
	text=base64,
	sort=base64,
	description={In informatica con il termine base64, si intede un sistema di codifica che consente la traduzione di dati binari in stringhe di testo ASCII cioè un insieme di codici per la codifica dei caratteri. I dati vengono rappresentati sulla base di sessantaquattro caratteri ASCII diversi}
}

\newglossaryentry{domg}{ 
	name=\glslink{DOM}{\textit{DOM}},
	text=Document Object Model,
	sort=dom,
	description={In informatica con il termine \emph{Document Object Model (DOM)} (ing. modello a oggetti del documento) in intede una forma di rappresentazione dei documenti strutturati in modo gerarchico. È lo standard ufficiale del \gls{W3C} per la rappresentazione di documenti strutturati in maniera da essere neutrali sia per la lingua che per la piattaforma}
}

\newglossaryentry{gdprg}{ 
	name=\glslink{GDPR}{\textit{GDPR}},
	text=General Data Protection Regulation,
	sort=gdpr,
	description={Per \emph{General Data Protection Regulation (GDPR)} (ing. Regolamento generale sulla protezione dei dati) in intede un regolamento dell'Unione europea in materia di trattamento dei dati personali e di \emph{privacy}, adottato il 27 aprile 2016, pubblicato sulla Gazzetta ufficiale dell'Unione europea il 4 maggio 2016 ed entrato in vigore il 24 maggio dello stesso anno ed operativo a partire dal 25 maggio 2018. Il testo affronta anche il tema dell'esportazione di dati personali al di fuori dell'UE e obbliga tutti i titolari del trattamento dei dati (anche con sede legale fuori dall'UE) che trattano dati di residenti nell'UE ad osservare e adempiere agli obblighi previsti.}
}

\newglossaryentry{sdkg}
{
	name=\glslink{sdk}{\textit{SDK}},
	text=Software Development Kit,
	sort=sdk,
	description={in informatica con il termine \emph{Software Development Kit (SDK)} (ing. pacchetto di sviluppo per \textit{software}) si intende una collezione di strumenti per lo sviluppo \emph{software} contenuti all'interno di un singolo pacchetto installabile all'interno del proprio sistema. Tutto ciò viene offerto per facilitare la creazione di applicazioni. Questi strumenti solitamente sono specifici per il particolare tipo di \emph{hardware}, sistema operativo e linguaggio di programmazione utilizzati per lo sviluppo \emph{software}}
}

\newglossaryentry{mock}{ 
	name={\textit{Mock}},
	text=mock,
	sort=mock,
	description={In informatica con il termine mock, si intede un oggetto che cerca di ripordure il comportamento di un oggetto reale in modo controllato, con l'obbiettivo di testare il comportamento di altri oggetti reali che dipendono dall'oggetto che si sta simulando con il mock}
}

\newglossaryentry{design pattern}{ 
	name={\textit{Design pattern}},
	text=design pattern,
	sort=design pattern,
	description={In informatica e specialmente nell'ambito dell'Ingegneria del Software con il termine design pattern, si intede di una descrizione o modello logico da applicare per la risoluzione di un problema che può presentarsi in diverse situazioni durante le fasi di progettazione e sviluppo del \emph{software}, ancor prima della definizione dell'algoritmo risolutivo della parte computazionale. È un approccio spesso efficace nel contenere o ridurre i costi per lo sviluppo del \emph{software}}
}


\begin{comment}
\newglossaryentry{e2eg}
{
	name=\glslink{test e2e}{E2E},
	text=Test End to End,
	sort=test End to End,
	description={Con il termine test end-to-end (end-to-end testing) si intende quell’attività di testing dell’interfaccia grafica vista dagli utenti del programma dall’inizio fino alla fine. In altre parole rappresenta una metodologia utilizzata per verificare se il flusso di un’applicazione si sta comportando come progettato dall’inizio fino alla fine senza che vengano rilevati dei errori che andrebbero a inficiare sulla qualità dell’applicazione stessa}
}
\newglossaryentry{AWMSg}
{
	name=\glslink{AWMS}{AWMS},
	text=Advanced Workforce Management System,
	sort=AWMS,
	description={È una soluzione software che utilizza algoritmi di \gls{machine learning}, per risolvere uno dei problemi cardine di un \gls{plant manager} ovvero, la pianificazione ottimale della forza lavoro che ha disposizione. L'obbiettivo principale della soluzione è quello di pianificare la persona giusta al posto giusto in base alle competenze tecniche possedute del lavoratore}
}

Parole da aggiungere


\end{comment}
 % database di termini
\makeglossaries


%**************************************************************
% Impostazioni di graphicx
%**************************************************************
\graphicspath{{immagini/}} % cartella dove sono riposte le immagini


%**************************************************************
% Impostazioni di hyperref
%**************************************************************
\hypersetup{
    %hyperfootnotes=false,
    %pdfpagelabels,
    %draft,	% = elimina tutti i link (utile per stampe in bianco e nero)
    %colorlinks=true,
    %linktocpage=true,
    pdfstartpage=1,
    pdfstartview=FitV,
    % decommenta la riga seguente per avere link in nero (per esempio per la stampa in bianco e nero)
    colorlinks=true, linktocpage=false, pdfborder={0 0 0}, pdfstartpage=1, pdfstartview=FitV,
    breaklinks=true,
    pdfpagemode=UseNone,
    pageanchor=true,
    pdfpagemode=UseOutlines,
    plainpages=false,
    bookmarksnumbered,
    bookmarksopen=true,
    bookmarksopenlevel=1,
    hypertexnames=true,
    pdfhighlight=/O,
    %nesting=true,
    %frenchlinks,
    urlcolor=SchoolColor,
    linkcolor=black,
    citecolor=webgreen,
    %urlcolor=Black, linkcolor=Black, citecolor=Black, %pagecolor=Black,
    pdftitle={\myTitle},
    pdfauthor={\textcopyright\ \myName, \myUni, \myFaculty},
    pdfsubject={},
    pdfkeywords={},
    pdfcreator={pdfLaTeX},
    pdfproducer={LaTeX}
}

%**************************************************************
% Impostazioni di itemize
%**************************************************************
\renewcommand{\labelitemi}{$\ast$}

%\renewcommand{\labelitemi}{$\bullet$}
%\renewcommand{\labelitemii}{$\cdot$}
%\renewcommand{\labelitemiii}{$\diamond$}
%\renewcommand{\labelitemiv}{$\ast$}


%**************************************************************
% Impostazioni di listings
%**************************************************************
\lstset{
	language=[LaTeX]Tex,%C++,
	keywordstyle=\color{RoyalBlue}, %\bfseries,
	basicstyle=\small\ttfamily,
	%identifierstyle=\color{NavyBlue},
	commentstyle=\color{Green}\ttfamily,
	stringstyle=\rmfamily,
	numbers=none, %left,%
	numberstyle=\scriptsize, %\tiny
	stepnumber=5,
	numbersep=8pt,
	showstringspaces=false,
	breaklines=true,
	frameround=ftff,
	frame=single
} 


%**************************************************************
% Impostazioni di xcolor
%**************************************************************
\definecolor{webgreen}{rgb}{0,.5,0}
\definecolor{webbrown}{rgb}{.6,0,0}
\definecolor{SchoolColor}{rgb}{0.71, 0, 0.106}
%%Colori per le tabelle
\definecolor{grigetto}{RGB}{234, 234, 234}
\definecolor{rossoep}{RGB}{164,60,59}
\definecolor{darkblue}{RGB}{59,77,95}
\definecolor{heavenly}{RGB}{74,199,253}
\definecolor{giallo}{RGB}{251,168,11}
\definecolor{verde}{RGB}{87,180,0}
\definecolor{grigio}{gray}{.7}
\definecolor{grigioChiaro}{gray}{.9}

\newcommand{\classdesc}[2]{\item[\textbf{#1:}] #2}
\newcommand{\intest}[1]{\multicolumn{1}{>{\columncolor{rossoep}}c}{\textbf{#1}}}
\newcolumntype{Z}{>{\centering\arraybackslash}X}
\renewcommand{\tabularxcolumn}[1]{>{\arraybackslash}m{#1}}

%**************************************************************
% Altro
%**************************************************************

\newcommand{\omissis}{[\dots\negthinspace]} % produce [...]

% eccezioni all'algoritmo di sillabazione
\hyphenation
{
    ma-cro-istru-zio-ne
    gi-ral-din
}

\newcommand{\sectionname}{sezione}
\addto\captionsitalian{\renewcommand{\figurename}{Figura}
                       \renewcommand{\tablename}{Tabella}}

\newcommand{\glsfirstoccur}{\ap{{[g]}}}
\newcommand{\g}[1]{\gls{#1}\textcolor{SchoolColor}{\ap{[g]}}} %parola da glossario
\newcommand{\acr}[1]{\texttt{\gls{#1}}}
\newcommand{\intro}[1]{\emph{\textsf{#1}}}

%**************************************************************
% Environment per ``rischi''
%**************************************************************
\newcounter{riskcounter}                % define a counter
\setcounter{riskcounter}{0}             % set the counter to some initial value

%%%% Parameters
% #1: Title
\newenvironment{risk}[1]{
    \refstepcounter{riskcounter}        % increment counter
    \par \noindent                      % start new paragraph
    \textbf{\arabic{riskcounter} #1}   % display the title before the 
                                        % content of the environment is displayed 
}{
    \par\medskip
}

\newcommand{\riskname}{Rischio}

\newcommand{\riskdescription}[1]{\textbf{\\Descrizione:} #1}

\newcommand{\risksolution}[1]{\textbf{\\Soluzione:} #1}

%**************************************************************
% Environment per ``use case''
%**************************************************************
\newcounter{usecasecounter}             % define a counter
\setcounter{usecasecounter}{0}          % set the counter to some initial value

%%%% Parameters
% #1: ID
% #2: Nome
\newenvironment{usecase}[2]{
    \renewcommand{\theusecasecounter}{\usecasename #1}  % this is where the display of 
                                                        % the counter is overwritten/modified
    \refstepcounter{usecasecounter}             % increment counter
    \vspace{10pt}
    \par \noindent                              % start new paragraph
    {\large \textbf{\usecasename #1: #2}}       % display the title before the 
                                                % content of the environment is displayed 
    \medskip
}{
    \medskip
}
%**************************************USE CASE**********************************************
\newcommand{\usecasename}{UC}

\newcommand{\usecaseactors}[1]{\textbf{\\Attori Principali:} #1. \vspace{4pt}}
\newcommand{\usecasepre}[1]{\textbf{\\Precondizioni:} #1. \vspace{4pt}}
\newcommand{\usecasedesc}[1]{\textbf{\\Scenario Principale:} #1 \vspace{4pt}}
\newcommand{\usecasepost}[1]{\textbf{\\Postcondizioni:} #1. \vspace{4pt}}
\newcommand{\usecasealt}[1]{\textbf{\\Scenario Alternativo:} #1. \vspace{4pt}}
\newcommand{\usecaseest}[1]{\textbf{\\Estensioni:} #1 \vspace{4pt}}
\newcommand{\usecaseflow}[1]{\textbf{\\Flusso di Eventi:} #1 \vspace{4pt}}
\newcommand{\usecasegen}[1]{\textbf{\\Generalizzazione:} #1 \vspace{4pt}}

%**************************************************************
% Environment per ``namespace description''
%**************************************************************

\newenvironment{namespacedesc}{
    \vspace{10pt}
    \par \noindent                              % start new paragraph
    \begin{description} 
}{
    \end{description}
   % \medskip
}


%*****************************************Chapter Style*************************************

\titleformat{\chapter}[display]
{\normalfont\scshape\Huge}
{\hspace*{-70pt}\textcolor{SchoolColor}{\textbf{\thechapter}} \textbf{\textcolor{SchoolColor}{ |}} ~#1}
{-15pt}
{\hspace*{-110pt}{\color{SchoolColor}\rule{\dimexpr\textwidth+80pt\relax}{3pt}}\Huge}
\titleformat{name=\chapter,numberless}[display]
{\normalfont\scshape\Huge}
{\hspace*{-70pt}#1}
{-15pt}
{\hspace*{-110pt}{\color{SchoolColor}\rule{\dimexpr\textwidth+80pt\relax}{3pt}}\Huge}
\titlespacing*{\chapter}{0pt}{0pt}{30pt}

\makeatletter
\renewcommand\@seccntformat[1]{\color{SchoolColor} {\csname the#1\endcsname}\hspace{0.5em}}
\makeatother

%***************************************Tree Directory***************************************


\definecolor{foldercolor}{RGB}{124,166,198}

\tikzset{pics/folder/.style={code={%
			\node[inner sep=0pt, minimum size=#1](-foldericon){};
			\node[folder style, inner sep=0pt, minimum width=0.3*#1, minimum height=0.6*#1, above right, xshift=0.05*#1] at (-foldericon.west){};
			\node[folder style, inner sep=0pt, minimum size=#1] at (-foldericon.center){};}
	},
	pics/folder/.default={20pt},
	folder style/.style={draw=foldercolor!80!black,top color=foldercolor!40,bottom color=foldercolor}
}

\forestset{is file/.style={edge path'/.expanded={%
			([xshift=\forestregister{folder indent}]!u.parent anchor) |- (.child anchor)},
		inner sep=1pt},
	this folder size/.style={edge path'/.expanded={%
			([xshift=\forestregister{folder indent}]!u.parent anchor) |- (.child anchor) pic[solid]{folder=#1}}, inner xsep=0.6*#1},
	folder tree indent/.style={before computing xy={l=#1}},
	folder icons/.style={folder, this folder size=#1, folder tree indent=3*#1},
	folder icons/.default={12pt},
}
%********************************************************************************************

                     % file con le impostazioni personali

\makeatletter
\def\Hy@colorlink#1{\begingroup\fontshape{it}\selectfont}%
\makeatother

% Tables	
\usepackage{color, colortbl}
\definecolor{Green}{rgb}{0.56, 0.93, 0.56}
\definecolor{Gray}{gray}{0.9}
\newcolumntype{C}[1]{>{\centering\let\newline\\\arraybackslash\hspace{0pt}}m{#1}} % Center text inside each cell
\setlength\arrayrulewidth{0.5pt} % Increasing table linewidth
\begin{document}
	%**************************************************************
	% Materiale iniziale
	%**************************************************************
	\frontmatter
	% !TEX encoding = UTF-8
% !TEX TS-program = pdflatex
% !TEX root = ../tesi.tex

%**************************************************************
% Frontespizio 
%**************************************************************

\begin{titlepage}
%\thiswatermark{\centering \transparent{0.1}\put(-275,-810){\includegraphics[scale=1]{logo-unipd}} }
\begin{center}

\begin{LARGE}
\textbf{\myUni}\\
\end{LARGE}

\vspace{10pt}

\begin{Large}
\textsc{\myDepartment}\\
\end{Large}

\vspace{10pt}

\begin{large}
\textsc{\myFaculty}\\
\end{large}

\vspace{30pt}
\begin{figure}[htbp]
\begin{center}
\includegraphics[scale=0.20]{logo-unipd}
\end{center}
\end{figure}
\vspace{11pt} 

\begin{LARGE}
\begin{center}
\textbf{\myTitle}\\
\end{center}
\end{LARGE}

\vspace{20pt} 

\begin{large}
\textsl{\myDegree}\\
\end{large}

\vspace{20pt} 

\begin{large}
\begin{flushleft}
\textit{Relatore}\\ 
\vspace{5pt} 
\profTitle \myProf
\end{flushleft}

\vspace{0pt} 

\begin{flushright}
\textit{Laureando}\\ 
\vspace{5pt} 
\myName
\end{flushright}
\end{large}

\vspace{20pt}

\line(1, 0){338} \\
\begin{normalsize}
\textsc{Anno Accademico \myAA}
\end{normalsize}

\end{center}
\end{titlepage} 
	\input{inizio-fine/colophon}
	%% !TEX encoding = UTF-8
% !TEX TS-program = pdflatex
% !TEX root = ../tesi.tex

%**************************************************************
% Dedica
%**************************************************************
\cleardoublepage
\phantomsection
\thispagestyle{empty}
\pdfbookmark{Dedica}{Dedica}

\vspace*{3cm}

\begin{center}
If something's important enough, you should try. Even if the probable outcome is failure.\\ \medskip
--- Elon Musk  
\end{center}

\medskip

\begin{center}
Dedicato a ...
\end{center}

	% !TEX encoding = UTF-8
% !TEX TS-program = pdflatex
% !TEX root = ../tesi.tex

%**************************************************************
% Sommario
%**************************************************************
\cleardoublepage
\phantomsection
\pdfbookmark{Sommario}{Sommario}
\begingroup
\let\clearpage\relax
\let\cleardoublepage\relax
\let\cleardoublepage\relax

\chapter*{Abstract}

Viviamo nell'era dei cosiddetti \emph{Big Data} dove grazie all'interconnessione un grande flusso di informazioni può essere ricavato da ogni possibile attività. \\
Non fa eccezione il calcio in cui da un paio d'anni, le società calcistiche si affidano a sistemi di analisi per produrre tattiche di gioco ma anche per effettuare \textit{scouting} di giocatori emergenti. Nel calcio moderno, perciò, numerose statistiche ad esempio il possesso della palla, il numero di tiri effettuati da una squadra ecc. vengono raccolte durante una partita di calcio.\\
Tale fatto scaturisce l'attenzione su un ulteriore tematica d'analisi: dato che si hanno a disposizione un gran numero di dati sulle prestazioni delle squadre nelle loro partite, è possibile individuare quali variabili vanno ad influenzare in modo significativo il successo o il fallimento sportivo delle singole squadre? \\
Da questo quesito nasce la tesi qui presentata. L’obbiettivo è quello di presentare un'analisi che risponda a tale quesito, attraverso l'utilizzo di tecniche di \textit{data mining}, in particolare lo sfruttamento di un modello di confronto a coppie per le partite di calcio in grado di tenere conto delle variabili esplicative specifiche per le partite. Il modello scelto per l’analisi sarà il modello \emph{Bradley-Terry} con le sue estensioni. Infine, verrà presentata un’applicazione di metodi di \textit{machine learning} ovvero, il K-Nearest-Neighbors (K-NN), la Support Vector Machine (SVM), il Decision Tree, la Random Forest e l'AdaBoost per la predizione dei risultati delle singole partite e l'individuazione delle \emph{features} più importanti. \\
Lo studio prenderà in considerazione i dati relativi alle partite della Serie A italiana della stagione 2021/2022.






%\vfill
%
%\selectlanguage{english}
%\pdfbookmark{Abstract}{Abstract}
%\chapter*{Abstract}
%
%\selectlanguage{italian}

\endgroup			

\vfill


	% !TEX encoding = UTF-8
% !TEX TS-program = pdflatex
% !TEX root = ../tesi.tex

%**************************************************************
% Ringraziamenti
%**************************************************************
\cleardoublepage
\phantomsection
\pdfbookmark{Ringraziamenti}{ringraziamenti}

\begin{flushright}{
	\slshape    
	``If something's important enough, you should try. Even if the probable outcome is failure.''} \\ 
	\medskip
   	--- Elon Musk %chi pensavi mettesi come citazione se non lui ;)
\end{flushright}


\bigskip

\begingroup
\let\clearpage\relax
\let\cleardoublepage\relax
\let\cleardoublepage\relax

\chapter*{Ringraziamenti}

\noindent \textit{Innanzitutto, vorrei esprimere la mia gratitudine al Prof.Annamaria Guolo, relatrice della mia tesi, per l'aiuto ed il sostegno fornitomi durante tutto il lavoro.}\\

\noindent \textit{Desidero ringraziare con affetto i miei genitori per il sostegno, per il grande aiuto che mi hanno dato e per essermi stati vicini in ogni momento durante gli anni di studio.}\\

\noindent \textit{Voglio inoltre ringraziare i miei amici per questi tre bellissimi anni trascorsi assieme e per avermi sempre sostenuto anche nei momenti più difficili.}\\

\bigskip

\noindent\textit{\myLocation, \myTime}
\hfill \myName

\endgroup


	\input{inizio-fine/indici}
	\cleardoublepage

%**************************************************************
% Materiale principale
%**************************************************************
\mainmatter
% !TEX encoding = UTF-8
% !TEX TS-program = pdflatex
% !TEX root = ../tesi.tex

%**************************************************************
\chapter{Introduzione}
\label{cap:introduzione}
%**************************************************************
%MEMO: Spiegazione del problema affrontato (il suo dominio) alcune applicazioni fatte nell'ambito delle comparazioni sportive, con maggior attenzione a qui studi con approccio statistico, esporre tecnologie usate e tools (Packages R ecc), motivazione scelta argomento della tesi e esposizione struttura della tesi(capitoli) TO DO
\intro{Il seguente capitolo descrive il problema scientifico affrontato in questa tesi e fornire un visione della letteratura di riferimento. Inoltre, il capitolo illustra le tecnologie utilizzate nella tesi, le motivazioni personali alla base della sua realizzazione e infine la struttura. } 

%**************************************************************
\section{Dominio del problema}

Negli ultimi anni lo sport del calcio è stato oggetto di un grande processo di rinnovamento tecnologico. Infatti, grazie alla spinta dell'evoluzione dei sistemi di comunicazioni, ora da una singola partita è possibile ottenere una grande quantità di informazioni in modo semplice, dato che lo sviluppo delle piattaforme di \emph{streaming} permette di usufruire di contenuti calcistici in ogni momento a livello globale. Questo rinnovamento tecnologico del calcio, con la conseguente disponibilità di dati riguardanti partite e giocatori, è dovuto anche all'introduzione di sensori e sistemi di tracciamento sofisticati all'interno del campo da calcio, come, ad esempio, la \emph{Goal-Line Technology} (vedi \textit{\cite{glt}}) oppure la Video Assistant Referee (VAR) (vedi \textit{\cite{var}}). Recentemente, durante la World Cup Qatar 2022, è stato introdotto il fuorigioco semi-automatico (vedi \textit{\cite{offside}}) composto da una grande quantità di sensori e telecamere per il tracciamento.\\
Data la disponibilità di una notevole quantità di dati e informazioni relative alle partite e ai giocatori, sono nati nuovi obbiettivi di studi. Tra questi, vi è lo \textit{scouting} di giocatori emergenti \autocite{vilela2018towards} oppure la scelta del ruolo più adatto per il giocatore in base alle sue caratteristiche \autocite{razali2017predicting} oppure quanto un giocatore è soggetto ad infortuni \autocite{theron2020use}. Oltre a questi, diversi studi si concentrano sull'analisi di informazioni relative alle squadre e alle partite nel loro complesso, ad esempio, il lavoro di \autocite{ley2019ranking} che utilizza dieci modelli statistici basati sulla forza stimata sulle partite di calcio, analizza il campionato inglese di calcio (Premier League) per produrre un nuovo \emph{ranking}.\\
Questa tesi si concentrerà sull'analisi delle partite, ossia l'interesse è l'identificazione dei fattori associati all'esito della partita. Infatti, molto spesso ci si pone l'interrogativo se una statistica, quale ad esempio il possesso della palla, o il numero di falli fatti o il numero di tiri fatti, sia rilevante per l'esito della partita, e in caso affermativo, con quale peso sia associata alla vittoria o al pareggio o alla sconfitta. Dunque, l'analisi sarà condotta sulle partite del campionato italiano della Serie A della stagione 2021/2022, utilizzando le statistiche più rilevanti, raccolte durante le partite di calcio. I dati utilizzati sono stati reperiti dal sito web \textit{\cite{fbref}}, il quale mette a disposizione un enorme quantità di statistiche riguardanti le partite delle maggiori leghe di calcio di più stagioni.\\
Fondamentale è la scelta delle metodologie da utilizzare per le analisi. Si è scelto l'utilizzo del modello Bradley-Terry \autocite{bradley1952rank} per l'interpretazione dei dati, l'individuazione di possibili legami tra le statistiche registrate durante una partita e l'esito della partita e infine, per scopi di predizione. Successivamente sono state utilizzate anche tecniche di \emph{machine learning} per implementare modelli matematici più complessi in grado di ottenere previsioni più accurate.
\section{Applicazione}
\textcite{bradley1952rank} hanno sviluppato un modello statistico utile per i confronti a coppie. L'obbiettivo è quello, per ogni confronto, di stabilire quale dei due oggetti confrontati sia il migliore sulla base di tratti latenti non osservati. Un esempio di tratto latente è l'effetto dell'ambiente dove avviene il confronto, che nell'ambito calcistico può essere tradotto come il vantaggio di giocare una partita in casa. \\
Successivamente il modello Bradley-Terry è stato modificato con diverse estensioni. Un'importante estensione si deve a \textcite{davidson1970extending}, il quale ha introdotto il pareggio nel confronto a coppie, elemento fondamentale per la nostra ricerca dato che il modello originario sviluppato da Bradley e Terry è binario (vittoria/sconfitta). In seguito grazie ai lavoro di \textcite{francis2010} e di \textcite{Turner2012Firth} è stata introdotta e approfondita l'inclusione di covariate per la valutazione nei confronti, ovvero l'utilizzo di attributi che descrivono i soggetti che eseguono i confronti tra oggetti. Nel nostro contesto, i soggetti sono rappresentanti dalle partite di calcio, mentre gli oggetti sono rappresentati dalle squadre di calcio. Successivi lavori da parte di \textcite{thurner2000policy} e di \textcite{mauerer2015modeling} introducono le covariate specifiche dell'oggetto e le covariate specifiche del soggetto e dell'oggetto.\\ %Nonostante non sia stata applicata all'interno di questa tesi si riporta l'esistenza di espansione del modello Bradley Terry elaborata da \textcite{cattelan2013dynamic} di una versione dinamica del modello ossia nel valutare i due oggetti viene presa in considerazione l'evoluzione che hanno avuto quest'ultimi.
Con l'introduzione delle covariate nei modelli di comparazioni a coppie aumentò la complessità dei modelli. Pertanto, sono state proposte in letteratura soluzioni basate sui metodi di regolarizzazione per ridurre la complessità dei modelli. Si veda ad esempio, \textcite{schauberger2019btllasso}, in cui svolgono l'analisi delle partite del campionato tedesco (Bundesliga) applicando il metodo LASSO come metodo di regolarizzazione.\\
Nell'ambito delle predizioni degli esiti delle comparazioni attraverso il modello Bradley-Terry si riporta l’interessante lavoro svolto da \textcite{kang2015poisson} in cui vengono svolte predizioni sulle partite del videogioco League of Legends (LOL). Per quanto riguarda i modelli predittivi di \emph{machine learning} \textcite{xu2021prediction} applica i metodi Decision Tree e Random Forest per la predizione degli esiti delle partite di calcio della Bundesliga.

\section{Tecnologie e Strumenti utilizzati}
Nella seguente sezione saranno illustrate le tecnologie e gli strumenti utilizzati durante il lavoro di tesi. 
\subsection{Tecnologie}
Le tecnologie utilizzate in questa tesi sono descritte di seguito.
\begin{itemize}
	\item \textbf{R} \autocite{R-language} è un linguaggio di programmazione per il calcolo statistico e l'analisi grafica. È stato sviluppato nel 1993 da Ross Ihaka e Robert Gentleman ed è diventato uno strumento molto popolare per l'analisi dei dati in molti campi, inclusa la scienza dei dati, l'economia, la genetica e la biologia computazionale. R offre un'ampia gamma di funzionalità per il trattamento dei dati, l'analisi statistica e la creazione di grafici e altre rappresentazioni visuali dei dati. Ad oggi è supportato dalla R Core Team e dalla R Foundation for Statistical Computing. È distribuito come software \emph{open source} e può essere facilmente esteso attraverso il \emph{download} di pacchetti di funzionalità aggiuntive sviluppati da una vasta comunità di utenti.\\
	Le librerie utilizzate sono indicate nell'Appendice \ref{cap:importR}.
	\item \textbf{Python} \autocite{van2003introduction} è un linguaggio di programmazione general-purpose, interpretato e ad alto livello. È stato sviluppato da Guido van Rossum negli anni '90 ed è mantenuto dalla Python Software Foundation. Il linguaggio Python è utilizzato in molti ambiti, come il web development, il \emph{machine learning} e l'automazione. È distribuito come software \emph{open source} e viene fornito con una grande quantità di librerie standard che espandono le sue funzionalità base.\\
	Le funzioni utilizzate sono indicate nell'Appendice \ref{cap:importPy}.
\end{itemize}


\begin{comment}
library(ggmosaic)
library(ggplot2)
library(gridExtra)
@article{marchiori2020secrets,
	title={Secrets of soccer: Neural network flows and game performance},
	author={Marchiori, Massimo and de Vecchi, Marco},
	journal={Computers \& Electrical Engineering},
	volume={81},
	pages={106505},
	year={2020},
	publisher={Elsevier}
}
\end{comment}

\subsection{Strumenti}
Gli strumenti utilizzati in questa tesi sono descritti di seguito.
\begin{itemize}
	\item \textbf{RStudio} (si veda \textit{\cite{rstudio}}) è un ambiente di sviluppo integrato (IDE) per il linguaggio di programmazione R. Fornisce un insieme di strumenti per facilitare la scrittura, il debugging e il \emph{testing} del codice R. RStudio include anche funzionalità per la visualizzazione e l'analisi dei dati, come il supporto per i grafici interattivi e la possibilità di eseguire il codice R direttamente nell'editor di testo. RStudio è distribuito come \emph{software} \emph{open source}. In questa tesi è stato utilizzato per implementare il modello Bradley-Terry in R.
	\item \textbf{PyCharm} (si veda \textit{\cite{pycharm}}) è un ambiente di sviluppo integrato (IDE) per il linguaggio di programmazione Python. Offre una serie di strumenti per facilitare la scrittura, il debugging e il \emph{testing} del codice Python. PyCharm include anche funzionalità per l'integrazione con altri strumenti e servizi comuni nello sviluppo web, come il supporto per il versionamento del codice con Git e il supporto per i \emph{framework} di sviluppo web come Django. PyCharm  è sviluppato da JetBrains. In questa tesi è stato utilizzato per implementare i modelli di \emph{machine learning} in Python.
\end{itemize}

\section{Motivazioni personali}
Durante il mio percorso di studio ho frequentato alcuni corsi legati al mondo dell'intelligenza artificiale e dello studio dei dati. Oltre a ciò, sono un appassionato dello sport del calcio. Nel seguire questo sport, a volte mi imbatto in articoli in cui sono riportate analisi dei dati e l'applicazione di algoritmi di \emph{machine learning} che provano a predire le classifiche finali dei maggiori campionati europei in fase di svolgimento. Inoltre, in seguito alla pandemia da COVID19, sempre più club calcistici hanno iniziato ad analizzare dati e statistiche per migliorare le loro prestazioni in campo e nello \emph{scouting}. Perciò, su spinta delle mie passioni e dalle recenti applicazioni precedentemente descritte, ho individuato nell'identificazione dei fattori associati all'esito di una partita, un campo di ricerca innovativo e interessante come lavoro di tesi di laurea magistrale.
\section{Struttura della tesi}
La struttura della tesi è riporta di seguito.
\begin{description}

\item[{\hyperref[cap:dataset]{Il secondo capitolo}}] descrive la raccolta dati e la struttura del dataset. 
\item[{\hyperref[cap:Analisi]{Il terzo capitolo}}] descrive l'analisi grafica dei dati e il \emph{preprocessing} dei dati. 
\item[{\hyperref[cap:BT]{Il quarto capitolo}}] descrive il modello Bradley-Terry e le sue estensioni che sono state utilizzante durante l'analisi.
\item[{\hyperref[cap:risultatiDM]{Il quinto capitolo}}] illustra i risultati registrati con il modello Bradley-Terry e le sue estensioni. Inoltre vengono riportate le predizioni eseguite dai vari modelli Bradley-Terry confrontate con le predizioni dei \emph{bookmakers}.
\item[{\hyperref[cap:ML]{Il sesto capitolo}}] descrive gli algoritmi di apprendimento automatico che sono stati utilizzanti durante l'analisi.
\item[{\hyperref[cap:RisML]{Il settimo capitolo}}] riporta le predizioni calcolate dagli algoritmi di apprendimento automatico.
\item[{\hyperref[cap:extraDM]{l'ottavo capitolo}}] descrive una nuova applicazione di un modello BT già utilizzato nel Capitolo \ref{cap:risultatiDM}, con una variabile risposta non più a tre ma a cinque categorie.
\item[{\hyperref[cap:precls]{Il nono capitolo}}] contiene una descrizione sui risultati riportati dai Capitoli \ref{cap:risultatiDM}, \ref{cap:RisML} e \ref{cap:extraDM}. 
\item[{\hyperref[cap:conclusioni]{Il decimo capitolo}}] riporta un riassunto di quanto è stato svolto, sottolineando possibili sviluppi del lavoro di tesi.
\end{description}





\begin{comment}
\begin{figure}[h]
	\begin{center}
		\includegraphics[scale=0.5]{Logo_azzurrodigite.png}
		\caption{Logo di AzzurroDigitale}
	\end{center}
\end{figure}	contenuto...
\end{comment}


%**************************************************************


%\gls{AWMS} \g{machine learning}


%\begin{description}
    
   % \item[{\hyperref[cap:descrizione-stage]{Il secondo capitolo}}] descrive in modo dettagliato lo stage svolto, indicandone obiettivi, prodotti attesi, pianificazione delle attività, strumenti e tecnologie utilizzate e motivazioni personali.
    
    
%\end{description}
             % Introduzione
% !TEX encoding = UTF-8
% !TEX TS-program = pdflatex
% !TEX root = ../tesi.tex

%**************************************************************
\chapter{Serie A 2021/2022 dataset }
%\label{cap:dataset}
%**************************************************************

\intro{Nel seguente capitolo verrà descritto in dettaglio la raccolta dati effettuata per costruire il dataset riguardante le partite di calcio della Serie A italiana della stagione 2021/2022 e di come tale dataset è strutturato descrivendone le variabili e i dati al suo interno, utilizzati per l'analisi descritta precedentemente.}\\

%**************************************************************
\section{Serie A 2021/2022}

L'analisi che è stata effettuata ha preso in considerazione le partite della Serie A italiana della stagione 2021/2022. La Serie A è un torneo che comprende 20 squadre sparse per tutta l'Italia, alcune anche della stessa città ad esempio, Milan e Inter sono due squadre di Milano. \\
Tale torneo è organizzato con una struttura Double-Round-Robin, dove ogni squadra affronta due volte le altre 19 avversarie del torneo. Vi è quindi una partita di andata e una di ritorno che in base al sorteggio della creazione del calendario delle partite decide quale delle due partite sarà giocata in casa oppure fuori casa (in casa dell'avversario). \\
Tale torneo nella stagione 2021/2022 è iniziato il 22 Agosto con Inter - Genoa e si è concluso il 22 Maggio con le partite Salernitana - Udinese e Venezia - Cagliari, per un totale 380 partite giocate suddivise in 38 turni dove ogni turno è composto da 10 partite.

\subsection{Ranking}
Le squadre di calcio sono classificate in base all'ordine dei punti che hanno totalizzato al termine della stagione. In un torneo calcistico, per ogni partita vinta la squadra vincente guadagna 3 punti, per ogni pareggio le due squadre avversarie guadagnano entrambe un punto, mentre per ogni sconfitta la squadra perdente non guadagna punti. Nel torneo della Serie A chi guadagna più punti vince il campionato, mentre chi si classifica tra le ultime tre retrocede alla lega inferiore, la Serie B, dove il posto delle tre squadre retrocesse verrà presso da tre squadre della Serie B che hanno guadagnato la promozione alla Serie A.\\ 
La classifica della stagione 2021/2022 è mostrata nella Tabella \ref{tab:ranking}.

	\begin{table}[!htb]%

	\renewcommand{\arraystretch}{1.7}
	\centering
	\begin{tabular}{c c c c}
		\hline	

		\textbf{Posizione} & \textbf{Squadra} & \textbf{Punti} & \textbf{ \% casa}  \\	
		\hline			
		1 & Milan & 86 & 0.47\\
		2 & Inter & 84 & 0.54\\
		3 & Napoli & 79 & 0.46\\
		4 & Juventus & 70 & 0.50\\
		5 & Lazio & 64 & 0.56\\
		6 & Roma & 63 & 0.57\\
		7 & Fiorentina & 62 & 0.66\\
		8 & Atalanta & 59 & 0.33\\
		9 & Hellas Verona & 53 & 0.57\\
		10 & Torino & 50 & 0.58\\
		11 & Sassuolo & 50 & 0.48\\
		12 & Udinese & 47 & 0.53\\
		13 & Bologna & 46 & 0.61\\
		14 & Empoli & 41 & 0.42\\
		15 & Sampdoria & 36 & 0.58\\
		16 & Spezia & 36 & 0.50\\
		17 & Salernitana & 31 & 0.48\\
		18 & Genoa & 30 & 0.50\\
		19 & Cagliari & 28 & 0.61\\
		20 & Venezia & 27 & 0.52\\
			\hline
		 & & & \\
	
	\end{tabular} \hbox{}

	\caption{La tabella mostra i punti guadagnati da ogni squadra con il loro piazzamento. Inoltre viene mostrata la percentuale di punti guadagnati in casa.} \label{tab:ranking}
\end{table}

%**************************************************************
\section{Costruzione del dataset}

Al giorno d'oggi, nelle partite di calcio professionistico viene raccolta un'enorme quantità di variabili. Ad esempio, per ogni squadra è noto il tempo in percentuale del possesso della palla o il numero di tiri in porta prodotto dalla squadra in una determinata partita. L'obiettivo principale di questo lavoro è determinare l'influenza di queste variabili specifiche della partita. \\
Per creare il dataset per tale scopo, sono state raccolte un gran numero di variabili che a primo avviso possono essere significative, tali dati sono stati offerti dal sito web FBref.\\
FBref è un sito web dedicato al tracciamento delle statistiche relative ai calciatori e alle squadre di calcio di tutto il mondo. FBref mette a disposizione i dati sotto forma di tabelle che possono essere modificate per mantenere solo i dati di nostro interesse, in più per rendere più facile l'esportazione, tali tabelle possono essere convertite in formato di CSV per poter essere poi trasportate in un file Excel.\\
\begin{figure}[!htb]
	\begin{center}
		\includegraphics[scale=0.40]{logo.png}
		\caption{Logo di FBref. link: \url{https://fbref.com}} \label{fig:filt}
	\end{center}
\end{figure}



Quindi per ogni squadra che ha partecipato alla stagione 2021/2022 di Serie A si è esportato per ogni partita giocata alcune variabili che ci interessavano, selezionando per prima cosa la macro aree dove si trovavano le variabili d'interesse e poi, modificando le tabelle per ottenere solo i dati di tali variabili. Ogni tabella generata veniva poi riconvertita in CSV per essere poi unita con tutte le altre in un file Excel che una volta completato, divenne il dataset per le nostre analisi. Per rendere più leggibile il file Excel, dato che le stringhe in CSV separavano i dati con il carattere separatore virgola, si è utilizzata la funzione di Excel "trasforma testo in colonne" per inserire tutti i dati in modo ordinato nelle celle del foglio Excel.

\subsection{Struttura dataset}
Il dataset risultate dalla raccolta dati è composto da 760 righe e 35 colonne. Ogni riga riguarda una specifica partita di calcio giocata dalla squadra indicata nella colonna \textsf{Team} contro la squadra indicata nella colonna \textsf{Vs}. Ogni riga perciò contiene informazioni riguardati solo la squadra indicata in \textsf{Team} fatta eccezioni per la data della partita (\textsf{Date}), il turno (\textsf{Round}), e gli spettatori (\textsf{Spec}). Quindi per ogni partita esistono due righe, una per ognuna delle due squadre coinvolte. Perciò ogni squadra appare nella colonna \textsf{Team} 38 e dato che si hanno 20 squadre si hanno perciò 760 righe totali. Per quanto riguarda le colonne se ne discuterà nella prossima sotto sezione. \\
La Tabella \ref{tab:db} mostra un breve estratto dei dati riguardanti le prime tre partite della stagione. 
	\begin{table}[!ht]%

	\renewcommand{\arraystretch}{1.7}
	\centering
	\begin{tabular}{c c c c c c c c c  }
		\hline	

		\textbf{Date} & \textbf{AtHome} & \textbf{Res} & \textbf{GF} & \textbf{GA} & \textbf{Team} & \textbf{Vs} & \textbf{Poss} & \textbf{...}   \\	
		\hline	
		21/08/2021 & TRUE & 1 & 4 & 0 & Inter & Genoa & 0,59 & ... \\
		... & ... & ... & ... & ... & ... & ... & ... & ... \\
		22/08/2021  & TRUE & 1 & 2 & 0 & Napoli & Venezia & 0,56 & ... \\
		... & ... & ... & ... & ... & ... & ... & ... & ...  \\
		23/08/2021  & FALSE & 1 & 1 & 0 & Milan & Sampdoria & 0,51 & ... \\		
		... & ... & ... & ... & ... & ... & ... & ... & ... \\
		21/08/2021  & FALSE & -1 & 0 & 4 & Genoa & Inter & 0,41 & ... \\
		... & ... & ... & ... & ... & ... & ... & ... & ...  \\
		22/08/2021  & FALSE & -1 & 0 & 2 & Venezia & Napoli & 0,44 & ... \\
		... & ... & ... & ... & ... & ... & ... & ... & ...  \\
		23/08/2021 1 & TRUE & 1 & 0 & 1 & Sampdoria & Milan & 0,49 & ... \\
		... & ... & ... & ... & ... & ... & ... & ... & ...  \\
		\hline
		& & & & & & & & \\
		
		
		
	\end{tabular} \hbox{}
	
	\caption{La tabella mostra un estratto del dataset utilizzato i cui dati sono stati ricavati da FBref.} \label{tab:db}
\end{table}


\subsection{Covariate}

Come scritto precedentemente all'interno del dataset sono presenti 35 colonne. Oltre alle già citate \textsf{Date}, \textsf{Round} e \textsf{Spec} che hanno solo un valore di completezza dei dati, le restanti 32 colonne saranno le possibili candidate a essere le covariate che costituiranno il modello. Ovviamente non è detto che tutte queste variabili saranno inserite nel modello perché prima di costruire un modello, ci sarà un analisi per verificare se sia sensato o no l'utilizzo di ognuna delle variabili verificando attraverso grafici (analisi grafica) e individuando possibili problemi di multicollinearità o di bassa significatività delle variabili.\\
%\pagebreak

Le possibili covariate sono le seguenti:
\begin{itemize}
	\item \textsf{AtHome}: Tale variabile indica se la squadra indicata sulla variabile \textsf{Team} gioca nel suo stadio, quindi in casa oppure fuori casa. Per indicare se la squadra gioca in casa viene messo come valore \texttt{TRUE} altrimenti \texttt{FALSE}. 
	
	Come mostrato nella terza colonna della tabella \ref{tab:ranking}, che indica in percentuale quante partite sono state vinte in casa per ogni singola squadra, ci sono 11 squadre che hanno avuto un leggero vantaggio nel giocare in casa le partite di calcio rispetto a altre sei squadre che hanno avuto l'effetto opposto, mentre le rimanti tre hanno avuto un effetto nullo. Alla luce di questo è stato deciso di inserire tale variabile per via del suo effetto nell'esito di una partita in generale.
	\item \textsf{Res}: Tale variabile indica se la squadra indicata sulla variabile \textsf{Team} ha vinto o ha pareggiato o ha perso. Per indicare se ha vinto viene inserito il valore 1, se ha pareggiato 0, altrimenti se ha perso -1. Chiaramente questa variabile sarà la nostra Y cioè la variabile risposta che il modello deve riuscire a prevedere.
	\item \textsf{GF}: Tale variabile indica il numero di gol fatti dalla squadra indicata sulla variabile \textsf{Team}. 
	
	Questa variabile è stata inserita perché può permettere di valutare la qualità della fase offensiva della squadra e quindi essere significativa ai fine dell'analisi.
	\item \textsf{GA}: Tale variabile indica il numero di gol subiti dalla squadra indicata sulla variabile \textsf{Team} e quindi fatti dalla squadra indicata nella variabile \textsf{Vs}. 
	
	Questa variabile è significativa perché subire pochi gol incide positivamente nell'esito della partita, infatti non espone la squadra a doversi sbilanciare in attacco per poter recuperare lo svantaggio e quindi non rischiare di subire altri gol dai avversari. Inoltre è un fatto riconosciuto che aver la miglior difesa del campionato porta con molta probabilità a vincere il campionato
	\item \textsf{Team}: Tale variabile indica il nome della squadra a cui i dati della riga fanno riferimento. É necessaria per il funzionamento del modello, nel prossimo paragrafo verrà approfondito il suo utilizzo nel modello.
	\item \textsf{Vs}: Tale variabile indica il nome della squadra avversaria. É necessaria per il funzionamento del modello, nel prossimo paragrafo verrà approfondito il suo utilizzo nel modello.
	\item \textsf{Poss}: Tale variabile indica in percentuale, la quantità di tempo di possesso della palla durante una partita di calcio della squadra indicata sulla variabile \textsf{Team}. Nel gioco del calcio con il termine “possesso palla” si intende un’azione manovrata di due o più giocatori che riescono a passarsi la palla evitando i contrasti degli avversari. In poche parole durante la partita, ogni volta che una squadra ha il dominio della palla si dice che questa squadra è in fase di “possesso palla”, quindi in questa variabile viene indicato quanto questa fase è durata nell'intera partita.\\
	Il metodo più comune utilizzato per calcolare il possesso palla di una squadra si basa sull'utilizzo di tre cronometri: uno per ciascuna formazione più uno per i tempi morti. Quando un giocatore della squadra A tocca un pallone che prima era in possesso della squadra B, il cronometro della squadra A parte e quello della squadra B si ferma e così via. Il terzo cronometro registra il tempo in tutte le situazioni di palla inattiva cioè ad esempio: rimesse laterali, calci di punizione ecc.. I tempi vengono poi trasformati in percentuali. Per una registrazione più sofisticata, si può utilizzare 22 cronometri, uno per ogni giocatore, in modo da registrare anche il possesso palla di ogni singolo giocatore per avere una registrazione più precisa.
	
	Tale variabile è stata inserita perché, la supremazia nel possesso palla è solitamente desiderabile e utile infatti si possono avere i seguenti vantaggi:
	\begin{itemize}
		\item Spingere l’avversario a muoversi verso la palla per allontanarlo dalla difesa della propria porta per poi sorprenderlo negli spazi lasciati incustoditi.   
		\item Modulare il ritmo della gara, ad esempio la squadra A sta vincendo con un gol di scarto e per non rischiare attacchi dalla squadra B, "congela" il risultato mantenendo il possesso della palla.
	\end{itemize}
	Il possesso palla però non garantisce certo la vittoria, infatti produrre un possesso palla "sterile" cioè senza che questo porti alla produzioni di azioni offensive, può esporre la squadra in possesso della palla a possibili contropiedi nel caso in cui perde la palla e quindi all'alto rischio di subito gol perché sbilanciata e non ben posizionata. Vedremmo di seguito quali variabili possono essere utili per capire se il possesso palla fatto dalla squadra è "sterile" oppure no.

	\item \textsf{Sh}: Tale variabile indica il numero di tiri totali fatti dalla squadra indicata sulla variabile \textsf{Team}. Quindi vengono conteggiati il numero di tiri in porta più i tiri fuori dalla porta. 
	
	Una squadra che effettua tanti tiri ha più probabilità di segnare un gol. Occorre pero capire quanto è precisa una squadra nel centrare la porta.
	\item \textsf{SoT}: Tale variabile indica il numero di tiri in porta totali fatti dalla squadra indicata sulla variabile \textsf{Team}. 
	
	Una squadra con un alto valore di tiri in porta è più probabile che possa segnare un gol. Tale variabile permette di capire quanto è precisa in combinazione con \texttt{Sh} la squadra di calcio nel centrare la porta nei suoi tiri.
	\item \textsf{G/Sh}: Tale variabile indica la proporzione tra gol e tiri fatti dalla squadra indicata sulla variabile \textsf{Team}. 
	
	Tale variabile perciò permette di capire quanto la produzioni di tiri della squadra è efficace o meno. Con \texttt{Sh} e \texttt{SoT} si riesce a valutare quanto è offensiva la squadra cioè, se essa gioca costantemente in attacco o utilizza la tattica difesa e contropiede. Inoltre permette di capire quanto la squadra è precisa nel effettuare i tiri in porta.
	\item \textsf{Saves}: Tale variabile indica il numero di parate fatte del portiere della squadra indicata sulla variabile \textsf{Team}. 
	
	La variabile è stata inserita perché permette di valutare se la squadra subisce tanti tiri dai avversari e la qualità del portiere nel salvare la squadra da un possibile gol subito.
	\item \textsf{PAtt}: Tale variabile indica il numero di tutti i passaggi tentati dai giocatori della squadra indicata sulla variabile \textsf{Team}. 
	
	Utile a capire quanto la squadra sia incline a tentare i passaggi. Si studierà nell'analisi se tale variabile è significativa ma sicuramente ha un maggior significato se messa a confronto con la percentuale di passaggi riusciti \texttt{PCmp\%}.
	\item\textsf{PCmp\%}: Tale variabile indica la percentuale di passaggi riusciti ai giocatori della squadra indicata sulla variabile \textsf{Team}. 
	
	Questa variabile è stata inserita perché permette di capire quanti passaggi sono andati a buon fine tra tutti quelli tentati e quindi qual'è la precisione dei giocatori della squadra.
	\item \textsf{SPAtt}: Tale variabile indica il numero di passaggi corti tentati dai giocatori della squadra indicata sulla variabile \textsf{Team}. Per passaggi corti si intendono quelli effettuati all'interno di una lunghezza tra i tre e 14 metri.
	
	Questa variabile è stata inserita per capire se un alto numero di passaggi corti possono essere determinanti ai fini dell'esito della partita. Ovviamente analogamente a \texttt{PAtt} occorre fare un confronto con la sua percentuale di passaggi corti riusciti \texttt{SPCmp\%}.
	\item \textsf{SPCmp\%}: Tale variabile indica la percentuale di passaggi corti riusciti ai giocatori della squadra indicata sulla variabile \textsf{Team}. 
	
	Questa variabile è stata inserita perché permette di capire quanti passaggi sono andati a buon fine tra tutti quelli tentati e quindi qual'è la precisione dei giocatori della squadra.
	\item \textsf{MPAtt}: Tale variabile indica il numero di passaggi medi tentati dai giocatori della squadra indicata sulla variabile \textsf{Team}. Per passaggi medi si intendono quelli effettuati all'interno di una lunghezza tra i 13 e 27 metri. Questi passaggi possono essere considerati come passaggi filtranti cioè un tipo di passaggio non diretto direttamente al proprio compagno di squadra ma verso un area del campo dove il compagno di squadra deve andare a prendere la palla, spesso questi passaggi vengono fatti per sorprendere la difesa avversaria e evitare che intercettino la palla. Nella Figura \ref{fig:filt} viene mostrato l'esecuzione di un passaggio filtrante.
	\begin{figure}[ht]
		\begin{center}
			\includegraphics[scale=0.51]{filtrante2.jpg}
			\caption{Esecuzione di un passaggio filtrante} \label{fig:filt}
		\end{center}
	\end{figure}

	Questa variabile è stata inserita per capire se un alto numero di passaggi medi possono essere determinanti ai fini dell'esito della partita. Ovviamente analogamente a \texttt{PAtt} occorre fare un confronto con la sua percentuale di passaggi corti riusciti \texttt{MPCmp\%}.

	\item \textsf{MPCmp\%}: Tale variabile indica la percentuale di passaggi medi riusciti ai giocatori della squadra indicata sulla variabile \textsf{Team}. Questa variabile è stata inserita perché permette di capire quanti passaggi sono andati a buon fine tra tutti quelli tentati e quindi qual'è la precisione dei giocatori della squadra.
	\item \textsf{LPAtt}: Tale variabile indica il numero di passaggi lunghi tentati dai giocatori della squadra indicata sulla variabile \textsf{Team}. Per passaggi corti si intendono quelli effettuati all'interno di una lunghezza superiore ai 27 metri. Questi passaggi possono essere considerati come lanci lunghi per cambi di gioco o per lanciare le punte, cioè i giocatori che giocano come attaccanti, in profondità. Una rappresentazione di passaggio lungo è mostrata nella Figura \ref{fig:cambio}.
	\begin{figure}[ht]
		\begin{center}
			\includegraphics[scale=0.53]{cambio-di-gioco.png}
			\caption{Esecuzione di un cambio di gioco} \label{fig:cambio}
		\end{center}
	\end{figure}
	
	Questa variabile è stata inserita per capire se un alto numero di passaggi lunghi possono essere determinanti ai fini dell'esito della partita. Ovviamente analogamente a \texttt{PAtt} occorre fare un confronto con la sua percentuale di passaggi corti riusciti \texttt{LPCmp\%}.

	\item \textsf{LPCmp\%}: Tale variabile indica la percentuale di passaggi lunghi riusciti ai giocatori della squadra indicata sulla variabile \textsf{Team}. 
	
	Questa variabile è stata inserita perché permette di capire quanti passaggi sono andati a buon fine tra tutti quelli tentati e quindi qual'è la precisione dei giocatori della squadra.
	\item \textsf{ToDefPen}: Tale variabile indica il numero di tocchi fatti dai giocatori della squadra indicata sulla variabile \textsf{Team} nella propria area di rigore. 
	
	Questa variabile è stata inserita perché può essere utile per capire come il possesso della palla viene gestito, cioè se vi è un alto numero di tocchi vuol dire che la squadra subisce molto la pressione della squadra avversaria, viceversa cerca di fare un gioco più offensivo. Questa variabile in combinazione con \textsf{ToDef3rd}, \textsf{ToMid3rd}, \textsf{ToAtt3rd} e \textsf{ToAttPen} permette di capire se il possesso della palla fatto della squadra è utile e porta benefici ai fini del risultato oppure è sterile. Inoltre si vuole capire attraverso l'analisi in che misura può influenzare il risultato della partita con un alto o un basso valore di numero di tocchi nella propria area di rigore la cui area nel campo da calcio è indicata nella Figura \ref{fig:penalty}.
	
	\begin{figure}[!ht]
		\begin{center}
			\includegraphics[scale=0.60]{rigore.jpg}
			\caption{In rosso l'area di rigore in un campo da calcio.} 
			\label{fig:penalty}
		\end{center}
	\end{figure}
	

	\item \textsf{ToDef3rd}: Tale variabile indica il numero di tocchi fatti dai giocatori della squadra indicata sulla variabile \textsf{Team} nella propria mediana o trequarti difensiva. 
	
	Questa variabile è stata inserita perché può essere utile per capire come il possesso della palla viene gestito, cioè se vi è un alto numero di tocchi vuol dire che la squadra cerca di mantenere il possesso palla creando poche azioni offensive, viceversa cerca di fare un gioco più offensivo. Questa variabile in combinazione con \textsf{ToDef3rd}, \textsf{ToMid3rd}, \textsf{ToAtt3rd} e \textsf{ToAttPen} permette di capire se il possesso della palla fatto della squadra è utile e porta benefici ai fini del risultato oppure è sterile. Inoltre si vuole capire attraverso l'analisi in che misura può influenzare il risultato della partita con un alto o un basso valore di numero di tocchi nella propria mediana la cui area nel campo da calcio è indicata nella Figura \ref{fig:def}.
	
		\begin{figure}[!ht]
		\begin{center}
			\includegraphics[scale=0.60]{mid.jpg}
			\caption{In rosso la mediana nel campo da calcio.} 
			\label{fig:def}
		\end{center}
	\end{figure}
	\item \textsf{ToMid3rd}: Tale variabile indica il numero di tocchi fatti dai giocatori della squadra indicata sulla variabile \textsf{Team} a centrocampo. 
	
	Questa variabile è stata inserita perché può essere utile per capire come il possesso palla viene gestito, cioè se vi è un alto numero di tocchi vuol dire che la squadra cerca di mantenere il possesso palla cercando di creare delle azioni offensive, viceversa cerca di fare un gioco più difensivo. Questa variabile in combinazione con \textsf{ToDef3rd}, \textsf{ToMid3rd}, \textsf{ToAtt3rd} e \textsf{ToAttPen} permette di capire se il possesso della palla fatto dalla squadra è utile e porta benefici ai fini del risultato oppure è sterile. Inoltre si vuole capire attraverso l'analisi in che misura può influenzare il risultato della partita con un alto o un basso valore di numero di tocchi a centrocampo la cui area nel campo da calcio è indicata nella Figura \ref{fig:cen}.
	
		\begin{figure}[!ht]
		\begin{center}
			\includegraphics[scale=0.60]{cen.jpg}
			\caption{In rosso il centrocampo nel campo da calcio.} 
			\label{fig:cen}
		\end{center}
	\end{figure}

	\item \textsf{ToAtt3rd}: Tale variabile indica il numero di tocchi fatti dai giocatori della squadra indicata sulla variabile \textsf{Team} a nella trequarti dell'avversario. 
	
	Questa variabile è stata inserita perché può essere utile per capire come il possesso della palla viene gestito, cioè se vi è un alto numero di tocchi vuol dire che la squadra cerca di mantenere il possesso palla per effettuare una pressione sulla squadra avversaria affinché si possano creare degli spazi per delle azioni offensive, viceversa cerca di fare un gioco molto più difensivo. Questa variabile in combinazione con \textsf{ToDef3rd}, \textsf{ToMid3rd}, \textsf{ToAtt3rd} e \textsf{ToAttPen} permette di capire se il possesso della palla fatto della squadra è utile e porta benefici ai fini del risultato oppure è sterile. Inoltre si vuole capire attraverso l'analisi in che misura può influenzare il risultato della partita con un alto o un basso valore di numero di tocchi nella trequarti dell'avversario la cui area nel campo da calcio è indicata nella Figura \ref{fig:treq}.
	
		\begin{figure}[!ht]
		\begin{center}
			\includegraphics[scale=0.60]{treq.jpg}
			\caption{In rosso la trequarti dell'avversario nel campo da calcio.} 
			\label{fig:treq}
		\end{center}
	\end{figure}

	\item \textsf{ToAttPen}: Tale variabile indica il numero di tocchi fatti dai giocatori della squadra indicata sulla variabile \textsf{Team} a nell'area di rigore dell'avversario. 
	
	Questa variabile è stata inserita perché può essere utile per capire come il possesso della palla viene gestito, cioè se vi è un alto numero di tocchi vuol dire che la squadra cerca di mantenere il possesso palla applicando un'alta pressione sulla squadra avversaria affinché si possano creare molte occasioni da gol in area, viceversa o la squadra subisce troppo la pressione dell'avversario oppure tende ad avere un gioco molto difensivo. Questa variabile in combinazione con \textsf{ToDef3rd}, \textsf{ToMid3rd}, \textsf{ToAtt3rd} e \textsf{ToAttPen} permette di capire se il possesso della palla fatto della squadra è utile e porta benefici ai fini del risultato oppure è sterile. Inoltre si vuole capire attraverso l'analisi in che misura può influenzare il risultato della partita con un alto o un basso valore di numero di tocchi nell'area di rigore dell'avversario.
	
	\item \textsf{ToDist}: Tale variabile indica la distanza totale, espressa in metri, in cui un giocatore della squadra indicata sulla variabile \textsf{Team}, si è mosso con la palla in qualsiasi direzione, controllandola con i piedi.
	
	Variabile che è stata inserita perché permette di ricavare se il possesso della palla sia stato statico cioè i giocatori si sono mossi poco senza avanzare, oppure no. Sarà di interesse analizzare se con un alto valore di metri percorsi con palla al piede, possa essere utile a ottenere la vittoria.
	\item \textsf{Fls}: Tale variabile indica il numero di falli dai giocatori della squadra indicata sulla variabile \textsf{Team}. 
	
	Questa variabile è stata inserita per capire se una squadra adotta un gioco più fisico/tattico. In questo caso sarà più propensa a interrompere il gioco della squadra avversaria e a commettere più falli. Si vuole perciò capire come questa variabile può andare ad influire sull'esito della partita, ricordando però che una che commette molti falli è più soggetta a ricevere cartellini gialli o rossi che condizionano la prestazione dei giocatori.
	\item \textsf{Fld}: Tale variabile indica il numero di falli subiti dai giocatori della squadra indicata sulla variabile \textsf{Team} da parte della squadra avversaria indicata sulla variabile \textsf{Vs}. 
	
	Si è deciso di inserire questa covariata perché un alto numero di falli può portare a molte interruzione della manovra di gioco e quindi permettere alla squadra avversaria di riorganizzarsi. Si vuole perciò capire come questa variabile può andare ad influire sull'esito della partita.
	\item \textsf{Off}: Tale variabile indica il numero di volte che la squadra indicata sulla variabile \textsf{Team} è finita in fuorigioco. Un calciatore si trova in posizione di fuorigioco quando una qualsiasi parte del suo corpo, fatta eccezione per braccia e mani perché non possono essere usate per controllare il pallone; si trova nella nella metà campo avversaria ed è più vicina alla linea di porta avversaria sia rispetto al pallone e sia rispetto al penultimo giocatore difendente avversario; tale penultimo avversario può essere anche il portiere, se un compagno di questi è più vicino di lui alla linea di porta. Una rappresentazione grafica del fuorigioco è mostrata nell'immagine \ref{fig:offside}.
	
	\begin{figure}[!ht]
		\begin{center}
			\includegraphics[scale=0.55]{var.png}
			\caption{Rappresentazione del fuorigioco} \label{fig:offside}
		\end{center}
	\end{figure}

	Tale variabile è stata inserita perché, se una squadra viene colta molte volte in fuorigioco allora il suo gioco sarà interrotto e darà un vantaggio alla squadra avversaria che farà ripartire la sua azione a suo favore.
	
	
	\item \textsf{Crs}: Tale variabile indica il numero di cross effettuati dalla squadra indicata sulla variabile \textsf{Team}. Un cross in italiano traversone, è un tipo di passaggio medio o lungo, solitamente effettuato sulle fasce laterali dell'area avversaria o comunque vicino all'area avversaria, che se eseguito permette al compagno di squadra posizionato vicino alla porta avversaria, di colpire la palla al volo di testa oppure di piede per segnare un possibile gol. Quindi se eseguito correttamente, il cross può diventare un assist, cioè l'ultimo passaggio per la realizzazione del gol. 
	
	\begin{figure}[!ht]
		\begin{center}
			\includegraphics[scale=0.43]{cross.jpg}
			\caption{Rappresentazione di un cross} \label{fig:cross}
		\end{center}
	\end{figure}
	
	Per tale motivo si è deciso di inserire una variabile specifica per questo tipo di passaggio nell'analisi. Una rappresentazione di un cross è mostrata nella Figura \ref{fig:cross}.
	\item \textsf{Int}: Tale variabile indica il numero di intercettazioni fatte dai giocatori della squadra indicata sulla variabile \textsf{Team}. Per intercettazione della palla si intende interrompere un passaggio della squadra avversaria entrando in possesso del pallone che era stato lanciato per un passaggio ma che una volta intercettato non è andato a buon fine cioè non è arrivato al compagno del giocatore avversario che ha effettuato il passaggio. 
	
	Quindi si è deciso di inserire questa variabile perché indica quante volte si è tolto il possesso della palla all'avversario interrompendone il suo gioco.
	\item \textsf{TklWin}: Tale variabile indica il numero di contrasti vinti dai giocatori della squadra indicata sulla variabile \textsf{Team}. Per contrasto si intende il tentativo da parte di un giocatore difendente di sottrarre il possesso della palla all'avversario. Quindi chi ha in possesso la palla viene attaccato da chi ne è privo. Se si riesce a prendere il pallone all'avversario allora si avrà vinto il contrasto. Si sottolinea che i contrasti vengono anche effettuati per allontanare dalle zone pericolose l'avversario. La Figura \ref{fig:tackle} mostra un contrasto di gioco.
	
	\begin{figure}[!ht]
		\begin{center}
			\includegraphics[scale=0.40]{tackle.jpg}
			\caption{Rappresentazione di un contrasto in scivolata} \label{fig:tackle}
		\end{center}
	\end{figure}
	
	Visto che tale intervento senza palla va a modificare il gioco dell'avversario, si è deciso di inserire i contrasti vinti come variabile. 
	
	\item \textsf{Recov}: Tale variabile indica il numero di palle vaganti recuperate dalla squadra indicata sulla variabile \textsf{Team}. Per palle vaganti si intendono quei palloni che a seguito di un contrasto di gioco, non sono stati recuperati dalla squadra che ha effettuato il contrasto ma chi ha subito il contrasto ne ha comunque perso il controllo. Si ha che nessuno ha in possesso il pallone e quindi si ha una palla vagante.
	
	Dato che questa variabile sembra essere legata al possesso del pallone sembra essere interessante per l'analisi.
	
\end{itemize}
\section{Preprocessing dei dati}
Nella sezione precedente si è descritto come si è costruito il dataset che verrà utilizzato per l'analisi e come esso è stato strutturato. Tale struttura ha il vantaggio di essere di facile interpretazione per un essere umano ma vi sono alcune criticità che non lo permettono di essere utilizzato correttamente all'interno del modello messo a disposizione dal pacchetto \texttt{BradleyTerry2}.\\ Dopo aver importato il dataset sul software R, sono state perciò necessarie apportare alcune modifiche attraverso la scrittura di codice che andasse a modificare la struttura del dataset per poi essere correttamente utilizzabile nel modello. \\

Innanzitutto il modello richiede per il suo funzionamento che le due variabili che indicano quali delle due squadra hanno partecipato alla partita in esame; devono essere o di tipo fattore oppure un \textsf{data.frame}. Una variabile fattore è un variabile non numerica, espressa in termini verbali ad esempio una categoria. Un \textsf{data.frame} è una lista di vettori, che devono avere tutti la stessa lunghezza, ma possono essere di tipo diverso: variabili nominali cioè fattori, variabili cardinali cioè vettori numerici; un \textsf{data.frame} può essere visto come una matrice ma con il tipo dei valori che può essere diverso.\\ Dato che le due variabili in questione \textsf{Team} e \textsf{Vs} erano solo di tipo \texttt{character} e si voleva inserire un collegamento che faccia capire al modello, quali valori sono legati alla squadra indicata in \textsf{Team} e quali in \textsf{Vs} nella stessa partita; si è deciso perciò di trasformare \textsf{Team} e \textsf{Vs} in \texttt{data.frame} inserendo al loro interno tutte le covariate descritte nella sezione precedente, ad esempio \textsf{Poss}, \textsf{Int} ecc..\\

Si sottolinea inoltre che sono state necessarie ulteriore modifiche per quanto riguarda la variabile \textsf{AtHome}; dato che al momento dell'importazione del dataset, i valori venivano interpretati come stringhe, è stato necessario trasformarli in valori logici con il commando \textsf{as.logical(soccern\$AtHome)}. Ciononostante pero il valore logico non era accettato dal modello ma era accettato un valore numerico per indicare se la squadra giocava in casa o no; si è quindi convertito il valore \texttt{TRUE} in 1 mentre FALSE in 0. 

\subsection{Codice per l'adattamento del dataset}
Di seguito viene mostrato il codice applicato per adeguare il dataset con le modifiche scritte precedentemente.

\begin{lstlisting}
PossVs <- c()
ShVs <- c()
ShTVs <- c()
G.ShVs <- c()
SavesVs <- c()
PAttVs <- c()
PCmp.Vs <- c()
SPAttVs <- c()
SPCmp.Vs <- c()
MPAttVs <- c()
MPCmp.Vs <- c()
LPAttVs <- c()
LPCmp.Vs <- c()
ToDefPenVs <- c()
ToDef3rdVs <- c()
ToMid3rdVs <- c()
ToAtt3rdVs <- c()
ToAttPenVs <- c()
ToDistVs <- c()
FlsVs <- c()
FldVs <- c()
OffVs <- c()
CrsVs <- c()
IntVs <- c()
TklWinVs <- c()
RecovVs <- c()
del <-c()
k <- 1
z <- 1
for(i in 1:nrow(soccern)){
   if(soccern$AtHome[i] == TRUE){
     for(j in 1:nrow(soccern)){
	 if((soccern$Team[j] == soccern$Vs[i]) && (soccern$Team[i] == soccern$Vs[j]) && (soccern$AtHome[j] == FALSE)){
		PossVs[k] <- soccern$Poss[j]
		ShVs[k] <- soccern$Sh[j]
		ShTVs[k] <- soccern$SoT[j]
		G.ShVs[k] <- soccern$G.Sh[j]
		SavesVs[k] <- soccern$Saves[j]
		PAttVs[k] <- soccern$PAtt[j]
		PCmp.Vs[k] <- soccern$PCmp.[j]
		SPAttVs[k] <- soccern$SPAtt[j]
		SPCmp.Vs[k] <- soccern$SPCmp.[j]
		MPAttVs[k] <- soccern$MPAtt[j]
		MPCmp.Vs[k] <- soccern$MPCmp.[j]
		LPAttVs[k] <- soccern$LPAtt[j]
		LPCmp.Vs[k] <- soccern$LPCmp.[j]
		ToDefPenVs[k] <- soccern$ToDefPen[j]
		ToDef3rdVs[k] <- soccern$ToDef3rd[j]
		ToMid3rdVs[k] <- soccern$ToMid3rd[j]
		ToAtt3rdVs[k] <- soccern$ToAtt3rd[j]
		ToAttPenVs[k] <- soccern$ToAttPen[j]
		ToDistVs[k] <- soccern$TotDist[j]
		FlsVs[k] <- soccern$Fls[j]
		FldVs[k] <- soccern$Fld[j]
		OffVs[k] <- soccern$Off[j]
		CrsVs[k] <- soccern$Crs[j]
		IntVs[k] <- soccern$Int[j]
		TklWinVs[k] <- soccern$TklWin[j]
		RecovVs[k] <- soccern$Recov[j]
		k <- k + 1
	   }      
	}
   }else{
	del[z] <- i
	z <- z + 1
   }
}
\end{lstlisting}
\bigskip
Con il codice precedente si ha l'obbiettivo di prendere le due righe di ogni partita e di unirle insieme formando un unica riga per ogni partita. Successivamente si elimineranno le righe delle partite giocate fuori casa (\textsf{AtHome} = FALSE) dalle squadre indicate in \textsf{Team} mentre le righe delle partite giocate in casa (\textsf{AtHome} = TRUE) dalle squadre indicate in \textsf{Team} conteranno il risultato della fusione.\\
Perciò si è creato un vettore vuoto per ogni covariata presente nel dataset, ad eccezione di \textsf{AtHome} che verrà gestita in un modo diverso. Il vettore \texttt{del} è il vettore che tiene traccia di quali righe saranno da eliminare. \texttt{k} è l'indice usato per scorre il dataset per trovare i dati dell'avversario; \texttt{z} l'indice usato per inserire un nuovo elemento nel vettore \texttt{del}.\\
Il primo ciclo \texttt{for} scorre tutto il dataset alla ricerca delle righe con i dati delle partite giocate in casa dalla squadra indicata in \texttt{Team}, infatti al suo interno il primo costrutto \texttt{if} controlla se la partita è in casa per \texttt{Team} se sì, parte un secondo ciclo \texttt{for} che anche esso scorre tutto il dataset per cercare la riga con la partita giocata della squadra indicata in \texttt{Vs}; giocata ovviamente fuori casa. Perciò all'interno del secondo ciclo \texttt{for} vi è un costrutto \texttt{if} che controlla se la j-esima riga si riferisce alla stessa partita indicata nella i-esima riga, se sì allora si salvano tutti i dati nei vettori e si incrementa l'indice \texttt{k}. Se il primo \texttt{if} da esito negativo allora si andrà a inserire l'indice dell'i-esima riga nel vettore \texttt{del} perché contiene informazioni di una partita giocata fuori casa dalla squadra indicata in \textsf{Team} e viene incrementato l'indice di uno \texttt{z}.\\

Di seguito vengono riportati i comandi fatti per applicare le modifiche al dataset.
\bigskip
\begin{lstlisting}
> soccern3 <- soccern2[-del,]
\end{lstlisting}
\bigskip
Con il precedente commando si va a creare un nuovo dataset con 380 righe, eliminando tutte quelle righe con valore \texttt{FALSE} su \textsf{AtHome}. 
\bigskip
\begin{lstlisting}
> soccern3$Team <- data.frame(team = soccern3$Team, GF = soccern3$GF, GA = soccern3$GA,  at.home = 1, Poss = soccern3$Poss, Sh = soccern3$Sh, SoT = soccern3$SoT, G.Sh = soccern3$G.Sh, Saves = soccern3$Saves, PAtt = soccern3$PAtt, PCmp. = soccern3$PCmp., SPAtt = soccern3$SPAtt, SPCmp. = soccern3$SPCmp., MPAtt = soccern3$MPAtt, MPCmp. = soccern3$MPCmp., LPAtt = soccern3$LPAtt, LPCmp. = soccern3$LPCmp., ToDefPen = soccern3$ToDefPen, ToDef3rd = soccern3$ToDef3rd, ToAtt3rd = soccern3$ToAtt3rd, ToAttPen = soccern3$ToAttPen, TotDist = soccern3$TotDist, Fls = soccern3$Fls, Fld = soccern3$Fld, Off = soccern3$Off, Crs = soccern3$Crs, Int = soccern3$Int, TklWin = soccern3$TklWin, Recov = soccern3$Recov)

\end{lstlisting}

\bigskip
Con il precedente commando si va a modificare \textsf{Team} rendendolo un \texttt{data.frame}, andando a inserire i dati della riga relativi alla squadra che gioca in casa. Si inserisce come chiave \texttt{team = soccern3\$Team} e si indica che la partita è in casa per la squadra di riferimento con \texttt{at.home = 1}.
\bigskip
\bigskip
\begin{lstlisting}
> soccern3$Vs <- data.frame(team = soccern3$Vs, GF = GFVs, GA = GAVs, at.home = 0, Poss = PossVs, Sh = ShVs, SoT = ShTVs, G.Sh = G.ShVs, Saves = SavesVs, PAtt = PAttVs, PCmp. = PCmp.Vs, SPAtt = SPAttVs, SPCmp. = SPCmp.Vs, MPAtt = MPAttVs, MPCmp. = MPCmp.Vs, LPAtt = LPAttVs, LPCmp. = LPCmp.Vs, ToDefPen = ToDefPenVs, ToDef3rd = ToDef3rdVs, ToAtt3rd = ToAtt3rdVs, ToAttPen = ToAttPenVs, TotDist = ToDistVs, Fls = FlsVs, Fld = FldVs, Off = OffVs, Crs = CrsVs, Int = IntVs, TklWin = TklWinVs, Recov = RecovVs)

\end{lstlisting}
\bigskip
Con il precedente commando si va a modificare \textsf{Vs} rendendolo un \texttt{data.frame}, andando a inserire i dati della riga relativi alla squadra che gioca fuori casa. Si inserisce come chiave \texttt{team = soccern3\$Vs} e si indica che la partita è fuori casa per la squadra \texttt{Vs} con \texttt{at.home = 0}.\\ Per quanto riguarda il resto dei dati vengo riportati attraverso l'inserimento dei vettori costruiti e riempiti precedentemente.\\

Si segnala inoltre che il dataset non contiene valori mancati dato che, in quei rari casi in cui venivano individui valori mancanti durante la raccolta, veniva ricercato il dato da altre fonti attendibili.

\section{Analisi grafica dei dati}
In questa sezione attraverso il supporto di grafici, si analizzerà graficamente i dati disponibili e le loro relazione per avere una prima visione dei dati raccolti. Si cercherà di: individuare possibili outliers o anomalie, quali distribuzioni hanno i dati ma soprattutto valutare le relazione tra covariate e variabile di risposta e tra due covariate, con lo scopo di individuare quali covariate possono essere significative per la variabile risposta e quali interazioni tra covariate emergono dall'analisi grafica.\\

Come primo passo dell'analisi, viene valutata la distribuzione delle classi della variabile risposta \texttt{Res} all'interno delle osservazioni disponibili. Tale distribuzione è mostrata nella Figura \ref{fig:res}.

\begin{figure}[htbp]
	\begin{center}
		\includegraphics[scale=0.28]{barRes.png}
		\caption{Barplot della distribuzione della variabile di risposta \texttt{Res}} \label{fig:res}
	\end{center}
\end{figure}

Come si può notare le classi sembrano ben distribuite dato che abbiamo 196 pareggi e 282 vittorie e altrettante sconfitte. Si ha quindi un campione abbastanza ampio e distribuito e corretto per le nostre analisi.\\

Aumentando il livello di dettaglio è di interessante analizzare come queste classificazione sono distribuite tra le varie squadre.

\begin{figure}[htbp]
	\begin{center}
		\includegraphics[height=8cm,width=15cm]{ResTeam.png}
		\caption{Barplot della distribuzione della variabile di risposta per squadra\texttt{Res}} \label{fig:team}
	\end{center}
\end{figure}

Nella Figura \ref{fig:team} si può notare come la distribuzione di vittorie, pareggi e sconfitte non è omogenea tra le squadre. Ovviamente è un risultato che ci si aspettava ma che sottolinea prima di tutto la correttezza dei dati ma soprattutto che vi è qualche elemento nascosto che ha determinato tale distribuzione.\\


Come secondo step si analizzerà le relazione tra variabile di risposta con alcune covariate.\\

La prima relazione che si analizza è quella con la variabile categorica \texttt{AtHome}. Nella Figura \ref{fig:AtHome} si può vedere che c'è una leggera variazione dei risultati tra la squadra che gioca in casa oppure no. Infatti c'è una leggera tendenza a favorire la vittoria per la squadra che gioca in casa piuttosto che la vittoria per la squadra fuori casa. Naturalmente non deve esserci alcuna variazione per quanto riguarda il pareggio dato che entrambe le squadre lo ottengono. Risulta perciò significativa la variabile \texttt{AtHome}.\\

\begin{figure}[htbp]
	\begin{center}
		\includegraphics[scale=0.40]{AtHomeRes.png}
		\caption{Mosaicplot che mostra la distribuzione degli esiti rispetto alle partite giocate in casa e fuori casa} \label{fig:AtHome}
	\end{center}
\end{figure}

Analizzando invece la relazione tra variabile di risposta e \texttt{Poss}, dalla Figura \ref{fig:Poss} si nota che tale variabile sembra essere significativa per l'esito. Infatti vi è un relazione positiva dove valori più alti di possesso palla sono registrati nel box della vittoria e ciò può portare a una maggiore probabilità di vittoria. Vi è una buona distribuzione dei dati, infatti le code sono simmetriche mentre vi è una variabilità quasi identica; si segnala solo che la mediana della sconfitta è più vicina al 3$^{\circ}$ quantile mentre quella della vittoria è più vicina al 1$^{\circ}$ quantile. Inoltre non vi sono presenti outliers.

\begin{figure}[htbp]
	\begin{center}
		\includegraphics[scale=0.40]{Poss.png}
		\caption{Boxplot della variabile risposta e della variabile numerica \texttt{Poss} } \label{fig:Poss}
	\end{center}
\end{figure}

La Figura \ref{fig:sot} mostra come si comporta la relazione con \texttt{SoT}. Come ci si aspetta si hanno valori più alti nella vittoria e valori molto più bassi nella sconfitta, si ha una buona distribuzione dei valori nella vittoria dato che le code sono simmetriche, per le altre due classi non c'è simmetria dato che ci sono valori più bassi rispetto a valori più alti. Vi sono inoltre alcuni outliners che si discostano dalla distribuzione di tutte e tre le classi dovuti al fatto che ci sono state squadre che hanno tirato molto in porta. Le mediane dei box pareggio e vittoria non sono equidistanti dai quantili ma più vicine al 1$^{\circ}$ quantile. Il box della sconfitta ha una bassa varianza. In conclusione avere un valore alto di tiri in porta sembra essere significativo ai fini della vittoria. Si segnala inoltre che si ottengono i stessi risultati anche con \texttt{Sh} solo con valori meno alti per la vittoria rispetto a \texttt{SoT}.\\

\begin{figure}[htbp]
	\begin{center}
		\includegraphics[scale=0.40]{SoT.png}
		\caption{Boxplot della variabile risposta e della variabile numerica \texttt{SoT} } \label{fig:sot}
	\end{center}
\end{figure}

La Figura \ref{fig:g} mostra come si comporta la relazione con \texttt{G/Sh}. Si nota che vi sono valori molto bassi ma leggermente più alti per la vittoria. La distribuzione non è buona perché le code sono asimmetriche infatti tutti i valori sono concentrati in basso e pochi verso la coda in alto, per di più c'è una bassa varianza tra i valori. Vi è la presenza di outliners dovuti a partite dove le squadre sono riuscite a ottenere il massimo da ogni tiro. I risultati mostrati nonostante la pessima distribuzione, sono comunque coerenti dato che non ci si aspetta dal rapporto tiri gol un numero alto ma comunque una tendenza che favorisca la vittoria.\\
 
\begin{figure}[htbp]
	\begin{center}
		\includegraphics[scale=0.40]{g.png}
		\caption{Boxplot della variabile risposta e della variabile numerica \texttt{G/Sh} } \label{fig:g}
	\end{center}
\end{figure}

La Figura \ref{fig:saves} mostra come si comporta la relazione con \texttt{Saves}. Come si può notare sembra che tale variabile sia poco significativa ai fini del risultato. Infatti c'è poca variazione tra una classe e l'altra dato che avere un alto numero di parate non è determinante a fini del risultato.\\

\begin{figure}[htbp]
	\begin{center}
		\includegraphics[scale=0.40]{saves.png}
		\caption{Boxplot della variabile risposta e della variabile numerica \texttt{Saves} } \label{fig:saves}
	\end{center}
\end{figure}

La Figura \ref{fig:pass} mostra come si comporta la relazione con \texttt{PAtt} e con \texttt{PCmp\%}. Per entrambi sembra significativo l'alto numero di passaggi tentati ma soprattutto quelli completati ai fini della vittoria. Nel primo boxplot la coda più in alto è più lunga rispetto alla coda in basso, quindi abbiamo valori più concentrati verso il basso che verso l'alto. Sempre nel primo boxplot il box della vittoria ha una maggiore variabilità rispetto ai altri due è varia di più avendo valori più alti; sia la mediana del box vittoria e sia quello del pareggio sono più vicine al 1${^\circ}$ quantile, viceversa quella della sconfitta. I dati nel primo boxplot sembrano essere coerenti con l'esito della partita dato che maggior numero di passaggi si prova ad effettuare maggiori sono le possibilità di vittoria, occorre pero sapere quanto è precisa la squadra.\\
Nel secondo boxplot si notano valori alti e molti outliners con valori bassi dovuti al fatto che ci sono state partite dove alcune squadre sono state poco precise nei passaggi. A differenza del primo boxplot il secondo boxplot ha molti valori alti, infatti la coda in alto e molto meno lunga rispetto alla coda in basso e le variabilità dei box sembrano essere uguali tra di loro; anche qui le code non sono simmetriche e quindi non c'è una buona distribuzione dei dati. Sorprendentemente sembra che avere un buona precisione pero non da la sicurezza di una vittoria, inoltre l'andamento prima scende da sconfitta a pareggio e poi sale da pareggio a vittoria.\\
Per quanto riguarda le variabili delle altre tipologie di passaggi si discostano di poco dai boxplot in Figura \ref{fig:pass}

\begin{figure}[htbp]
	\begin{center}
		\includegraphics[scale=0.50]{pass.png}
		\caption{Boxplot della variabile risposta e della variabile numerica \texttt{PAtt} e \texttt{PCmp\%}  } \label{fig:pass}
	\end{center}
\end{figure}
\bigskip
La Figura \ref{fig:defp} mostra come si comporta la relazione con \texttt{ToDefPen}. Come si può notare questa non è per nulla significativa per la variabile risposta, infatti non c'è una minima variazione e i box hanno tutti la stessa varianza. Tale esito può essere giustificato dal fatto che le squadre cercano di rimane fuori il più possibile dalla propria area di rigore per non portare troppo vicino alla porta l'avversario. Da ciò quindi tale variabile non sarà inserita nel modello.\\

\begin{figure}[htbp]
	\begin{center}
		\includegraphics[scale=0.40]{def.png}
		\caption{Boxplot della variabile risposta e della variabile numerica \texttt{ToDefPen} } \label{fig:defp}
	\end{center}
\end{figure}

La Figura \ref{fig:defp} mostra come si comporta la relazione con \texttt{ToAttPen}. Contrariamente quanto visto con la Figura \ref{fig:defp} qui si nota una certa variazione da una un box e l'altro, infatti vi è una tendenza positiva che porta ad aver valori più alti in caso di vittoria. Si ha una maggior varianza per quanto riguarda la vittoria rispetto ai altri due esiti e la distribuzione di tutti e tre è abbastanza bilanciata se non che la coda più bassa è leggermente meno lunga rispetto all'altra coda; la mediana invece è equilibrata. Si nota inoltre che vi sono alcuni outliners segno che alcune squadre in qualche partite, si sono particolarmente rese note nel produrre un quantitativo di tocchi maggiore rispetto alla distribuzione, ciò pero non sembra influenzare l'esito.\\

Per quanto riguarda \texttt{ToDef3rd}, \texttt{ToMid3rd} e \texttt{ToAtt3rd}, esse si comportano come \texttt{ToAttPen}. Perciò è stato omesso il loro grafico.\\
 
\begin{figure}[htbp]
	\begin{center}
		\includegraphics[scale=0.40]{att.png}
		\caption{Boxplot della variabile risposta e della variabile numerica \texttt{ToAttPen} } \label{fig:att}
	\end{center}
\end{figure}

Nella Figura \ref{fig:falli} vengono mostrati gli andamenti delle variabile dei falli, \texttt{Fls} e \texttt{Fld}. Nel boxplot a sinistra si può notare che i valori più alti sono nel box del pareggio mentre sono presenti valori più bassi nel box vittoria. Ciò fa pensare che subire molti falli può impedire la vittoria alla squadra che li subisce. Per quanto riguarda la distribuzione sembra essere buona; c'è minor varianza per quanto riguarda la sconfitta. \\

\begin{figure}[htbp]
	\begin{center}
		\includegraphics[scale=0.40]{falli.png}
		\caption{A sinistra il boxplot della variabile risposta e della variabile numerica \texttt{Fls} e a destra il boxplot della variabile risposta e della variabile numerica \texttt{Fld} } \label{fig:falli}
	\end{center}
\end{figure}

Nel secondo boxplot si può notare che i valori più alti sono presenti sia sul pareggio e sia sulla vittoria e sempre qui si ha una maggior distribuzione rispetto alla sconfitta. Sembra perciò che dal grafico si può intuire che se la squadra non commette dei falli allora sarà più soggetta a perdere.\\

La Figura \ref{fig:int} mostra come si comporta la relazione con \texttt{Int}. Sorprendentemente valori più alti sono registrati nella sconfitta, anche se la mediana risulta essere più vicina al 1 $^{\circ}$ quantile sottolineando che vi è un maggior numero di valori bassi piuttosto che alti. La mediana dei restanti esiti invece e ben equilibrata ma il pareggio risulta avere meno variabilità. Sembra perciò che effettuare troppi intercettazioni dei passaggi avversari contrariamente da quanto si pensi sembra essere controproducente per la vittoria. Si segnala inoltre la presenza di alcuni outliners con valori alti di intercettazioni, che si discostano dalle distribuzioni.\\

\begin{figure}[htbp]
	\begin{center}
		\includegraphics[scale=0.40]{int.png}
		\caption{Boxplot della variabile risposta e della variabile numerica \texttt{Int}} \label{fig:int}
	\end{center}
\end{figure}

La Figura \ref{fig:tkl} mostra come si comporta la relazione con \texttt{TklWin}. Come si può notare, vincere più contrasti possibili evita di subire una sconfitta. Infatti vi sono valori più alti in pareggio e vittoria oltre a una maggiore varianza rispetto alla sconfitta. Nello specifico pero si nota che, nella distribuzione dei valori vi sono maggior valori alti nella vittoria rispetto al pareggio, graficamente lo si vede dalla mediana che nel pareggio è più vicina al 1$^{\circ}$ quindi a valori più bassi e lo si nota anche dalla coda più bassa che è meno lunga rispetto a quella in alto; invece la mediana della vittoria risulta più vicina al 3$^{\circ}$ oltre ad avere la coda in alto più corta rispetto a quella in basso. Vi è inoltre qualche outliners con valori più alti di contrasti vinti ma sembrano non influenzare la classificazione.\\

\begin{figure}[htbp]
	\begin{center}
		\includegraphics[scale=0.40]{tklwin.png}
		\caption{Boxplot della variabile risposta e della variabile numerica \texttt{TklWin}} \label{fig:tkl}
	\end{center}
\end{figure}

\begin{figure}[htbp]
	\begin{center}
		\includegraphics[scale=0.40]{recov.png}
		\caption{Boxplot della variabile risposta e della variabile numerica \texttt{Recov}} \label{fig:recov}
	\end{center}
\end{figure} 

Infine la Figura \ref{fig:recov} mostra come si comporta la relazione con \texttt{Recov}. Per entrambe le classi la distribuzione sembra più sbilanciata verso valori bassi quindi ad una loro maggior presenza, infatti entrambe le code più in basso sono più corte rispetto a quelle più in alto che sono più lunghe. Per quanto riguarda la mediana sembra per entrambe le classi equidistante dai quantili. Si nota che il pareggio presenta minor varianza rispetto alle altre due classi ma valori più alti soprattutto nei confronti della vittoria oltre ad averne anche di più rispetto alle altre classi. Sembra perciò che un eccessivo numero di recuperi non porti alla vittoria. Si nota inoltre che vi sono numerosi outliners sopratutto per il pareggio.


             % Dati
% !TEX encoding = UTF-8
% !TEX TS-program = pdflatex
% !TEX root = ../tesi.tex

%**************************************************************
\chapter{Analisi dei dati}
%\label{cap:flow engine}
%**************************************************************
\intro{Nel seguente capitolo verrà illustrata la fase di preprocessing e le analisi grafiche dei dati. Le analisi verranno svolte usando il linguaggio di programmazione di R \autocite{R}. }\\

%*************************************************************

\section{Preprocessing dei dati}
Dopo aver importato il dataset utilizzando il linguaggio di programmazione R \autocite{R}, il primo step da effettuare durante il prepocessing è individuare e risolvere possibili anomalie nei dati.
Il dataset è stato importato in modo che la prima riga contenga l'intestazione, mentre le restanti righe tutte le osservazioni. Il comando usato per importare il dataset è il seguente:\\

\begin{lstlisting}[language=R]
> soccer<-read.xlsx("SerieA.xlsx", 1, header=TRUE)
\end{lstlisting}
\bigskip
Il dataset non ha valori mancanti. Questo è stato possibile grazie a FBref che ha messo a disposizione dati quasi sempre completi; in quei rari casi di mancanza di dati sono stati reperiti manualmente da altre fonti altrettanto attendibili.\\ 
Sono state inoltre tolte le variabili \texttt{Date} e \texttt{Round}.\\
Il passo successivo è stato controllare che le variabili fossero interpretate correttamente. \texttt{Team} e \texttt{Vs} vengono interpretate erroneamente come tipo \texttt{character}. \texttt{Team} e \texttt{Vs} devono essere interpretate come un fattore cioè è un valore non numerico, espresso in termini verbali, ad esempio una categoria; quindi ogni squadra sarà un livello del fattore. Analogamente, \texttt{AtHome} è stata fatta trasformata in un fattore a due livelli. Invece, \texttt{Res} è stata trasformata in un fattore ordinato con i livelli: -1 = sconfitta <  0 = pareggio < 1 = vittoria.


\section{Analisi grafica dei dati}
In questa sezione attraverso il supporto di grafici, si analizzerà graficamente i dati disponibili e le loro relazione per avere una prima visione dei dati raccolti. Si valuteranno le relazione tra covariate e la variabile di risposta, le relazioni tra due covariate. Tutto ciò per individuare quali covariate possano essere significative per la variabile risposta e quali interazioni emergono dall'analisi grafica.\\

Come primo passo, è stata valutata la distribuzione della variabile risposta \texttt{Res}, come è mostrato in Figura \ref{fig:res}.

\begin{figure}[htbp]
	\begin{center}
		\includegraphics[scale=0.40]{barRes.png}
		\caption{Barplot della distribuzione della variabile di risposta \texttt{Res}} \label{fig:res}
	\end{center}
\end{figure}

Si può notare come le classi sembrino ben distribuite, dato che abbiamo 196 pareggi e 282 vittorie e altrettante sconfitte. Si ha quindi un campione abbastanza ampio, distribuito e privo di classi povere.\\

 La Figura \ref{fig:team} mostra la distribuzione delle vittorie, dei pareggi e delle sconfitte per ogni squadra.

\newpage
\paperwidth=\pdfpageheight
\paperheight=\pdfpagewidth
\pdfpageheight=\paperheight
\pdfpagewidth=\paperwidth
\headwidth=\textheight

\begingroup 
\vsize=\textwidth
\hsize=\textheight


\pagestyle{empty}
\begin{figure}[htbp]
	\begin{center}
		\includegraphics[height = 15cm, width = 25cm]{ResTeam.png}
		\caption{Barplot della distribuzione della variabile di risposta per squadra\texttt{Res}} \label{fig:team}
	\end{center}
\end{figure}

\textwidth=\hsize
\textheight=\vsize

\endgroup
\newpage
\paperwidth=\pdfpageheight
\paperheight=\pdfpagewidth
\pdfpageheight=\paperheight
\pdfpagewidth=\paperwidth
\headwidth=\textwidth


\subsection{Relazione tra la variabile risposta e le covariate}

La prima relazione che si analizza riguarda la variabile categorica \texttt{AtHome}. Nella Figura \ref{fig:AtHome} viene riportato il mosaicplot tra la variabile risposta e \texttt{AtHome}. Tale grafico è un particolare tipo di diagramma a barre impilate che mostra la relazione che c'è tra due fattori. Il numero di colonne è uguale al numero livelli della variabile inserita sull'asse orizzontale. L'altezza delle barre in verticale, invece, è proporzionale al numero di osservazioni della variabile inserita sull'asse verticale per ciascun livello della variabile nell'asse orizzontale.
In sostanza, il mosaicplot è una rappresentazione grafica di una tabella di contingenza che permette un confronto visivo tra gruppi. Nella Figura \ref{fig:AtHome} c'è una leggera variazione dei risultati tra la squadra che gioca in casa e l'avversaria, infatti per le squadre che giocano in casa, c'è una maggior presenza di vittorie e di minor sconfitte. Naturalmente non c'è alcuna variazione per il pareggio dato che entrambe le squadre lo ottengono.

\begin{figure}[htbp]
	\begin{center}
		\includegraphics[scale=0.45]{AtHomeRes.png}
		\caption{Mosaicplot che mostra la distribuzione degli esiti rispetto alle partite giocate in casa e fuori casa} \label{fig:AtHome}
	\end{center}
\end{figure}

Nella Figura \ref{fig:Poss} viene riportato il boxplot della distribuzione della variabile \texttt{Poss} rispetto ai valori della variabile risposta \texttt{Res}. Il boxplot è un grafico che consente di visualizzare il centro e la distribuzione dei dati. Inoltre, può essere un strumento visivo per la verifica della normalità o per l'identificazione di possibili outlier. Dal grafico si nota che \texttt{Poss} sembra essere significativa per l'esito. Infatti i valori crescono dal boxplot della sconfitta al boxplot della vittoria. C'è una buona distribuzione dei dati perché la lunghezza dei baffi per ogni boxplot è simmetrica. Si segnala che la mediana della sconfitta è più vicina al 3$^{\circ}$ quantile mentre la mediana della vittoria è più vicina al 1$^{\circ}$ quantile. Non sono presenti outliers.\\

\begin{figure}[htbp]
	\begin{center}
		\includegraphics[scale=0.50]{Poss.png}
		\caption{Boxplot della distribuzione della variabile \texttt{Poss} rispetto ai valori della variabile risposta \texttt{Res}} \label{fig:Poss}
	\end{center}
\end{figure}

Nella Figura \ref{fig:sot} viene riportato il boxplot della distribuzione della variabile \texttt{SoT} rispetto ai valori della variabile risposta \texttt{Res}. Valori più alti sono presenti nella vittoria mentre valori molto più bassi sono presenti nella sconfitta. C'è una buona distribuzione dei valori nella vittoria dato che i baffi sono simmetrici, viceversa per le altri due boxplot non c'è simmetria infatti, il baffo inferiore è molto più corto rispetto al baffo superiore, segno che la maggior parte dei valori sono bassi e simili tra loro. Inoltre alcuni outliers si discostano dalla distribuzione di tutti e tre i boxplot, questo perché ci sono state squadre che hanno tirato molte volte in porta. Le mediane dei boxplot pareggio e vittoria non sono equidistanti dai quantili ma più vicine al 1$^{\circ}$ quantile. Il boxplot della sconfitta ha una bassa varianza. In conclusione, avere un valore alto di tiri in porta sembra essere utile ai fini della vittoria.\\

Per la relazione tra la variabile risposta e la variabile \texttt{Sh}, si ha un boxplot molto simile al boxplot mostrato nella Figura \ref{fig:Poss}. Il grafico di \texttt{Sh} rispetto al grafico di \texttt{Poss}, ha degli outliers e la mediana della sconfitta non è equidistante dai quantili ma più vicina al 1$^{\circ}$ quantile.\\

\begin{figure}[htbp]
	\begin{center}
		\includegraphics[scale=0.50]{SoT.png}
		\caption{Boxplot della distribuzione della variabile \texttt{SoT} rispetto ai valori della variabile risposta \texttt{Res} } \label{fig:sot}
	\end{center}
\end{figure}

Nella Figura \ref{fig:g} viene riportato il boxplot della distribuzione della variabile \texttt{G/Sh} rispetto ai valori della variabile risposta \texttt{Res}. Si nota che ci sono valori molto bassi ma leggermente più alti per la vittoria. La distribuzione non è buona perché i baffi sono asimmetrici infatti, tutti i valori sono concentrati in basso e pochi verso il baffo superiore, segno che la maggior parte dei valori sono bassi e simili tra loro. C'è una bassa varianza tra i valori. C'è la presenza di outliers perché alcune squadre sono riuscite a ottenere il massimo da ogni tiro. I risultati mostrati, nonostante la distribuzione, sono comunque coerenti dato che non ci si aspetta dal rapporto tiri-gol un numero alto ma comunque una tendenza che favorisca la vittoria.\\

\begin{figure}[htbp]
	\begin{center}
		\includegraphics[scale=0.50]{g.png}
		\caption{Boxplot della distribuzione della variabile \texttt{G/Sh} rispetto ai valori della variabile risposta \texttt{Res} } \label{fig:g}
	\end{center}
\end{figure}

Nella Figura \ref{fig:saves} viene riportato il boxplot della distribuzione della variabile \texttt{Saves} rispetto ai valori della variabile risposta \texttt{Res}. Come si può notare sembra che \texttt{Saves} sia poco significativa ai fini del risultato. Infatti c'è poca variazione tra un boxplot e l'altro perché sembra che avere un alto numero di parate non è determinante a fini del risultato.\\

\begin{figure}[htbp]
	\begin{center}
		\includegraphics[scale=0.50]{saves.png}
		\caption{Boxplot della distribuzione della variabile \texttt{Saves} rispetto ai valori della variabile risposta \texttt{Res} } \label{fig:saves}
	\end{center}
\end{figure}

La Figura \ref{fig:pass} viene riportato a sinistra il boxplot della variabile numerica \texttt{PAtt} rispetto ai valori della variabile risposta \texttt{Res} e a destra il boxplot della variabile numerica \texttt{PCmp\%} rispetto ai valori della variabile risposta \texttt{Res}. Per entrambi sembra significativo l'elevato numero di passaggi tentati ma soprattutto quelli completati ai fini della vittoria. Nel grafico a sinistra, nel secondo e terzo boxplot il baffo superiore è più lungo rispetto al baffo inferiore, segno che molti valori sono bassi e simili tra loro, viceversa il primo boxplot ha una buona distribuzione perché i baffi sono simmetrici. Il boxplot della vittoria ha una maggiore varianza rispetto agli altri due e in più ha valori più alti; sia la mediana del boxplot della vittoria e sia quello del pareggio sono più vicine al 1$^{\circ}$ quantile, viceversa quella della sconfitta. I dati nel primo boxplot sembrano essere coerenti con l'esito della partita perché, maggior numero di passaggi si prova ad effettuare, maggiori sono le possibilità di vittoria. Occorre però sapere quanto è precisa la squadra e questo lo si può scoprire con la variabile \texttt{PCmp\%}\\

Nel grafico a destra, si notano valori alti e molti outliers con valori bassi dovuti al fatto che ci sono state partite dove alcune squadre sono state poco precise nei passaggi. I baffi superiori di tutti e tre i boxplot sono molto meno lunghi rispetto ai baffi inferiori segno che molti valori sono alti e simili tra loro, inoltre, le varianze dei box sembrano essere uguali tra di loro. Sorprendentemente l'andamento invece di essere sempre crescente, prima scende da sconfitta a pareggio e poi sale da pareggio a vittoria.\\

Per la relazione tra la variabile risposta e la variabile \texttt{SPAtt}, si ha un grafico molto simile al grafico a sinistra della Figura \ref{fig:pass}. Il grafico di \texttt{SPAtt} rispetto al grafico di \texttt{PAtt}, ha un maggior numero di outliers soprattutto per la sconfitta rispetto al grafico \texttt{PAtt} inoltre, c'è una minor varianza per tutti i tre boxplot oltre a valori più bassi in generale, questo è naturale perché \texttt{PAtt} contiene tutti i passaggi tentati e non solo quelli corti.\\

Per la relazione tra la variabile risposta e la variabile \texttt{SPCmp\%}, si ha un grafico molto simile al grafico a destra della Figura \ref{fig:pass}. Il grafico di \texttt{SPCmp\%} rispetto al grafico di \texttt{PCmp\%}, il boxplot della sconfitta ha una maggior varianza, viceversa per la vittoria, che ha una minor varianza.\\

Per la relazione tra la variabile risposta e la variabile \texttt{MPAtt}, si ha un grafico molto simile al grafico a sinistra della Figura \ref{fig:pass}. Il grafico di \texttt{MPAtt} rispetto al grafico di \texttt{PAtt}, il boxplot della sconfitta ha una maggior varianza. In generale i valori sono più bassi rispetto al grafico di \texttt{PAtt} ma questo è naturale perché \texttt{PAtt} contiene tutti i passaggi tentati e non solo quelli medi.\\

Per la relazione tra la variabile risposta e la variabile \texttt{MPCmp\%}, si ha un grafico molto simile al grafico a destra della Figura \ref{fig:pass}. Il grafico di \texttt{MPCmp\%} rispetto al grafico di \texttt{PCmp\%}, ha valori più alti e molti più outliers, inoltre i baffi inferiore dei boxplot della sconfitta e della vittoria sono più corti.\\

Per la relazione tra la variabile risposta e la variabile \texttt{LPAtt}, si ha un grafico molto simile al grafico a sinistra della Figura \ref{fig:pass}. Il grafico di \texttt{LPAtt} rispetto al grafico di \texttt{PAtt}, ha per il boxplot della sconfitta valori più bassi rispetto agli boxplot del pareggio e della vittoria inoltre, il boxplot del pareggio ha una maggior varianza valori mentre il boxplot della vittoria ha una minor varianza.\\
In generale i valori sono più bassi rispetto al grafico di \texttt{PAtt} ma questo è naturale perché \texttt{PAtt} contiene tutti i passaggi tentati e non solo quelli lunghi.\\

Per la relazione tra la variabile risposta e la variabile \texttt{LPCmp\%}, si ha un grafico molto simile al grafico a destra della Figura \ref{fig:pass}. Il grafico di \texttt{LPCmp\%} rispetto al grafico di \texttt{PCmp\%}, ha valori più bassi, la distribuzione dei valori per il boxplot della sconfitta è ben equilibrata perché i baffi sono della stessa lunghezza e in più la mediana è equidistante dai due quantili, analogamente anche il boxplot del pareggio ha una distribuzione equilibrata ma con più varianza e una mediana equidistante dai quantili.\\

\begin{figure}[htbp]
	\begin{center}
		\includegraphics[height=8cm,width=13cm]{pass.png}
		\caption{A sinistra il boxplot della variabile numerica \texttt{PAtt} rispetto ai valori della variabile risposta \texttt{Res} e a destra il boxplot della variabile numerica \texttt{PCmp\%} rispetto ai valori della variabile risposta \texttt{Res}} \label{fig:pass}
	\end{center}
\end{figure}

Nella Figura \ref{fig:defp} viene riportato il boxplot della distribuzione della variabile \texttt{ToDefPen} rispetto ai valori della variabile risposta \texttt{Res}. Si nota che non c'è nessuna variazione dei tre boxplot, oltre ad avere la stessa varianza. L'esito può essere giustificato dal fatto che le squadre cercano di rimane fuori il più possibile dalla propria area di rigore per non portare troppo vicino alla porta l'avversario. Da ciò si può ipotizzare che \texttt{ToDefPen} non è significativa per la variabile risposta. Prima di escluderla si andrà ad analizzare se c'è qualche interazione con altre variabili che la fanno diventare significativa.\\

\begin{figure}[htbp]
	\begin{center}
		\includegraphics[scale=0.50]{def.png}
		\caption{Boxplot della distribuzione della variabile \texttt{ToDefPen} rispetto ai valori della variabile risposta \texttt{Res} } \label{fig:defp}
	\end{center}
\end{figure}

Nella Figura \ref{fig:att} viene riportato il boxplot della distribuzione della variabile \texttt{ToAttPen} rispetto ai valori della variabile risposta \texttt{Res}. Contrariamente quanto visto con la Figura \ref{fig:defp} qui si nota una certa variazione tra i boxplot infatti, i valori crescono dal boxplot della sconfitta fino al boxplot della vittoria. C'è una maggior varianza per il boxplot della vittoria rispetto agli altri due boxplot. Per tutti e tre i boxplot i baffi inferiori sono leggermente meno lunghi rispetto ai baffi superiori, segno che i valori sono bassi e simili tra loro infatti, ci sono alcuni outliers sopra al baffo superiore, segno che alcune squadre in qualche partita, si sono particolarmente rese note nel produrre un quantitativo di tocchi maggiore rispetto alla distribuzione, ciò però non sembra influenzare l'esito. Le mediane sono equidistanti.\\

Per la relazione tra la variabile risposta e la variabile \texttt{ToDef3rd}, si ha un grafico molto simile a quello mostrato nella Figura \ref{fig:att}. Il grafico di \texttt{ToDef3rd} rispetto al grafico di \texttt{ToAttPen}, ha un minore numero di outliers soprattutto per il boxplot del pareggio, tale boxplot ha inoltre una varianza simile al boxplot della sconfitta. Il boxplot della vittoria invece, ha una distribuzione ben equilibrata.\\

Per la relazione tra la variabile risposta e la variabile \texttt{ToMid3rd}, si ha un grafico molto simile a quello mostrato nella Figura \ref{fig:att}. Il grafico di \texttt{ToMid3rd} rispetto al grafico di \texttt{ToAttPen}, ha un minore numero di outliers e la varianza del boxplot della sconfitta è molto simile alla mediana del boxplot del pareggio ma con la mediana più vicina al 3$^{\circ}$ quantile.\\

Per la relazione tra la variabile risposta e la variabile \texttt{ToAtt3rd}, si ha un grafico molto simile a quello mostrato nella Figura \ref{fig:att}. Il grafico di \texttt{ToAtt3rd} rispetto al grafico di \texttt{ToAttPen}, ha una minor varianza in generale per tutti e tre i boxplot e una distribuzione sbilanciata verso valori più bassi dato che tutti i baffi inferiori sono più corti rispetto ai baffi superiori. L'andamento però rimane lo stesso presente nella Figura \ref{fig:att}.\\

\begin{figure}[htbp]
	\begin{center}
		\includegraphics[scale=0.50]{att.png}
		\caption{Boxplot della distribuzione della variabile \texttt{ToAttPen} rispetto ai valori della variabile risposta \texttt{Res} } \label{fig:att}
	\end{center}
\end{figure}

Nella Figura \ref{fig:falli} vengono riportati a sinistra il boxplot della variabile numerica \texttt{Fls} rispetto ai valori della variabile risposta \texttt{Res} e a destra il boxplot della variabile numerica \texttt{Fld} rispetto ai valori della variabile risposta \texttt{Res}. Nel boxplot a sinistra si può notare che i valori più alti sono nel boxplot del pareggio e della vittoria ma nel boxplot del pareggio ci sono più valori alti. Ciò fa ipotizzare che subire molti falli può impedire la vittoria alla squadra che li subisce. Per quanto riguarda la distribuzione sembra essere buona; c'è una minor varianza per quanto riguarda il boxplot della sconfitta. \\

\begin{figure}[htbp]
	\begin{center}
		\includegraphics[height=8cm,width=13cm]{falli.png}
		\caption{A sinistra il boxplot della variabile numerica \texttt{Fls} rispetto ai valori della variabile risposta \texttt{Res} e a destra il boxplot della variabile numerica \texttt{Fld} rispetto ai valori della variabile risposta \texttt{Res}} \label{fig:falli}
	\end{center}
\end{figure}

Nel secondo boxplot si hanno valori valori più alti nel boxplot della vittoria e una maggior varianza rispetto al boxplot della sconfitta. Sembra perciò che dal grafico si può intuire che se la squadra non commette dei falli allora sarà più soggetta a perdere.\\

Per la relazione tra la variabile risposta e la variabile \texttt{Off}, si ha un grafico molto simile a quello mostrato nella Figura \ref{fig:saves}. Il grafico di \texttt{Off} rispetto al grafico di \texttt{Saves}, ha un numero minore di valori per il boxplot della sconfitta rispetto agli altri due boxplot inoltre, le mediane del boxplot della sconfitta e del pareggio sono attaccate al 1$^{\circ}$ quantile.\\

Per la relazione tra la variabile risposta e la variabile \texttt{Crs}, si ha un grafico molto simile a quello mostrato nella Figura \ref{fig:int}. Il grafico di \texttt{Crs} rispetto al grafico di \texttt{Saves}, ha per il boxplot della sconfitta maggior varianza e il baffo inferiore dei boxplot della sconfitta e della vittoria sono più corti rispetto ai baffi superiori.\\

Nella Figura \ref{fig:int} viene riportato il boxplot della distribuzione della variabile \texttt{Int} rispetto ai valori della variabile risposta \texttt{Res}. Sorprendentemente valori più alti sono registrati nel boxplot della sconfitta, anche se la mediana risulta essere più vicina al 1$^{\circ}$ quantile sottolineando che c'è un maggior numero di valori bassi piuttosto che alti. Le mediane dei restanti boxplot invece, sono ben equilibrate ma il boxplot del pareggio risulta avere meno varianza. Sembra perciò che effettuare troppi intercettazioni dei passaggi avversari contrariamente da quanto si pensi sia controproducente per la vittoria. Si segnala inoltre la presenza di alcuni outliers con valori alti di intercettazioni, che si discostano dalle distribuzioni.\\

\begin{figure}[htbp]
	\begin{center}
		\includegraphics[scale=0.50]{int.png}
		\caption{Boxplot della distribuzione della variabile \texttt{Int} rispetto ai valori della variabile risposta \texttt{Res}} \label{fig:int}
	\end{center}
\end{figure}

Nella Figura \ref{fig:tkl} viene riportato il boxplot della distribuzione della variabile \texttt{TklWin} rispetto ai valori della variabile risposta \texttt{Res}. Come si può notare, vincere più contrasti possibili evita di subire una sconfitta. Infatti ci sono valori più alti nei boxplot del pareggio e della vittoria rispetto al boxplot della sconfitta. Nello specifico però si nota che: nella distribuzione ci sono maggior valori alti nella vittoria rispetto al pareggio, graficamente lo si vede dalla mediana che nel boxplot del pareggio è più vicina al 1$^{\circ}$ quindi ha valori più bassi e lo si nota anche dal baffo inferiore che è meno lungo rispetto a quello superiore viceversa, la mediana del boxplot della vittoria risulta più vicina al 3$^{\circ}$ oltre ad avere il baffo superiore più corto rispetto a quello inferiore. C'è inoltre qualche outliers con valori più alti di contrasti vinti ma sembrano non influenzare la classificazione.\\

\begin{figure}[htbp]
	\begin{center}
		\includegraphics[scale=0.50]{tklwin.png}
		\caption{Boxplot della distribuzione della variabile \texttt{TklWin} rispetto ai valori della variabile risposta \texttt{Res}} \label{fig:tkl}
	\end{center}
\end{figure}

\begin{figure}[htbp]
	\begin{center}
		\includegraphics[scale=0.50]{recov.png}
		\caption{Boxplot della distribuzione della variabile \texttt{Recov} rispetto ai valori della variabile risposta \texttt{Res}} \label{fig:recov}
	\end{center}
\end{figure} 

Infine nella Figura \ref{fig:recov} viene riportato il Boxplot della distribuzione della variabile \texttt{Recov} rispetto ai valori della variabile risposta \texttt{Res}. Per entrambi i boxplot la distribuzione sembra più sbilanciata verso valori bassi quindi ad una loro maggior presenza, infatti entrambe i baffi inferiori sono più corti rispetto a quelli superiori. Per quanto riguarda la mediana sembra equidistante dai quantili per entrambi i tre boxplot. Si nota che il boxplot del pareggio presenta minor varianza rispetto agli altri due boxplot ma valori più alti soprattutto nei confronti del boxplot della vittoria. Sembra perciò che un eccessivo numero di recuperi non porti alla vittoria. Si nota inoltre che ci sono numerosi outliers.

\subsection{Analisi possibili interazioni} 
Per concludere l'attività di preprossening, non resta che analizzare le relazioni tra le covariate per individuare possibili interazioni tra di loro che possono influenzare la variabile risposta. Chiaramente dato che ci sono più di trenta variabili e dunque, un grandissimo numero di combinazioni, non si sono esaminate tutte le relazioni ma sono state selezionate solo alcune per l'analisi, basandosi su teorie calcistiche esaminate durante la fase di studio del problema.\\


Per l'analisi delle interazioni si sono utilizzati i grafici di dispersione. Un grafico di dispersione mostra la relazione tra due variabili continue. A tali grafici si è inserito una terza variabile, la variabile risposta \texttt{Res}, dove ogni punto è colorato in tre possibili colori che rappresentano una delle tre categorie di \texttt{Res}. Di conseguenza il grafico permette di visualizzare se le categorie sono ben separati e quindi se un'interazione può spiegare l'andamento dei punti della variabile risposta.\\
Inoltre è stato utilizzato l'indice di correlazione, che indica la forza dell'associazione lineare espressa in valori compresi tra -1 e 1. Tale misura permette di escludere da subito alcune relazioni tra variabili se l'indice è troppo alto o basso, infatti, le relazioni troppo forti vanno escluse perché può presentarsi il fenomeno della collinearità. 
La collinearità è quel fenomeno che va a nasconde il legame tra le variabili e la variabile risposta, a causa di un legame troppo forte tra le covariate.

Nella Figura \ref{fig:cor} viene mostrato il valore della correlazione per ogni possibile relazione tra variabili numeriche. Si nota che ci sono molte relazioni che hanno un valore di correlazione molto vicino a 1, in basso a sinistra del grafico. Ad esempio notiamo che la variabile \texttt{SPCmp\%} ha una relazione molto forte con la variabile \texttt{PCmp\%} (correlazione = 0.82), ciò è coerente perché, la variabile \texttt{SPCmp\%} contiene solo i passaggi corti completati mentre \texttt{PCmp\%} contiene tutti i tipi di passaggi completati, ne consegue che la ridondanza dei dati causa questa alta correlazione. Analogamente la stessa motivazione la si può applicare tra la variabile \texttt{PAtt} e la variabile \texttt{SPAtt} (correlazione = 0.91). Perciò tale motivazione è applicabile a tutte le variabili relative ai passaggi completati o relative ai passaggi tentati.

\begin{figure}[htbp]
	\begin{center}
		\includegraphics[scale=0.47]{Rplot.png}
		\caption{Grafico delle correlazioni di ogni coppia di variabili}  \label{fig:cor}
	\end{center}
\end{figure}

Di seguito si riporteranno le interazioni che sono state individuate come significative.\\ 

	%$\dfrac{\sum_{i=1}^{N}(x_{i}-\tau_{x})(y_{i}-\tau_{y})}{N}$

Sono state individuate le seguenti tre interazioni con la variabile \texttt{Sh}:
\begin{itemize}
	\item Interazione tra la variabile \texttt{Sh} e la variabile \texttt{ToAttPen}. È ragionevole ipotizzare che il numero di tocchi fatti nell'area di rigore avversaria possano creare azioni da tiro. È quindi possibile che tra le due variabili possa esserci una relazione. La Figura \ref{fig:shpen} mostra una relazione  positiva tra le due variabili infatti, quando aumenta la variabile \texttt{Sh} aumenta anche la variabile \texttt{ToAttPen} e viceversa. Sono distinguibili tre differenti gruppi che rappresentano le tre categorie della variabile risposta, inoltre la correlazione tra le due variabili non è troppo alta (0.72). Ne consegue che un'interazione tra la variabile \texttt{Sh} e la variabile \texttt{ToAttPen}, sembra essere significativa rispetto alla variabile risposta.
	\begin{figure}[htbp]
		\begin{center}
			\includegraphics[scale=0.50]{sh-toattpen.png}
			\caption{Scatterplot della distribuzione della variabile \texttt{Sh} rispetto ai valori della variabile \texttt{ToAttPen}}  \label{fig:shpen}
		\end{center}
	\end{figure}

\item Interazione tra la variabile \texttt{Sh} e la variabile \texttt{G/Sh}. È ragionevole ipotizzare che ci sia un legame naturale tra tiri fatti e rapporto tiri-gol. La Figura \ref{fig:shgol} mostra una relazione negativa tra le due variabili infatti, quando aumenta la variabile \texttt{Sh} diminuisce anche la variabile \texttt{G/Sh} e viceversa. Sono distinguibili tre differenti gruppi che rappresentano le tre categorie della variabile risposta infatti, i punti della categoria vittoria sono più in alto mentre i punti delle categorie pareggio e sconfitta più in basso. Inoltre la correlazione tra le due variabili non è bassa (-0.15). Ne consegue che un'interazione tra la variabile \texttt{Sh} e la variabile \texttt{G/Sh}, sembra essere significativa rispetto alla variabile risposta.

\begin{figure}[htbp]
	\begin{center}
		\includegraphics[scale=0.60]{sh-g.sh.png}
		\caption{Scatterplot della distribuzione della variabile \texttt{Sh} rispetto ai valori della variabile \texttt{G/Sh}}  \label{fig:shgol}
	\end{center}
\end{figure}

\item Interazione tra la variabile \texttt{Sh} e la variabile \texttt{Poss}. Generalmente è possibili ipotizzare che il possesso della palla possa favorire nel effettuare i tiri. Infatti, la Figura \ref{fig:shposs} mostra una relazione positiva tra le due variabili, quando aumenta la variabile \texttt{Sh} aumenta anche la variabile \texttt{Poss} e viceversa. Sono distinguibili tre differenti gruppi che rappresentano le tre categorie della variabile risposta, inizialmente i vari punti sono mescolati tra di loro ma, con l'avanzamento emergono le direzioni di ogni categoria infatti, i punti della categoria vittoria vanno più verso destra mentre i punti delle categoria sconfitta si spostano verso l'alto senza tendere verso destra, i punti della categoria pareggio pareggio invece, si muovono in mezzo ai punti delle altre due categorie. La correlazione tra le due variabili non è alta (0.51). Ne consegue che un'interazione tra la variabile \texttt{Sh} e la variabile \texttt{Poss}, sembra essere significativa rispetto alla variabile risposta.

\begin{figure}[htbp]
	\begin{center}
		\includegraphics[scale=0.60]{poss-sh.png}
		\caption{Scatterplot della distribuzione della variabile \texttt{Sh} rispetto ai valori della variabile \texttt{Poss}}  \label{fig:shposs}
	\end{center}
\end{figure}
\end{itemize}


Sono state individuate le seguenti tre interazioni con la variabile \texttt{ToMid3rd}:
\begin{itemize}
	\item Interazione tra la variabile \texttt{ToMid3rd} e la variabile \texttt{LPAtt}. Si suppone che tra le due variabili ci sia una relazione perché molti lanci lunghi per le punte partono proprio del centrocampo. La Figura \ref{fig:midl} mostra un andamento un po' a "nuvola" ma comunque, è possibile individuare una relazione positiva tra le due variabili infatti, quando aumenta la variabile \texttt{ToMid3rd} aumenta anche la variabile \texttt{LPAtt} e viceversa. Sono distinguibili tre differenti gruppi che rappresentano le tre categorie della variabile risposta, inizialmente i vari punti sono mescolati tra di loro ma, successivamente i punti della categoria vittoria vanno molto in alto mentre i punti della categoria sconfitta rimangono molto più bassi muovendosi verso destra, invece i punti della categoria pareggio anche essi vanno verso destra ma rimanendo più alti rispetto ai punti della categoria sconfitta. La correlazione tra le due variabili non è alta (0.45). Ne consegue che un'interazione tra la variabile \texttt{ToMid3rd} e la variabile \texttt{LPAtt}, sembra essere significativa rispetto alla variabile risposta.
	\begin{figure}[htbp]
		\begin{center}
			\includegraphics[scale=0.60]{mid-lpatt.png}
			\caption{Scatterplot della distribuzione della variabile \texttt{ToMid3rd} rispetto ai valori della variabile \texttt{LPAtt}}  \label{fig:midl}
		\end{center}
	\end{figure}
	
	\item Interazione tra la variabile \texttt{ToMid3rd} e la variabile \texttt{PCmp\%}. Per le stesse ragioni illustrate nel punto precedente si ipotizza una relazione tra le variabili. La Figura \ref{fig:midp} mostra una relazione positiva tra le due variabili infatti, quando aumenta la variabile \texttt{ToMid3rd} aumenta anche la variabile \texttt{PCmp\%}, con un'andamento simile ad una funzione esponenziale. Sono distinguibili tre differenti gruppi che rappresentano le tre categorie della variabile risposta, dove i punti più in alto sono della categoria del pareggio, leggermente più sotto ci sono i punti della vittoria che pero verso la fine del grafico raggiungono i valori più alti, e infine i punti della sconfitta. La correlazione tra le due variabili non è alta (0.66). Ne consegue che un'interazione tra la variabile \texttt{ToMid3rd} e la variabile \texttt{PCmp\%}, sembra essere significativa rispetto alla variabile risposta.
	
	\begin{figure}[htbp]
		\begin{center}
			\includegraphics[scale=0.60]{mid-pcmp.png}
			\caption{Scatterplot della distribuzione della variabile \texttt{ToMid3rd} rispetto ai valori della variabile \texttt{PCmp\%}}  \label{fig:midp}
		\end{center}
	\end{figure}
\end{itemize}


Infine sono state individuate le seguenti interazioni:
\begin{itemize}
	\item Interazione tra la variabile \texttt{TotDist} e la variabile \texttt{PCmp\%}. Naturalmente per effettuare i passaggi e completarli è possibile farlo solo se ci si muove con la palla. La Figura \ref{fig:totdistpcmp} mostra una relazione positiva tra le due variabili infatti, quando aumenta la variabile \texttt{TotDist} aumenta anche la variabile \texttt{PCmp\%}, con un'andamento simile ad una funzione esponenziale. Sono distinguibili tre differenti gruppi che rappresentano le tre categorie della variabile risposta, dove i punti più in alto sono della categoria del pareggio, leggermente più sotto ci sono i punti della vittoria e infine i punti della sconfitta. La correlazione tra le due variabili non è troppo alta (0.75). Ne consegue che un'interazione tra la variabile \texttt{TotDist} e la variabile \texttt{PCmp\%}, sembra essere significativa rispetto alla variabile risposta.
	
	\begin{figure}[htbp]
		\begin{center}
			\includegraphics[scale=0.50]{TotDist-PCmp.png}
			\caption{Scatterplot della distribuzione della variabile \texttt{TotDist} rispetto ai valori della variabile\texttt{PCmp\%}}  \label{fig:totdistpcmp}
		\end{center}
	\end{figure}
	\item Interazione tra la variabile \texttt{PAtt} e la variabile \texttt{PCmp\%}. Data la loro naturale correlazione si ipotizza che ci sia un'interazione. La Figura \ref{fig:pp} mostra una relazione positiva tra le due variabili infatti, quando aumenta la variabile \texttt{PAtt} aumenta anche la variabile \texttt{PCmp\%}, con un'andamento simile ad una funzione esponenziale. Sono distinguibili tre differenti gruppi che rappresentano le tre categorie della variabile risposta, dove i punti più in alto sono della categoria del pareggio, leggermente più sotto ci sono i punti della vittoria e infine i punti della sconfitta. La correlazione tra le due variabili non è troppo alta (0.74). Ne consegue che un'interazione tra la variabile \texttt{PAtt} e la variabile \texttt{PCmp\%}, sembra essere significativa rispetto alla variabile risposta.
	
	\begin{figure}[htbp]
		\begin{center}
			\includegraphics[scale=0.50]{PAtt-PCmp.png}
			\caption{Scatterplot della distribuzione della variabile \texttt{PAtt} rispetto ai valori della variabile \texttt{PCmp\%}}  \label{fig:pp}
		\end{center}
	\end{figure}

\item Interazione tra la variabile \texttt{ToDefPen} e la variabile \texttt{ToAttPen}. Come ci si può aspettare la Figura \ref{fig:defatt} mostra una relazione negativa tra le due variabili, quando aumenta la variabile \texttt{ToDefPen} diminuisce anche la variabile \texttt{ToAttPen} e viceversa. Sono distinguibili tre differenti gruppi che rappresentano le tre categorie della variabile risposta infatti, i punti della categoria vittoria sono quelli più distanti dallo zero mentre i punti delle categorie pareggio e sconfitta sono più vicini allo zero. Inoltre la correlazione tra le due variabili non è bassa (-0.45). Ne consegue che un'interazione tra la variabile \texttt{ToDefPen} e la variabile \texttt{ToAttPen}, sembra essere significativa rispetto alla variabile risposta.

\begin{figure}[htbp]
	\begin{center}
		\includegraphics[scale=0.60]{def.att.png}
		\caption{Scatterplot della distribuzione della variabile \texttt{ToDefPen} rispetto ai valori della variabile \texttt{ToAttPen}}  \label{fig:defatt}
	\end{center}
\end{figure}

\end{itemize}

\begin{comment}
	\subsection{Collinearità}
	Per collinearità si intende quel fenomeno per il quale se più variabili esplicative altamente correlate vengono inserite nel modello, allora la loro alta correlazione andrà a nasconde la loro associazione con la variabile risposta. La soluzione per risolvere questo problema è quella di scegliere soltanto una sola variabile della relazione da inserire nel modello.\\
	Nella Figura \ref{fig:cor} viene mostrato il valore della correlazione per ogni possibile interazione tra variabile numeriche.\\
	Come si può notare una maggior correlazione tra le variabile è concentra nella prima parte del triangolo. Dal grafico possiamo vedere come tutte le interazioni che sono state descritte nella sottosezione precedente abbiano un alta correlazione ma non eccessivamente alta. Si hanno i seguenti valori di correlazione:
	\begin{itemize}
		\item Le interazioni con la variabile \texttt{Sh}:
		\begin{itemize}
			\item L'interazione tra \texttt{Sh} e \texttt{SoT} ha come valore 0,70 tale da giustificare l'inserimento dell'interazione.
			\item L'interazione tra \texttt{Sh} e \texttt{ToAtt3rd} ha come valore 0,72 tale da giustificare l'inserimento dell'interazione.
			\item L'interazione tra \texttt{Sh} e \texttt{ToAttPen} ha come valore 0,82 è un valore alto con un possibile rischio di collinearità. Tale valore però giustifica l'inserimento dell'interazione.
		\end{itemize}
		\item Le interazioni con la variabile \texttt{Poss}:
		\begin{itemize}
			\item L'interazione tra \texttt{Poss} e \texttt{PAtt} ha come valore 0,88 è un valore molto alto con un possibile rischio di collinearità. Tale valore però giustifica l'inserimento dell'interazione.
			\item L'interazione tra \texttt{Poss} e \texttt{PAtt} ha come valore 0,81 è un valore alto con un possibile rischio di collinearità. Tale valore però giustifica l'inserimento dell'interazione.
		\end{itemize}
		\item Le interazioni con la variabile \texttt{TotDist}:
		\begin{itemize}
			\item L'interazione tra \texttt{TotDist} e \texttt{PAtt} ha come valore 0,87 è un valore molto alto con un possibile rischio di collinearità. Tale valore però giustifica l'inserimento dell'interazione.
			\item L'interazione tra \texttt{TotDist} e \texttt{PCmp\%} ha come valore 0,75 tale da giustificare l'inserimento dell'interazione.
		\end{itemize}
		
		\item L'interazione tra \texttt{ToAtt3rd} e \texttt{PAttPen} ha come valore 0,79 tale da giustificare l'inserimento dell'interazione.
		\item L'interazione tra \texttt{PAtt} e \texttt{PCmp\%} ha come valore 0,74 tale da giustificare l'inserimento dell'interazione.
	\end{itemize}  
	
	
	Nella sottosezione precedente si poteva pensare di inserire interazioni abbastanza naturali ad esempio: \texttt{PAtt}*\texttt{SPAtt}, \texttt{PAtt}*\texttt{MPAtt} e, \texttt{PCmp\%}*\texttt{MPCmp\%} e \texttt{PAtt}*\texttt{LPCmp\%}. Tali interazioni però sono composte da variabili che hanno un alta correlazione tra loro, e quindi si ha il rischio di incombere in un problema di collinearità. Una correlazione cosi alta era prevedibile dato che c'è una ridondanza dei dati tra le variabili. In questa fase dell'analisi non si hanno abbastanza elementi per poter scegliere quale variabile tenere e quale no perciò tale scelta verrà rinviata alla fase di modellazione.\\
	Il grafico suggerisce alcune interazioni che non sono state descritti ad esempio:\\ \texttt{Poss}*\texttt{ToAtt3rd}, \texttt{Poss}*\texttt{SPAtt},  \texttt{TotDist}*\texttt{ToAtt3rd}, \texttt{PCmp\%}*\texttt{SPAtt} e \texttt{PCmp\%}*\texttt{MPAtt}.
	Tali interazioni saranno analizzate durante la fase di modellazione per verificare se effettivamente sono significative per il modello.\\
	Infine si nota una buona correlazione tra \texttt{ToDefPen} e \texttt{ToDef3rd}, l'interazione può essere inserita perché va a giustificare il fatto che la variabile \texttt{ToDefPen} combinata con \texttt{ToDef3rd} diventa significativa per la variabile risposta....
\end{comment}




%\begin{figure}[htbp]
%	\begin{center}
%		\includegraphics[scale=0.70]{cov.png}
%		\caption{Grafico riassuntivo delle variabili rimaste dopo il Prepossesing}  \label{fig:cov}
%	\end{center}
%\end{figure}
\begin{comment}
	

\section{Ulteriori modifiche del dataset}

Nelle sezioni precedenti si è descritto come si è costruito il dataset e come esso è stato strutturato. Tale struttura ha il vantaggio di rendere il dataset di facile interpretazione, ma deve essere riadattato per poter utilizzare le funzioni del pacchetto \texttt{BradleyTerry2}.\\ 
Sono state apportare le seguenti modifiche.\\

Innanzitutto il modello richiede che le due variabili \texttt{Team} e \texttt{Vs} siano di tipo fattore oppure costituiscano un \textsf{data.frame}. Un \textsf{data.frame} è una raccolta di vettori di osservazioni, che devono avere tutti la stessa lunghezza, ma possono essere di tipo diverso: variabili nominali (fattori) o variabili numeriche.
Le variabili \textsf{Team} e \textsf{Vs} sono state trasformate in \texttt{data.frame} in modo da poter inserire al loro interno tutte le covariate descritte nella sezione precedente.

Inoltre  i valori della variabile \texttt{AtHome} sono stati converti in 1 (se \texttt{TRUE}) mentre in 0 (se \texttt{FALSE}).

\end{comment}






             % Analisi grafica
% !TEX encoding = UTF-8
% !TEX TS-program = pdflatex
% !TEX root = ../tesi.tex
%**************************************************************
\chapter{Il modello Bradley-Terry}
%\label{cap:archittettura del sistema AWMS}
%**************************************************************

\intro{Nel seguente capitolo verranno introdotti differenti modelli per il confronto a coppie, iniziando con il modello Bradley-Terry versione base fino a presentare tutte le sue estensioni usate per l'analisi trattata. Infine verrà illustrata la penalizzazione applicata.}


\section{Modello Bradley-Terry base}
Il modello Bradley-Terry \autocite{bradley1952rank} è un modello probabilistico che permette di predire il risultato di un confronto a coppie, dove per confronto a coppie si intende un processo di comparazione tra due oggetti $\alpha_{i}$ e $\alpha_{j}$ con i,j $\in$ \{1,...,n\}, appartenenti a un set di oggetti \{$\alpha_{1},....,\alpha_{n}$\}, dove gli oggetti sono l'entità che vengono confrontate tra di loro.

Formalmente, dato un set di oggetti \{$\alpha_{1},....,\alpha_{n}$\}, un set di parametri $ \{\gamma_{1}, ..., \gamma_{n}\}$ che rappresentano ciascuno l'abilità/forza del i-esimo oggetto e la variabile casuale associata al risultato del confronto a coppie Y$_{i,j}$ con i<j $\in$ \{1,...,n\}, la probabilità che il risultato sia $\alpha_{i}$ $\succ$ $\alpha_{j}$ è

\begin{align} 
	P(\alpha_{i} \succ \alpha_{j}) = P(Y_{i,j} = 1) = \frac{exp(\gamma_{i} - \gamma_{j})}{1 + exp(\gamma_{i} - \gamma_{j})} \label{for:3.1} 
\end{align}

Il risultato $\alpha_{i}$ $\succ$ $\alpha_{j}$ può essere letto come "l'oggetto $\alpha_{i}$ è preferito all'oggetto $\alpha_{j}$","$\alpha_{i}$ batte l'oggetto $\alpha_{j}$" oppure "$\alpha_{i}$ è migliore dell'oggetto $\alpha_{j}$". La variabile casuale è di tipo binario cioè Y$_{i,j}$ = 1 se l'oggetto $\alpha_{i}$ è preferito all'oggetto $\alpha_{j}$ e Y$_{i,j}$ = 0 se l'oggetto $\alpha_{j}$ è preferito all'oggetto $\alpha_{i}$. I parametri $\gamma_{i}$ sono stimati dal modello attraverso la massima verosimiglianza.
È necessario imporre un vincolo per identificare gli oggetti. Tali vincoli possono essere, il vincolo di somma $ \sum_{i=1}^{n} \gamma_{i} = 0 $ oppure il vincolo dell'oggetto di riferimento.
Per il vincolo dell'oggetto di riferimento si intende che viene fissato $\gamma_{i} = 0$ per un oggetto $\alpha_{i}$ $\in$ \{1, ..., n\}, mentre il valore dei parametri $\gamma_{j}$ degli altri oggetti $\alpha_{j}$ sarà la differenza rispetto all'oggetto di riferimento $\alpha_{i}$.\\
Il modello precedentemente descritto è chiamato modello non strutturato, inoltre il modello base non considera covariate e, in generale, non presta alcuna attenzione all'eterogeneità causata dai soggetti dei confronti.\\

Il modello può essere alternativamente espresso in forma di logit lineare:

\begin{align}
		logit(\alpha_{i} \succ \alpha_{j}) =  log\left( \frac{P( \alpha_{i} \succ \alpha_{j})}{P( \alpha_{j} \succ \alpha_{i})} \right) = log\left(\frac{exp(\gamma_{i})}{exp(\gamma_{j})}\right) = \gamma_i - \gamma_j \label{for:4.1}
\end{align}


\section{Modello Bradley-Terry con categorie di risposta ordinate}\label{sez:4.2}
In molti contesti di comparazione tra oggetti, è possibile che sia richiesto di dare una scala di preferenza tra un oggetto e un altro, ossia la variabile casuale deve avere K possibili categorie di risposta con K > 2. Inoltre tali scelte devono avere un ordine di preferenza, dal risultato meno gradevole al più gradevole per l'i-esimo oggetto, ad esempio si preferisce il pareggio piuttosto che perdere. Perciò il modello \ref{for:4.1} che ha una variabile casuale binaria non è adeguato. 

%Supponiamo che due oggetti $\alpha_{i}$ e $\alpha_{j}$ siano confrontati e che la preferenza ora non sia più espressa i termini di: preferisco $\alpha_{i}$ al posto di $\alpha_{j}$ o viceversa ma, attraverso una scala di preferenza, ad esempio dando una forte preferenza a $\alpha_{i}$ rispetto a $\alpha_{j}$ o una leggera preferenza a $\alpha_{i}$ rispetto a $\alpha_{j}$ o non dando nessuna preferenza o preferendo leggermente $\alpha_{j}$ rispetto a $\alpha_{i}$ oppure preferire fortemente $\alpha_{j}$ rispetto a $\alpha_{i}$. 
%Quindi dal modello descritto nella precedente sezione si passa da due classi di preferenza a cinque classi di preferenza.\\
Avere K categorie di risposta ordinate con K > 2 è di interesse per le comparazioni calcistiche dato che non è sufficiente stimare la probabilità di vittoria o sconfitta ma deve essere obbligantemente preso in considerazione anche il pareggio come risultato.\\

Modelli che consentono un numero generale di categorie K, sono stati proposti da \autocite{bradley1952rank} a \autocite{tutz1986bradley} . In particolare \autocite{tutz1986bradley} mostrò come due modelli per l'analisi di dati ordinati possono essere adattati per i confronti a coppie.\\

Il primo modello presentato è detto a collegamento cumulativo e sfrutta la rappresentazione nelle variabili latenti. In generale, data la variabile continua casuale latente $Z_{i,j}$ sia K il numero di gradi della scala di preferenza e siano $\theta_{1} $ < $\theta_{2}$ < .... < $\theta_{K-1}$ le soglie tale che Y$_{i,j} = k$ quando $\theta_{k-1} < Z_{i,j} < \theta_{k}$. Allora:
\begin{align}
	P(Y_{i,j}\leq k) =  \frac{exp(\theta_{k} + \gamma_{i} - \gamma_{j})}{1 + exp(\theta_{k} + \gamma_{i} - \gamma_{j})} \label{for:3.2.1}
\end{align}

con k $\in$ \{1,....,K\} che indica le possibili categorie di risposta. I parametri $\theta_{k}$ rappresentano le cosiddette soglie per le singole categorie di risposta, che determinano la preferenza per le specifiche categorie. In particolare, Y$_{i,j} = 1$ rappresenta la massima preferenza per un oggetto \textit{i} rispetto a un oggetto \textit{j}.\\
In generale vi è imposta una simmetria del modello in modo che valga: $P(Y_{i,j} = k) = P(Y_{j,i} = K - k + 1)$. È quindi necessario che le soglie siano ristrette a $\theta_{k}$ = -$\theta_{K-k}$ e se, K è dispari, $\theta_{K/2}$ = 0; per garantire che le probabilità siano simmetriche cioè il risultato opposto abbia la stessa probabilità di verificarsi. Per garantire che le probabilità siano non negative per le singole categorie di risposta vi è imposta la seguente limitazione: $-\infty$ = $\theta_{0} < \theta_{1} < ... < \theta_{K-1} < \theta_{K} = \infty$. Dato che la soglia per l'ultima categoria è fissata a $\theta_{K} = \infty$ allora $P(Y_{i,j} \leq K)$ = 1. Si sottolinea che le soglie sono parametri che vanno stimate dai dati. Inoltre, la probabilità di una singola categorie di risposta può essere derivata dalla differenza tra categorie adiacenti cioè:
\begin{center}
	  $P(Y_{i,j} = k)$ = $P(Y_{i,j} \leq k)$ - $P(Y_{i,j} \leq k - 1)$.
\end{center}

Il modello ha anche una rappresentazione logit lineare ed è la seguente:
\begin{align}
	logit(Y_{i,j}\leq k) =  \theta_{k} + \gamma_i - \gamma_j 
\end{align}

Il secondo modello invece proposto da \autocite{agresti1992analysis} è detto modello a categorie adiacenti. In questo caso il collegamento è applicato alle probabilità di risposte adiacenti piuttosto che alle probabilità cumulative, riducendosi così al modello Bradley-Terry quando sono consentite solo due categorie mentre quando sono consentite solo tre categorie, si riduce al modello proposto da \autocite{davidson1970extending} che verrà presentato di seguito al prossimo paragrafo.\\
Il modello a categorie adiacenti è più semplice da interpretare rispetto ai modelli a collegamenti cumulativi poiché la probabilità si riferisce a un determinato risultato anziché a raggruppamenti di risultati. \\
Perciò dal modello proposto da \autocite{davidson1970extending}, sia $\theta$ il parametro stimato dai dati che indica quanto è auspicabile la non preferenza, allora:

\begin{align}
	P(Y_{i,j} = 2 | Y_{i,j} \not = 0) =  \frac{exp(\gamma_{i} - \gamma_{j})}{1 + exp(\gamma_{i} - \gamma_{j})}, \label{for:4.5}
\end{align}
	
\begin{align}
	P(Y_{i,j} = 1) =  \frac{\theta \sqrt{exp(\gamma_{i}) * exp(\gamma_{j})}}{exp(\gamma_{i}) + exp(\gamma_{j}) + \theta\sqrt{exp(\gamma_{i}) * exp(\gamma_{j})}}, \label{for:4.6}
\end{align}

\begin{align}	
	P(Y_{i,j} = 0 | Y_{i,j} \not = 1) =  \frac{exp(\gamma_{j} - \gamma_{i})}{1 + exp(\gamma_{j} - \gamma_{i})}\label{for:4.7}
\end{align}

Si è riportato la modellazione di tutti e tre i possibili risultati, con $\gamma_{n}$ che rappresenta la forza degli oggetti in comparazione. La probabilità che l'oggetto $\alpha_{i}$ batta l'oggetto $\alpha_{j}$ è rappresentata da \hyperref[for:4.5]{(4.5)}, mentre la probabilità che l'oggetto $\alpha_{j}$ batta l'oggetto $\alpha_{i}$ è rappresentata da  \hyperref[for:4.7]{(4.7)}. Sia \hyperref[for:4.5]{(4.5)} e sia \hyperref[for:4.7]{(4.7)} rimangono uguali alla probabilità \hyperref[for:3.1]{(4.2)} descritta precedentemente. Invece, per la probabilità che l'oggetto $\alpha_{i}$ pareggi con l'oggetto $\alpha_{j}$ \hyperref[for:4.6]{(4.6)}, viene aggiunto il parametro $\theta$. Il parametro $\theta$ rappresenta quanto è auspicabile il pareggio. \\

\section{Bradley–Terry Model con effetti dell'ordine} \label{sez:4.3}
Nel modello descritto nella sezione \ref{sez:4.2}, è necessario imporre la simmetria tra le categorie di risposta. Purtroppo la simmetria imposta risulta essere non adeguata in alcuni contesti. Tra questi vi è anche il calcio poiché l'ordine dei oggetti (le squadre) conta. Infatti in una partita di calcio, la prima squadra che viene indicata tra le due squadre, è quella che gioca in casa, dove teoricamente dovrebbe avere un vantaggio sull'avversario. Perciò, il presupposto che le categorie di risposta siano simmetriche non vale più. \\
Un possibile modello riadattato al problema esposto è il seguente:

\begin{align} 
	P(\alpha_{i}\succ \alpha_{j}) = P(Y_{i,j} = 1) = \frac{exp(\delta + \gamma_{i} - \gamma_{j})}{1 + exp(\delta + \gamma_{i} - \gamma_{j})} \label{for:3.8}. 
\end{align}

L'effetto dell'ordine (il vantaggio di giocare in casa in ambito calcistico) viene trattato come un parametro $\delta$. Se $\delta$ > 0 allora viene attribuito un vantaggio all'oggetto $\alpha_{i}$; aumentando la probabilità che vinca il confronto o nel caso di categorie di risposta ordinate, di avere un risultato superiore rispetto all'oggetto $\alpha_{j}$. Chiaramente il peso di $\delta$ deve essere stimato dai dati.\\

Invece un modello con categorie di risposta ordinate riadatto è il seguente:

\begin{align}
	P(Y_{i,j}\leq h) =  \frac{exp(\delta + \theta_{h} + \gamma_{i} - \gamma_{j})}{1 + exp(\delta + \theta_{h} + \gamma_{i} - \gamma_{j})} \label{for:3.9}
\end{align}

Il modello \hyperref[for:3.8]{(4.8)} e il modello \hyperref[for:3.9]{(4.9)} , hanno anche una rappresentazione logit lineare e sono le seguenti:\\

Per \hyperref[for:3.8]{(3.8)}

\begin{align}
	logit(\alpha_{i} \succ \alpha_{j}) =  \delta + \gamma_i - \gamma_j 
\end{align}

Per \hyperref[for:3.9]{(3.9)}

\begin{align}
	logit(\alpha_{i} \succ \alpha_{j}) =  \delta + \theta_{h} + \gamma_i - \gamma_j 
\end{align}
	
\section{Bradley–Terry Model con variabili esplicative}
È stato presentato un modello che valutasse il grado di preferenza per un oggetto $\alpha_{i}$ rispetto a un oggetto $\alpha_{j}$, senza considerare nessuna covariata. Tale modello risulta essere inutile, dato che siamo interessati a capire quali covariate possono influenzare il risultato della comparazione. Prima di esporre il modello con covariate, è necessario fare una distinzione tra soggetti e oggetti e successivamente distinguere i tre tipi di covariate di un confronto a coppie, ovvero: le covariate specifiche al soggetto $x_p$, le covariate specifiche all'oggetto $z_i$ e infine le covariate specifiche al soggetto e all'oggetto $z_pi$ per i soggetti \emph{p}, \emph{p} = 1,.....,m e gli oggetti $\alpha_{i}$, \emph{i} = 1,....,n.\\
Gli oggetti sono le entità che vengono confrontate in un confronto a coppie. I soggetti invece, sono le unità che stabiliscono la preferenza tra gli oggetti in un confronto a coppie. Nel calcio gli oggetti sono le squadre di calcio, mentre i soggetti sono le partite di calcio dove avviene la comparazione tra le squadre.\\

Di seguito vengono illustrate le tre tipologie di covariate in un confronto a coppie:
\begin{itemize}
	\item \texttt{specifiche al soggetto}: Caratterizzano i soggetti che eseguono i confronti tra oggetti, e quindi queste covariate variano solo tra soggetti. Ad esempio nel calcio, covariate come il numero spettatori o il meteo sono specifiche al soggetto. Perciò, sia $x_p$ un vettore di covariate specifiche al soggetto, $\beta_i$ il peso stimato delle covariate per ogni oggetto $\alpha_{i}$ e $\beta_{i0}$ l'intercetta, allora l'abilità $\gamma_{pi}$ dell'oggetto $\alpha_{i}$ nel soggetto \emph{p} sarà
	\begin{center}
		$ \gamma_{pi}$ = $\beta_{i0} + x^{T}_{p}\beta_i$.
	\end{center}

	Con l'inclusione di covariate specifiche al soggetto, il modello è in grado di spiegare l'eterogeneità sui soggetti. Le covariate specifiche al soggetto nei confronti a coppie sono state considerate da \autocite{francis2010} a \autocite{Turner2012Firth}.
	\item \texttt{specifiche all'oggetto}: Caratterizzano gli oggetti che vengono confrontati ma, non variano tra i soggetti ma tra gli oggetti. Nel calcio una covariata specifica all'oggetto può essere il valore di mercato della rosa della squadra di calcio. Un loro utilizzo lo si può trovare in \autocite{schauberger2017}.
	Perciò, sia $z_{i}$ un vettore di covariate specifiche all'oggetto, $\tau$ il peso uguale per tutti gli oggetti e $\beta_{i0}$ l'intercetta,  
	allora l'abilità $\gamma_{i}$ dell'oggetto $\alpha_{i}$ sarà
	\begin{center}
		$\gamma_{pi}$ = $ \gamma_{i}$ = $\beta_{i0} + z^{T}_{i}\tau$.
	\end{center}
	Il peso $\tau$ è un parametro globale, che insieme a $z_{i}$ rappresenta l'abilità spiegata delle covariate mentre $\beta_{i0}$ rappresenta la parte dell'abilità non spiegata dalle covariate. 
	\item \texttt{specifiche al soggetto e all'oggetto}: Questi tipi di covariate possono variare sia per oggetti e sia per i soggetti, ad esempio nel calcio il possesso palla è una covariata che varia per ogni singola squadra e per ogni singola partita. Tali variabili vengono approfondite da \autocite{thurner2000policy} a \autocite{mauerer2015modeling}. Perciò, sia 
	$z_pi$ un vettore di covariate specifiche al soggetto e all'oggetto, $\eta_i$ il peso stimato delle covariate per ogni oggetto, $\beta_{i0}$ l'intercetta, allora l'abilità $\gamma_{pi}$ dell'oggetto $\alpha_{i}$ nel soggetto \emph{p} sarà
		\begin{center}
		$ \gamma_{pi}$ = $\beta_{i0} + z^{T}_{pi}\eta_i$.
	\end{center}
	Contrariamente alle coviariate specifiche al soggetto, le covariate specifiche al soggetto e all'oggetto posso essere modellate con un effetto globale, quindi $\gamma_{pi}$ sarà
		\begin{center}
			$ \gamma_{pi}$ = $\beta_{i0} + z^{T}_{pi}\tau$
		\end{center}
	 dove $\tau$ rappresenta il peso stimato delle covariate. Come si può notare il parametro $\tau$ non ha alcun indice, questo perché l'effetto della covariate è uguale su tutti gli oggetti.
\end{itemize}






Nei vari punti presentati precedentemente, veniva aggiunto il parametro $\beta_{i0}$. Tale parametro è l'intercetta che è un parametro specifico all'oggetto. Tale parametro spiegata la maggior parte della forza dell'oggetto, infatti le covariate possono essere viste come estensioni contenenti effetti aggiuntivi dell'abilità dell'oggetto che non sono spiegati dall'intercetta. In tal senso, gli effetti della covariata possono aiutare a spiegare i risultati (imprevisti) di un soggetto che non possono essere completamente spiegati esclusivamente dall'intercetta.\\
Nella Sezione \ref{sez:4.3}, viene presentato l'effetto dell'ordine degli oggetti in competizione. Invece dell'effetto d'ordine globale $\delta$ , che è uguale per tutti gli oggetti, è possibile specificare l'effetto d'ordine specifico per ogni oggetto $\alpha_i$, quindi $\delta_i$.\\
Nella Tabella \ref{tab:type} vengono riassunti tutti i tipi di covariate e tutte le possibili parametrizzazioni che possono essere applicate.\\
Quindi, il parametro abilità $\gamma_{pi}$ di un oggetto $\alpha_i$ con \emph{i} = 1,....,n su un soggetto \emph{p}, \emph{p} = 1,.....,m non è altro che una combinazione lineare dei parametri precedentemente spiegati. Da ciò si ottiene il modello capace di utilizzare le covariate. Tale modello viene chiamato modello strutturato e fa parte dei \emph{generalized linear models} (GLMs). Riprendendo il modello \ref{for:3.9} può essere riadatto nella seguente forma
\begin{align}
	P(Y_{p(i,j)}\leq h) =  \frac{exp(\delta + \theta_{h} + \beta_{i0} - \beta_{j0} + x^T_{pi}\eta_i - x^T_{pj}\eta_j)}{1 + exp(\delta + \theta_{h} + \beta_{i0} - \beta_{j0} + x^T_{pi}\eta_i - x^T_{pj}\eta_j)} \label{for:4.9}
\end{align}

	\begin{table}[!htb]%
	
	\renewcommand{\arraystretch}{1.7}
	\centering
	\begin{tabular}{c c c c c}
		\hline	
		
		\textbf{Tipo di covariate} & \textbf{Tipo di effetto} & \textbf{$\gamma_{pi}$ =}& \textbf{$\gamma_{pj}$ =} & \textbf{$\gamma_{p(ij)}$ = $\gamma_{pi}$ $-$ $\gamma_{pj}$} \\	
		\hline			
		Intercetta & Spec. all'oggetto & $\beta_{i0}$ & $\beta_{j0}$ & $\beta_{i0} - \beta_{j0}$\\
		Effetto dell'ordine & Globale & + $\delta$ &  & + $\delta$ \\
		Effetto dell'ordine & Spec. all'oggetto &  + $\delta_i$ &  &  + $\delta_i$\\
		Spec. al soggetto $x_p$ & Spec. all'oggetto & + $x^T_p\beta_i$ & + $x^T_p\beta_j$ & + $x^T_p(\beta_i - \beta_j)$\\
		Spec.all'oggetto  $z_i$ & Globale & + $z^T_{i}\tau$ & + $z^T_{si}\tau$ & + ($z_{i} - z_{j})^T\tau$\\
		Spec. al soggetto e all'oggetto $z_pi$ & Globale & + $z^T_{pi}\tau$ & + $z^T_{pj}\tau$ & + ($z_{pi} - z_{pj})^T\tau$\\
		Spec. al soggetto e all'oggetto $z_pi$ & Spec. all'oggetto & + $x^T_{pi}\eta_i$ & + $x^T_{pj}\eta_i$& + $x^T_{pi}\eta_i$ $-$ $x^T_{pj}\eta_j$\\
		\hline
		
		
	\end{tabular} \hbox{}
	
	\caption{La Tabella riassuntiva di tutti i tipi di covariate e di tutte le possibili parametrizzazioni applicabili.} \label{tab:type}
\end{table}

\section{Stima e penalizzazione}
È importante considerare che con l'inserimento di un elevato numero di covariate si ha un aumento di complessità del modello. Dato che si utilizza un modello lineare, un eccessivo livello di complessità può portare a problemi di identificabilità ed efficienza. Infatti includendo soltanto una covariata specifica al soggetto e all'oggetto, questa ha un peso pari a \emph{n} covariate dove \emph{n} sono il numero di oggetti in considerazione. Oltretutto per ogni oggetto c'è la sua intercetta, perciò è necessario limitare il più possibile la complessità del modello. La soluzione è utilizzare metodi di \emph{shrinkage} che includono termini di penalizzazione nelle procedure di stima. L'obiettivo è quello di ottenere un modello con una moderata complessità utilizzando solo i parametri realmente necessari. \\
Con l'inclusione della penalizzazione dei termini il modello potrebbe migliorare o leggermente peggiorare, ma la variabilità associata alle stime sarà minore. C'è perciò un trade-off di cui occuparsi, infatti più è forte la penalità inserita, più sarà elevata la varianza perché molte informazioni sulle variabili vengono perse. Non si massimizzerà la verosimiglianza ma la verosimiglianza penalizzata 
\begin{center}
	$ l(\varepsilon)_{p}$ = $l(\varepsilon) - \lambda J(\varepsilon)$
\end{center}
dove $l(\varepsilon)$ è la log verosimiglianza con $\varepsilon$ che rappresenta il vettore contenente tutti i parametri del modello. $J(\varepsilon)$ è un termine di penalizzazione. Il parametro $\lambda$ è il parametro di Turing che stabilisce quanto forte deve essere la penalizzazione sui parametri. \\
Per eseguire la penalizzazione è necessario trasformare in scale comparabili tutte le covariate.\\
Sono state utilizzate solo alcune modalità di penalizzazione tra quelle disponibili, quindi verranno esposte solo quelle effettivamente utilizzate. In \autocite{schauberger2019btllasso} vi è una trattazione completa di tutte le penalizzazioni applicabili.\\
Come metodo di penalizzazione verrà applicato l'\emph{Adaptive Lasso} proposto da \autocite{zou2006}. Il metodo riduce i coefficienti ed esegue una selezione delle covariate applicando penalità di tipo $L_1$ per le differenze di coefficienti, di seguito verrà illustrato come è stato applicato.\\

Nel modello si è inserito l'effetto partita in casa come parametro con effetto specifico all'oggetto $\delta_i$, la penalizzazione risultate è data dalle differenze assolute tra tutti i confronti.

\begin{center}
	$ P(\delta_1,....\delta_m)_{\delta}$ = $\sum_{i<j}|\delta_i - \delta_j|$ 
\end{center}

È importante sottolineare che se ci sono molte differenze pari a zero, si ottengono gruppi di oggetti (nel nostro caso squadre) con un effetto identico della covariata penalizzata e che quindi la covariata deve avere un effetto globale piuttosto che specifico all'oggetto. Quindi con la penalizzazione è possibile capire quale tipo di effetto è più opportuno applicare.\\
Dato che non vi sono dubbi che l'effetto casa sia determinante per l'esito di una partita di calcio \autocite{lago2016home}, non verrà applicata nessun altra penalizzazione.\\
La penalizzazione per tutte le altre covariate (specifiche al soggetto e all'oggetto) è la seguente 
\begin{center}
	$ P_{\eta}(\eta_1,....\eta_m)$ = $\sum^{m}_{p=1}\sum_{i<j}|\eta_{ip} - \eta_{jp}| + \sum^{m}_{p=1}\sum^{n}_{i<j}|\eta_{ip}|$.
\end{center}

Rispetto alla penalizzazione precedente è stata aggiunta una penalizzazione al valore assoluto delle covariate. Questo perché non sappiamo in anticipo se una variabile è influente oppure no.\\

Le penalizzazione illustrate precedentemente se combinate permettono di ottenere il parametro $J(.) = P_\delta(.) + P_\eta(.)$.
\subsection{Scelta del parametro di Turing}
Un punto cruciale per le tecniche di \emph{shrinkage} è la determinazione del parametro di Turing ottimo $\lambda$, cioè il grado di penalizzazione che ci da il miglior trade-off. Per farlo ci si affiderà alla \emph{K-Fold Cross-Validation} (con k = 10), che sceglierà la miglior $\lambda$ rispetto alla metrica \emph{ranked probability score} (RPS).\\
Il RPS \autocite{gneiting2007strictly} per categorie di risposte ordinate $y \in \{1,....,K\}$  misura quanto siano buone le previsioni espresse come distribuzioni di probabilità rispetto ai valori osservati.
Il RPS può essere cosi espresso 
\begin{center}
	$ RPS(y,\pi(k))$ = $\sum^{K}_{k = 1}(\pi(k) -  \mathds{1} (y \le k))^2$ 
\end{center}
dove $\pi(k)$ rappresenta la probabilità cumulativa $\pi(k)$ = $P(y \le k)$. A differenza delle altre possibili misure dell'errore, ad esempio la devianza, il RPS tiene conto dell'ordine di preferenza.
$\mathbb{}$



\begin{comment}
	Si necessita perciò di un modello che tenga conto anche di variabili esplicative inserite durante l'analisi. \\
	Sia x$_{i}$=($x_{i1},....x_{iK}$) il vettore di K variabili esplicative per un certo oggetto \textit{i} e $\beta$ = ($\beta_{1},....\beta_{P}$) il vettore dei pesi stimati per ogni variabile presente in x$_{i}$, allora si ha che il parametro abilità $\alpha_{i}$ di un certo oggetto \textit{i} è uguale a:
	
	\begin{center}
		\begin{large}
			$\alpha_{i}$ = $\beta_{1}x_{i1}$ + .... + $\beta_{P}x_{iP}$      con i=1,....,n
		\end{large}
		
	\end{center}
	
	Si ha quindi che il parametro abilità $\alpha_{i}$ per un certo oggetto \textit{i} è una combinazione lineare di variabili.\\
	Il modello è stato presentato per la prima volta da \autocite{springall1973response}; tale modello viene chiamato modello strutturato.\\
	
	Grazie a questo modello se vi sono covariate che hanno un legame con la variabile risposta, tanto da influenzarne l'esito con quest'ultima allora, sarà possibile inserirle nel modello. Nel caso calcistico tali covariate possono essere il possesso della palla o il numero di falli fatti.
\end{comment}
 
	             % Modello BT
% !TEX encoding = UTF-8
% !TEX TS-program = pdflatex
% !TEX root = ../tesi.tex

%**************************************************************
\chapter{Flussi conversazionali prodotti}
\label{cap:flussi di conversazione}
%**************************************************************

\intro{In questo capitolo verrà descritto il lavoro che è stato fatto di analisi, progettazione e implementazione dei flussi conversazioniali per Azzurra creati durante lo stage.}\\

%**************************************************************
\section{Analisi dei requisiti}
\subsection{Descrizione del problema}
Durante lo stage è stato deciso, in comune accordo con il tutor aziendale, di costruire due flussi conversazionali per il \g{bot} Azzurra, nello specifico:
\begin{itemize}
	\item \textbf{DeskBooking}: Questo flusso conversazionale consiste nel gestire le prenotazioni di un posto a sedere. Deve esserci la possibilità di richiedere una nuova prenotazione, visualizzare la lista delle proprie prenotazioni ed infine, scannerrizzare un \g{QR code} messo nel posto a sedere, per controllare se il lavoratore può usufruire e riscattare tale posto. C'è quindi bisogno di integrare un lettore di \g{QR code} in Azzurra per poter fare il controllo che consiste nell'aprire la fotocamera e scannerizzare il \g{QR code} che verrà usato per il controllo ed infine comunicare l'esito della verifica al lavoratore;
	\item \textbf{Planning}: Questo flusso conversazionale consiste nel far visualizzare al lavoratore il lavoro che deve svolgere. Deve esserci la possibilità di richiedere la visualizzazione del lavoro pianificato di uno specifico giorno oppure la possibilità di vedere il lavoro pianificato per tutta la settimana.
\end{itemize}
\subsection{Requisiti}
Ogni requisito sarà strutturato come segue:
\begin{itemize}
	\item Identificativo: \textbf{R[Importanza][Tipologia][Codice]}\\
	Dove:
	\begin{itemize}
		\item \textbf{Importanza:}
		\begin{itemize}
			\item \textbf{1}: Requisito obbligatorio, vincolante in quanto primario e fondamentale;
			\item \textbf{2}: Requisito desiderabile, non strettamente necessario ma che porta valore aggiunto riconoscibile;
			\item \textbf{3}: Requisito opzionale, relativamente utile.
		\end{itemize}
		\item \textbf{Tipologia:}
		\begin{itemize}
			\item \textbf{F}: Funzionale, definisce una funzione di un sistema di uno o più dei suoi componenti;
			\item \textbf{Q}: Qualitativo, definisce un requisito per garantire la qualità per un certo aspetto del prodotto;
			\item \textbf{P}: Prestazionale, definisce un requisito che garantisce efficienza prestazionale nel prodotto;
			\item \textbf{V}: Vincolo, definisce un requisito volto a far rispettare un dato vincolo.
		\end{itemize}
		\item \textbf{Codice:} Viene utilizzato per identificare univocamente il requisito tramite un numero progressivo.\\
	\end{itemize}
\end{itemize}
Dopo un’analisi del problema sono stati individuati i seguenti requisiti
\begin{table}[h]%
	\rowcolors{2}{grigetto}{white}
		\renewcommand{\arraystretch}{1.5}
	\centering
	\begin{tabularx}{\textwidth}{c X}
		\hline	
		\rowcolor{heavenly}
		\intest{Codice} &  \intest{Descrizione} \\	
		\hline			
		R1F1 & Il lavoratore deve poter accedere alla funzionalità di "Prenotazione posto".\\
		R1F2 & Il lavoratore deve poter inserire una nuova prenotazione di un posto a sedere.\\
		R1F3 & Il lavoratore deve poter visualizzare le sue prenotazioni.\\
		R1F4 & Il lavoratore deve poter scansionare il \g{QR code} per poter usufruire del posto prenotato.\\
		R1F5 & Il lavoratore deve poter inserire la data in cui vuole prenotare il posto a sedere se disponibile.\\
		R1F6 & Il lavoratore deve poter inserire l'ora di inizio della prenotazione desiderata.\\
		R1F7 & Il lavoratore deve poter inserire l'ora in cui finisce la prenotazione desiderata.\\
		R1F8 & Il lavoratore deve poter inserire la stanza del posto a sedere che desidera prenotare.\\
		R1F9 & Il lavoratore deve poter inserire il posto a sedere che desidera prenotare se disponibile.\\
		R1F10 & Il lavoratore deve poter visualizzare il messaggio di conferma se la prenotazione del posto a sedere è andata a buon fine.\\
		R1F11 & Il lavoratore deve poter visualizzare il messaggio d'errore se non è stato possibile inserire la nuova prenotazione.\\
		R1F12 & Il lavoratore deve poter visualizzare le sue prenotazioni del giorno corrente.\\
		R1F13 & Il lavoratore deve poter visualizzare le sue prenotazioni del giorno successivo.\\
		R1F14 & Il lavoratore deve poter visualizzare le sue prenotazioni di uno specifico giorno.\\
		R1F15 & Il lavoratore deve poter inserire la data del giorno in cui vuole vedere le prenotazioni.\\
		\hline	
	\end{tabularx} \hbox{}
	\caption{Tabella del tracciamento dei requisiti}
\end{table}%


\begin{table}[h]%
	\rowcolors{2}{grigetto}{white}
		\renewcommand{\arraystretch}{1.5}
	\centering
	\begin{tabularx}{\textwidth}{c X}
		\hline		
		\rowcolor{heavenly}
		\intest{Codice} &  \intest{Descrizione} \\	
		\hline
		R1F16 & Il lavoratore, per ogni prenotazione, deve poter visualizzare l'ora di inizio della prenotazione.\\
		R1F17 & Il lavoratore, per ogni prenotazione, deve poter visualizzare l'ora in cui finisce la prenotazione.\\		
		R1F18 & Il lavoratore, per ogni prenotazione, deve poter visualizzare la stanza della prenotazione.\\
		R1F19 & Il lavoratore, per ogni prenotazione, deve poter visualizzare il posto della prenotazione.\\	
		R1F20 & Il lavoratore, dopo avere scannerizzato il \g{QR code} del posto a sedere, deve ricevere un messaggio di conferma che lo informa che può usufruire del posto a sedere.\\
		R1F21 & Il lavoratore, dopo avere scannerizzato il \g{QR code} del posto a sedere, deve ricevere un messaggio d'errore che lo informa che non può usufruire del posto a sedere.\\
		R1F22 & Il lavoratore deve poter visualizzazione la pianificazione di uno specifico giorno.\\
		R1F23 & Il lavoratore deve poter visualizzazione la pianificazione della settimana corrente.\\
		R1F24 & Il lavoratore deve poter inserire la data del giorno in cui vuole vederne la pianificazione del lavoro a lui assegnato.\\
		R1F25 & Il lavoratore deve poter visualizzare la data del giorno del lavoro pianificato.\\
		R1F26 & Il lavoratore deve poter visualizzare l'ora d'inizio del lavoro pianificato.\\
		R1F27 & Il lavoratore deve poter visualizzare l'ora in cui termina il lavoro pianificato.\\
		R1F28 & Il lavoratore deve poter visualizzare il lavoro che è stato pianificato per essere svolto.\\
		R1V1 & Per implementare i flussi conversazionali devono essere usati Angular e Ionic.\\
		R1V2 & Per gestire la fotocamera per la lettura del \g{QR code} deve essere usato il \emph{plugin} di Cordova, QR Scanner.\\
		\hline	
	\end{tabularx} \hbox{}
	\caption{Tabella del tracciamento dei requisiti}
\end{table}%
\clearpage
\section{Progettazione}
Dopo aver individuato i requisiti che descrivono i flussi da costruire, sono passato alla progettazione dei flussi. Come spiegato nel capitolo precedente i flussi conversazionali sono un insieme di blocchi che svolgono determinate funzioni a seconda del tipo di appartenenza. Grazie a ciò, la progettazione dei due flussi è iniziata con l'inserimento dei blocchi corretti e il collegamento tra di essi, ottenendo così due diagrammi che rappresentano i due flussi dove vengono visualizzati quali passi deve fare il \g{bot} Azzurra durante la conversazione con l'utente umano. Successivamente ho progettato i metodi da aggiungere a quelli esistenti per poter creare i messaggi nel modo corretto.\\
Di seguito vengono illustrati i diagrammi dei flussi fatti.

\subsection{Gestione delle prenotazioni dei posti}
Nei seguenti diagrammi viene visualizzato l'insieme dei blocchi che fanno parte del flusso conversazionale DeskBooking. Per comodità si è deciso di rappresentare il flusso attraverso tre diagrammi più piccoli.

\begin{figure}[h]
	\centering
	\includegraphics[scale=0.25]{chatbot/chatbot1.png}
	\caption{Diagramma per l'inserimento di una nuova prenotazione del flusso DeskBooking}\label{fig:ins}
\end{figure}

La Figura~\ref{fig:ins} rappresenta il ramo del flusso Deskbooking dedicato all'inserimento di una nuova prenotazione. È così composto:
\begin{enumerate}
	\item Il flusso inizia con un blocco ASK che chiede all'utente se vuole inserire una nuova prenotazione o visualizzare le sue prenotazione oppure scansionare un \g{QR code};
	\item Nel caso in cui l'utente voglia inserire una nuova prenotazione, attraverso un blocco ASK, viene chiesta la data che vuole inserire per la prenotazione. Viene progettato che l'inserimento della data viene fatta attraverso il DATEPICKER;
	\item Successivamente viene chiesta l'ora di inizio per la prenotazione tramite un blocco ASK. L'inserimento dell'ora avviene tramite il TIMEPICKER;
	\item Viene rifatta la stessa operazione del punto precedente ma chiedendo l'ora in cui finisce la prenotazione;
	\item Terminato il punto precedente, l'utente ha inserito l'intervallo di tempo all'interno del quale desidera effettuare una prenotazione di un posto a sedere. Attraverso il blocco CALLFUN viene chiesto ad Azzurra.io quali stanze con posti liberi sono disponibili per la data e l'intervallo inseriti dall'utente;
	\item Se non ci sono stanze con posti liberi o l'operazione di richiesta va in errore, attraverso un blocco SAY viene informato l'utente della situazione e ricomincia l'esecuzione dall'inizio del flusso;
	\item Se invece ci sono stanze con posti liberi, attraverso il blocco PROC vengono formattati i dati ricevuti in modo da poterli mostrare in una forma adatta alla situazione. In questo caso viene chiesto di creare una mappa con chiave contenente il nome della stanza e il numero dei posti e come valore l'identificativo della stanza. Quando si vorrà mostrare questi dati, verrà solo mostrato la chiave dei dati;
	\item Tramite il blocco ASK vengono mostrate le stanze disponibili mostrando i dati secondo la formattazione fatta al punto precedente;
	\item L'utente sceglie la stanza è viene controllato se nel frattempo è ancora disponibile e chiede quali posti a sedere sono liberi;
	\item Se avviene un errore di connessione o non ci sono più posti liberi si torna al punto 6 spiegato precedentemente;
	\item I dati ricevuti vengono formattati attraverso il blocco PROC creando la mappa con chiave contenente ora di inizio, orario di terminazione, nome stanza e nome posto a sedere, mentre come valore conterrà l'identificativo del posto a sedere, l'orario di inizio e di fine;
	\item Viene chiesto all'utente, attraverso un blocco ASK, di scegliere uno dei posti a sedere liberi;
	\item Viene contattato Azzurra.io con il blocco CALLFUN per inserire la nuova prenotazione;
	\item Se avviene un errore nell'inserimento viene eseguito il punto 6;
	\item Se l'operazione va a buon fine viene comunicato all'utente l'esito positivo dell'operazione, ricordandogli i dati della prenotazione e di scansionare il \g{QR code} per riscattare il posto a sedere. Il flusso poi termina.
\end{enumerate}

\begin{figure}[h]
	\centering
	\includegraphics[scale=0.27]{chatbot/chatbot2.png}
	\caption{Diagramma per la visualizzazione delle prenotazioni del flusso DeskBooking}\label{fig:vis}
\end{figure}

La Figura~\ref{fig:vis} rappresenta il ramo del flusso Deskbooking dedicato alla visualizzazione delle prenotazioni. È così composto:
\begin{enumerate}
	\item Il flusso inizia con un blocco ASK che chiede all'utente se vuole inserire una nuova prenotazione o scansionare un \g{QR code};
	\item Nel caso in cui l'utente voglia visualizzare le sue prenotazioni, viene chiesto attraverso un blocco ASK se vuole sapere le prenotazione del giorno corrente o del giorno successivo o di un altro giorno;
	\item Nel caso l'utente voglia vedere le sue prenotazioni del giorno corrente o del giorno successivo verrà fatta una richiesta ad Azzurra.io per ottenere le prenotazioni della data inserita dall'utente. La richiesta viene fatta attraverso il blocco CALLFUN;
	\item Se invece l'utente vuole vedere le sue prenotazioni di una data diversa dal giorno corrente o successivo, attraverso un blocco ASK viene chiesta la data che vuole inserire per la visualizzazione. L'inserimento della data viene fatta attraverso il DATEPICKER, successivamente si esegue il punto 3 per la richiesta;
	\item Se ci sono prenotazioni queste vengono mostrate all'utente, se invece avviene un errore o non ci sono prenotazioni fatte da lui, verrà avvisato di tale evento. Dopo questo passo il flusso termina.\\
\end{enumerate}

\begin{figure}[h]
	\centering
	\includegraphics[scale=0.275]{chatbot/chatbot3.png}
	\caption{Diagramma per lo scansionamento del \g{QR code} del flusso DeskBooking}\label{fig:qrcode}
\end{figure}

La Figura~\ref{fig:qrcode} rappresenta il ramo del flusso Deskbooking dedicato allo scansionamento del \g{QR code}. È così composto:

\begin{enumerate}
	\item Il flusso inizia con un blocco ASK che chiede all'utente se vuole inserire una nuova prenotazione o scansionare un \g{QR code};
	\item Nel caso in cui l'utente voglia scansionare un \g{QR code} per riscattare il suo posto a sedere prenotato, viene chiesto attraverso un blocco ASK di aprire lo scannerizzatore di \g{QR code}. Viene usato QRSCANNER per leggere il \g{QR code};
	\item Viene chiesto ad Azzurra.io attraverso il blocco CALLFUN, se il posto a sedere può essere usato dall'utente;
	\item Se l'esito è positivo, viene comunicato all'utente che può usufruire del posto fino al termine della prenotazione. Termina così il flusso;
	\item Se l'esito è negativo, viene informato l'utente che non può usare il posto a sedere in quel momento. Termina così il flusso;
\end{enumerate}

\subsection{Visualizzazione della pianificazione}
Nel seguente diagramma viene mostrato l'insieme dei blocchi che fanno parte del flusso conversazionale Planning.

\begin{figure}[h]
	\centering
	\includegraphics[scale=0.29]{chatbot/chatbot4.png}
	\caption{Diagramma per la visualizzazione della pianificazione del flusso Planning}\label{fig:plan}
\end{figure}

La Figura~\ref{fig:plan} rappresenta il ramo del flusso Planning dedicato alla visione della pianificazione del lavoro da svolgere.
\begin{enumerate}
	\item Il flusso inizia con un blocco ASK che chiede all'utente di quale giorno vuole vedere la pianificazione. L'inserimento della data viene fatta attraverso il DATEPICKER;
	\item Viene fatta richiesta a Azzurra.io, utilizzando il blocco CALLFUN, di trovare la pianificazione del giorno indicato dall'utente;
	\item Se non viene trovato nulla allora l'utente viene avvisato tramite un blocco SAY che non c'è niente di pianificato e il flusso termina;
	\item Se invece c'è una pianificazione disponibile per il giorno indicato dall'utente, viene visualizzata attraverso un blocco SAY;
	\item Dopo il punto precedentemente descritto viene chiesto con un blocco SAY se si vuole sapere la pianificazione di tutta la settimana;
	\item Se l'utente risponde no il flusso termina;
	\item Se l'utente risponde sì viene fatta richiesta a Azzurra.io, utilizzando il blocco CALLFUN, di trovare la pianificazione della settimana corrente;
	\item Se la richiesta va buon fine viene mostrata la pianificazione della settimana, altrimenti viene mostrato un messaggio. In entrambi i casi il flusso poi termina.
\end{enumerate}
\clearpage

\section{Codifica}
Per implementare i due flussi si sono utilizzati i \g{framework} Angular e Ionic. Grazie a Angular si è potuto strutturare un’applicazione web come una gerarchia di componenti quindi, attraverso il linguaggio TypeScript, si è gestita l'\emph{application logic} mentre con \gls{HTML} e \gls{CSS} si è gestita la \emph{presentation logic}. Purtroppo solo l'uso di Angular non basta per poter sviluppare un'applicazione \emph{mobile} infatti solo utilizzando Angular si può sviluppare una applicazione web. Si è quindi usato Cordova, un \g{framework} che permette di sviluppare un'applicazione \emph{mobile} multi-piattaforma e offre \g{api} per accedere alle funzionalità native del dispositivo, ad esempio la fotocamera. Cordova infatti incapsula l'applicazione web e la esegue localmente all’interno di un’\g{applicazione nativa} che può interagire con le funzionalità del dispositivo. Per sfruttare le funzionalità di Angular e di Cordova assieme, è stato usato il \g{framework} Ionic che permette di creare un ambiente integrato che semplifica lo sviluppo di applicazioni offrendo anche componenti grafiche ottimizzate per i dispositivi \emph{mobile}.\\

Per quanto riguarda la codifica, per prima cosa si è implementato una configurazione \g{JSON} per ogni flusso, dove si sono codificati i vari blocchi progettati, utilizzando la sintassi spiegata nel precedente capitolo. Una volta scritte le due configurazioni si è dovuto aggiornare il \emph{main flow} aggiungendo nel primo blocco che viene eseguito, cioè un blocco ASK dove viene chiesto che funzionalità si vuole eseguire, due BlockItem per indicare le due nuove funzionalità offerte dai due flussi prodotti. Oltre alle due nuove scelte, nel \emph{main flow} sono stati aggiunti due blocchi JUMP per permettere di mandare in esecuzione i due nuovi flussi quando l'utente ne richiede l'esecuzione, subito dopo la selezione della funzionalità desiderata da parte dell'utente.\\

Per poter creare i tre Widget, DATEPICKER, TIMEPICKER e QRSCANNER, nel createActions() ho aggiunto tre metodi per ognuno dei tre Widget, che vengono chiamati da createActions() in base al tipo di Widget da creare. In questi tre nuovi metodi viene impostato il testo che devono mostrare e nel caso dei DATEPICKER e TIMEPICKER viene impostato anche il formato del giorno e dell'ora. Per ognuno di questi metodi ho creato un metodo specifico per ogni Widget che si occupa della creazione e dell'apertura, in particolare per il metodo che crea il QRSCANNER, \_openQRcode(), viene utilizzato il ModalController di Ionic per creare la classe dove è definita l'interfaccia grafica e i metodi per il funzionamento di QRSCANNER. Il ModalController di Ionic permette di aprire una nuova finestra, sopra a quella corrente, per visualizzare la componente Ionic definita nella nuova finestra, in questo caso la classe che implementa il lettore di \g{QR code}. Una volta finito di usare la nuova finestra, essa viene chiusa e si ritorna alla finestra precedente che sarà nello stato precedente all'apertura della nuova finestra. Ho dovuto perciò implementare la classe che gestisce il lettore \g{QR code}, denominata CameraComponent, dove al suo interno richiama il \emph{plugin} di Cordova, QR Scanner. QR Scanner è un \g{api} che permette di accedere alla fotocamera del dispositivo e di scansionare i \g{QR code}. Vengono quindi definiti due metodi in CameraComponent, un metodo per l'apertura della fotocamera, la lettura del \g{QR code} e la chiusura della fotocamera che successivamente invia l'eventuale valore letto al ModalController. Nel CameraComponent viene definito anche il suo aspetto grafico mostrato nella Figura~\ref{fig:qrc}.\\
 
  Nel ChatComponent ho implementato il metodo \_initGenericQRCode() il quale aspetta di ricevere il valore letto dal lettore di \g{QR code} che richiama quindi il metodo sendReply() di AzzurraService per dare inizio al il processo di creazione del messaggio dell'utente umano spiegato nel capitolo precedente. \\
  
  Per quanto riguarda i metodi per il DATEPICKER e per il TIMEPICKER, essi sono molto simili a quelli per gestire il lettore \g{QR code}, con la differenza che i metodi analoghi a \_openQRcode(), \_openDatePicker() e \_openTimePicker(), non utilizzano il ModalComponent ma l'ion-datetime, una componente grafica offerta da Ionic che può essere configurato per chiedere una data oppure un intervallo temporale; il risultato viene mostrato nella Figura~\ref{fig:date}. Quindi in questi due metodi ho definito come si devono presentare i due \emph{picker} e quindi non c'è stato bisogno di un \emph{component} apposito che li gestisca come per il QRSCANNER. 
\section{Risultati}

Nella Figura~\ref{fig:planning} viene mostrata la \emph{chat} tra Azzurra e l'utente per la visualizzazione della pianificazione, sia per un singolo giorno (prima figura) sia per tutta la settimana (seconda figura).\\

\begin{figure}[h]
	\begin{center}
		\includegraphics[scale=0.17]{day.png}\hfil
		\includegraphics[scale=0.17]{week.png}
		\caption{Richiesta di visualizzazione della pianificazione}\label{fig:planning}
	\end{center}
\end{figure}

Nella Figura~\ref{fig:QRc} viene mostrata la \emph{chat} tra Azzurra e l'utente per la richiesta di visualizzazione del prenotazione dell'utente (prima figura) e la richiesta di scannerizzare il \g{QR code} per usufruire del posto prenotato (seconda figura).\\

\begin{figure}[h]
	\begin{center}
		\includegraphics[scale=0.17]{visDB.png}\hfil
		\includegraphics[scale=0.17]{qrc.png}
		\caption{Richiesta di visualizzazione delle prenotazioni e scannerizzazione di un QR code}\label{fig:QRc}
	\end{center}
\end{figure}

\begin{figure}[h]
	\begin{center}
		\includegraphics[scale=0.167]{DB1.png}\hfill
		\includegraphics[scale=0.167]{DB2.png}\hfill
		\includegraphics[scale=0.167]{DB3.png}
		\caption{Richiesta di inserimento di una nuova prenotazione}\label{fig:DB}
	\end{center}
\end{figure}
\clearpage
Nella Figura~\ref{fig:QRc} viene mostrata la \emph{chat} tra Azzurra e l'utente per la richiesta di inserimento di una nuova prenotazione. Viene perciò richiesto il giorno e l'intervallo di tempo in cui fare la prenotazione, in quale stanza deve essere il posto da prenotare e quale posto a sedere vuole prenotare tra quelli disponibili secondo i parametri inseriti.

\section{Considerazioni}
\label{cap:cons1}
I risultati ottenuti hanno soddisfatto tutti i requisiti stabili e sono stati soddisfacenti per il tutor aziendale. Dai risultati precedentemente elencati si sono fatte delle considerazioni su come potessero essere migliorati in termini di \emph{user experience}. L'\g{architettura} che supporta Azzurra permette di tenere lo storico dei messaggi e lo stato della conversazione per una certa durata, in modo da rendere migliore l'interazione tra utente e Azzurra. Un ulteriore miglioramento che può essere adottato è l'introduzione dei comandi vocali. Al momento l'inserimento dei dati da parte dell'utente avviene principalmente tramite comandi \emph{touch}. Talvolta questa interazione può risultare limitante per alcuni utenti ad esempio con limitazioni alla vista, all'uso delle mani o con il \emph{device} con schermo troppo piccolo. Grazie ai comandi vocali invece queste limitazioni risulterebbero annullate. \\

Dalla possibile introduzione dei comandi vocali nascono però altre considerazioni. Il riconoscimento vocale deve essere sufficientemente accurato da capire ciò che dice l'utente perché altrimenti può provocare una sensazione negativa di irritazione o frustrazione nell'utente dovuta al fatto che ciò che dice non viene capito dall'applicazione e quindi non offrire una buona \emph{user experience}. In sintesi, un possibile modo per migliorare l'\emph{user experience} è l'introduzione dei comandi vocali con un'accurata valutazione a priori di rischi e costi da affrontare.

				% Risultati BT
% !TEX encoding = UTF-8
% !TEX TS-program = pdflatex
% !TEX root = ../tesi.tex

%**************************************************************
\chapter{Metodi di Machine Learning}
\label{cap:ML}
%**************************************************************
\intro{Questo capitolo illustrerà i metodi di \emph{Machine Learning} che sono stati utilizzati per la predizione degli esiti delle partite di calcio della Seria A italiana della stagione 2021/2022. Purtroppo, non è stato possibile applicare metodi di \emph{Machine learning} che corrispondessero al modello \emph{Bradley-Terry} perché, nonostante esistano metodi in \emph{Machine learning} che forniscono modelli basati sul modello \emph{Bradley-Terry}, essi non sono in grado di gestire l'esito del pareggio ma solo un esito binario. Ne consegue che tali metodi non sono adatti per contesti come il calcio ma ad altri tipi di sport dove il pareggio non è previsto come il \emph{baseball}. I metodi di \emph{Machine learning} considerati sono: il K-Nearest-Neighbors (K-NN), la Support Vector Machine (SVM), gli alberi di decisione per la classificazione, la Random Forests e in fine l'Adaboost.
}
\section{Componenti essenziali}
In questa sezione vengono definite alcune misure e tecniche che sono necessarie per il funzionamento dei metodi di \emph{Machine Learning} applicati.
\subsection{Distanza di Minkowski}
La \textit{\cite{minkdist}} è una misura utilizzata per la valutazione della distanza ovvero, nel nostro contesto della somiglianza tra due punti in spazio di \textit{n}-dimensioni. La distanza di Minkowski di ordine \emph{d} tra due punti A = (a$_1$,...a$_n$) e B = (b$_1$,...b$_n$) vale
\begin{center}
	$Dist(A,B) =  \left(\sum_{i = 1}^{n}|a_i-b_i|^d\right)^{1/d} $
\end{center}

Si sottolinea che quando l'ordine d = 1, la distanza utilizzata è la \textit{\cite{manhattan}} ovvero la distanza tra due punti è la somma del valore assoluto delle differenze delle loro coordinate. Quando l'ordine d = 2 è applicata la \textit{\cite{euclidea}} dove la distanza tra due punti è la lunghezza del segmento con agli estremi i due punti d'interesse.
Tale misura sarà utilizzata nel metodo K-Nearest-Neighbors (K-NN).
\subsection{Funzione kernel}
Nel contesto dell'apprendimento automatico, la \textit{\cite{kernel}} permette di trasformare uno spazio di input non linearmente separabile in uno nuovo spazio delle istanze di input detto \emph{feature space} di dimensione superiore rispetto a quello originale tale da diventare linearmente separabile. Per spazio linearmente separabile si intende che esiste un iperpiano in grado di separare correttamente i dati in due gruppi distinti. Perciò aumentando la dimensionalità dello spazio d'interesse è possibile trovare la dimensione opportuna che permetta di separare linearmente i dati. Tale applicazione è chiamata kernel trick. Perciò, una funzione kernel è una funzione \emph{K} che per ogni \emph{x}, \emph{y} $\in \chi$ dove $\chi$ è lo spazio di input di dimensione \emph{n}, vale 
\begin{center}
	$K(x,y) =  \langle\psi(x),\psi(y)\rangle $.
\end{center}
Dove $\psi$ è la funzione che mappa i punti di uno spazio di dimensione \emph{n} in uno spazio di dimensione \emph{m} con \emph{m>n}, invece, $\langle . \rangle$ indica il prodotto scalare.\\
Nelle nostre predizioni saranno usati questi kernel:
\begin{itemize}
	\item Linear kernel: è la funzione precedentemente definita.
	\item Polynomial kernel: $K(x,y) =  \left(1 + \sum_{i = 1}^{p}x_iy_i\right)^{d} $ dove \emph{p} è il numero di istanze di input presenti in $\chi$ mentre \emph{d} la dimensione del spazio (l'ordine).
	\item Gaussian Radial Basis kernel (RBF): $K(x,y) = exp(-\gamma||x-y||^2) $ con $\gamma=\frac{1}{2\sigma^2}$ mentre $\sigma$ è un paramento libero. 
\end{itemize}

Nella Figura \ref{fig:kernel} è mostrato un esempio di applicazione della funzione kernel.\\

\begin{figure}[h]
	\begin{center}
		\includegraphics[scale=0.50]{kernel.png}
		\caption{Esempio grafico della funzione kernel $\gamma$ mappa i punti di uno spazio d'input in uno feature space di dimensione maggiore e linearmente separabile.
		} 
		Source: \url{https://towardsdatascience.com/the-kernel-trick-c98cdbcaeb3f}\label{fig:kernel}
	\end{center}
\end{figure}

La funzione kernel sarà utilizza nella Support Vector Machine (SVM).

\subsection{Bootstrap}
In statistica e nell'apprendimento automatico, per \textit{\cite{bootstrap}} si intende una tecnica di ricampionamento per la generazione di un insieme di campioni di \emph{m} osservazioni contenute da un dataset di dimensione \emph{n}. Ogni estrazione è casuale e con rimpiazzo, cioè un’osservazione può essere presente in più campioni. Tale tecnica è utilizzata per produrre un insieme di campioni che siano il più possibile rappresentativi e indipendenti tra di loro.\\

Nella Figura \ref{fig:bootstrap} viene mostrato un esempio della procedura di Bootstrap

\begin{figure}[h]
	\begin{center}
		\includegraphics[scale=0.60]{bootstrap1.png}
		\caption{Esempio grafico della procedura di Bootstrap.
		} 
		Source: \url{https://blog.paperspace.com/bagging-ensemble-methods/}\label{fig:bootstrap}
	\end{center}
\end{figure}

\subsection{Bagging}
Il Bagging \autocite{breiman1996bagging} detto anche \emph{Bootstrap Aggregation Approch}, è una tecnica \emph{ensemble learning} di tipo parallelo che dalla mediazione di più predizioni fatte da un insieme di classificatori deboli ottiene un'unica predizione finale. È di tipo parallelo perché va a sfruttare l'indipendenza dei classificatori. La procedura applicata è la seguente:
\begin{itemize}
	\item Creazione di k campioni utilizzando la tecnica di Bootstrap.
	\item Per ogni campione viene allenato un classificatore.
	\item Viene prodotta una predizione per ogni classificatore allenato.
	\item Le predizioni ottenute vengono mediate ottenendo un predizione finale.
\end{itemize} 
Una tecnica per mediare è ad esempio, il \emph{voting} dove la classe più predetta sarà il risultato dalla predizione finale.Inoltre, si utilizza il Bootstrap per rendere i classificatori indipendenti tra di loro.
Perciò l'obbiettivo del Bagging è quello di creare un classificatore modello di gestire un'elevata varianza dei dati in modo efficiente grazie al parallelismo.\\
Nella Figura \ref{fig:bagging} viene illustrato graficamente la procedura di Bagging

\begin{figure}[h]
	\begin{center}
		\includegraphics[scale=0.50]{Ensemble_Bagging.png}
		\caption{Esempio grafico della procedura di Bagging.
		} 
		Source: \url{https://www.analyticsvidhya.com/blog/2020/02/what-is-bootstrap-sampling-in-statistics-and-machine-learning/}\label{fig:bagging}
	\end{center}
\end{figure}

\subsection{Boosting}
Il Boosting \autocite{freund1996experiments} è una tecnica \emph{ensemble learning} di tipo sequenziale che sfrutta la dipendenza tra i classificatori usati. Sostanzialmente l'algoritmo inizialmente allena un classificatore debole con tutto il \emph{dataset} a disposizione. Successivamente per raffinare la predizione vengono allenati in sequenza nuovi classificatori che apprendono da tutto ciò che è stato appreso dal classificatore precedente e dal l'intero \emph{dataset}.\\
La procedura completa è la seguente:
\begin{itemize}
	\item Viene utilizzato l'intero \emph{dataset} per allenare un classificatore debole.
	\item Vengono ripesati gli esempi di \emph{training} dando un peso maggiore a quei esempi a cui che è stata sbagliata la classificazione, viceversa per gli esempi classificati correttamente.
	\item Ripetere per n volte i passi precedenti con un nuovo classificatore con i pesi aggiornati.
	\item Combinare tutte le ipotesi semplici in un unico classificatore accurato per ottenerne il risultato finale.
\end{itemize}
Perciò con l'aggiornamento dei pesi si presuppone che i classificatori successivi non andranno a commettere gli stessi errori dei classificatori precedenti.\\
L'obbiettivo del Boosting è concentrare i propri sforzi nel creare un classificatore adatto a gestire un'elevata distorsione anziché un'elevata varianza dei dati. Infatti, partendo da un classificatore debole e migliorandolo in modo sequenziale, si consente ai classificatori successivi di imparare dagli errori precedentemente commessi, riducendo la distorsione dei dati. Inoltre, il Boosting è la resistenza agli effetti dell'\emph{overfitting}.
Purtroppo, il Boosting risulta molto sensibile ai valori anomali e inoltre, dato che le operazioni di addestramento di ogni classificatore avvengo in modo sequenziale non sarà possibile utilizzare il parallelismo per risparmiare tempo di calcolo.\\
Nella Figura \ref{fig:boosting} viene illustrato graficamente la procedura di Boosting

\begin{figure}[h]
	\begin{center}
		\includegraphics[scale=0.55]{boosting.png}
		\caption{Esempio grafico della procedura di Boosting.
		} 
		Source: \url{https://www.section.io/engineering-education/boosting-algorithms-python/}\label{fig:boosting}
	\end{center}
\end{figure}

\section{K-Nearest-Neighbors}
Il \emph{K-Nearest-Neighbors} (K-NN) \autocite{dasarathy1991nearest} è un algoritmo di apprendimento automatico di tipo supervisionato che permette la classificazione delle istanza ricevute in input. Inoltre, ne esiste una sua versione per problemi di regressione. Il K-NN assume che tutte le istanze corrispondano a punti in uno spazio di dimensionalità \emph{n} e utilizza la prossimità tra i vari punti per classificarli, ossia classifica l'istanza da classificare con la classe maggiormente presente tra i punti attorno all'istanza da classificare. I punti attorno all'istanza da classificare sono detti vicini. Fondamentale, perciò, è l'utilizzo di una qualche tecnica di misurazione della distanza per individuare chi sono i vicini dell'istanza da classificare, ossia di calcolare la distanza tra il nostro punto d'interesse con tutti gli altri punti. La misura di distanza utilizza è la distanza di Minkowski definita nella sezione precedente. Misurati tutti i punti, occorre stabile poi quanti dei punti presenti devono essere considerati vicini, ovvero il cosiddetto parametro \emph{k}. Il valore di \emph{k} è un iperparametro dell'algoritmo che stabilisce di considerare solo i \emph{k} punti più vicini all'istanza da classificare. Per esempio, se \emph{k = 3}, si considerano i tre punti più vicini e si classifica l'istanza con la classe più frequente tra i tre punti considerati. È importante scegliere il corretto valore di \emph{k} poiché valori diversi possono portare a \emph{overfitting} nel caso si considerino troppi vicini, o \emph{underfitting} nel caso si considerino pochi vicini. Infatti, con valori più bassi di \emph{k} può verificarsi un'elevata varianza e una distorsione bassa, mentre con valori più grandi di \emph{k} può verificarsi un'elevata distorsione e una varianza bassa. È importante analizzare la composizione del \emph{dataset} per scegliere il corretto valore di \emph{k} questo perché, ad esempio, se ci sono tante istanze con valori anomali o rumore è probabile che funzioneranno meglio con valori più alti di \emph{k}. Una buona soluzione per la scelta dell'iperparametro \emph{k} ma anche del tipo di ordine \emph{p} della distanza da utilizzare è la \emph{Cross Validation}.\\
Nella Figura \ref{fig:knn} è mostrato un esempio di applicazione dell'algoritmo K-NN.\\
\begin{figure}[h]
	\begin{center}
		\includegraphics[scale=0.40]{knn.png}
		\caption{Esempio grafico dell'algoritmo \emph{K-Nearest-Neighbors}.
			L'istanza da classificare è indicata con il punto interrogativo (?).
			Il primo passo dell'algoritmo è quello di calcolare tutte le distanze. Dopo di che, considera solo i \emph{k = 3} punti più vicini all'istanza (?). Infine l'algoritmo classifica con la classe B l'istanza (?).
		} 
		Source: \url{https://www.ibm.com/topics/knn#:~:text=The%20k%2Dnearest%20neighbors%20algorithm%2C%20also%20known%20as%20KNN%20or,of%20an%20individual%20data%20point.}\label{fig:knn}
	\end{center}
\end{figure}

Il K-NN è un algoritmo di classificazione non parametrico, ovvero non fa alcuna assunzione sulla forma della distribuzione dei dati. Inoltre, dato che è un algoritmo di apprendimento supervisionato le istanze d'input sono nella forma $(x, f(x))$. Nella fase di 
addestramento si limita soltanto a memorizzare i dati di \emph{training}, dato che li utilizza direttamente per fare predizione. Purtroppo, però la fase di predizione può essere lenta poiché è necessario calcolare la distanza di ogni osservazione dall'istanza da classificare, il che può essere computazionalmente costoso se si hanno molti dati.

\section{Support Vector Machine}
La \emph{Support Vector Machine} (SVM) \autocite{GHOLAMI2017515} è un algoritmo di apprendimento automatico di tipo supervisionato applicabile in contesti di classificazione. La SVM considera le istanze del \emph{dataset} come punti in uno spazio di dimensionalità \emph{n} è il suo obbiettivo è quello di costruire l'iperpiano ottimo che separi in due classi le osservazioni. L'iperpiano ottimo viene scelto in modo tale da ottenere il maggior margine possibile tra le due classi, ovvero il maggior spazio possibile tra le osservazioni di ciascuna classe e l'iperpiano. Dal nome di quest'algoritmo deriva dall'utilizzo dei vettori detti vettori di supporto. Questi vettori sono le istanze che si trovano più vicino all'iperpiano ovvero quelli più difficili da classificare e quindi danno un grosso contributo alla costruzione dell'iperpiano rispetto alle altre osservazioni. Perciò per massimizzare la distanza tra l'iperpiano e i punti di entrambe le classi, occorre risolvere un problema di ottimizzazione vincolata, ovvero minimizzare la funzione di perdita secondo certe condizioni. Perciò, vale 
\begin{align*}
	\text{min} & \frac{1}{2} \|\mathbf{w}\|^2 + C \, \sum_{i=1}^{n} (\xi_i) \\
	\text{tale che} & 
	\begin{cases}
		y_i(\mathbf{w \cdot x}_i - b) \geq  1 - \xi_i \\
		\xi_i \geq 0, i=\{1,...,n\}
	\end{cases} \, .
\end{align*}
Dove $\|\mathbf{w}\|$ è il vettore direzione, l'iperparametro \emph{C} è un parametro di regolarizzazione, il quale permette di gestire il \emph{trade-off} tra massimizzazione del margine e perdita, consentendo di controllare la complessità del modello e quindi a prevenire l'\emph{overfitting}. La variabile $\xi_i$ è l'errore commesso. La formula $\mathbf{w \cdot x}_i - b$ è la distanza algebrica tra l'iperpiano scelto e il punto più vicino.\\
Nella Figura \ref{fig:svm} è mostrato un esempio di applicazione dell'algoritmo SVM.\\
\begin{figure}[h]
	\begin{center}
		\includegraphics[scale=0.80]{svm.jpg}
		\caption{Esempio grafico dell'algoritmo \emph{Support Vector Machine}. I punti sulle linee trattegiate indicano i vettori di supporto, mentre la retta al centro indica l'iperpiano ottimo di separazione.
		} 
		Source: \url{https://www.sciencedirect.com/science/article/pii/B9780128113189000272}\label{fig:svm}
	\end{center}
\end{figure}
L'algoritmo SVM è grado di gestire anche spazi d'input non linearmente separabili, grazie all'utilizzo della funzione kernel definita nella sezione precedente.\\
Tramite la \emph{Cross Validation} si sceglierà il valore più opportuno per l'iperparametro \emph{C} e il tipo di kernel da applicare.

\section{Decision Tree}
Un Decision Tree \autocite{} è un algoritmo di apprendimento automatico di tipo supervisionato e non parametrico che utilizza una struttura ad albero per produrre le proprie predizioni. Tale albero contiene un insieme di nodi in cui per ogni nodo vi è un test su un'attributo dell'osservazione da classificare; perciò, ad ogni nodo ci sarà una scelta da compiere in base al valore contenuto dell'attributo, che porterà verso un nuovo ramo oppure a una foglia. Le foglie contengono i risultati della classificazione. L'approccio utilizzato per la costruzione dell'albero di decisione è di tipo \emph{greedy} cioè, ogni scelta effettuata su un nodo è l'opzione più conveniente in quel momento. L'albero viene costruito in modalità \emph{top-down} e i passi sono i seguenti
\begin{itemize}
	\item viene creata la radice \emph{T} dell'albero,
	\item se le osservazioni dell'insieme \emph{D} sono tutte della stessa classe \emph{k}, allora viene ritornata la radice \emph{T} classificata con la classe \emph{k},
	\item se le osservazioni non hanno attributi che li descrivono, allora viene ritornata la radice \emph{T} classificata con la classe di maggior presenza tra le osservazioni,
	\item viene scelto un'attributo \emph{a}, in base a una specifica regola, 
	\item viene partizionato \emph{D} a seconda dei \emph{m} valori che può assumere l'attributo \emph{a} 
	\item vengono creati ricorsivamente i sotto-alberi dall'albero con radice \emph{T} senza l'attributo \emph{a} ripetendo i passi appena descritti.
\end{itemize} 
Un iperparametro di quest'algoritmo è la regola per la decisione di quale attributo testare in un nodo. Ne esistono alcune ma in questa applicazione useremo le seguenti due regole.
\begin{itemize}
	\item Cross Entropy: \begin{align*}
		I_E =	- \sum_{k=1}^{m} p_k log(p_k).
		\end{align*} 
	\item Gini Index: \begin{align*}
			I_G = 1 - \sum_{k=1}^{m} p_{k}^2.
		\end{align*} 
\end{itemize}

\begin{align*}
	G(D,a) = I_x - \sum_{v\in V(a)} \frac{|D_a = v|}{D}I_x(D_a=v)
\end{align*} 



             % Modelli ML
\chapter{Risultati dei Algoritmi di machine learning}
\label{cap:RisML}
%**************************************************************
\intro{Questo capitolo illustrerà i risultati ottenuti dai algoritmi K-Nearest-Neighbors (K-NN),  Support Vector Machine (SVM), Decision Tree, Random Forest e infine AdaBoost.  
}

\section{Premesse}
Si sottolinea che, purtroppo, i metodi di \emph{machine learning} non consentono un'analisi interpretativa dei dati come i metodi di \emph{data mining}. Per questo verranno presentati solo i risultati delle predizioni con le relative metriche.\\
Durante la fase di \emph{preprocessing}, i dati sono stati standardizzati attraverso la funzione \textsf{StandardScaler()} del linguaggio Python, ovvero tutti i dati per ogni \emph{feature} sono stati centrati per ottenere media 0 e varianza 1. Questa operazione è stata eseguita per poter rendere le \emph{features} comparabili tra loro. Come già verificato nel Capitolo \ref{cap:Analisi} non ci sono valori mancanti. Durante la fase di \emph{feature selection} si utilizzano le stesse \emph{features} del modello Bradley-Terry escludendo ancora il numero di gol fatti dalla squadra in casa e dagli ospiti. Questo perché, durante una prima fase iniziale di verifica degli algoritmi, i metodi Decision Tree e Random Forest, grazie a queste \emph{feature}, ottenevano un’accuratezza pari a 1. Infatti, sapere il numero di gol segnati in una partita indica implicitamente l'esito della partita stessa. Quindi per non rendere inutili le altre \emph{feature}, il numero di gol segnati dalla squadra in casa e dagli ospiti non sono stati presi in considerazione. Ciononostante, si sottolinea la bontà dei due algoritmi che sono riusciti a trovare la correlazione precedentemente illustrata. Quindi le 26 \emph{features} scelte diventeranno 52, una metà delle quali si riferiscono alla squadra in casa, in cui ogni features viene indicata con il prefisso \textsf{Home\_}, mentre l'altra metà si riferisco alla squadra ospite, in cui ogni features viene indicata con il prefisso \textsf{Away\_}. L'attività di predizione verrà condotta suddividendo il \emph{dataset} in un insieme di training (80\%) e uno di test (20\%). La funzione utilizzata per rendere utilizzabile il \emph{dataset} è indicata nella Sezione \ref{code:a9}.

\section{Ulteriori metriche}
La capacità predittiva di un modello sarà valutata con le metriche illustrate nel Capitolo \ref{cap:risultatiDM} e con la seguente metrica.
\begin{itemize}
	
	\item \textsf{Area Under the Curve}. La Area Under the Curve (AUC) rappresenta l'area al di sotto della curva ROC, che è una rappresentazione grafica del rapporto tra specificità e sensibilità.
	L'AUC è una metrica che va da 0 a 1, dove un valore più alto indica una migliore prestazione di classificazione. L'AUC permette di confrontare le performance di classificatori diversi, poiché è indipendente dalla soglia di classificazione e fornisce un riassunto delle prestazioni del classificatore.
\end{itemize}

\section{K-Nearest-Neighbors}
L'algoritmo K-Nearest-Neighbors è risultato essere semplice da applicare, ottenendo complessivamente discreti risultati in fase di predizione. Per scegliere i valori migliori per gli iperparametri si è applicato la K-Fold Cross Validation con \emph{k = 10}.\\
Gli iperparametri valutati sono stati i seguenti:
\begin{itemize}
	\item \textsf{n\_neighbors}. Indica il numero massimo di vicini da considerare. Si sono verificati i valori da 3 fino a 163 vicini con un aumento unitario di 1.
	\item \textsf{p} indica il parametro potenza della distanza di Minkowski. Si sono verificati i valori \emph{p=1} (Manhattan distance) e \emph{p=2} (Euclidean distance).
\end{itemize}

Nella Figura \ref{fig:knnCV} viene mostrato l'andamento della Cross Validation per ogni valore degli iperparametri.
\begin{figure}[h]
	\begin{center}
		\includegraphics[scale=0.35]{knnCV.png}
		\caption{Grafico dell'andamento della media dell'accuratezza per ogni valore dell'iperparametro \textsf{n\_neighbors} e per ogni tipo di metrica di distanza utilizzata durante l'applicazione della Cross Validation con 10 fold per il modello K-Nearest-Neighbors. Ogni punto è un classificatore con un certo numero di vicini. La linea blu indica l'andamento con la distanza di Manhattan, mentre la linea arancione l'andamento con la distanza euclidea.
		} 
		\label{fig:knnCV}
	\end{center}
\end{figure}

Quello che si può notare dalla figura è che inizialmente, con pochi vicini, la distanza euclidea risulta essere migliore nella fase di training nel \emph{validation} \emph{set}, ma, con l'aumentare del numero dei vicini, la distanza di Manhattan ottiene risultati migliori. \\
Secondo la Cross Validation i valori migliori sono stati: il valore 30 come numero di vicini e la distanza euclidea come metrica della distanza da utilizzare. L'accuratezza ottenuta dal modello migliore nel \emph{validation} \emph{set} è stata di 0.582.\\
Nella fase di predizione si sono ottenute le predizioni mostrate nella Figura \ref{fig:knnpre} con le relative metriche presentate nella Figura \ref{fig:knnmetrics}.

\begin{figure}[h]
	\begin{center}
		\includegraphics[scale=0.60]{tabknn.png}
		\caption{Tabella di confusione del modello K-Nearest-Neighbors con\textsf{ n\_neighbors} = 30 e \textsf{p} = 2. La classe 0.0 indica la vittoria della squadra in casa, la classe 1.0 indica il pareggio tra le due squadre, la classe 2.0 indica la vittoria della squadra ospite.
		} 
		\label{fig:knnpre}
	\end{center}
\end{figure}

\begin{figure}[h]
	\begin{center}
		\includegraphics[scale=0.60]{metricknn.png}
		\caption{Grafico delle misurazione durante la fase di predizione del modello K-Nearest-Neighbors con\textsf{ n\_neighbors} = 30 e \textsf{p} = 2.
		} 
		\label{fig:knnmetrics}
	\end{center}
\end{figure}
I risultati ottenuti sono discreti. Infatti, l'accuratezza del modello nella fase di predizione è di 0.58. Il modello ha molta difficoltà a riconoscere quando un’osservazione è di classe pareggio dato che la sensibilità è pari 0.16 ovvero, delle 19 osservazioni effettivamente di classe pareggio solo tre vengono riconosciute come tali. Purtroppo, quando le osservazioni vengono etichettate con il pareggio, molto spesso viene commesso un errore di classificazione, dato che la precisione è pari a 0.33 ovvero, solo tre osservazioni su nove %che sono state etichettate con la classe pareggio
sono effettivamente di classe pareggio. Il modello, commettendo relativamente un numero di errori basso, ha una specificità pari a 0.89. Risultati discreti si ottengono per la classe vittoria della squadra in casa e vittoria della squadra ospite. La precisione del modello nella classe vittoria della squadra in casa è pari a 0.63. Infatti, 13 osservazioni vengono etichettate erroneamente con la classe vittoria della squadra in casa. La specificità risulta pari a 0.71 mentre la sensibilità della classe vittoria della squadra in casa è pari a 0.73 ovvero, otto osservazioni su trenta non sono state identificate di classe vittoria della squadra in casa. La precisione della classe vittoria della squadra ospite è pari 0.59 poiché sono stati commessi tanti errori di classificazione. Infatti, 13 osservazioni su 32 sono state etichettate erroneamente con la classe vittoria della squadra ospite. Invece la sensibilità registrata per la classe vittoria della squadra ospite è pari a 0.70 dato che solo otto osservazioni non sono state riconosciute appartenenti alla classe vittoria della squadra ospite. Analogamente, anche la specificità della classe vittoria della squadra ospite e buona dato che è pari a 0.73. Perciò, il modello riesce ad individuare quasi tutte le osservazioni di classe vittoria della squadra in casa o di classe vittoria della squadra ospite, anche se per entrambe le due classi ha una discreta precisione a causa di molte istanze di classe pareggio etichettate erroneamente.\\
Risulta non essere particolarmente adatto l'algoritmo K-Nearest-Neighbors per quest'analisi, a causa della grande diversità tra partita e partita. Ciononostante, si ottengono discreti risultati tenendo conto del fatto delle poche osservazioni messe a disposizione.
\section{Support Vector Machine}
L'algoritmo Support Vector Machine ha ottenuto dei buoni risultati durante la fase di predizione. Analogamente all'algoritmo K-Nearest-Neighbors, si è applicata la K-Fold Cross Validation con \emph{k = 10} per individuare i valori migliori per gli iperparametri.\\
Gli iperparametri valutati sono stati i seguenti:
\begin{itemize}
	\item \textsf{C}. Indica la forza applicata della penalità L2. Si sono verificati valori che vanno da 0.1 a 1.5 con un aumento unitario di 0.1.
	\item \textsf{kernel}. Indica quale funzione di kernel utilizzare. Si sono verificati i valori kernel = linear ovvero il linear kernel, kernel = rbf ovvero Gaussian Radial Basis kernel (RBF) e kernel = poly ovvero il polynomial kernel.
\end{itemize}

Nella Figura \ref{fig:svcCV} viene mostrato l'andamento della Cross Validation per ogni valore degli iperparametri.
\begin{figure}[]
	\begin{center}
		\includegraphics[scale=0.35]{svcCV.png}
		\caption{Grafico dell'andamento della media dell'accuratezza per ogni valore dell'iperparametro C e per ogni funzione kernel utilizzata durante l'applicazione della Cross Validation con 10 fold per il modello Support Vector Machine. Ogni punto è un classificatore con un certo valore di C. La linea blu indica l'andamento con la funzione linear kernel mentre la linea verde l'andamento con la funzione RBF e	la linea arancione l'andamento con la funzione polynomial kernel.
		} 
		\label{fig:svcCV}
	\end{center}
\end{figure}
Dal grafico si nota che, dal punto di vista dell'accuratezza registrata nel \emph{validation} \emph{set}, il kernel di tipo lineare ottiene le prestazioni migliori. Infatti, vediamo che la linea blu è costantemente al di sopra rispetto alle due restanti kernel a parità del valore in \textsf{C}. Invece, il polynomial kernel non ha buone prestazioni. Infatti, si registra che la linea arancione è costantemente al di sotto delle altre due linee, perché ha valori molto più bassi di accuratezza. Secondo la Cross Validation i risultati migliori sono stati il valore 1.4 per l'iperparametro \textsf{C} e il Linear kernel come funzione \textsf{kernel} da utilizzare. L'accuratezza ottenuta dal modello migliore nel \emph{validation} \emph{set} è stata di 0.789. Nella fase di predizione si sono ottenute le predizioni mostrate nella Figura \ref{fig:tabsvc} con le relative metriche presentate nella Figura \ref{fig:svcmetrics}.
\begin{figure}[]
	\begin{center}
		\includegraphics[scale=0.60]{tabsvc.png}
		\caption{Tabella di confusione del modello Support Vector Machine con \textsf{C} = 1.4 e \textsf{kernel} = linear. La classe 0.0 indica la vittoria della squadra in casa, la classe 1.0 indica il pareggio tra le due squadre, la classe 2.0 indica la vittoria della squadra ospite.
		} 
		\label{fig:tabsvc}
	\end{center}
\end{figure}
\begin{figure}[]
	\begin{center}
		\includegraphics[scale=0.60]{metricsvc.png}
		\caption{Grafico delle misurazione durante la fase di predizione del modello Support Vector Machine con \textsf{C} = 1.4 e \textsf{kernel} = linear.
		} 
		\label{fig:svcmetrics}
	\end{center}
\end{figure}
Nonostante le poche osservazioni disponibili e l'elevato numero di \emph{features} utilizzate, i risultati ottenuti sono buoni. Infatti, l'accuratezza delle predizioni è pari a 0.78. Il modello riesce a riconoscere buona parte delle osservazioni di classe pareggio. Infatti, la sensibilità per la classe pareggio è pari a 0.68 ovvero, solo sei osservazioni di classe pareggio su 19 non sono state riconosciute come tali. Tuttavia, la precisione scende a 0.59. Infatti, nove osservazioni sono state etichettate erroneamente dal modello con la classe pareggio. Ciononostante, quest'errore non è molto grande dato che la specificità è pari a 0.84. Risultati migliori si ottengono per la classe vittoria della squadra in casa e vittoria della squadra ospite. La precisione per la classe della vittoria della squadra in casa è pari a 0.92. Infatti, solo due osservazioni sono state etichettate erroneamente dal modello con la classe vittoria della squadra in casa. Di conseguenza anche la specificità ha un valore molto alto ovvero pari a 0.95. Tuttavia, sette osservazioni di classe vittoria della squadra in casa non sono state identificate correttamente, infatti la sensibilità è pari a 0.77. Invece, la sensibilità della classe vittoria della squadra ospite è pari a 0.85 ovvero, solo quattro osservazioni non vengo identificate di classe vittoria della squadra ospite. Si rileva una precisione pari a 0.79 ovvero delle 29 osservazioni classificate con la classe vittoria della squadra ospite solo sei risultano essere di un'altra classe. Buone prestazioni anche per la specificità della classe vittoria della squadra ospite che è pari a 0.87.\\
L'algoritmo Support Vector Machine per quest'analisi si è rivelato particolarmente soddisfacente, nonostante la grande diversità da partita a partita e le poche osservazioni messe disposizione (380 partite).


\section{Decision Tree}

Con l'applicazione dell'algoritmo Decision Tree sono stati registrati dei buoni risultati durante la fase di predizione. Analogamente all'algoritmo K-Nearest-Neighbors, si è applicato la K-Fold Cross Validation con \emph{k = 10} per individuare i valori migliori per gli iperparametri.\\
Gli iperparametri valutati sono stati i seguenti:
\begin{itemize}
	\item \textsf{max\_depth}. Indica la profondità massima dell'albero di decisione. Si sono verificati valori che vanno da 3 fino a 52 con un aumento unitario di 1. Si sottolinea che 52 è il numero di \emph{feature} che sono presenti nel \emph{dataset}. Inoltre, questo parametro permette di controllare l'\emph{overfitting} ovvero, se le prestazioni peggiorano durante il training, i rami più profondi dell'albero vengono tagliati.
	\item \textsf{criterion}. Indica la regola di decisione utilizzata per la creazione dell'albero di decisione. Si sono verificate le regole Gini Index e Cross Entropy.
	\item \textsf{min\_samples\_split}. Indica il numero minimo di istanze richieste per dividere un nodo interno. Se la condizione non viene rispettata allora il nodo sarà una foglia attuando una potatura dell'albero di decisione.
\end{itemize}
Nella Figura \ref{fig:dtCV}  viene mostrato l'andamento della Cross Validation per ogni valore degli iperparametri.
\begin{figure}[h]
	\begin{center}
		\includegraphics[scale=0.40]{dtCV.png}
		\caption{Il grafico in alto a sinistra indica l'andamento della media dell'accuratezza per ogni valore dell'iperparametro \textsf{criterion}. Il grafico in alto a destra indica l'andamento della media dell'accuratezza per ogni valore dell'iperparametro \textsf{min\_samples\_split}. Il grafico in basso indica l'andamento della media dell'accuratezza per ogni valore dell'iperparametro \textsf{max\_depth}. Entrambi i grafici sono l'applicazione della Cross Validation con 10 fold per il modello Decision Tree. 
		} 
		\label{fig:dtCV}
	\end{center}
\end{figure}
Dal grafico in alto a sinistra si nota che tra le due regole la migliore in termini di accuratezza media registrata nel \emph{validation set} è stata la Cross Entropy. Nel grafico in alto a destra i valori 3,4,6,7 e 9, assunti dall'iperparametro \textsf{min\_samples\_split} hanno fatto registrare le accuratezze medie più alte.\\
Nel grafico in basso vi è un andamento abbastanza irregolare. Si nota un forte decadimento delle prestazioni con l'aumento del valore del limite della profondità, soprattutto con valori superiore a 5. Secondo la Cross Validation i valori migliori sono stati il valore 3 sia per l'iperparametro \textsf{max\_depth} e sia per l'iperparametro \textsf{min\_samples\_split} e la Cross Entropy come regola di decisione da utilizzare. L'accuratezza ottenuta dal modello migliore nel \emph{validation} \emph{set} è stata di 0.684.\\
Nella fase di predizione si sono ottenute le predizioni mostrate nella Figura \ref{fig:tabdt} con le relative metriche presentate nella Figura \ref{fig:dtmetrics}.
\begin{figure}[h]
	\begin{center}
		\includegraphics[scale=0.60]{tabdt.png}
		\caption{Tabella di confusione del modello Decision Tree con \textsf{max\_depth} = 3, \textsf{min\_samples\_split} = 3 e \textsf{criterion} = entropy. La classe 0.0 indica la vittoria della squadra in casa, la classe 1.0 indica il pareggio tra le due squadre, la classe 2.0 indica la vittoria della squadra ospite.
		} 
		\label{fig:tabdt}
	\end{center}
\end{figure}
\begin{figure}[]
	\begin{center}
		\includegraphics[scale=0.60]{metricdt.png}
		\caption{Grafico delle misurazione durante la fase di predizione del modello Decision Tree con \textsf{max\_depth} = 3, \textsf{min\_samples\_split} = 3 e \textsf{criterion} = entropy.
		} 
		\label{fig:dtmetrics}
	\end{center}
\end{figure}
Nonostante le poche osservazioni disponibili i risultati ottenuti sono buoni. Infatti, l'accuratezza delle predizioni è di 0.71. Il modello ha comunque qualche difficoltà a identificare le osservazioni della classe pareggio. Infatti, la sensibilità è pari a 0.26. Nonostante il modello etichetti un'osservazione con la classe pareggio in poche occasioni, esso sbaglia poco. Ci sono solo due osservazioni etichettate erroneamente con la classe pareggio, quindi la precisione è pari a 0.71. Ovviamente la specificità nella classe pareggio è molto alta dato che il modello poche volte sbaglia ad etichettare con la classe pareggio. Buoni risultati si ottengono sia per la classe vittoria della squadra in casa sia per la classe vittoria della squadra ospite. Infatti, per la prima classe c'è una sensibilità pari a 0.93, ovvero, soltanto due osservazioni di classe vittoria della squadra in casa non sono state identificate come tali. Le prestazioni però calano nella precisione dato che undici osservazioni sono state etichettate erroneamente con la classe vittoria della squadra in casa, registrando una precisione pari a 0.72. Di conseguenza anche la specificità è calata rispetto alla sensibilità con un valore pari a 0.76. La classe vittoria della squadra ospite ha una sensibilità pari a 0.78. Infatti, sei osservazioni di classe vittoria della squadra ospite non sono state identificate come tali. Anche la precisione rimane in linea con le prestazioni della sensibilità con un valore pari a 0.70, in cui nove osservazioni su trenta sono state etichettate erroneamente con la classe vittoria della squadra ospite. Nonostante ciò, la specificità è pari a 0.82.\\
Dato che l'algoritmo Decision Tree non è un algoritmo a scatola chiusa (\emph{black box}), è possibile capire quali sono state le sue scelte analizzando l'albero di decisione che ha prodotto per svolgere le sue predizioni. Nella Figura \ref{fig:dttree} viene mostrato l'albero di decisione prodotto e utilizzato durante la fase di predizione.
\begin{figure}[h]
	\begin{center}
		\includegraphics[height = 10cm, width = 16cm]{treedt.png}
		\caption{L'albero di decisione del modello Decision Tree con \textsf{max\_dept}h = 3 e \textsf{criterion} = entropy. Per ogni nodo non foglia c'è un test che indica quale ramo scegliere asseconda del valore contenuto nell'attributo testato. Il parametro \textsf{entropy} indica l'entropia misurata. Il parametro \textsf{samples} indica il numero di istanze che soddisfano i test precedenti. Nel parametro \textsf{value} vengono riportati il numero di istanze presenti per ognuna delle tre classi. Il parametro \textsf{class} indica la classe di maggioranza nel nodo. Il colore indica la classe di maggioranza con una tonalità differente asseconda della frequenza della classe di maggioranza.
		} 
		\label{fig:dttree}
	\end{center}
\end{figure}
Nell'albero mostrato in Figura \ref{fig:dttree}, per ogni nodo non foglia c'è un test che indica quale ramo scegliere a seconda del valore contenuto nell'attributo testato. Il test consiste semplicemente nel verificare se l'attributo ha un valore minore o uguale oppure maggiore rispetto a una certa soglia. Questo tipo di test permette di gestire attributi con valori continui utilizzando una soglia di decisione scelta opportunamente dell'algoritmo. Per tutti i nodi è presente il parametro \textsf{entropy} che indica l'entropia misurata. Il parametro \textsf{samples} che indica il numero di istanze che soddisfano i test precedenti. Il parametro \textsf{value} che il numero di istanze presenti per ognuna delle tre classi. Il parametro \textsf{class} che indica la classe di maggioranza nel nodo, mentre il colore indica la classe di maggioranza con una tonalità differente a seconda della frequenza della classe di maggioranza. Quando viene raggiunto un nodo foglia si classifica l'osservazione con la classe di maggioranza dato che non ci sono più attributi che permettono di distinguere le istanze.\\
Osservando l'albero, le \emph{feature} legate ai tiri si confermano di grande importanza non solo i modelli Bradley-Terry visti nel Capitolo \ref{cap:risultatiDM}, ma anche per il modello Decision Tree.\\
In conclusione, l'algoritmo Decision Tree ottiene delle buone prestazioni pur rimanendo semplice senza diventare troppo complesso.

\section{Random Forest}
Con l'applicazione dell'algoritmo Random Forest si sono registrati dei buoni risultati durante la fase di predizione. Analogamente all'algoritmo K-Nearest-Neighbors, si è applicato la K-Fold Cross Validation con \emph{k = 10} per individuare i valori migliori per gli iperparametri.\\
Gli iperparametri valutati sono stati i seguenti:
\begin{itemize}
	\item \textsf{criterion}. Indica la regola di decisione utilizzata per la creazione degli alberi di decisione. Si sono verificate le regole Gini Index e Cross Entropy.
	\item \textsf{max\_depth}. Indica la profondità massima degli alberi di decisione utilizzati. Si sono verificati valori che vanno da 3 fino a 52 con un aumento unitario di 1. Si sottolinea che 52 è il numero di \emph{feature} che sono presenti nel \emph{dataset}. Inoltre, questo parametro permette di controllare l'\emph{overfitting} ovvero, se le prestazioni peggiorano durante il training, i rami più profondi dell'albero vengono tagliati.
	\item \textsf{n\_estimators}. Indica il numero massimo di classificatori impiegati per il Bagging. Si sono verificati valori che vanno da 3 fino a 126 classificatori con un aumento unitario di 1.
	\item \textsf{min\_samples\_split}. Indica il numero minimo di istanze richieste per dividere un nodo interno. Se la condizione non viene rispettata allora il nodo sarà una foglia attuando una potatura dell'albero di decisione.
\end{itemize}
Nella Figura \ref{fig:rfCV} viene mostrato l'andamento della Cross Validation per ogni valore degli iperparametri.
\begin{figure}[]
	\begin{center}
		\includegraphics[scale=0.40]{rfCV.png}
		\caption{Il grafico in alto a sinistra indica l'andamento della media dell'accuratezza per ogni valore dell'iperparametro \textsf{criterion}. Il grafico in alto a destra indica l'andamento della media dell'accuratezza per ogni valore dell'iperparametro \textsf{min\_samples\_split}. Il grafico al centro indica l'andamento della media dell'accuratezza per ogni valore dell'iperparametro \textsf{max\_depth}. Il grafico in basso indica l'andamento della media dell'accuratezza per ogni valore dell'iperparametro \textsf{n\_estimators}. Tutti i grafici sono l'applicazione della Cross Validation con 10 fold per il modello Random Forest. 
		} 
		\label{fig:rfCV}
	\end{center}
\end{figure}
Dal grafico in alto a sinistra si nota che tra le due regole la migliore in termini di accuratezza media registrata nel \emph{validation set} è stata la Cross Entropy. Nel grafico in alto a destra con il valore 7 nell'iperparametro \textsf{min\_samples\_split} il modello ottiene l'accuratezza media più alta. Nel grafico al centro vi è un andamento abbastanza irregolare. Si nota che con il valore 17 dell'iperparametro \textsf{max\_depth} vi è un aumento improvviso del valore dell'accuratezza media del modello. Infine, anche nel grafico in basso vi è un andamento abbastanza irregolare. Nonostante un'iniziale tendenza di aumento dell'accuratezza. Quando l'iperparametro \textsf{n\_estimators} assume il valore 65 vi è un picco di prestazione dell'accuratezza media.\\
Secondo la Cross Validation i valori migliori sono stati per l'iperparametro \textsf{criterion} la Cross Entropy, il valore 17 per l'iperparametro \textsf{max\_depth}, per l'iperparametro \textsf{min\_samples\_split} il valore 7 e per l'iperparametro \textsf{n\_estimators} il valore 65 come numero di classificatori da impiegare.
L'accuratezza ottenuta dal modello migliore nel \emph{validation} \emph{set} è stata di 0.753.\\
Nella fase di predizione si sono ottenute le predizioni mostrate nella Figura \ref{fig:tabrf} con le relative metriche presentate nella Figura \ref{fig:rfmetrics}.
\begin{figure}[h]
	\begin{center}
		\includegraphics[scale=0.60]{tabrf.png}
		\caption{Tabella di confusione del modello Random Forest con \textsf{criterion} = gini, \textsf{max\_depth} = 5, \textsf{n\_estimators} = 102 e \textsf{min\_samples\_split} = 9. La classe 0.0 indica la vittoria della squadra in casa, la classe 1.0 indica il pareggio tra le due squadre, la classe 2.0 indica la vittoria della squadra ospite.
		} 
		\label{fig:tabrf}
	\end{center}
\end{figure}

\begin{figure}[]
	\begin{center}
		\includegraphics[scale=0.60]{metricrf.png}
		\caption{Grafico delle misurazione durante la fase di predizione del modello Random Forest con \textsf{criterion} = gini, \textsf{max\_depth} = 5, \textsf{n\_estimators} = 102 e \textsf{min\_samples\_split} = 9.
		} 
		\label{fig:rfmetrics}
	\end{center}
\end{figure}
Nonostante le poche osservazioni disponibili i risultati ottenuti sono buoni. Infatti, l'accuratezza delle predizioni e di 0.72. Anche in questo modello si sono riscontrate alcune difficoltà nell'identificare le istanze della classe pareggio. La sensibilità rilevata sulla classe pareggio è pari a 0.37, perché solo sette osservazioni su 19 osservazioni di classe pareggio sono state identificate con la classe pareggio. Tuttavia, le prestazioni migliorano con la precisione del modello nell'etichettare un'istanza con la classe pareggio, infatti delle undici osservazioni classificate con la classe pareggio solo quattro sono risultate essere di una classe differente. Di conseguenza anche la specificità sulla classe pareggio ha un valore alto pari a 0.92. Si registrano prestazioni migliori con le classi vittoria della squadra in casa e vittoria della squadra ospite. Per la classe vittoria della squadra in casa si ha una sensibilità pari a 0.90 ovvero, solo tre osservazioni su 30 non sono state riconosciute di classe vittoria della squadra in casa. Tuttavia, si registra un calo di prestazioni nella precisione nell'etichettare correttamente le istanze di classe vittoria della squadra in casa. Infatti, si registra un valore pari a 0.77 perché otto osservazioni sono state classificate erroneamente con la classe vittoria della squadra in casa. Di conseguenza anche la specificità sulla classe della vittoria della squadra in casa cala, raggiungendo un valore pari a 0.82. Buone prestazioni del modello si rilevano nell'identificare le istanze di classe vittoria della squadra ospite, con una sensibilità pari a 0.78. Purtroppo, si registra una precisione più bassa ovvero pari a 0.70, con nove osservazioni classificate erroneamente con la classe vittoria della squadra ospite. Infine, la specificità del modello sulla classe vittoria della squadra ospite è pari a 0.81.\\
Dato che l'algoritmo Random Forest utilizza i classificatori Decision Tree è possibile capire quali sono le \emph{features} più importanti secondo il modello per produrre una predizione dell'esito di una partita. Perciò, nella Figura \ref{fig:rftree} viene mostrato per ogni \emph{feature} la sua importanza misurata sull'impurità.
\begin{figure}[]
	\begin{center}
		\includegraphics[height = 10cm, width = 14cm]{rfCV5.png}
		\caption{Il grafico riporta per ogni \emph{feature} la sua importanza basata sull'impurità.
		} 
		\label{fig:rftree}
	\end{center}
\end{figure}
Osservando i risultati ottenuti essi sono coerenti con quanto visto nell'albero del Decision tree. Le \emph{features} legate ai tiri si confermano di grande importanza, così come i passaggi completati di tutti i tipi. Invece, poco significativi sono il numero di parate ma anche il numero di fuorigioco, intercetti e contrasti vinti. Perciò, in sostanza sembrano essere poco significative le \emph{features} legate alla difesa, fatta eccezione per il numero di recuperi che sembra essere significativo. Inoltre, le \emph{features} riguardanti i falli quindi numero di falli subiti e numero di falli fatti sono poco importanti per il modello, inoltre come già visto nei modelli BT anche qui il possesso della palla è poco importante. Per quanto riguarda le \emph{features} riguardanti i tocchi nelle diverse zone del campo, esse sono mediamente importanti. Non sorprendentemente, la più importante è il numero di tocchi fatti nell'area di rigore avversaria. \\
In conclusione, nonostante le poche osservazioni disponibili, l'algoritmo Random Forest ottiene nel complesso delle buone prestazioni, nonostante qualche difficoltà nell’identificazione dei pareggi.

\section{AdaBoost}
Con l'applicazione dell'algoritmo AdaBoost si sono registrati dei buoni risultati durante la fase di predizione. Analogamente all'algoritmo K-Nearest-Neighbors, si è applicato la K-Fold Cross Validation con \emph{k = 10} per individuare i valori migliori per gli iperparametri.\\
Gli iperparametri valutati sono stati i seguenti:
\begin{itemize}
	\item \textsf{n\_estimators}. Indica il numero massimo di classificatori impiegati per il Boosting. Si sono verificati valori che vanno da 3 fino a 126 classificatori con un aumento unitario di 1.
	\item \textsf{learning\_rate}. Indica il peso applicato ad ogni classificatore per ogni iterazione del Boosting. Si sono verificati valori che vanno da 0.1 fino a 1.0 con un aumento unitario di 0.1. Più l'\textsf{learning\_rate} è elevato più sarà il contributo di ciascun classificatore.
\end{itemize}
Esiste un \emph{trade-off} tra i parametri \textsf{learning\_rate} e \textsf{n\_estimators}. Nella Figura \ref{fig:adaCV} viene mostrato l'andamento della Cross Validation per ogni valore degli iperparametri.
\begin{figure}[h]
	\begin{center}
		\includegraphics[scale=0.35]{adaCV.png}
		\caption{Grafico dell'andamento della media dell'accuratezza per ogni valore dell'iperparametro \textsf{n\_estimators} e per ogni valore dell'iperparametro \textsf{learning\_rate} utilizzati durante l'applicazione della Cross Validation con 10 fold per il modello AdaBoost. Ogni punto utilizza un certo numero classificatori mentre il colore indica il peso dell'\emph{learning rate}.
		} 
		\label{fig:adaCV}
	\end{center}
\end{figure}
Dal grafico, notiamo un andamento abbastanza irregolare per quasi tutti i tassi d'apprendimento, eccetto per il modello con il tasso d'apprendimento pari a 0.1, il quale inizialmente ottiene prestazioni migliori con il salire del numero dei classificatori. Tutti i tassi d'apprendimento con un numero elevato di classificatori calano il loro valore di accuratezza nel \emph{validation set}. Nonostante il modello con \textsf{learning\_rate} pari a 0.1 abbia costantemente valori più alti, il modello con \textsf{learning\_rate} pari a 0.5 ha segnato un'accuratezza più alta anche se, con l'aumentare del numero di classificatori, le sue prestazioni calano di molto. \\
Perciò, secondo la Cross Validation i valori migliori sono stati il valore 19 per l'iperparametro \textsf{n\_estimators} e il valore 0.5 per l'iperparametro \textsf{learning\_rate}. L'accuratezza ottenuta dal modello migliore nel \emph{validation} \emph{set} è stata di 0.78.\\
Nella fase di predizione si sono ottenute le predizioni mostrate nella Figura \ref{fig:tabada} con le relative metriche presentate nella Figura \ref{fig:adametrics}.
\begin{figure}[h]
	\begin{center}
		\includegraphics[scale=0.60]{tabada.png}
		\caption{Tabella di confusione del modello AdaBoost con \textsf{n\_estimators} = 19 e \textsf{learning\_rate} = 0.5. La classe 0.0 indica la vittoria della squadra in casa, la classe 1.0 indica il pareggio tra le due squadre, la classe 2.0 indica la vittoria della squadra ospite.
		} 
		\label{fig:tabada}
	\end{center}
\end{figure}

\begin{figure}[h]
	\begin{center}
		\includegraphics[scale=0.60]{metricada.png}
		\caption{Grafico delle misurazione durante la fase di predizione del modello AdaBoost con \textsf{n\_estimators} = 19 e \textsf{learning\_rate} = 0.5.
		} 
		\label{fig:adametrics}
	\end{center}
\end{figure}
Nonostante le poche osservazioni disponibili i risultati ottenuti sono buoni. Infatti, l'accuratezza delle predizioni è di 0.72. Anche questo modello ha delle difficoltà a identificare le istanze di classe pareggio. La sensibilità del modello sulla classe pareggio è pari a 0.58 ovvero, undici osservazioni su 19 di classe pareggio sono state identificate come tali. Purtroppo, il modello non è abbastanza soddisfacente nell’etichettare le istanze con la classe pareggio perché la precisione è pari a 0.46. Di conseguenza anche la specificità ne risente, con un valore pari a 0.77. Tuttavia, le prestazioni sono decisamente migliori con le classi vittoria della squadra in casa e vittoria della squadra ospite. Infatti, per la classe vittoria della squadra in casa la sensibilità è pari a 0.83, cioè solo cinque osservazioni su 30 non sono state riconosciute di classe vittoria della squadra. Analogamente, la specificità della classe vittoria della squadra in casa ha un valore molto alto ovvero, 0.93. Inoltre, il modello risulta essere molto preciso nel classificare correttamente le istanze con la classe vittoria della squadra in casa. Infatti, il modello ha commesso solo tre errori ottenendo una precisione pari a 0.89. Buone prestazioni si sono registrate anche nell'identificare le istanze di classe vittoria della squadra ospite da parte del modello. La sensibilità è pari a 0.70, mentre la precisione è pari a 0.79, con solo cinque osservazioni classificate erroneamente con la classe vittoria della squadra ospite. Buona anche la specificità del modello sulla classe vittoria della squadra ospite che risulta essere pari a 0.89.\\
In conclusione, nonostante le poche osservazioni messe a disposizione, l'algoritmo AdaBoost ottiene nel complesso delle buone prestazioni nonostante qualche difficoltà nell’identificazione dei pareggi.

\section{Comparazione dei algoritmi}
Grazie alla metrica AUC è possibile confrontare le performance degli algoritmi utilizzati. 
Nella Figura \ref{fig:auc} vengono riportate le AUC che sono state misurate durante la fase di predizione di ogni algoritmo utilizzato.

\begin{figure}[h]
	\begin{center}
		\includegraphics[scale=0.30]{auc.png}
		\caption{Grafico a barre in cui viene riportata la Area Under the Curve (AUC) registrata durante la fase di predizione dei algoritmi K-Nearest-Neighbors (K-NN),  Support Vector Machine (SVM), Decision Tree, Random Forest e AdaBoost.  
		} 
		\label{fig:auc}
	\end{center}
\end{figure}
Dalle misurazioni ottenute si evince che l'AUC più alta è stata registrata nell'algoritmo SVM con un valore pari a 0.82. Infatti, la SVM ha una buona accuratezza perché è l'algoritmo che riesce a identificare con maggior precisione la classe pareggio, la quale è risultata molto ostica da identificare per gli algoritmi trattati. Con una AUC leggermente inferiore, l'algoritmo AdaBoost si classifica come secondo con un valore pari a 0.78. Analogamente alla SVM, l'AdaBoost si distingue dai algoritmi con prestazioni inferiori grazie alle discrete prestazioni nell'identificazione della classe pareggio. L'algoritmo con la terza AUC più alta è stato il Random Forest con una AUC pari a 0.77 e con prestazioni simili all'algoritmo AdaBoost.
In quarta posizione si piazza l'algoritmo Decision Tree con una AUC pari a 0.75. Nonostante le buone prestazioni registrate nell'identificazione delle classi vittoria della squadra in casa e vittoria della squadra ospite, l'algoritmo Decision Tree paga una minor AUC a causa delle brutte prestazioni registrate nell'identificazione della classe pareggio. Infine, l'algoritmo con la più bassa AUC misurata è il K-NN con una AUC pari a 0.65. Infatti, l'algoritmo K-NN ha registrato delle pessime prestazioni nell'identificare le istanze di classe pareggio e delle discrete prestazioni nell'identificare le istanze delle altre due classi.\\
In conclusione, il problema presentato risulta essere troppo complesso per l'algoritmo K-NN: la strategia di classificare una nuova istanza con la classe di maggioranza dei k-vicini, seppur semplice, non è risultata efficace. Viceversa, la SVM, grazie all'utilizzo della funzione kernel, riesce a gestire la complessità dei dati ottenendo delle buone prestazioni in fase di predizione.
             % Modelli ML
% !TEX encoding = UTF-8
% !TEX TS-program = pdflatex
% !TEX root = ../tesi.tex

%**************************************************************
\chapter{Discussione e comparazione dei risultati}
\label{cap:risFin}
%**************************************************************
\intro{In questo capitolo si discuteranno i risultati ottenuti dai modelli di Bradley-Terry e dai modelli di Machine Learning. Si concluderà con una comparazione tra i modelli di Data Mining e quelli di Machine Learning.
}
%**************************************************************
\section{Discussione risultati dei modelli Bradley-Terry}
Come illustrato nel Capitolo \ref{cap:risultatiDM} sono state applicate quattro versione del modello Bradley-Terry. Con la prima versione, ovvero il modello Bradley-Terry standard \autocite{bradley1952rank} (\hyperref[for:3.9]{4.9}) è stato possibile costruire una base da cui partire per ottenere risultati sempre più esplicativi e precisi. Infatti, dalla applicazione delle covariate nelle comparazioni ovvero dal modello (\ref{for:5.1}), è stato possibile approfondire in che modo le variabili esplicative influenzano l'esito di una partita. Successivamente per abbassare la complessità del modello e per aumentarne la flessibilità, con il modello (\ref{for:4.9}) è stato introdotto e applicato il metodo di regolarizzazione LASSO. Infine, per approfondire gli effetti delle singole variabili esplicative asseconda della squadra in esame è stato applicato il modello \ref{for:5.2} in cui viene tolto l'effetto dell'intercetta.\\
Dai risultati ottenuti e dalle analisi condotte è possibile concludere quanto segue. Nel campionato italiano, ai fini della vittoria o, in generale, dell'ottenimento di buoni risultati è rilevante per una squadra adottare un comportamento tattico, in particolare, dell'utilizzo della costruzione dal basso e giocare prevalentemente nella propria metà campo. 
È importante che la squadra adotti un comportamento meno propenso a controllare il pallone per lungo tempo perché sia il possesso della palla \texttt{Poss} e sia la distanza percorsa con la palla \texttt{TotDist} non sono significavi. Inoltre, la squadra deve essere più propensa a giocare maggiormente la palla nella propria area di difesa per evitare contropiedi perché sia le stime del numero di tocchi in area di rigore \texttt{ToDefPen}, sia il numero di tocchi nella trequarti difensiva \texttt{ToDef3rd} e sia il numero di tocchi a centrocampo \texttt{ToMid3rd} sono associati ad un aumento della probabilità di vittoria. Avere perciò una buona difesa è fondamentale. La fase offensiva non deve essere troppo lunga in termini di possesso della palla. Infatti, il numero di tocchi fatti nella trequarti offensiva \texttt{ToAtt3rd} porta ad avere una diminuzione delle probabilità di vittoria ma se si fanno i giusti passaggi per entrare nell'area di rigore avversaria mantenendo sempre un possesso palla breve si aumentano le probabilità di vittoria come visto nella stima del numero di tocchi in area di rigore avversaria \texttt{ToAttPen}. Dalle analisi emerge che uno sbilanciamento verso la fase offensiva porta una forte diminuzione alle probabilità di vittoria. Considerando i casi di Inter e Atalanta, la prima si dimostra essere una dalle squadre che più tira in generale (\texttt{Sh}) e in porta (\texttt{SoT}). Alti valori sono presenti anche per l'Atalanta. Entrambe però mantengono troppo il controllo del pallone nell'area avversaria. Infatti, per entrambe le squadre si riscontrano notevoli diminuzioni della probabilità di vittoria a causa della stima del parametro di \texttt{ToAtt3rd}. Peggio ancora per l'Atalanta, che ha un gioco particolarmente offensivo (vedi \textit{\cite{ataGioco}}), che le fa ottenere una diminuzione della probabilità della vittoria dalla stima del parametro \texttt{ToAttPen}. Questo perché il prolungato controllo del pallone la porta a esporsi e a subire contropiede. Si è parlato spesso di contropiedi nella nostra analisi. Quello che emerge sempre in tema di fase offensiva è che, il numero di tiri è relativamente basso, fatto dimostrato dal notevole aumento della probabilità di vittoria portato della stima del rapporto gol/tiri \texttt{G/Sh}. Di conseguenza le squadre attaccano poco e, quando lo fanno, cercano di massimizzare la loro fase offensiva. Infatti, le partite nel campionato italiano spesso finiscono con un massimo di due o tre gol segnati. Pertanto, l'efficacia di un azione offensiva che porta al gol e la carenza di azioni offensive portano \texttt{Sh}, \texttt{SoT} ma soprattutto \texttt{G/Sh} ad assumere un elevato contributo nel determinare la vittoria.\\
Concludendo la trattazione sulla fase offensiva, si illustra quale sia il miglior modo di attaccare che emerge dai modelli. Si sa che il contropiede è efficace, ma allo stesso tempo difficile da attuare per via del comportamento delle squadre a non sbilanciarsi. Una valida alternativa che emerge è il lancio lungo che parte dall'area compressa tra l'area di rigore della squadra fino a centrocampo ed arriva nell'area avversaria. Infatti, la stima del parametro del numero di passaggi lunghi tentati \texttt{LPAtt} è associata ad una crescita della probabilità di vittoria. L'utilizzo di passaggi filtrati \texttt{MPCmp\%} non è una buona tattica. Analogamente anche i cross \texttt{Crs} non danno benefici. Anzi, causano svantaggi, e ancora una volta ne rimangono penalizzate Inter e soprattutto Atalanta che con il suo gioco sfrutta molto le fasce (vedi \textit{\cite{ataGioco}}). In conclusione, è importante sottolineare che un atteggiamento troppo speculativo o difensivo da parte della squadra non porta alla vittoria. Questo è il caso del Venezia, classificatosi come ultimo, e che ha ottenuto benefici dalle covariate \texttt{ToDefPen} e \texttt{ToDef3rd} ma non dalle variabili esplicative offensive. Dall'analisi emerge che per ottenere la vittoria una squadra debba mantenere un comportamento tattico e giocare prevalentemente nella propria metà campo.\\
Infine, ogni modello ha prodotto delle predizioni dei risultati di alcune partite. Si sono confrontate le prestazioni dei quattro modelli durante la fase di test tra loro insieme alle predizioni fatte dai \emph{bookmakers}. Quello che si evince è che tutti i modelli Bradley-Terry hanno prestazioni migliori delle predizioni dei \emph{bookmakers}, segno che le informazioni vengo utilizzate correttamente. Purtroppo, dato che i modelli di Data Mining mirano a interpretare le relazioni tra i dati e non a imparare caratteristiche molto complesse dei dati, non sono adatti alle predizioni. Infatti, delle discrete prestazioni dal punto di vista dell'accuratezza, precisione, sensibilità e specificità. Il modello Bradley-Terry che ha ottenuto le migliori prestazioni in fase di predizione è stato (\ref{for:4.9}).
\section{Discussione risultati dei modelli di Machine Learning}

\section{Confronto}             % confronti 
% !TEX encoding = UTF-8
% !TEX TS-program = pdflatex
% !TEX root = ../tesi.tex

%**************************************************************
\chapter{Conclusioni}
\label{cap:conclusioni}
%**************************************************************
MEMO Riassunto del lavoro/risultati ottenuti, possibili estensione e migliorie che possono essere apportate. Sottolineare che alcune variabili possono avere un peso differente a seconda della lega in cui si svolge la partita, (ad esempio Premier league è un campionato più fisico con alti ritmi rispetto alla Serie A che è più "tattica") TO DO
%**************************************************************             % Conclusioni
%**************************************************************
\chapter{Conclusioni}
\label{cap:conclusioni}
%**************************************************************

%**************************************************************
In questa tesi, partendo dalla domanda posta inizialmente "Cosa influenza il successo o il fallimento delle singole squadre durante una partita di calcio?" si sono analizzate le possibili relazioni tra l'esito di una partita di calcio e le principali statistiche raccolte durante la partita. L'utilizzo di un modello di confronto a coppie, ovvero il modello Bradely-Terry è un naturale strumento di analisi che è stato applicato sui dati che si riferiscono agli incontri della Serie A italiana 2021/2022.\\
\begin{comment}
	L'analisi, attraverso strumenti grafici, inizia con lo studio dei dati individuando le possibili relazioni tra l'esito della partita e le singole 29 variabili esplicative, e le possibili interazioni tra covariate. Successivamente ad alcune operazioni di \emph{prepocessing}, l'analisi è continuata con la modellazione dei modelli Bradley-Terry. La modellazione parte con il modello Bradley-Terry standard per stimare l'abilità delle singole squadre e l'effetto di giocare in casa per poi spingersi sempre più in profondità, introducendo 26 variabili esplicative fino a analizzarne il loro effetto sulle partite per ogni singola squadra. Successivamente, per verificare che l'uso dei dati è stato svolto correttamente, si è svolto l'attività di predizione con i modelli confrontando le loro predizioni con le predizioni fatte dai \emph{bookmakers}. Per rendere il lavoro di tesi più completo e approfondito, l'analisi è passata allo studio delle predizioni fatte da metodi di apprendimento automatico ovvero, il K-Nearest-Neighbors (K-NN), la Support Vector Machine (SVM), il Decision Tree, la Random Forest e l'AdaBoost. Dopo un breve confronto tra i vari metodi di \emph{machine learning} dal punto di vista delle prestazioni registrate durante la fase di predizione, i metodi di apprendimento automatico sono stati utilizzati come riferimento dal punto di vista dell'accuratezza, della precisione, della sensibilità e della specificità, per valutare la bontà dei modelli BT in fase di predizione. Successivamente, l'analisi prosegue confrontando i metodi Decision Tree e Random Forest con i modelli BT, sull'identificazione delle statistiche che influenzano l'esito di una partita. Infine, per un maggior approfondimento, è stato riapplicato il modello BT con covariate specifiche per ogni partita e per ogni squadra, estendendo la variabile risposta da tre a cinque categorie, ottenendo un risultato più raffinato ma in linea con quanto ricavato precedentemente.\\
\end{comment}
I risultati ricavati dai cinque modelli di Bradley-Terry e dai algoritmi Decision Tree e Random Forest, mettono in luce l'importanza dei tiri, l'effetto di giocare in casa, il mantenimento del controllo della palla nell'area della squadra in possesso della palla e dell'utilizzo di lanci lunghi al fine di una vittoria. Viceversa, viene individuato che i cross, i passaggi filtranti e un alto numero di tocchi del pallone nell'area dell’avversario sono associati negativamente all'esito della partita. Per le squadre con la maggior abilità stimata subire un fuorigioco incide negativamente sul loro successo sportivo. Infine, viene ricavato che il possesso della palla non viene associato all'esito della partita.\\
Chiaramente quanto ricavato vale per la stagione 2021/2022 del campionato italiano di Serie A. È naturale quindi che l'analisi possa essere estesa su un’altra stagione della Serie A oppure ad gli altri campionati europei quali, la Premier League, la Liga spagnola, la Ligue 1 e la Bundesliga. Occorre fare attenzione al fatto che sia nella Ligue 1 e nella Bundesliga negli ultimi anni Paris Saint Germain e Bayern München hanno monopolizzato la vittoria del campionato. Quindi c'è da aspettarsi ampi dislivelli di abilità tra queste squadre e le altre dei rispettivi campionati. Per quanto riguarda la Premier League è possibile aspettarsi una stima delle abilità più bilanciata, ma soprattutto una stima fortemente positiva riguardo al possesso della palla grazie allo stile di gioco proposto dall'allenatore del Manchester City, Pep Guardiola (vedi \textit{\cite{futbol}}), vincitore di quattro delle ultime cinque edizioni del campionato inglese.\\
Sicuramente d'interesse potrebbe essere l'applicazione di un modello Bradley-Terry dinamico \autocite{cattelan2013dynamic}, ovvero che vada a valutare la variazione dell'abilità di ogni singola squadra durante la stagione sportiva, in modo da individuare possibili fenomeni che possano ripercuotersi positivamente o negativamente sulle prestazioni delle squadre.\\
Naturalmente, oltre a valutare l'abilità di una squadra durante la stagione un'ulteriore estensione dell'analisi è considerare più di una stagione \autocite{tsokos2019modeling}, sebbene ci sia un atteso aumento di complessità computazionale. \\
Occorre sottolineare che nelle analisi svolte, le covariate vengono incorporate in modo lineare nelle abilità delle squadre nelle specifiche partite. L'aggiunta di effetti non lineari attraverso l'utilizzo di metodi di \emph{machine learning} non supervisionati come ad esempio, il Clustering \autocite{dunn1974well}, può essere una possibile estensione del lavoro. Infatti come riportato nel lavoro svolto da \textcite{shin2014novel}, il Clustering è utilizzato per individuare caratteristiche non lineari comuni tra squadre di calcio come ad esempio, le strategie di gioco. Infatti, gli algoritmi di apprendimento non supervisionato hanno l'obiettivo di raggruppare e interpretare i dati basandosi solo sull'input ricevuto, attraverso l'individuazione di \emph{features} non lineari. Questi algoritmi permettono di migliorare l'analisi ma con lo svantaggio di un maggior costo computazionale da sostenere.\\
Certamente anche l'aggiunta di altre covariate come, ad esempio, la distanza percorsa dai giocatori o il numero di calci d'angolo battuti, potrebbe permettere di individuare nuove statistiche chiave che determinano l'esito della partita.\\

			
\chapter{Appendice A}

\section{Codice di adattamento dataset per il trasferimento dati} \label{sec:a1}
\begin{lstlisting}[language=R]
PossVs <- c()
ShVs <- c()
ShTVs <- c()
G.ShVs <- c()
PAttVs <- c()
PCmp.Vs <- c()
SPAttVs <- c()
SPCmp.Vs <- c()
MPAttVs <- c()
MPCmp.Vs <- c()
LPAttVs <- c()
LPCmp.Vs <- c()
ToDef3rdVs <- c()
ToMid3rdVs <- c()
ToAtt3rdVs <- c()
ToAttPenVs <- c()
ToDistVs <- c()
FlsVs <- c()
FldVs <- c()
CrsVs <- c()
IntVs <- c()
TklWinVs <- c()
RecovVs <- c()
del <-c()
k <- 1
z <- 1
for(i in 1:nrow(soccern)){
	if(soccern$AtHome[i] == TRUE){
		for(j in 1:nrow(soccern)){
			if((soccern$Team[j] == soccern$Vs[i]) && (soccern$Team[i] == soccern$Vs[j]) && (soccern$AtHome[j] == FALSE)){
				PossVs[k] <- soccern$Poss[j]
				ShVs[k] <- soccern$Sh[j]
				ShTVs[k] <- soccern$SoT[j]
				G.ShVs[k] <- soccern$G.Sh[j]
				PAttVs[k] <- soccern$PAtt[j]
				PCmp.Vs[k] <- soccern$PCmp.[j]
				SPAttVs[k] <- soccern$SPAtt[j]
				SPCmp.Vs[k] <- soccern$SPCmp.[j]
				MPAttVs[k] <- soccern$MPAtt[j]
				MPCmp.Vs[k] <- soccern$MPCmp.[j]
				LPAttVs[k] <- soccern$LPAtt[j]
				LPCmp.Vs[k] <- soccern$LPCmp.[j]
				ToDef3rdVs[k] <- soccern$ToDef3rd[j]
				ToMid3rdVs[k] <- soccern$ToMid3rd[j]
				ToAtt3rdVs[k] <- soccern$ToAtt3rd[j]
				ToAttPenVs[k] <- soccern$ToAttPen[j]
				ToDistVs[k] <- soccern$TotDist[j]
				FlsVs[k] <- soccern$Fls[j]
				FldVs[k] <- soccern$Fld[j]
				CrsVs[k] <- soccern$Crs[j]
				IntVs[k] <- soccern$Int[j]
				TklWinVs[k] <- soccern$TklWin[j]
				RecovVs[k] <- soccern$Recov[j]
				k <- k + 1
			}      
		}
	}
	else{
		del[z] <- i
		z <- z + 1
	}
}
\end{lstlisting}
\bigskip
\section{Codice per la creazione del data.frame Team} \label{sec:a2}
\begin{lstlisting}[language=R]
	> soccern3$Team <- data.frame(team = soccern3$Team, GF = soccern3$GF, GA = soccern3$GA,  at.home = 1, Poss = soccern3$Poss, Sh = soccern3$Sh, SoT = soccern3$SoT, G.Sh = soccern3$G.Sh, PAtt = soccern3$PAtt, PCmp. = soccern3$PCmp., SPAtt = soccern3$SPAtt, SPCmp. = soccern3$SPCmp., MPAtt = soccern3$MPAtt, MPCmp. = soccern3$MPCmp., LPAtt = soccern3$LPAtt, LPCmp. = soccern3$LPCmp., ToDef3rd = soccern3$ToDef3rd, ToAtt3rd = soccern3$ToAtt3rd, ToAttPen = soccern3$ToAttPen, TotDist = soccern3$TotDist, Fls = soccern3$Fls, Fld = soccern3$Fld, Crs = soccern3$Crs, Int = soccern3$Int, TklWin = soccern3$TklWin, Recov = soccern3$Recov)
\end{lstlisting}
\bigskip
\section{Codice per la creazione del data.frame Vs} \label{sec:a3}
\begin{lstlisting}[language=R]
	> soccern3$Vs <- data.frame(team = soccern3$Vs, GF = GFVs, GA = GAVs, at.home = 0, Poss = PossVs, Sh = ShVs, SoT = ShTVs, G.Sh = G.ShVs, PAtt = PAttVs, PCmp. = PCmp.Vs, SPAtt = SPAttVs, SPCmp. = SPCmp.Vs, MPAtt = MPAttVs, MPCmp. = MPCmp.Vs, LPAtt = LPAttVs, LPCmp. = LPCmp.Vs, ToDef3rd = ToDef3rdVs, ToAtt3rd = ToAtt3rdVs, ToAttPen = ToAttPenVs, TotDist = ToDistVs, Fls = FlsVs, Fld = FldVs, Crs = CrsVs, Int = IntVs, TklWin = TklWinVs, Recov = RecovVs)
\end{lstlisting}
 %R
\chapter{Codice in Python}
\section{createTable}
Codice per la creazione del dataset utilizzabile con i metodi di machine learning.
\begin{lstlisting}[language=Python, caption={Codice per la creazione del dataset utilizabile con i metodi di machine learning.}, captionpos=b, label=code:a9]
	def createTable(df, dfM): 
		for i in range(0, df.shape[0], 1):
			if df.AtHome[i] == True: 
				for j in range(0, df.shape[0], 1): 
					if df.Team[j] == df.Vs[i] and df.Team[i] == df.Vs[j] and df.AtHome[j] == False:
					dfM2 = pd.DataFrame({'Date': [df.Date[i]],
						'Round': [df.Round[i]],
						'AtHome': [df.AtHome[i]],
						'Result': [df.Res[i]],
						'G_Home': [df.GF[i]],
						'G_Away': [df.GA[i]],
						'Home_Team': [df.Team[i]],
						'Away_Team': [df.Vs[i]],
						'Home_Poss': [df.Poss[i]],
						'Home_Sh': [df.Sh[i]],
						'Home_SoT': [df.SoT[i]],
						'Home_G/Sh': [df['G/Sh'][i]],
						'Home_Saves': [df.Saves[i]],
						'Home_PAtt': [df.PAtt[i]],
						'Home_PCmp%': [df['PCmp%'][i]],
						'Home_SPAtt': [df.SPAtt[i]],
						'Home_SPCmp%': [df['SPCmp%'][i]],
						'Home_MPAtt': [df.MPAtt[i]],
						'Home_MPCmp%': [df['MPCmp%'][i]],
						'Home_LPAtt': [df.LPAtt[i]],
						'Home_LPCmp%': [df['LPCmp%'][i]],
						'Home_ToDefPen': [df.ToDefPen[i]],
						'Home_ToDef3rd': [df.ToDef3rd[i]],
						'Home_ToMid3rd': [df.ToMid3rd[i]],
						'Home_ToAtt3rd': [df.ToAtt3rd[i]],
						'Home_ToAttPen': [df.ToAttPen[i]],
						'Home_TotDist': [df.TotDist[i]],
						'Home_Fls': [df.Fls[j]],
						'Home_Fld': [df.Fld[i]],
						'Home_Off': [df.Off[i]],
						'Home_Crs': [df.Crs[j]],
						'Home_Int': [df.Int[j]],
						'Home_TklWin': [df.TklWin[i]],
						'Home_Recov': [df.Recov[i]],
						'Away_Poss': [df.Poss[j]],
						'Away_Sh': [df.Sh[j]],
						'Away_SoT': [df.SoT[j]],
						'Away_G/Sh': [df['G/Sh'][j]],
						'Away_Saves': [df.Saves[j]],
						'Away_PAtt': [df.PAtt[j]],
						'Away_PCmp%': [df['PCmp%'][j]],
						'Away_SPAtt': [df.SPAtt[j]],
						'Away_SPCmp%': [df['SPCmp%'][j]],
						'Away_MPAtt': [df.MPAtt[j]],
						'Away_MPCmp%': [df['MPCmp%'][j]],
						'Away_LPAtt': [df.LPAtt[j]],
						'Away_LPCmp%': [df['LPCmp%'][j]],
						'Away_ToDefPen': [df.ToDefPen[j]],
						'Away_ToDef3rd': [df.ToDef3rd[j]],
						'Away_ToMid3rd': [df.ToMid3rd[j]],
						'Away_ToAtt3rd': [df.ToAtt3rd[j]],
						'Away_ToAttPen': [df.ToAttPen[j]],
						'Away_TotDist': [df.TotDist[j]],
						'Away_Fls': [df.Fls[j]],
						'Away_Fld': [df.Fld[j]],
						'Away_Off': [df.Off[j]],
						'Away_Crs': [df.Crs[j]],
						'Away_Int': [df.Int[j]],
						'Away_TklWin': [df.TklWin[j]],
						'Away_Recov': [df.Recov[j]]})
		dfM = dfM.append(dfM2, ignore_index=True) 
	return dfM
	
\end{lstlisting}

\section{Librerie}
\begin{itemize}
	\item \textbf{Sklearn}
	\item \textbf{Matplotlib}
	\item \textbf{NumPy}
	\item \textbf{Pandas}
\end{itemize} %Python
\appendix                               
%5\input{capitoli/capitolo-A}             % Appendice A

%**************************************************************
% Materiale finale
%**************************************************************
\backmatter
%\printglossary[type=acronym, title={Acronimi e abbreviazioni}]
%\printglossary[type=main, title={Glossario}]
% !TEX encoding = UTF-8
% !TEX TS-program = pdflatex
% !TEX root = ../tesi.tex

%**************************************************************
% Bibliografia
%**************************************************************
% Bibliografia
%**************************************************************

\cleardoublepage
\chapter{Bibliografia}
\nocite{*}
\DeclareFieldFormat[article]{volume}{\mkbibbold{#1}}
\DeclareFieldFormat[article]{pages}{#1}
\DeclareFieldFormat[article]{number}{\mkbibparens{#1}}
\DeclareFieldFormat[article]{title}{#1} 

\defbibfilter{papers}{
	type=article or
	type=book or
	type=manual
}

% Stampa i riferimenti bibliografici
%\printbibliography[heading=subbibliography,title={Bibliographical references},type=article]
\printbibliography[heading=subbibliography,filter=papers]
\chapter{Sitography}

% Stampa i siti web consultati
%\printbibliography[heading=subbibliography,title={Websites consulted},type=online]
\printbibliography[heading=subbibliography,type=online]

\end{document}
